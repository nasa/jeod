\section{Commands and Environments}
\subsection{Commands}
\begin{description}
\item[\command{addmodel}\cmdarg{NAME}\cmdarg{dir}\cmdarg{name}\cmdarg{title}]
  \hfill [dynenvupfront.sty] \\
  Word
\item[\command{backmatter}\cmdopt{options}\cmdarg{list,of,files}]
  \hfill [dynenvmatter.sty] \\
\item[\command{boilerplatechapterone\cmdarg{description}\cmdarg{history}}]
  \hfill [dynenvboilerplate.sty] \\
  Argument \#1 - Contents of section 1.1 \\
  Argument \#2 - Table entries for history table. \\
  See section~\ref{sec:chapterone} for details.
\item[\command{boilerplateinventory}] \hfill [dynenvboilerplate.sty] \\
  Word
\item[\command{boilerplatemetrics}] \hfill [dynenvboilerplate.sty] \\
  Word
\item[\command{boilerplatetraceability}] \hfill [dynenvboilerplate.sty] \\
  Word
\item[\command{escapeus}\cmdarg{text\_with\_underscores}]
  \hfill [dynenvupfront.sty] \\
  Word
\item[\command{iflabeldefined}\cmdarg{label}\cmdarg{commands}]
  \hfill [dynenvboilerplate.sty] \\
  Argument \#1 - Label to be tested.\\
  Argument \#2 - Code to be expanded. \\
  Expands second argument if label specified by first argument is defined.
\item[\command{JEODHOME}] \hfill [paths.def] \\
  Path to \$JEODHOME.
\item[\command{frontmatter}\cmdopt{options}\cmdarg{abstractOrSummary}]
  \hfill [dynenvmatter.sty] \\
\item[\command{inspection}\cmdarg{name}] \hfill [dynenvreqt.sty] \\
  Words.
\item[\command{longentry}] \hfill [dynenvboilerplate.sty] \\
  Causes broken lines in a |longtable| environment to be printed with a
  hanging indent.
\item[\command{mainmatter}\cmdopt{options}\cmdarg{list,of,files}]
  \hfill [dynenvmatter.sty] \\
\item[\command{MODELDIR}] \hfill [paths.def] \\
  Model directory, with underscores escaped.
\item[\command{MODELDIRx}] \hfill [dynenv.sty] \\
  |{\MODELDIR\xspace}|
\item[\command{MODELDOCS}] \hfill [paths.def] \\
  Relative path to model documentation directory.
\item[\command{MODELGROUP}] \hfill [paths.def] \\
  Model group (same as type except for utils).
\item[\command{MODELGROUPx}] \hfill [dynenv.sty] \\
  |{\MODELGROUPx\xspace}|
\item[\command{ModelHistory}] \hfill [\inanglebrackets{model\_name}.sty] \\
  History table entries.
\item[\command{MODELHOME}] \hfill [paths.def] \\
  Relative path to model directory.
\item[\command{MODELNAME}] \hfill [paths.def] \\
  Model directory, underscores not escaped.
\item[\command{MODELPATH}] \hfill [paths.def] \\
  Path to model directory from \$JEODHOME.
\item[\command{MODELPATHx}] \hfill [dynenv.sty] \\
  |{\MODELPATH\xspace}|
\item[\command{ModelPrefix}] \hfill [\inanglebrackets{model\_name}.sty] \\
  One-word prefix for requirements, etc.
\item[\command{MODELTITLE}] \hfill [paths.def] \\
  All-caps model name command.
\item[\command{MODELTITLEx}] \hfill [dynenv.sty] \\
  |{\MODELTITLE\xspace}|
\item[\command{MODELTYPE}] \hfill [paths.def] \\
  Model type (e.g., dynamics, environment).
\item[\command{MODELTYPEx}] \hfill [dynenv.sty] \\
  |{\MODELTYPE\xspace}|
\item[\command{simpletracetable}\cmdarg{one}\cmdarg{two}\cmdarg{three}]
  \hfill [dynenvreqt.sty] \\
  Words.
\item[\command{requirement}\cmdarg{name}] \hfill [dynenvreqt.sty] \\
  Words.
\item[\command{subrequirement}\cmdarg{name}] \hfill [dynenvreqt.sty] \\
  Words.
\item[\command{test}\cmdarg{name}] \hfill [dynenvreqt.sty] \\
  Words.
\item[\command{traceref}\cmdarg{label}]
  \hfill [dynenvreqt.sty] \\
  Words.
\item[\command{tracerefrange}\cmdarg{label1}\cmdarg{label2}]
  \hfill [dynenvreqt.sty] \\
  Words.
\item[\command{tracetable}\cmdarg{one}\cmdarg{two}\cmdarg{three}]
  \hfill [dynenvreqt.sty] \\
  Words.
\end{description}

\subsection{Environments}
\begin{description}
\item[\option{abstract}] \hfill [dynenvmatter.sty] \\
  The abstract environment defined in |report.cls| issues an almost
  harmless |\titlepage| command,  the harm being that it resets the page
  number to one (or rather, roman i). Redefining this environment avoids
  this problem.
\item[\option{codeblock}] \hfill [dynenvcode.sty] \\
  The codeblock environment is a specialization of the Verbatim environment.
  Use this environment for code to be printed inline with the text. The
  printed code will be indented with respect to the current indentation level.
\item[\option{codebox}] \hfill [dynenvcode.sty] \\
  The codebox environment is a specialization of the Verbatim environment.
  Use this environment for code to be printed in an exhibit. The code will
  be framed in a box and is indented slightly with respect to page boundaries.
\item[\option{description:}] \hfill [dynenvreqt.sty] \\
  The description: environment is a specialization of the description
  environment intended for use with requirements.
\item[\option{exhibit}] \hfill [dynenvcode.sty] \\
  The exhibit environment is similar to the table environment.
  Exhibits, like tables, can have captions and labels. The caption should
  be placed above the exhibited items. The exhibit will be listed as a part
  of the list of tables.
\item[\option{stretchlongtable}] \hfill [dynenvboilerplate.sty] \\
  A | stretchlongtable | is a |longtable| environment that takes an optional
  |arraystretch| (default: 1.2). If specified, the |arraystretch| value must be
  specified in angle brackets and must precede the optional arguments to the
  |longtable| environment.
\end{description}
