%%%%%%%%%%%%%%%%%%%%%%%%%%%%%%%%%%%%%%%%%%%%%%%%%%%%%%%%%%%%%%%%%%%%%%%%%%%%%%%%
% BodyAction_intro.tex
% Intro chapter of the BodyAction part of the Body Action Model
%
%%%%%%%%%%%%%%%%%%%%%%%%%%%%%%%%%%%%%%%%%%%%%%%%%%%%%%%%%%%%%%%%%%%%%%%%%%%%%%%%
\chapter{Introduction}\label{ch:BodyAction:intro}

\section{Purpose and Objectives of the \ModelDesc Base Classes}
The \ModelDesc defines two base classes: BodyAction and BodyActionMessages.
The latter class encapsulates identifiers for use with the MessageHandler.
All of the instantiable \ModelDesc classes ultimately derive from the BodyAction
base class. This part of the document describes these base classes.

The BodyAction base class by itself does not change a single aspect of
a MassBody or DynBody object. Making changes to a MassBody, DynBody, or derived class object is the
responsibility of the various classes that derive from the base BodyAction
class. What the base class does provide is a common framework for describing
actions to be  performed on bodies.

\section{Part Organization}
This part of the \ModelDesc document is organized along the
lines described in section \ref{sec:overview:docorg}. It
comprises the following chapters in order:

\begin{description}
\item[Introduction] -
This introduction describes the objective and purpose of the base classes.

\item[Product Requirements] -
The \ModelDesc Base Classes Product Requirements chapter describes
the requirements on the model base classes
and the requirements levied on classes that derive from the BodyAction class.

\item[Product Specification] -
The \ModelDesc Base Classes Product Specification chapter describes
the underlying theory, architecture, and design of the
BodyAction class.

\item[User Guide] -
Describes the use of the BodyAction class.
The BodyAction User Guide chapter is organized in
the following sections:
\begin{itemize}
 \item Analysts or users of simulations (Analysis).
 \item Integrators or developers of simulations (Integration).
 \item Model Extenders (Extension).
\end{itemize}

\item[Inspection, Verification, and Validation] -
The BodyAction class is not an instantiable class.
As a result, all of the BodyAction class requirements involve inspection only.
\end{description}
