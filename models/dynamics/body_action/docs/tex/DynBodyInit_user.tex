%%%%%%%%%%%%%%%%%%%%%%%%%%%%%%%%%%%%%%%%%%%%%%%%%%%%%%%%%%%%%%%%%%%%%%%%%%%%%%%%
% DynBodyInit_user.tex
% User Guide for the DynBodyInit and derived classes.
%
%%%%%%%%%%%%%%%%%%%%%%%%%%%%%%%%%%%%%%%%%%%%%%%%%%%%%%%%%%%%%%%%%%%%%%%%%%%%%%%%

\chapter{User Guide}\label{ch:\modelpartid:user}
The following assumes the scheme outlined in
section~\ref{subsec:recommended_practice}.

\section{Analysis}

The following examples illustrate various uses of the sub-model to
initialize one of the vehicles. The examples assume
a simulation S\_define file that declares simulation objects
named `launchpad', `target', `chaser', and `dynamics'.
The launchpad, target, and chaser objects
each contain a DynBody object name `dyn\_body'
and a number of objects of a type derived from DynBodyInit.
The dynamics object contains a DynManager object named `manager'
and a BodyAction pointer named `body\_action\_ptr'.
See the Integration section below for a detailed description of the
S\_define file.

\subsection{DynBodyInitOrbit}
The DynBodyInitOrbit class initializes translational state
using a variety of orbital elements sets.
The following specifies the orbit in terms of
semi-major axis, eccentricity, inclination,
right ascension of ascending node, argument of periapsis, and the time
since periapsis passage.

\begin{verbatim}
target.orb_init.dyn_subject     = &target.dyn_body;
target.orb_init.action_name = "target.orb_init";

target.orb_init.reference_ref_frame_name = "Earth.inertial";
target.orb_init.body_frame_id = "composite_body";

target.orb_init.set = DynBodyInitOrbit::SmaEccIncAscnodeArgperTimeperi;
target.orb_init.orbit_frame_name = "Earth.inertial";
target.orb_init.planet_name      = "Earth";

target.orb_init.arg_periapsis {d}       =  100.582445989;
target.orb_init.eccentricity            =    0.00129073350;
target.orb_init.inclination {d}         =   51.670450765;
target.orb_init.ascending_node {d}      =   49.708417385;
target.orb_init.semi_major_axis {km}    = 6732.90120152;
target.orb_init.time_periapsis {s}      = 4581.96167293;

dynamics.body_action_ptr = &target.orb_init;
call dynamics.dynamics.manager.add_body_action (dynamics.body_action_ptr);
\end{verbatim}

\subsection{DynBodyInitTranState}
The DynBodyInitTranState class initializes translational state
given the state relative to some known reference frame.
The following initializes the chaser relative to the target
using a point-to-point state specification
and initializes the target given an Earth-centered, Earth-fixed
state specification.

Note the order in which these two initializations are specified.
The chaser cannot be initialized until the target is initialized.
There is nothing wrong with adding initializers in an out-of-order
sequence as is done in this example. The Dynamics Manager will
sort things out.

\begin{verbatim}
chaser.trans_init.subject = &chaser.dyn_body.mass;
chaser.trans_init.action_name = "chaser.trans_state_tpoint_cpoint";

chaser.trans_init.reference_ref_frame_name = "target.attach_point";
chaser.trans_init.body_frame_id = "attach_point";

chaser.trans_init.position[0] {M}   = 100.0, 0.0, 5.0;
chaser.trans_init.velocity[0] {M/s} = -1.0, 0.0, 0.0;

dynamics.body_action_ptr = &chaser.trans_init;
call dynamics.dynamics.manager.add_body_action (dynamics.body_action_ptr);


target.trans_init.dyn_subject = &target.dyn_body;
target.trans_init.action_name = "target.trans_state_pfix_body";

target.trans_init.reference_ref_frame_name = "Earth.pfix";
target.trans_init.body_frame_id = "composite_body";

target.trans_init.position[0] {M}   =
    5406298.5700, -2074684.5600,  3426540.0300;
target.trans_init.velocity[0] {M/s} =
    -706.0655810,  5764.3071620,  4587.2461630;

dynamics.body_action_ptr = &target.trans_init;
call dynamics.dynamics.manager.add_body_action (dynamics.body_action_ptr);
\end{verbatim}

\subsection{DynBodyInitRotState}
The DynBodyInitRotState class initializes rotational state
given the state relative to some known reference frame.
The following initializes the target attitude rate
given an Earth-centered inertial state specification.

\begin{verbatim}
target.rot_init.subject = &target.dyn_body.mass;
target.rot_init.action_name = "target.rate_state_inertial_body";

target.rot_init.reference_ref_frame_name = "Earth.inertial";
target.rot_init.body_frame_id = "composite_body";

target.rot_init.state_items = DynBodyInitRotState::Rate;
target.rot_init.ang_velocity[0] {d/s}  =
    0.002,
    0.006 - 0.06556131568278,
   -0.003;

dynamics.body_action_ptr = &target.rot_init;
call dynamics.dynamics.manager.add_body_action (dynamics.body_action_ptr);
\end{verbatim}


\subsection{DynBodyInitLvlhState}
The DynBodyInitLvlhState class initializes state
given the state relative to some rectilinear or curvilinear LVLH
reference frame. The classes DynBodyInitLvlhTransState and
DynBodyInitLvlhRotState derive from DynBodyInitLvlhState and only set
translational and rotational aspects of the state; DynBodyInitLvlhState
can set the full state.

The DynBodyInitTranState and DynBodyInitRotState classes cannot
be used with these LVLH frames because derived frames
are not known at initialization time. If supplied, the DynBodyInitLvlhState class
uses an existing user-specified LVLH reference frame. Otherwise, it constructs
a temporary one to be used only for initialization.

The following initializes the chaser's full state with respect to the
target LVLH frame.

\begin{verbatim}
target.lvlh.subject_name    = "target.composite_body"
target.lvlh.planet_name     = "Earth"

# Choose Rectilinear or CircularCurvilinear
chaser.lvlh_init.lvlh_type = trick.LvlhType.Rectilinear;
# chaser.lvlh_init.lvlh_type = trick.LvlhType.CircularCurvilinear;
chaser.lvlh.subject_name    = "chaser.composite_body"
chaser.lvlh.planet_name     = "Earth"

chaser.lvlh_init.subject = chaser.dyn_body.mass
chaser.lvlh_init.action_name = "chaser_lvlh_init"
chaser.lvlh_init.planet_name = "Earth"
chaser.lvlh_init.set_lvlh_frame_object (target.lvlh)
chaser.lvlh_init.position  = trick.attach_units( "m",[ -100.0, 25.0, 7.5])
chaser.lvlh_init.velocity  = trick.attach_units( "m/s",[ 0.0, 0.0, 1.0])

chaser.lvlh_init.orientation.data_source    = trick.Orientation.InputEulerRotation
chaser.lvlh_init.orientation.euler_sequence = trick.Orientation.Yaw_Pitch_Roll
chaser.lvlh_init.orientation.euler_angles   = trick.attach_units( "degree",[ 0.0, 0.0, 0.0])
chaser.lvlh_init.ang_velocity               = trick.attach_units( "degree/s",[ 0.0, 0.0, 4.5])

dynamics.dyn_manager.add_body_action (chaser.lvlh_init);
\end{verbatim}


\subsection{DynBodyInitNedState}

The DynBodyInitNedState class initializes state
given the state relative to North-East-Down reference frame.
The classes DynBodyInitNedTransState and DynBodyInitNedRotState
derive from DynBodyInitNedState and only set translational and
rotational aspects of the state; DynBodyInitNedState can set the
full state.

Like the LVLH frames described above,
North-East-Down frames are derived frames.
The DynBodyInitNedState class
constructs the North-East-Down reference frame
so that it can be used for initialization.

The following initializes the full state of the launchpad
with respect to a specified point on the Earth
and initializes the chaser's translational state relative to the launchpad.

\begin{verbatim}
launchpad.ned_init.dyn_subject = &launchpad.dyn_body;
launchpad.ned_init.action_name = "launchpad.ned_state_ref_struct";

launchpad.ned_init.reference_ref_frame_name = "Earth.inertial";
launchpad.ned_init.body_frame_id = "structure";
launchpad.ned_init.planet_name = "Earth";

launchpad.ned_init.ref_point.altitude {M}  =   3.0;
launchpad.ned_init.ref_point.latitude {d}  =  28.6082;
launchpad.ned_init.ref_point.longitude {d} = -80.6040;
launchpad.ned_init.altlatlong_type = NorthEastDown::elliptical;

launchpad.ned_init.set_items = RefFrameItems::Pos_Vel_Att_Rate;
launchpad.ned_init.position[0] {M}   =  0.0,  0.0, 10.0;
launchpad.ned_init.velocity[0] {M/s} =  0.0,  0.0,  0.0;
launchpad.ned_init.orientation.data_source = Orientation::InputEulerRotation;
launchpad.ned_init.orientation.euler_sequence = Pitch_Yaw_Roll;
launchpad.ned_init.orientation.euler_angles[0] {d} = 0.0, 0.0, 0.0;
launchpad.ned_init.ang_velocity[0] {d/s} = 0.0, 0.0, 0.0;

dynamics.body_action_ptr = &launchpad.ned_init;
call dynamics.dynamics.manager.add_body_action (dynamics.body_action_ptr);


chaser.ned_trans_init.dyn_subject = &chaser.dyn_body;
chaser.ned_trans_init.action_name = "chaser.trans_ned_launch_pad_struct";

chaser.ned_trans_init.reference_ref_frame_name = "Earth.inertial";
chaser.ned_trans_init.body_frame_id = "structure";
chaser.ned_trans_init.planet_name = "Earth";
chaser.ned_trans_init.ref_body_name = "launchpad";
chaser.ned_trans_init.altlatlong_type = NorthEastDown::elliptical;

chaser.ned_trans_init.position[0] {M}   = 0.0, 0.0,  -40.0;
chaser.ned_trans_init.velocity[0] {M/s} = 0.0, 0.0, -100.0;

dynamics.body_action_ptr = &chaser.ned_trans_init;
call dynamics.dynamics.manager.add_body_action (dynamics.body_action_ptr);
\end{verbatim}

\section{Integration}

The main concern of the simulation integrator with respect to this
model is making enough initializers available to the simulation users
so that the users can easily initialize the vehicles in the simulation.
Segments of the S\_define file used in the above examples are shown below.
The launchpad and chaser simulation objects are not shown. The launchpad is a
very pared-down version of the target (it only needs a DynBodyInitNedState
object) while the chaser has the same set of initializers as does the target.
This is most likely overkill. Most simulations will not need the plethora
of initializers used here.
\begin{verbatim}
/**
 * sim object target
 */
sim_object {

   //
   // Data structures
   //
   dynamics/dyn_body:         DynBody dyn_body;
   dynamics/body_action:      MassBodyInit mass_init;

   dynamics/body_action:      DynBodyInitOrbit orb_init;
   dynamics/body_action:      DynBodyInitTransState trans_init;
   dynamics/body_action:      DynBodyInitRotState rot_init;
   dynamics/body_action:      DynBodyInitLvlhState lvlh_init;
   dynamics/body_action:      DynBodyInitLvlhTransState lvlh_trans_init;
   dynamics/body_action:      DynBodyInitLvlhRotState lvlh_rot_init;
   dynamics/body_action:      DynBodyInitNedState ned_init;
   dynamics/body_action:      DynBodyInitNedTransState ned_trans_init;
   dynamics/body_action:      DynBodyInitNedRotState ned_rot_init;

   ...

} target;


/**
 * sim object chaser -- not shown; it is a replicate of the target
 */

/**
 * sim object dynamics
 */
sim_object {

   //
   // Data structures
   //
   dynamics/dyn_manager: DynManager          manager;
   dynamics/dyn_manager: DynManagerInit      manager_init;
   dynamics/body_action: BodyAction        * body_action_ptr;
   sim_services/include: INTEGRATOR          integ;
   utils/message:        TrickMessageHandler msg_handler;

   //
   // Jobs registered for input file activation.
   //
   (0.0, environment) dynamics/dyn_manager:
   dynamics.manager.add_body_action (
      Inout BodyAction * body_action = dynamics.body_action_ptr);

   //
   // Initialization jobs
   //
   P_MNGR (initialization) dynamics/dyn_manager:
   dynamics.manager.initialize_model (
      Inout INTEGRATOR *     integ = &dynamics.integ,
      In    DynManagerInit & init  =  dynamics.manager_init);

   P_BODY (initialization) dynamics/dyn_manager:
   dynamics.manager.initialize_simulation ();

   ...

} dynamics;
\end{verbatim}


\section{Extension}
Extensibility was one of the key drivers in the development of this sub-model.

Suppose you as a simulation integrator find that your users insist on
having the ability to initialize state with respect to the inertial frame that
is instantaneously colinear and co-moving with some reference vehicle
reference frame. JEOD does not supply this capability, so how to supply it?

The answer is to create a class that derives from DynBodyInit.
This new class will need data member(s) that
specify the reference vehicle, the target reference frame on that vehicle,
and possibly the states the user wants to initialize.
The new class will also need to override four virtual member functions.

The {\tt initialize()} method should check for errors in the specification
of the reference vehicle and the reference reference frame and then should
call the {\tt DynBodyInit::initialize()} method.

The default {\tt initializes\_what()} method says that the initializer
initializes nothing. You will certainly want to override that. Exactly
what state elements are to be initialized by this new class is between you
and your users. The overridden {\tt initializes\_what()} method needs to
reflect this decision.

This new class needs to wait until the reference position and velocity are
known. The default {\tt is\_ready()} method does not know this.
You will need to override the {\tt is\_ready()} method to make it wait.
To conform with the model requirements, the overridden method needs to
query {\tt DynBodyInit::is\_ready()} before it returns {\tt true}.

Finally, you will need to override the {\tt apply()} method. One approach is to
\begin{enumerate}
\item Create a new reference frame that represents this collinear and co-moving
inertial frame. A simply way to do this is to declare a RefFrame object as
a local variable. Note that a reference frame initializes to having a null
translation and rotational with respect to its parent.
\item Give this frame a name such as ``vehicle.inertial'', with the
vehicle name substituted for vehicle.
Various JEOD methods do not like reference frames that don't have a name.
\item (Temporarily) attach this new reference frame to the reference frame tree.
In this case, attaching the frame to the reference body's integration frame
will make that collinear and co-moving requirement easy to satisfy.
\item Copy the position and velocity from the reference reference frame to
this new frame. Voila! This is now an inertial frame that is collinear and
co-moving with the reference vehicle.
\item Point the inherited {\tt reference\_ref\_frame} data member to this
newly created frame.
\item Call the inherited {\tt apply\_user\_inputs()} method to interpret the
user inputs in terms of this frame.
\item Call the overridden {\tt DynBodyInit::apply()} method to initialize
the subject vehicle state.
\item Reset the inherited {\tt reference\_ref\_frame} data member to NULL.
The item to which it points is about to go out of scope.
\item Return. The reference frame created by declaring a RefFrame variable
will go out of scope and be destroyed. The RefFrame destructor automatically
removes the frame from the reference frame tree.
\end{enumerate}
