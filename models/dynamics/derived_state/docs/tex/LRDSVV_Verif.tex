%%%%%%%%%%%%%%%%%%%%%%%%%%%%%%%%%%%%%%%%%%%%%%%%%%%%%%%%%%%%%%%%%%%%%%%%%%%%%%%%%
%
% Purpose:  Verification part of V&V for the LVLH relative derived state model
%
%
%%%%%%%%%%%%%%%%%%%%%%%%%%%%%%%%%%%%%%%%%%%%%%%%%%%%%%%%%%%%%%%%%%%%%%%%%%%%%%%%

% \section{Verification}

%%% code imported from old template structure
%\inspection{<Name of Inspection>}\label{inspect:<label>}
% <description> to satisfy
% requirement \ref{reqt:<label>}.

A simulation was developed to test the \LRDSDesc. The simulation
includes two vehicles, and LVLH relative derived states, both rectilinear and
curvilinear, are computed. A software program is provided to check the derived
state outputs in the generic cases.  Two run cases are
provided; one generic which tests most aspects of the model which can be
independently verified by the software program and a second case which allows
verification of the rotational curvilinear state by inspection.

\test{Verification of \LRDSDesc\ Output Data for Rectilinear LVLH}\label{test:RLVLH}

\begin{description}
\item{Purpose:}\newline
To demonstrate that the output of the \LRDSDesc\ in the rectilinear case is
consistent with existing LVLH code.
\item{Requirements:}\newline
Satisfactory conclusion of this test satisfies requirement \ref{reqt:RLVLH}

\item{Procedure:}\newline
The data from the rectilinear LVLH output was compared against the
\LVLHDesc\ for the following features:
\begin{enumerate}
 \item  {Position.}
 \item  {Velocity.}
 \item  {Angular Velocity.}
 \item {Calculation of Transformation matrix elements.}
\end{enumerate}

\item{Predictions:}
The values should be identical.

\item{Results:}\ \newline
The outputs were identical.
\end{description}

\test{Verification of \LRDSDesc\ Output Data for Curvilinear LVLH}\label{test:CLVLH}

\begin{description}
\item{Purpose:}\newline
To demonstrate that the output of the \LRDSDesc\ in the curvilinear case
matches theoretical predictions.
\item{Requirements:}\newline
Satisfactory conclusion of this test satisfies requirement \ref{reqt:CLVLH}

\item{Procedure:}\newline
The data from the curvilinear output was compared against that of a computer
program which used an independent method for calculating the following features:
\begin{enumerate}
 \item  {Position.}
 \item  {Velocity.}
 \item  {Angular Momentum.}
\end{enumerate}

\item{Predictions:}
The values should match to machine precision.

\item{Results:}\ \newline
The outputs matched to machine precision.

\item {Hand Calculation of Transformation Matrix}

A second run case was included in which the transformation from curvilinear
LVLH to the subject frame could easily be calculated by hand. The output of
the \LRDSDesc\ was identical to the hand calculations. The case was constructed
in such a way that the curvilinear and rectilinear frames would differ by a
$90$ degree rotation, thus accentuating the effectiveness of the test.
\end{description}
