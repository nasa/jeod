\inspection{Top-level Inspection}
\label{inspect:TLI}
This document structure, the code, and associated files have been inspected, and together satisfy requirement~\ref{reqt:toplevel}.

\inspection{Design Inspection}
\label{inspect:design}
Table~\ref{tab:design_inspection} summarizes the key elements of the
implementation of the \ModelDesc that satisfy requirements levied on the model.
By inspection, the \ModelDesc satisfies
requirements~\ref{reqt:mass_requirements} to~\ref{reqt:attach_detach}.

\begin{longtable}{||l @{\hspace{4pt}} p{1.38in} |p{3.95in}|}
\caption{Design Inspection}
\label{tab:design_inspection} \\[6pt]
\hline
\multicolumn{2}{||l|}{\bf Requirement} & \bf{Satisfaction}
\\ \hline\hline
\endfirsthead

\caption[]{Design Inspection (continued from previous page)} \\[6pt]
\hline
\multicolumn{2}{||l|}{\bf Requirement} & \bf{Satisfaction}
\\ \hline\hline
\endhead

\hline \multicolumn{3}{r}{{Continued on next page}} \\
\endfoot

\hline
\endlastfoot

\ref{reqt:mass_requirements} & Mass &
  The class DynBody utilizes functionality from the MassBody class through
  friendship. The friendship is not inheritable, thereby granting DynBody
  objects full access MassBody functionality while conserving core
  behavior for user-derived classes.
\tabularnewline[4pt]
\ref{reqt:integ_frame} & Integration Frame &
  The DynBody class data member \verb+integ_frame+ points to the integration
  frame for a particular DynBody object. This member is set at initialization
  time by name. Upon attachment as a child to another DynBody object, the
  integration frame for the child bodies are set to that of the root body.
  The integration frame can be changed dynamically via the
  \verb+switch_integration_frames()+ member function.
\tabularnewline[4pt]
\ref{reqt:state_representation} & State Representation &
  The class BodyRefFrame derives from the RefFrame class.
  The inheritance is public, thereby granting external users of a DynBody
  object full access to the public RefFrame functionality of the BodyRefFrame
  objects contained in a DynBody object.
  A DynBody object contains three BodyRefFrame objects that represent the
  structure frame, core body frame, and the composite body frame.
  In addition to these three basic frames, a DynBody object contains an
  STL list of BodyRefFrame pointers that represent additional points
  of interest associated with the DynBody object.
\tabularnewline[4pt]
\ref{reqt:staged_initialization} & Staged Initialization &
  The BodyRefFrame \verb+initialized_items+ data member indicates which
  elements of a BodyRefFrame object's state have been set.
  The \ModelDesc methods described in section~\ref{sec:detailed_state_prop}
  ensure that a BodyRefFrame object's \verb+initialized_items+ properly
  reflect which of the BodyRefFrame  object's state elements have been set.
\tabularnewline[4pt]
\ref{reqt:eom} & Equations of Motion &
  This requirement has four sub-requirements that specify the treatment of
  forces, torques, translational acceleration, and rotational acceleration.
  Sections~\ref{sec:detailed_force_torque} to~\ref{sec:detailed_eom}
  describe the treatment of forces and torques in the \ModelDesc and
  how these lead to the development of the equations of motion.
\tabularnewline[4pt]
\ref{reqt:state_integ_prop} &
  \raggedright State Integration \\ and Propagation &
  As described in section~\ref{sec:detailed_state_integ}, the DynBody
  class provides the ability to integrate the composite body frame associated
  with the root body of a composite body or of an isolated (unconnected) body.
  The integrated state is propagated throughout a DynBody tree as described
  in section~\ref{sec:detailed_state_prop}.
\tabularnewline[4pt]
\ref{reqt:vehicle_points} & Vehicle Points &
  The DynBody class mirrors the MassBody \verb+add_mass_point()+ method.
  This creates a BodyRefFrame that corresponds to the MassPoint,
  thereby making these registered points of interest have a corresponding
  state. State is propagated to all reference frames associated with a DynBody
  object, including these vehicle points. The DynBody method
  \verb+compute_vehicle_point_derivatives()+ computes accelerations
   for a specific vehicle point as required.
\tabularnewline[4pt]
\ref{reqt:attach_detach} & Attach/Detach &
  As described in section ~\ref{sec:detailed_attach_detach}, the DynBody
  class implements attach/detach member functions in addition to the Massbody attach/detach methods
  to provide the the required capability to attach and detach DynBody objects and to do
  so in a manner consistent with the laws of physics. The DynBody class also provides attachment methods for attaching
  to a RefFrame object
\tabularnewline[4pt]
\end{longtable}

\inspection{Mathematical Formulation}
\label{inspect:math}
The algorithmic implementations of the methods that provide the functionality of
the \ModelDesc follow the mathematics described in
section~\ref{sec:mathematics}.

By inspection, the \ModelDesc satisfies
requirements~\ref{reqt:eom}, \ref{reqt:vehicle_points}
and~\ref{reqt:state_integ_prop}.
