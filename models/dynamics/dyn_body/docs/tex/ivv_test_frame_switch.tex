\test{Frame Switch}\label{test:frame_switch}
\begin{description}
\item[Background]
A driving requirement for JEOD was to provide the ability to switch the
reference frame in which a vehicle's state is integrated during the course of a
simulation run. An Apollo-style mission was chosen as the reference mission for
this capability. The vehicle's integration frame must be switched from
Earth-centered to Moon-centered inertial to avoid an otherwise inevitable loss of
accuracy in the integrated state somewhere along the translunar trajectory.

\item[Test description]
This test represents a 100 second interval of the translunar trajectory of the
Apollo 8 mission near the point of the transition to the Moon's sphere of
influence. The test verifies that the frame switching mechanisms maintain state
to within numerical precision.

\item[Test directory]
{\tt models/dynamics/body\_action/verif/SIM\_frame\_switch} \\
The simulation \Sdefine file defines a vehicle and three gravitational bodies,
the Sun, Earth and Moon. Two input files are provided that
have identical initial conditions. One 
causes the vehicle to switch to Moon-centered inertial
when the vehicle enters the Moon's gravitational sphere of influence;
the other leaves the vehicle in Earth-centered inertial throughout.

\item[Success criteria]
The frame switch is to maintain a continuous state with respect to the common
parent of the former and new integration frames. In the case of a switch from
the Earth-centered inertial to Moon centered-inertial frames, this common
parent is the Earth-Moon barycenter frame. The difference between the Earth-Moon
barycenter relative states in the two runs must be attributable to numerical
differences for the test to pass.

\item[Test results]
The transition from Earth-centered inertial to Moon-centered inertial occurs 45.5
seconds into the simulation. At this time, the position and velocity vectors of
the vehicle relative to the Earth-Moon barycenter are
$[298463.448, -116959.258, -55769.040]$ km and
$[0.934601239, -0.199088249, -0.297865412]$ km/s.
The switched and unswitched runs show an exact match in position;
the velocities differ in {\emph y} and {\emph z} by $5.7\eneg{17}$ km/s.
These small differences are clearly numerical.

The simulation ends 54.5 seconds later. At this time,
the position and velocity vectors of
the vehicle relative to the Earth-Moon barycenter are
$[298514.379, -116970.107, -55785.274]$ km and
$[0.934431436, -0.199064620, -0.297866911]$ km/s.
The differences between the switched and unswitched runs at this point
in time are
$[1.1\eneg{8}, 4.3\eneg{9}, 1.1\eneg{9}]$ km and
$[1.8\eneg{10}, -1.2\eneg{10}, -1.\eneg{10}]$ km/s.
These differences are no longer purely numerical but are still quite small.
Some of these small differences result from
the way the simulation is constructed and used,
but some result from the fact that at this stage a Moon-centered inertial
frame is a better choice than an Earth-centered inertial frame.

The test passes.

\item[Applicable requirements]
This test demonstrates the partial or complete satisfaction of the
following requirements:
\begin{itemize}
\item \ref{reqt:integ_frame}. Demonstrating this capability
is the primary purpose of this test.
\item \ref{reqt:state_representation}. States are properly converted to
the new integration frame.
\item \ref{reqt:eom}. Accelerations are properly calculated
in the new integration frame.
\item \ref{reqt:state_integ_prop}. The equations of motion are properly
integrated.
\end{itemize}

\end{description}
