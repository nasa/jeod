
\chapter{Product Requirements}\hyperdef{part}{reqt}{}\label{ch:reqt}

\requirement{Project Requirements}
\label{reqt:toplevel}
\begin{description}
\item[Requirement:]\ \newline
  This model shall meet the JEOD project requirements specified in the
  \hyperref{file:\JEODHOME/docs/JEOD.pdf}{part1}{reqt}{JEOD} top-level document.

\item[Rationale:]\ \newline
  This is a project-wide requirement.

\item[Verification:]\ \newline
  Inspection
\end{description}


\requirement{Mass}
\label{reqt:mass_requirements}
\begin{description}
\item[Requirement:]\ \newline
  The \ModelDesc shall satisfy all requirements levied on the
  \hypermodelrefinside{MASS}{part}{reqt}.

\item[Rationale:]\ \newline
  A dynamic body has the properties of a mass body, including mass,
  moment of inertia, and center of mass.

\item[Verification:]\ \newline
  Inspection, test
\end{description}


\requirement{Integration Frame}
\label{reqt:integ_frame}
\begin{description}
\item[Requirement:]\ \newline
  The \ModelDesc shall provide the ability to initially specify and to
  dynamically change the reference frame with respect to which the
  states associated with a dynamic body are referenced and are integrated.

\item[Rationale:]\ \newline
  The integration frame is the connection between a dynamic body and
  the outside world.

  That each independent vehicle can have its own integration frame and
  that this frame can be changed during the course of a simulation run
  are driving requirements for JEOD.

  Changing integration frames needs to be done within the confines of the
  laws of physics.

\item[Verification:]\ \newline
  Inspection, test
\end{description}



\requirement{State Representation}
\label{reqt:state_representation}
\begin{description}
\item[Requirement:]\ \newline
  \subrequirement{State Representation}\label{reqt:state_rep_sub}.
    The \ModelDesc shall provide the ability to represent the states
    (positions, velocities, orientations, and angular velocities)
    of local reference frames (origin and axes) associated with a dynamic body
    with respect to the body's integration frame.
  \subrequirement{Required Frames}\label{reqt:state_rep_details}.
    The \ModelDesc shall represent the states of the object's\begin{itemize*}
    \item Structural origin / structural axes,
    \item Core center of mass / body axes,
    \item Composite center of mass / body axes, and
    \item Mass point frames registered as object frames of interest.
    \end{itemize*}

\item[Rationale:]\ \newline
  This requirement gives a dynamic body a connection to the outside world.

\item[Verification:]\ \newline
  Inspection, test
\end{description}


\requirement{Staged Initialization}
\label{reqt:staged_initialization}
\begin{description}
\item[Requirement:]\ \newline
  \subrequirement{Separate Initializations}\label{reqt:partial_init}.
    The \ModelDesc shall provide the ability to set one or more element of the
    state (position, velocity, orientation, and angular velocity) of one of the
    reference frames associated with a dynamic body.
  \subrequirement{Partial Propagation}\label{reqt:partial_prop}.
    The \ModelDesc shall provide the ability to propagate the partially set
    states for a given dynamic body to all reference frames associated with
    that body, subject to the mathematical limits regarding the ability to
    perform the propagation.
  \subrequirement{Initialized State Elements Query}\label{reqt:init_query}.
    The \ModelDesc shall provide the ability to query whether
    specific elements (position, velocity, orientation, and/or angular velocity)
    of the state of a reference frame associated with a dynamic body
    have been initialized.

\item[Rationale:]\ \newline
  Vehicle states must be initialized before they can be used.

\item[Verification:]\ \newline
  Inspection, test
\end{description}


\requirement{Equations of Motion}
\label{reqt:eom}
\begin{description}
\item[Requirement:]\ \newline
  \subrequirement{Forces and Torques}\label{reqt:force_torque}.
    The \ModelDesc shall provide the ability to individually represent
    and collectively assess the forces and torques that act on a dynamic body,
    including forces and torques acting on dynamic bodies attached to
    the dynamic body in question.
  \subrequirement{Non-transmitted forces and torques}\label{reqt:no_xmit}.
    The model shall provide the ability to identify some forces and torques
    as being non-transmittable to the parent dynamic body.
  \subrequirement{Translational Acceleration}\label{trans_accel}.
    The model shall provide the ability to calculate the translational
    acceleration of a dynamic body that results from the collective forces
    acting on the dynamic body, including gravitation.
  \subrequirement{Rotational Acceleration}\label{rot_accel}.
    The model shall provide the ability to calculate the rotational
    acceleration of a dynamic body that results from the collective torques
    acting on the dynamic body, including the fictitious inertial torque.

\item[Rationale:]\ \newline
  Properly representing the equations of motion is a prerequisite
  to computing the time-varying nature of a dynamic body's state.

\item[Verification:]\ \newline
  Inspection, test
\end{description}


\requirement{State Integration and Propagation}
\label{reqt:state_integ_prop}
\begin{description}
\item[Requirement:]\ \newline
  \subrequirement{State Integration}\label{reqt:state_integ}.
    The \ModelDesc shall provide the ability to integrate at least one of a
    dynamic body's reference frames over time based on the dynamical equations
    of motion for that body.
  \subrequirement{State Propagation}\label{reqt:state_prop}.
    The \ModelDesc shall provide the ability to propagate a dynamic body's
    non-integrated states such that the propagated states, local frames,
    and integrated states are consistent to within numerical precision.

\item[Rationale:]\ \newline
  Properly integrating vehicle state over time has been a driving requirement
  for JEOD since its inception.

\item[Verification:]\ \newline
  Inspection, test
\end{description}


\requirement{Vehicle Points}
\label{reqt:vehicle_points}
\begin{description}
\item[Requirement:]\ \newline
  \subrequirement{Vehicle Point Registration}\label{reqt:point_spec}.
    The \ModelDesc shall provide the ability to register local reference frames
    associated with mass points as object frames of interest.
  \subrequirement{Vehicle Point Acceleration}\label{reqt:point_accel}.
    The \ModelDesc shall provide the ability to compute the translational and
    rotational acceleration of a local reference frame associated with a mass
    point, including the non-gravitational translational acceleration.

\item[Rationale:]\ \newline
  This requirement derives from reqt~\ref{reqt:state_rep_details}.

\item[Verification:]\ \newline
  Inspection, test
\end{description}


\requirement{Attach/Detach}
\label{reqt:attach_detach}
\begin{description}
\item[Requirement:]\ \newline
  \subrequirement{Dynamic Body Attach/Detach}\label{reqt:dyn_attach_detach}.
    The \ModelDesc shall provide the ability to attach and detach dynamic
    bodies in a manner consistent with the conservation laws.
  \subrequirement{Kinematic Body Attach/Detach}\label{reqt:kin_attach_detach}
    The \ModelDesc shall provide the ability to attach and detach dynamic
    bodies to any reference frame kinematically.

\item[Rationale:]\ \newline
  The \MASS attach/detach requirements do not mention the
  conservation laws, so this functionality must be introduced in \ModelDesc in order to provide capabilities for docking,
  berthing, jettision, etc. operations common to spaceflight missions. \newline \newline
  Additionally, the ability to attach a dynamic body to a reference frame such that it is rigidly fixed to it is a 
  valuable simulation capability. In this case, the conservation laws for attach or detach may be ignored because their
  effects are negligible or the attachment acts as a low-fidelity attachment to constrain the vehicle. This is a feature
  the user can decide to use and the \ModelDesc should provide this capability.

\item[Verification:]\ \newline
  Inspection, test
\end{description}
