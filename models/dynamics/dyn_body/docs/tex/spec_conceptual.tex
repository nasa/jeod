\section{Conceptual Design}
The primary purpose of the \ModelDesc is to represent dynamic aspects of
spacecraft in a simulation.
The model's DynBody class is the base class for such representations.
A DynBody has properties such as mass that are intrinsic to the body itself
as well as extrinsic properties such as the state of the body with
respect to the outside world. External forces and torques act to change
the body's state over time.

The forces and torques that act on a DynBody are obviously \hasa
relationships.\footnote{A \hasa relationship entails containment while a \isa
relationship entails specialization. A car \isa kind of vehicle but \hasa carburetor and four tires. Sometimes the distinction between \isa and \hasa
is not as clear as in this simple example. For example, the very first design
of JEOD 2.0 used a \hasa relationship for the association between a DynBody
object and a MassBody object, whereas JEOD 3.0 used a DynBody \isa MassBody approach. Ultimately, these design decisions should be made in order to balance code legibility, use cases, and user feedback.}
Multiple forces and torques can be present and can vary in number
and nature from body to body.
State is also a \hasa relationship as a DynBody has multiple states.
The body's center of mass and structural origin, for example, have potentially
distinct locations with respect to the outside world.

A spacecraft has mass properties such as mass, a center of mass, and
moment of inertia.
These intrinsic properties of a DynBody are represented by the
\hypermodelref{MASS} MassBody class. The relationship
between a DynBody and a MassBody can be either \hasa or \isa.
In JEOD 1.5, the relationship between an OrbitalBody
and a MassBody was \hasa by necessity. JEOD 1.5 was implemented in C,
which does not offer \isa relationships. That limitation does not
exist in JEOD 2.0. Making the relation \isa offered the advantage of overridden
functions over a \hasa relationship. The JEOD 2.0 DynBody class
inherited from the JEOD 2.0 MassBody class. In JEOD 4.0, the relationship was
reverted to a \hasa relationship with friendship. This removed the need for
constant checks to differentiate between a MassBody and a DynBody, and provides
 extension stability by preventing users from overriding protected JEOD
 functionality.

The enhanced flexibility introduced with recent changes to
the \hypermodelref{INTEGRATION} \INTEGRATION are enabled by abstracting the
sorts of things which JEOD can integrate in the er7\_utils::IntegrableObject
class. Now a DynBody \isa er7\_utils::IntegrableObject.  In addition, the
DynBody class now provides the capability to associate additional
IntegrableObjects which are integrated along with the DynBody state.

\subsection{Key Concepts}\label{sec:key_concepts}
This section describes the key concepts that form the conceptual foundation of
the \ModelDesc.

\subsubsection{Mass}\label{sec:key_mass}
As noted above, a DynBody \hasa MassBody.
MassBody objects have mass properties and can be attached to/detached from
other MassBody objects. MassBody objects represent mass properties in the
form of instances of the MassProperties class.
The MassProperties class derives from the MassPoint class.

A MassBody object defines three mass properties objects, and hence three
MassPoints, one for each of the object's \begin{inparaenum}[(1)]
\item structural origin/structural axes,
\item core center of mass/body axes, and
\item composite center of mass/body axes.
\end{inparaenum}
The MassPoints class defines a rudimentary kind of reference frame basis. These
rudimentary reference frames include an origin and a set of orthogonal
right-handed axes, but exclude the derivatives information found in the
\hypermodelref{REFFRAMES}.
A basic MassBody object lacks a connection to the
outside world. The DynBody class adds that connection to a MassBody object.
DynBody mass attributes exist in the MassBody \verb+DynBody::mass+  member.
MassBody's referring to the mass contents of a DynBody contain a pointer DynBody \verb+MassBody::dyn_owner+ member.
Pure mass items, which do not refer to a dynamic body, should have a \verb+NULL+ value for this member.

\subsubsection{State}
The DynBody class makes the connection to the outside world in the form of
`state'.  Multiple states, in fact. The \ModelDesc uses the BodyRefFrame
class to represent state. A BodyRefFrame \isa RefFrame plus a pointer
to a MassPoint object. The model maintains a BodyRefFrame for each of the
MassPoint objects associated with a MassBody object.

A RefFrame object contains links and states. The links connect RefFrame
objects to one another. The state describes the
translational and rotational relation between the frame and its parent frame.
All active reference frames in a simulation are connected to one another to
form a tree structure. The problem at hand is to connect the reference frames
associated with a DynBody object to this larger, simulation-wide reference
frame tree.

\subsubsection{Inertial Frames}
The root node of the simulation reference frame tree is, by
assumption, a true Newtonian inertial frame of reference. Other reference
frames in the tree loosely qualify as being `inertial' frames. These so-called
inertial frames are not truly inertial frames; their origins are accelerating
with respect to the root frame. What qualifies these frames as being `inertial'
frames is that they are not rotating with respect to the root frame,
they are connected to the root frame either directly or through other
`inertial' frames, and they are subject to gravitation forces only.
Non-rotating planet-centered reference frames, for example, qualify as
`inertial' reference frames in JEOD. (And elsewhere. This is a very
widespread and very convenient abuse of notation.)

Any non-inertial frame can be made to look like an inertial frame of reference
(\ie $F=ma$) by means of fictional forces and torques. These fictional forces
and torques are accounting tricks. The only such trick needed to make the JEOD
inertial frames appear to be true inertial frames is to account for the
inertial force due to frame acceleration, and that acceleration is only
gravitational. The \hypermodelref{GRAVITY} performs these accounting tricks.
From the perspective of this model, these pseudo-inertial frames of reference
can be treated as inertial frames in which $F=ma$ is valid.

\subsubsection{Integration Frame}
The connection between a DynBody object's DynBodyFrame objects and the outside
world is through the DynBody object's integration frame. The integration frame
must be one of the simulation's inertial frames. All of the DynBodyFrame objects
associated with a given DynBody object share the DynBody object's integration
frame as a common parent.

A driving requirement for JEOD was to provide the ability for different
vehicles in a multi-vehicle simulation to have their states represented in
and integrated in different frames of reference.
For example, using the Moon-centered inertial frame as the
integration frame would best suit a vehicle orbiting the Moon, but the
Earth-centered inertial frame would best suit a vehicle in low Earth orbit.
The \ModelDesc accomplishes this goal by enabling each root DynBody object to
have its own integration frame. This allows state to be represented in and
integrated in the frame best-suited to the object in question.

Another driving requirement for JEOD was to provide the ability to switch
integration frames for a given vehicle during the course of a simulation run.
For example, a vehicle in transit from the Earth to the Moon should be able
to switch the vehicle's integration frame from Earth-centered inertial to
Moon-centered inertial at some point during the transit. The \ModelDesc
accomplishes this goal as well. A DynBody object's integration frame can
be changed dynamically in a manner that is consistent with the laws of physics.

\subsubsection{Attach/Detach}\label{sec:key_attach_detach}
MassBody objects can be attached to/detached from one another. So can
DynBody objects. This capability is essential for modeling the docking and
undocking of spacecraft. The composite mass properties change when MassBody
objects attach or detach. That plus more happens when DynBody objects attach
and detach. Vehicle states change, and these changes must be consistent
with the laws of physics.

The \ModelDesc attach mechanism first queries both the child and parent regarding the validity of the proposed
attachment. If both the child and parent assent to the attachment, the transform from the root body of the child
to the desired attach point is calculated. If the child is already a root body, the transform is simply the mass point
transform or the user-provided transform. The child is told to establish the linkages between
child-and-parent/parent-and-child while the parent is told to update properties using the \MASS attach mechanisms.

Dynbody objects may also be attached to a RefFrame object kinematically. This capability is primarily for sim
operations or to assume the capabilities of a low-fidelity attachment model. As a result, momentum and energy
conservation issues are ignored and the dynamic state is simply held rigid w.r.t. its parent RefFrame.

A detachment proceeds similarly. First there is a check for validity,
and only then is the child told to sever linkages and the parent is told
to update properties. The DynBody calls the \MASS detach mechanisms.

\subsubsection{Initialization}\label{sec:key_initialization}
Initializing a DynBody object is a rather complex process. The zeroth stage
of the initialization process occurs when the object is constructed.
For example, the association between the core properties and core state is
immutable and can be established at construction time.

The next stage of initialization involves preparing the object for the truly
complex stages that follow. The body is registered with the dynamics manager,
as are the standard body reference frames associated with the body. The
body's integration frame is set and the standard body reference frames are
linked to this integration frame. Note that while the links between the
DynBody object are established in this phase, much work remains to be done.
The body is unattached and its mass properties, mass points/vehicle points, and
states remain uninitialized at this stage.

These yet-to-be initialized aspects of a DynBody must be initialized in the
proper order. The recommended approach is to first initialize the mass
properties and mass points of all MassBody objects, then perform any
initialization-time attachments, and finally initialize state. This is
the approach taken when the body action-based initialization scheme is used
to initialize mass bodies and dynamic bodies.
This approach involves a series of interactions
between the \hypermodelref{DYNMANAGER} and the \hypermodelref{BODYACTION}.
The \BODYACTION in turn interacts with
several other models, including this model, while performing its actions.

With this approach, the simulation's dynamics manager performs, in stages,
the queued BodyAction objects that derive from the MassBodyInit class,
then the queued BodyAction objects that derive from the BodyAttach class,
and finally the queued BodyAction objects that derive from the DynBodyInit
class.
The first stage establishes the mass properties and the attach points needed to
properly perform the next two stages. The attachments made in the second stage
(potentially) reduce the number of independent states that need to be set.
The final stage initializes states.

\subsubsection{State Initialization}\label{sec:key_state_initialization}
Properly initializing the state of all of the vehicles in a simulation
has been an ongoing and rather vexing issue.
The code that performed this initialization in pre-2.0 releases of JEOD was
fragile, hard to extend, and overly complex. JEOD 2.0 solved this problem by
spreading the responsibility for state initialization over several models.

The recommended approach is to use the \BODYACTION to initialize states.
That model provides several classes that derive from the DynBodyInit class.
It is these DynBodyInit derivative classes that initialize DynBody states.
Each DynBodyInit object sets one or more elements (position, velocity,
orientation, and angular velocity) of a reference frame associated with the
DynBody object that is the subject of the DynBodyInit object.
The to-be-set state elements are specified with respect to some other
basis reference frame, which need not be the subject DynBody object's
integration frame. For example, rendezvous simulations often initialize the
state of one vehicle given the relative state with some other target vehicle.

Certain elements of the basis reference frame must already have been
established before the values of the desired state elements can be determined.
These required elements form a set of prerequisites for setting the desired
state elements. Those prerequisites are sometimes satisfied immediately, as is
the case when the basis is a planetary frame. Difficulties can arise when the
basis is a vehicle frame. Suppose the reference object for a given
DynBodyInit object is some vehicle frame for which the requisite state elements
have not been set. One option would be to simply declare a user error when this
happens. That is not the course taken by JEOD.
Instead, the DynBodyInit object in question is put on hold until the reference
vehicle frame is initialized to the extent needed by the DynBodyInit object.

Each BodyAction object, including the DynBodyInit objects, provides an
indication of whether the object is ready to perform its action. The dynamics
manager queries each enqueued object regarding its readiness before telling
the object to perform its action. The readiness test for a DynBodyInit object
includes a check of whether the prerequisite state elements have been set.
The dynamics manager repeatedly loops over the queued DynBodyInit objects,
stopping only when either all such objects have been applied or until
no such object indicates that it is ready to be applied.

This model participates in the exchange as well. The DynBody class provides
mechanisms
to query which state elements have been set,
to set specific state elements, and
to propagate the set states to frames associated with the DynBody object,
including those owned by DynBody objects attached to the subject DynBody
object.


\subsubsection{Forces and Torques}
In addition to the acceleration due to gravity,
a spacecraft can be subject to many external forces and torques.
For example a spacecraft's thrusters, atmospheric drag, and radiation pressure
exert forces and torques on the vehicle. The source and nature of these forces
and torques is more or less irrelevant to this model.
Forces and torques are each categorized as effector, environmental, and
non-transmitted. This categorization is external to this model; all this model
sees is a set of three Standard Template Library (STL) vectors of
CollectForces and a set of three STL vectors of CollectTorques.

The distinction between the effector and environmental forces and torques is
somewhat arbitrary and exists solely as an aid to the user.\footnote{From a
JEOD perspective, things would be a bit easier had the effector and
environmental  forces been lumped together as transmitted forces, with the
effector and environmental torques similarly lumped together as transmitted
torques.}
On the other hand, the distinction between the non-transmitted forces and
torques and the effector and environmental forces and torques is significant.
For a non-root DynBody object, the effector and environmental forces and
torques are transmitted to the DynBody object to which the non-root DynBody is
attached. The non-transmitted forces and torques are not transmitted to the
parent (hence the name).

The reason for having these non-transmitted forces and torques is again one of
user convenience. Consider the force due to atmospheric drag on a composite
vehicle. While this composite drag force is the vector sum of the drag forces
on the individual vehicles that comprise the composite, the close proximity of
the components of the composite vehicle makes those individual drag forces
different from what they would be in the absence of those other vehicles.
Enabling users to mark certain forces and torques as non-transmittable
simplifies the users' calculation of those forces and torques that are affected
by the presence of or connection to other vehicles. Users can simply ignore
those complicating factors and calculate those forces and torques as if other
vehicles were not present.

The net force and torque on a root body form the basis for the equations of
motion for that body. It is the net forces and torques that are of the
greatest interest to JEOD. JEOD uses the superposition principle to compute the
net force acting on a DynBody object. The net force on a root body is the
vector sum of all of the forces acting directly on that body, including that
body's non-transmitted forces, plus the sum of all transmittable forces acting
on bodies attached to the root body. Calculating the net torque is a bit more
complex. In addition to the torques specified through the collect vectors,
torques due to non-centerline forces must be taken into account. The
forces in the collect vectors are assumed to act through the center of mass
of the DynBody object that contains those collect vectors. When a DynBody object is
attached as a child to some other DynBody object, the centerline forces on the
child body become non-centerline forces when transmitted to the parent body.

\subsubsection{Wrenches}
JEOD 3.4 introduces the concept of a wrench. A wrench comprises an external force
and an external torque, with the the force applied at a specified point on the
dynamic body. The torque induced by the force should not be included in the
torque portion of a wrench; JEOD calculates this torque. For example, a typical
chemical thruster model that uses a wrench will have the torque be zero.
On the other hand, a model of an ion thruster will have a non-zero torque.
The difference is that the exhaust from a chemical thruster typically does not
have a curl while the exhaust from an ion thruster does. Even if an ion thruster's
line of action passes directly through the vehicle's center of mass, the thruster
imparts a torque on the vehicle due to the swirling exhaust.

\subsubsection{Collect Mechanism}
The above discussion on forces and torques ignored how the STL collect vectors
are formed. How these vectors are populated is not an issue of concern in the
narrow view of the DynBody class taken by itself. From the perspective of the
DynBody class, those vectors are populated by some external agent.
This is a concern to the model taken as a whole, and even more importantly,
it is a concern to the user.

Prior to JEOD 2.0, Trick users used Trick's \emph{collect} mechanism to
identify the forces and torques in play. This approach was replaced for a
number of reasons. Instead, JEOD and Trick developers worked
together to create the Trick 07 \emph{vcollect} mechanism. Compared to the
\emph{collect} mechanism, this new mechanism has the potential for greater
safety (type, unit, and dimensionality),
takes better advantage of capabilities provided with the C++ language,
and is easier to implement outside of the Trick environment.

The wrenches introduced in JEOD 3.4 can also be collected via the vcollect
mechanism. As of JEOD 3.4, only the StructureIntegratedDynBody provides the
ability to collect and process wrenches. This capability may be migrated upward
in future releases of JEOD.

\subsubsection{Gravity}
The above discussion on forces intentionally skirted over the issue of
gravitational acceleration. JEOD has always treated gravitation as a topic
conceptually distinct from the other forces that act on a dynamic body.
One reason for doing so is that the \hypermodelref{GRAVITY} computes
gravitational acceleration directly, and acceleration (rather than force)
is what is ultimately needed to formulate the equations of motion.

More importantly, gravitation truly is distinct from the external forces.
Unlike any other force, the gravitational force acting on an object cannot be
detected directly by any device. An accelerometer, for example, measures the
acceleration due to the net non-gravitational force. Even though JEOD assumes
Newtonian physics, it is a good idea to look at gravitation from the
perspective of the more accurate general relativistic point of view.
Gravitation is not a force in general relativity. It is something quite
different from forces and is best treated as such.

The earlier discussion on integration frames noted that the integration
frame can be a not-quite inertial frame. The \GRAVITY folds the acceleration
of a DynBody's integration frame into the total gravitational acceleration
presented to a DynBody object. As far as the DynBody object is concerned,
the integration frame \emph{is} an inertial frame of reference; Newton's
second law can be employed to integrate a DynBody object's state.

\subsubsection{Equations of Motion}\label{sec:key_eom}
The external forces and gravitational acceleration form the basis for the
translational equations of motion for a dynamic body, while the external torques
form the basis for the body's rotational equations of motion. Those equations
of motion are second order differential equations; they must be integrated over
time.

\subsubsection{State Integration}\label{sec:key_state_integ}
This state integration must be performed numerically due to the varying and complex nature of the forces and torques. This model uses the \INTEGRATION to
perform the numerical integration.

The model only integrates the root body of a tree of attached DynBody objects
and for that body, integrates only one reference frame. This is the body's
integrated frame. Exactly which frame is integrated is a decision made by a
class that derives from the DynBody class. The DynBody class provided integrates the root body's composite body frame.
The StructureIntegratedDynBody class (introduced with JEOD 3.4) integrates the
root body's structural frame.

\subsubsection{State Propagation}\label{sec:key_state_prop}
A DynBody contains multiple reference frames. All of these frames must reflect
the current state of the vehicle. Since only one frame is integrated, the other
frames must be made consistent with this integrated frame. This is performed
by propagating the results of the integration to all frames associated
with the DynBody object, including the DynBody objects attached as children
to the root DynBody object. This propagation is performed using the geometry
information embodied in the MassBody objects contained in the DynBody objects.

\subsubsection{Point Acceleration}
A reference frame state contains zeroth and first derivatives. Second
derivatives are not a part of the state. Some vehicle sensors such as
accelerometers need acceleration data, and they need the acceleration that
would be sensed at the point where the accelerometer is located.
An accelerometer senses acceleration due to non-gravitational forces.
The model provides an acceleration propagation capability to meet this
need.
