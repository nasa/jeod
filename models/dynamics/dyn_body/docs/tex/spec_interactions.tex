\section{Interactions}

\subsection{JEOD Models Used by the \ModelDesc}
The \ModelDesc uses the following JEOD models:
\begin{itemize}
\item\hypermodelref{DYNMANAGER}. The \ModelDesc uses the \DYNMANAGER as
  \begin{inparaenum}[(1)]
  \item a name register of DynBody objects,
  \item a name register and subscription manager of RefFrame objects, and
  \item a name register and validator of integration frames.
  \end{inparaenum}
\item\hypermodelref{MASS}. The \ModelDesc uses the \MASS through friendship to 
    access protected functionality. The DynBody class \hasa MassBody.
\item\hypermodelref{GRAVITY}. The \ModelDesc relies on the \GRAVITY to
  compute gravitational acceleration. To accomplish this end, each DynBody
  object contains a GravityInteraction object. That class is a part of the
  \GRAVITY. A GravityInteraction indicates which gravitational bodies in the
  simulation have an influence on a DynBody object and contains the
  gravitational accelerations and gravity gradients as computed by the
  \GRAVITY.
\item\hypermodelref{INTEGRATION}. The \ModelDesc uses the \INTEGRATION to
  perform the state integration. This model merely sets things up for the
  \INTEGRATION so that it can properly perform the integration.
With the migration of most integration functionality to the er7\_utils
package, the DynBody class now extends the er7\_utils::IntegrableObject class.
For a more detailed explanation, see the \INTEGRATION documentation.
\item\hypermodelref{MATH}. The \ModelDesc uses the \MATH to operate on
  vectors and matrices.
\item\hypermodelref{MEMORY}. The \ModelDesc uses the \MEMORY to allocate
  and deallocate memory.
\item\hypermodelref{MESSAGE}. The \ModelDesc uses the \MESSAGE to generate
  error and debug messages.
\item\hypermodelref{NAMEDITEM}. The \ModelDesc uses the \NAMEDITEM to
  construct names.
\item\hypermodelref{QUATERNION}. The \ModelDesc uses the \QUATERNION
  to operate on the quaternions embedded in the RefFrame objects and to
  compute the quaternion time derivative.
\item\hypermodelref{REFFRAMES}. The \ModelDesc makes quite extensive use of the
  \REFFRAMES. A DynBody's integration frame is a RefFrame object, and each of
  the body-based reference frames contained in a DynBody object is a
  BodyRefFrame object. The BodyRefFrame \isa RefFrame. The \ModelDesc uses
  the \REFFRAMES functionality to compute relative states.
\item\hypermodelref{SIMINTERFACE}. All classes defined by the \ModelDesc
  use the\\
  \verb+JEOD_MAKE_SIM_INTERFACES+ macro defined by the \SIMINTERFACE to provide
  the behind-the-scenes interfaces needed by a simulation engine such as Trick.
\end{itemize}


\subsection{Use of the \ModelDesc in JEOD}
The following JEOD models use the \ModelDesc:
\begin{itemize}
\item\hypermodelref{BODYACTION}. The \BODYACTION provides several classes that
  operate on a DynBody object. The DynBodyInit class and its derivatives
  initialize a DynBody object's state. The DynBodyFrameSwitch class provides
  a convenient mechanism for switching a DynBody object's integration frame.
  The BodyAttach and BodyDetach classes indirectly operate on a DynBody object
  through the \ModelDesc attach/detach mechanisms.
\item\hypermodelref{DERIVEDSTATE}. The primary purpose of the \DERIVEDSTATE
  is to express a DynBody state in some other form.
\item\hypermodelref{DYNMANAGER}. The \DYNMANAGER drives the integration of
  all DynBody objects registered with the dynamics manager.
\item\hypermodelref{GRAVITY}. The \GRAVITY computes the gravitational
  acceleration experienced by a DynBody object.
\item\hypermodelref{GRAVITYTORQUE}. The \GRAVITYTORQUE computes the gravity
  gradient torque exerted on a DynBody object, with the resultant torque
  expressed in the DynBody object's structural frame.
\end{itemize}


\subsection{Interactions with Trick}\label{sec:spec_interactions_trick}
The \ModelDesc interacts with Trick in both an ordinary and extraordinary
manner. Many models are designed to be used in the Trick environment in the
form of data declarations and function calls placed in an \Sdefine file. The
\ModelDesc is no different from other models in this regard.

The model also interacts with Trick in an extraordinary manner.
The \emph{vcollect} mechanism was designed as an object-oriented
replacement for the old-style \emph{collect} mechanism. The syntax for the
Trick \emph{vcollect} statement is
\verb+vcollect container converter { [item [,item]...]};+

The \verb+container+ is an STL sequence container object (an STL deque, list,
or vector) that implements the \verb+push_back+ method.
The \verb+converter+ is a function, possibly overloaded, that converts each
\verb+item+ in the item list into a form suitable for pushing onto the end of
the \verb+container+.
A \verb+vcollect+ statement results in a series of calls to
\verb+container.push_back()+. For example, the following code is from the
SIM\_force\_torque verification simulation \Sdefine:

\begin{verbatim}
vcollect body2.body.collect.collect_environ_forc CollectForce::create {
   body2.environ_forc1,
   body2.environ_forc2
};
\end{verbatim}

The generated code that corresponds to this \verb+vcollect+ statement is

\begin{verbatim}
trick->body2.body.collect.collect_environ_forc.push_back(
   CollectForce::create( trick->body2.environ_forc1 ) );
trick->body2.body.collect.collect_environ_forc.push_back(
   CollectForce::create( trick->body2.environ_forc2 ) );
\end{verbatim}

\subsection{Interaction Requirements on the \MASS}
The concepts discussed in section~\ref{sec:key_mass} conform with the
requirements levied on the \MASS.
The DynBody class includes a public attribute MassBody class such that 
the MassBody class must declare friendship with the DynBody class to properly 
propagate states. The MassBody must also include a constant DynBody pointer 
populated at construction to ensure that restricted operations are not applied 
to a DynBody's attribute MassBody.
These designs are particularly necessary for the attach and detach mechanisms 
as discussed in section~\ref{sec:key_attach_detach}. The attach and detach 
mechanisms must be, and were, designed with such that extension of the DynBody 
class will not interupt functionality. 

\subsection{Interaction Requirements on the \DYNMANAGER}
The \ModelDesc uses the \DYNMANAGER as a \begin{inparaenum}[(1)]
\item a name register of DynBody objects,
\item a name register and subscription manager of RefFrame objects, and
\item a name register and validator of integration frames.
\end{inparaenum}
All of these dependencies place functional requirements on the \DYNMANAGER.
The initialization concepts described in sections~\ref{sec:key_initialization}
and~\ref{sec:key_state_initialization} also place functional requirements on
the \DYNMANAGER.

The integration scheme implemented in the \ModelDesc implicitly assumes that a
higher-level function controls the integration. The model's \verb+integrate()+
methods cannot be called as integration-class jobs from a Trick \Sdefine file.
The higher-level function that controls the integration is a part of the
\DYNMANAGER, and this functionality is addressed as requirements on that model.

\subsection{Interaction Requirements on the \BODYACTION}
The initialization concepts described in sections~\ref{sec:key_initialization}
and~\ref{sec:key_state_initialization} place functional requirements on
the \BODYACTION.

\subsection{Interaction Requirements on the \INTEGRATION}
The \INTEGRATION is required to, and was designed to, integrate state and time
separately. This requirement was levied on the \INTEGRATION precisely
because of the way in which integration is performed within JEOD.
