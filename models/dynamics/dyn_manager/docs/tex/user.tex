%%%%%%%%%%%%%%%%%%%%%%%%%%%%%%%%%%%%%%%%%%%%%%%%%%%%%%%%%%%%%%%%%%%%%%%%%%%%%%%%
% user.tex
% User guide for the Dynamics Manager Model
%
%
%%%%%%%%%%%%%%%%%%%%%%%%%%%%%%%%%%%%%%%%%%%%%%%%%%%%%%%%%%%%%%%%%%%%%%%%%%%%%%%%

%----------------------------------
\chapter{User Guide}\hyperdef{part}{user}{}
\label{ch:user}
%----------------------------------
This chapter describes how to use the \ModelDesc from the
perspective of a simulation user, a simulation developer,
and a model developer.


\section{Analysis}
This section addresses use of the \ModelDesc from the perspective
of a simulation analyst. The model largely works behind the scenes.
With the exception of the body action list described below,
little interaction between the model and the typical simulation
user is required.

\subsection{The BodyAction List}\label{sec:user_analysis_body_action}

The recommended practice regarding the initialization of DynBody
objects in a JEOD-based Trick simulation is to use the
\hypermodelref{BODYACTION} in conjunction with the \ModelDesc.
The simulation's DynManager object maintains a list
of \ModelDesc objects. Calling the DynManager's {\tt add\_body\_action()}
method adds an item to the list. The DynManager draws objects from this list
at the appropriate time and only does so when the objects indicate
that they are ready to be executed. This mechanism disconnects the
execution order of the enqueued model objects from the order in
which the simulation objects are declared and from the order
in which the objects are added to the list.

This separation between initialization and declaration order
was first addressed in JEOD 1.4/1.5, but the C-based solution
involved some rather convoluted and very inflexible code.
The C++ based JEOD 2.0 solution to this problem makes for
a very extensible and simpler set of code. Users are strongly
recommended to take advantage of this flexibility.

\subsection{Initialization}\label{sec:user_analysis_initialization}
The DynManager member function \verb+initialize_model()+ takes two arguments:
a pointer to a Trick INTEGRATOR structure
and a reference to a DynManagerInit object.
The DynManagerInit is a simple object. It contains three data members that
are used to set values in the simulation's dynamic manager that must
remain constant for the duration of the simulation run. These three data
members are:\begin{description}
\item[{\tt EphemerisMode mode}]
  Specifies the mode in which the dynamics manager will operate.
  The default is ephemeris-driven mode, and this is the context under which
  realistic simulations will operate. Following the guideline that simple
  problems should be simple to specify, JEOD offers two other simplified modes,
  empty space and single planet mode. There is but one integration frame
  in these two simple modes. The two differ in that the planet in single
  planet mode presumably has a gravitational field that affects dynamics
  while gravitation is nonexistent in empty space mode.
\item[{\tt std::string central\_point\_name}]
  This parameter names the central point and is used only when the manager
  operates in empty space or single planet mode.
\item[{\tt IntegratorConstructor * integ\_constructor}]
  The integrator constructor specifies the integration technique that is
  to be used to integrate all of the dynamic bodies in a JEOD simulation.
  See the \hypermodelref{INTEGRATION} for details.
\end{description}

If the DynManagerInit object's \verb+ integ_constructor+ pointer is null
(which it is by default), the dynamics manager uses the \INTEGRATION factory constructor to create an integrator constructor based on the \verb+option+
in the INTEGRATOR structure pointer passed to the \verb+initialize_model()+.

\subsection{Controlling Dynamics}\label{sec:user_analysis_dynamics}
The DynManager \verb+deriv_ephem_update+ data member determines whether
ephemerides are updated at the derivative rate or only as scheduled.
Setting this element to true will cause the ephemeris models to be updated at
the derivative rate. Doing so makes the calculations of the perturbative
gravitational accelerations more accurate but at the expense of an increased
computational burden. The default setting for this data member is false,
which means ephemerides are updated at the scheduled rate as specified
in the simulation's \Sdefine file.

\section{Integration}\label{sec:user_integration}
This section addresses use of the \ModelDesc, first from the perspective of
a simulation integrator and then from the perspective of a model developer
who wants to use some of the functionality provided by this model.

\subsection{\Sdefine-Level Integration}

The DynManager class provides 48 publicly accessible methods, 50 counting
the default constructor and destructor. Only eight of these methods are expected
to be used in a simulation \Sdefine file. These eight methods are listed in
table~\ref{tab:user_sdefine_methods}.

\begin{table}[htp]
\label{tab:user_sdefine_methods}
\centering
\caption{DynManager \Sdefine-level Methods}
\vspace{1ex}
\begin{minipage}{\textwidth}
\centering
\begin{tabular}{||l|l|p{3.41in}|}
\hline
{\bf Method} & {\bf Job Class} & {\bf Description}
\\ \hline\hline
\verb+add_body_action+ &
  environment\footnote{Typically specified as a zero-rate job. Zero-rate jobs
    are not be called on a regular basis. They can however be called from the
    input file, which is the intended use of this method.} &
  \raggedright Adds a BodyAction to the dynamic manager's list of body actions.
    The dynamics manager will eventually process this newly-queued action.
  \tabularnewline[6pt]
\verb+compute_derivatives+ &
  derivative\footnote{It may be desirable to also call this as scheduled jobs if
    you are logging the accelerations of a DynBody object and want to see the
    derivatives reflect the current time.\label{fn:guide_deriv_jobs}} &
  \raggedright Causes each root dynamic body registered with the dynamics
    manager to formulate the equations of motion for that body. This includes
    accumulating the collected forces and torques that on the body. If your
    simulation has gravitational forces, make sure you call \verb+gravitation()+
    before calling \verb+ compute_derivatives()+.
  \tabularnewline[6pt]
\verb+gravitation+ &
  derivative\footref{fn:guide_deriv_jobs} &
  \raggedright Causes the gravity model to compute the gravitational
    acceleration and gravity gradient for each root dynamic body registered with
    the dynamics manager.
  \tabularnewline[6pt]
\verb+initialize_model+ &
  initialization &
  \raggedright Prepares the dynamics manager for the initializations that
    are to follow. This method should run fairly early in the initialization
    process.
  \tabularnewline[6pt]
\verb+initialize_simulation+ &
  initialization &
  \raggedright Initializes the ephemerides and performs the initialization-time
    body actions registered with the dynamics manager.
  \tabularnewline[6pt]
\verb+integrate+ &
  integration &
  \raggedright Causes each root dynamic body registered with the dynamics
    manager to integrate the equations of motion for that body and
    updates time to reflect the time at the end of the intermediate step.
  \tabularnewline[6pt]
\verb+perform_actions+ &
  environment &
  \raggedright Performs all of the body actions registered with the
    dynamics manager that are ready to be applied.
  \tabularnewline[6pt]
\verb+update_ephemerides+ &
  environment &
  \raggedright Makes each ephemeris model registered with the dynamics manager
    update itself to reflect the current simulation time.
  \tabularnewline[6pt]
\hline
\end{tabular}
\end{minipage}
\end{table}

\subsubsection{Initialization Jobs}\label{sec:user_integration_initialization}
As noted in table~\ref{tab:user_sdefine_methods}, the model provides
two methods that are expected to be used in an \Sdefine file as
integration class jobs. These are \verb+initialize_model()+ and
\verb+initialize_simulation()+.
The first method, \verb+initialize_model()+, prepares the simulation's
DynManager object for all of the other initializations that will soon follow.
See section~\ref{sec:user_analysis_initialization}
for a description of this method.
The second method, \verb+initialize_simulation()+, takes no arguments.
This method
initializes the gravity controls for each of the simulation's dynamic bodies,
completes the initialization of the simulation's ephemeris models, and
initializes and applies body actions that have been registered with the
dynamics manager and are ready to run at initialization time.

The only initializations that should occur prior to the call to
\verb+initialize_model()+ are initializations of the time model.
Initializations that depend on dynamic bodies have fully-initialized states
should be performed after the call to \verb+initialize_simulation()+.
The bulk of the calls to initialization class jobs should be sandwiched
between the calls to the two DynManager initialization-time methods.
The best way to ensure that jobs will take place as described above
in a Trick-based simulation is to use the Trick prioritization scheme.

Two issues dictate this arrangement:\begin{itemize}
\item Those intermediate initialization jobs' use of the dynamics manager
  requires that the dynamics manager itself be properly initialized.
  That means these tasks must be performed after the call to
  \verb+initialize_model()+.
\item  Which planetary ephemerides are active is determined by those frames
  that have a non-zero subscription count. Initialization-time jobs that
  subscribe to ephemerides-based reference frames must be performed prior
  to the call to \verb+initialize_simulation()+ to ensure that the
  simulation's reference frame tree is constructed as planned.
\end{itemize}

\subsubsection{Derivative and Integration Jobs}
\label{sec:user_integration_dynamics}
The model provides two methods,
\verb+gravitation()+ and \verb+compute_derivatives()+,
that should be called as derivative class jobs and
one method, \verb+integrate()+,
that should be called as an integration class job.
The call to \verb+gravitation()+ must precede the call to
\verb+compute_derivatives()+ to ensure proper computation of accelerations.

As noted in table~\ref{tab:user_sdefine_methods}, it may be
desirous to also call the two derivative class jobs as scheduled or
logging class jobs. Doing so will not affect simulation dynamics, but it
will make the logged accelerations be consonant with the other logged values.


\subsubsection{Scheduled Jobs}\label{sec:user_integration_scheduled}
The model provides three methods that are expected to be called as
scheduled jobs. The method \verb+update_ephemerides()+ updates the
simulation's ephemeris models to reflect the current time. All simulations
that involve multiple gravitational bodies should include a call to
this method in the simulation's \Sdefine file.

The method \verb+perform_actions()+ performs queued body actions that
are ready to be applied. Some simulations only use the \BODYACTION
to initialize a simulation's dynamic bodies. These simulations do not
need to call \verb+perform_actions()+. A call to this method is needed
if the \BODYACTION is to be used to perform asynchronous operations.

Finally, the mechanism by which body actions are registered with the
simulation's dynamics manager is via calls to \verb+add_body_action()+.
JEOD models do not invoke this method. The recommended practice is to call this
method as a zero rate scheduled job in the simulation's \Sdefine file.
Because the calling rate is zero, the generated \verb+S_source+ file will not
call this method.
The method can however be invoked from a simulation input file,
and this is exactly how this method is intended to be used.

\subsection{Using the Model in a User Defined Model}
The public DynManager methods that are intended for use by model developers is
more-or-less orthogonal to the set intended for use at the \Sdefine level.
In particular, using \verb+add_body_action()+ and \verb+perform_actions()+ as
methods invoked by a user model is not a recommended practice.

The methods intended for use in user-defined methods fall into one of
three categories.\begin{itemize}
\item Lookup services.
  The lookup methods include\begin{itemize}
  \item {\tt find\_ephem\_item()}.
    Finds the EphemItem
    registered with the dynamics manager whose name matches the supplied value.
  \item {\tt find\_ephem\_angle()}, {\tt find\_ephem\_point()}, and
        {\tt find\_ephem\_planet()}.
    Specializations of \verb+find_ephem_item()+
    that restrict the lookup to specializations of EphemItem.
  \item {\tt find\_planet()}.
    Finds the Planet
    registered with the dynamics manager whose name matches the supplied value.
  \item {\tt find\_ref\_frame()}.
    Finds the RefFrame
    registered with the dynamics manager whose name matches the supplied value.
  \item {\tt find\_dyn\_body()}.
    Finds the DynBody
    registered with the dynamics manager whose name matches the supplied value.
  \end{itemize}
\item Registration services. Most of the registration methods are intended
  for use within JEOD. The one exception is \verb+add_ref_frame()+.
  JEOD cannot possibly anticipate every reference frame that users might
  want to use. For example, JEOD does not support the Mean of 1950 (M50)
  reference frame. A user-defined model can implement this frame and
  register it with the dynamics manager.
\item Frame subscription services.
  The frame subscription methods are\begin{itemize}
  \item {\tt subscribe\_to\_frame()}
    Increments a frame's subscription count.
  \item {\tt unsubscribe\_to\_frame()}
    Decrements a frame's subscription count.
  \item {\tt frame\_is\_subscribed()}
    Checks whether a frame's subscription count is positive.
  \end{itemize}
  Two versions of each method are defined. One takes a name of a reference
  frame as an argument; the other takes a reference to a RefFrame object.
\end{itemize}

\section{Extension}
The model is not designed to be extensible.