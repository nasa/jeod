%%%%%%%%%%%%%%%%%%%%%%%%%%%%%%%%%%%%%%%%%%%%%%%%%%%%%%%%%%%%%%%%%%%%%%%%
%
% Purpose:
%
%  
%
%%%%%%%%%%%%%%%%%%%%%%%%%%%%%%%%%%%%%%%%%%%%%%%%%%%%%%%%%%%%%%%%%%%%%%%%%

\begin{abstract}

The \ModelDesc\ maintains the mass properties of all non-planetary massive
objects in the simulation.  Mass-Body objects that also have a dynamic state are
represented as Dyn-Body objects, Dyn-Body being an inheritance from Mass-Body.
Consequently, the most frequently encountered Mass Body objects
are the Dyn-Body-based simulation vehicle(s); this model accurately maintains the mass
properties of such entities throughout processes such as mass depletion due to
fuel usage, the attaching and detaching of one Mass Body to/from another, and
the relative motion of individual massive components of a composite body
comprising multiple Mass Body elements (e.g. rotation of a solar panel on a vehicle).

When the dynamic state of a composite vehicle (one comprising multiple
components) is integrated, only one integration is performed for the entire
composite unit; the component parts are not integrated directly.
It is the \ModelDesc that is responsible for computing the overall mass and inertia
of the composite body that will be used in determining the dynamic response to
applied forces and torques.

\end{abstract}
