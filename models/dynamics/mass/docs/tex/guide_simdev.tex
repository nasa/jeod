\subsection{S\_define File}
Below is an example of how to include the \ModelDesc\ in an S\_define file
\subsubsection{Instantiation}

Instantiation of a MassBody is straightforward:

\paragraph{Trick07}
\begin{verbatim}
   dynamics/mass:    MassBody       body;
\end{verbatim}

\paragraph{Trick10}
\begin{verbatim}
   MassBody       body;
\end{verbatim}

Typically, the body requires some state information, in which case the
DynBody is used:

\paragraph{Trick07}
\begin{verbatim}
   dynamics/dyn_body:    DynBody       body;
\end{verbatim}
\paragraph{Trick10}
\begin{verbatim}
   DynBody       body;
\end{verbatim}


\subsubsection{Initialization}
To initialize the mass properties of a MassBody it is STRONGLY recommended
that the Body Action MassBodyInit be used.  Examples of usage are given in the
previous section (\textref{Instruction for Simulation
Users}{sec:guide_sim_user}) and in the Body Action documentation~(\cite{dynenv:BODYACTION}).

This body action will also initialize points associated with the MassBody if
they are allocated.  Allocation can be performed, for example, in the input
file right along with the declaration of the values.  In the S\_define, the
body-action itself needs to be instantiated.
\paragraph{Trick07}
\begin{verbatim}
   dynamics/body_action: MassBodyInit mass_init;
\end{verbatim}
\paragraph{Trick10}
\begin{verbatim}
   MassBodyInit mass_init;
\end{verbatim}

\subsubsection{Manipulation}
Performing actions on the body (such as attaching to another body, moving,
detaching, etc.) is also best handled by the MassBody Attach / Detach
sub-model of the Body Action model (see
documentation~\cite{dynenv:BODYACTION}).

Important points regarding attach and detach mechanisms:
\begin{itemize}
 \item When attaching using mass points, the axes of the mass points will be
 aligned on the z-axis, and anti-aligned on the x- and y- axes.  An assistive
 visual guide is to imagine two vehicles approaching with the x-axes of the
 two mass-points pointing at one another and z-axes aligned.
 \item When attaching using the offset and orientation method, the values
 specified are:
 \begin{itemize}
  \item The location of the child structure point with respect to the parent
  structure point, expressed in the parent structural-axes.
  \item The transformation matrix from the parent structural-axes to the child
  structural-axes.
 \end{itemize}
 \item When commanding an attach of object A to object B, the mass-tree will
 actually attach the root of the mass-tree containing object A to object B.
 Object A will only be attached to B in the mass-tree if A was originally the
 root of its own tree.
 \item When using the simple detach, remember that:
 \begin{itemize}
  \item The mass-tree will be severed immediately above the detached object.
  \item The mass-tree does not always represent reality.
  \item Consequence is that there are situations where if object A was
  commanded attached to object B, then object A commanded detached, the
  mass-tree would not return to its original form.  The user must be aware of
  the effect on the mass-tree of all attach and detach commands.
 \end{itemize}
\end{itemize}


\subsection{Utilizing the MassBody}
The two strengths of the \ModelDesc are in
\begin{itemize}
 \item Maintaining the properties of compound bodies through attach and detach
 processes.
 \item Providing anchor-points between which the relative state (position and
 orientation only) can be measured.
\end{itemize}

The maintenance of compound properties is performed automatically, as long as
the \textit{update\_flag} is set.  To set this flag, call
\textit{set\_update\_flag()} on the affected body farthest from the root of
the
tree.  This method will propagate up the tree, setting the flags for all
bodies between the body of interest and the root body (inclusive).
Note that this method is automatically called in the following situations:
\begin{itemize}
 \item At initialization of this MassBody object.
 \item At destruction of this MassBody object.
 \item A mass is attached to this body.
 \item This mass is detached from its parent (call made before links to parent
 have been canceled)
 \item A mass is moved (i.e. reattached to this mass).
\end{itemize}

The relative state (position and orientation only) of ``this'' mass point can
be found with respect to, and expressed in the axes-set associated with, any
other mass point in the same tree with a call to
\textit{compute\_relative\_state}.  This method takes two arguments:
\begin{enumerate}
 \item The other mass point (MassBasicPoint),
 \item The MassPointState to be populated with the relative state data.
\end{enumerate}

This method is distinct from the one provided in the Reference Frames model,
which produces a full relative state (including velocity and angular rate).
