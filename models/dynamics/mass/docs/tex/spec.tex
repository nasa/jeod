\chapter{Product Specification}\hyperdef{part}{spec}{}\label{ch:spec}
%----------------------------------

\section{Conceptual Design}
\subsection{The Basic MassBody}
The \ModelDesc\ must maintain an up-to-date record of the mass and inertia
tensor (the mass properties) of massive objects.  The inertia tensor
is referenced to a coordinate system with origin at
the center of mass and set of axes, the \textit{body-axes}, oriented as
desired.
Because a stand-alone massive block
is neither very interesting nor useful, additional characteristics are
incorporated to enhance the Mass Body, while maintaining its fundamental
purpose.

\subsection{Adding Dynamic State Information}
First, it is often desirable, though not essential, to monitor and propagate
the
dynamic state of a massive object.  This feature is provided through
composition,
whereby a MassBody object can be contained in a DynBody object~\cite{dynenv:DYNBODY},
with the dynamic state features.
Once again though, having a dynamically active massive object that just moves
subject to central forces is not particularly interesting or useful.

JEOD must be able to realistically propagate the response of a vehicle to
external forces and torques. The rotational response to an applied torque, and
the translational response to an applied force are both easy, but the
rotational response to an applied force requires knowledge of the relative
position vector between the center of mass of the object and the point at
which
the force is applied.  Conventionally, locations of objects within a vehicle
are defined in the \textit{structural} reference frame.  Consequently, a
MassBody also defines a point
(a MassBasicPoint, see \textref{Points of Interest}{sec:MassPoint}) called the
\textit{structure\_point}, relative to which all
positions can be defined or determined.  A new coordinate system, with origin
at the structure-point and with a set of axes called the
\textit{structural-axes}, is used
for enumeration of the position vectors.  The structural-axes and body-axes
are oriented and positioned relative to one another.

\paragraph {Aside on Structure-Body Orientation}
Typically, this relative orientation is straightforward, such as by
having the axes aligned, or with a 180-degree yaw, or 90-degree pitch, or
similar. See \textref{Composite and Core Properties}{sec:core_comp_prop} for
more details.

\paragraph {Aside on Axes and Reference Frames}
The two coordinate systems now defined are both further developed into
full-up reference frames -- with states all of their own -- in the
DynBody~\cite{dynenv:DYNBODY} class.
There, the structure-axes and body-axes become known as the structural and
body reference frames (respectively), and retain their orientations and their
origins at the structure-point and center of mass (respectively).

\subsection{Mass-Mass Attachments}
Even with a dynamic state, the object is still very limited in its
versatility.
A very significant additional feature, which by itself adds sufficient
complexity to justify the need for a separate model just to maintain the mass
properties, is the ability to connect
two (or more) massive objects to make compound objects.  When each of these
massive objects can be attaching or detaching to/from the compound object, or
changing its mass properties, or
moving with respect to the other mass entities in the compound object, then
the
maintenance of the compound mass properties becomes non-trivial and
necessitates a dedicated model for proper treatment.

To handle multiple massive objects connected together, we use a tree
structure, refered to as the \textit{mass-tree}.
Every MassBody is in one -- and only one -- tree; some trees are ``atomic'',
having only one member,
and some are more complex.  Each tree represents the elemental massive objects
that are physically connected together to make a compound massive object.
There may be more than one tree in any simulation, it is not necessary that
all MassBody objects be connected.

At the base (or top, depending on how it is visualized) of the tree is the
\textit{root} body,
which (unless it is atomic) has one or more \textit{children}.  Each child has
one (exactly one) \textit{parent}, and possibly one or more children of its
own.  There are two important restrictions on how these objects are positioned
in the tree:
\begin{itemize}
 \item Because the dynamic state is propagated from parent to child, a DynBody
 can never be a child of a non-dynamic MassBody.
 \item There can be no circular attachments (e.g. a situation in which A is a
 child of B, which is a child of C, which is a child of A, is forbidden.)
\end{itemize}
An important consequence is that the mass tree cannot always
reflect reality.  It is entirely feasible that two objects directly connected
on the mass tree have no physical connection; the mass tree is purely a
mathematical construct and should never be relied upon as an illustration of
physical connectivity.

Recall that each MassBody has a coordinate system associated with its
structure-point; when two MassBody objects are joined, those coordinate
systems should be linked.  Therefore, each
structure-point is given an orientation and position
specification that identifies its relative state with respect to its parent's
structure-point axes.  The orientation is represented as a quaternion and as a
transformation matrix.

Note that the root MassBody objects have no parent, thus it would correctly be
inferred that they have no defined orientation or position.  However, realize
that if a base-class
MassBody were at the root of the tree, then the entire tree
must comprise MassBody objects (as opposed to DynBody objects) since DynBody
objects cannot be children of MassBody objects.  Thus, the entire tree has
no state, thus absolute orientation and position are not defined, and all that
is necessary
is relative orientation and position of the elemental MassBody objects with
respect to one
another.  When (as is typical) the root is actually a DynBody, the structural
reference frame that is developed from the structure-point is given a parent,
thus the orientation and position of the structural reference frame, and thus
of the structure-point, are defined.

\subsection{Core and Composite Properties}\label{sec:core_comp_prop}
Consider a compound object, comprising multiple MassBody objects.  The
\ModelDesc must provide the mass properties of this single entity, but it must
also retain the mass properties of the individual components, and the mass
properties of all sub-trees:
\begin{itemize}
 \item Mass-loss would be applied to elemental components, and then
 incorporated
 into the compound object, so elemental properties must be retained so that
 they can be manipulated.
 \item Motion or detachment of elemental components -- or of a section of the
 mass tree --
 requires that the mass properties of that moving entity be known so that
 the mass and inertia can be subtracted out (and added back in the case of
 motion).
\end{itemize}
Thus, every parent object in the tree must, at all times, keep information on
itself, and on that part of the tree of which it is the head (i.e. itself, and
all of its children, children of children, etc.).  Thus, we keep two sets of
mass properties associated with each MassBody:
\begin{enumerate}
 \item \textbf{Core Properties} are those properties associated with the
 center of mass of the elemental body.
 \item \textbf{Composite Properties} are those properties associated with the
 center of mass of that part of the mass tree of which it is the head.
\end{enumerate}

To simplify the determination of whether a MassBody needs a set of composite
properties, we just provide this capacity to all MassBody objects; in the case
of a body without children, the core properties and composite properties are
equivalent.

With each set of properties comes a different center of mass, and consequently
a different set of body-axes.  A Mass Point is automatically created for each
center of mass, thereby setting
the relative position of each set of body-axes relative to the structural
axes, and the relative orientation of each of
the body-axes sets with respect to the structural-axes.  Note that both sets
of body-axes have the same orientation.

 \paragraph{Aside on locations of variables}
The relative position is stored as the respective properties' position (e.g.
\textit{composite\_properties.position}). The relative orientation of each of
the body-axes sets with respect to the structural-axes is specified as a
transformation matrix or quaternion-set (e.g.
\textit{core\_properties.T\_parent\_this}).  The rationale for this choice is
expounded in \textref{Detailed Design}{sec:detailed_design}.

\paragraph{Aside for DynBody use}

In a DynBody, there are three distinct reference frames - the
core-body frame, the composite-body frame, and the structural frame.  The
structural frame has its origin at the structure-point, while the body frames
have their origin at the respective centers of mass.  Therefore, the position
of a body frame with respect to the structural frame in a DynBody is equal to
the respective properties' position value (e.g.
\textit{core\_properties.position}).  By analogy, the orientation of the body
frames with respect to the structural frame is the same as that of the
body-axes with respect to the structural-axes. For example, the transformation
from the structural frame to the composite-body frame is
\textit{composite\_properties.T\_parent\_this}.

\subsection{Points of Interest}\label{sec:MassPoint}
There are numerous instances where some point on a body is of particular
interest, and its position needs to be well defined - the location of an
antenna, or a sensor, the point at which another object is attached, etc.  In
many applications, it is also desirable to specify an axes-set at the point so
that other positions can be defined with respect to the point.  Consequently,
we include an orientation in the defining data of each point.

We have two types of points - the basic \textit{MassBasicPoint}, and the more
commonly used \textit{MassPoint} (which is, essentially, a MassBasicPoint with
a name; the name provides the user with a way to identify the mass point).

Every MassBody starts with three MassBasicPoints, and we have already
considered them:
\begin{enumerate}
  \item The structure-point; axes are the structure-axes.
  \item The composite-properties point; axes are the body-axes with origin at
  the composite center of mass.
  \item The core-properties point; axes are the body-axes with origin at the
  core center of mass.
\end{enumerate}

Any MassBody can then be given any number of additional, user-specified,
MassPoint instances.  Every MassPoint gets added to the mass-tree as a child
of the structure-point of the same MassBody (obviously, excepting the
structure-point itself, which is a child of the structure-point of the parent
MassBody, and thereby a sibling to other MassPoint instances of the parent
MassBody).


\subsection {Summary}
A MassBody comprises three components:
\begin{enumerate}
 \item A MassBasicPoint, called the \textit{structure\_point} that provides:
 \begin{enumerate}
  \item A position and orientation relative to the parent body (if it exists)
  \item A reference point and axes (\textit{structural-axes}), from which
  further measurements may be made (such as to define the location of the
  center of mass).
 \end{enumerate}
 \item A MassProperties object, \textit{core\_properties}, that provides:
   \begin{enumerate}
   \item Mass
   \item Inertia tensor, referenced to a specified axes-set, the
   \textit{body-axes}.
   \item Position of the center of mass with respect to its structure-point,
   expressed in the structural-axes.
   \item The orientation of the body-axes with respect to the
   structural-axes.
   \end{enumerate}
   for the stand-alone MassBody.
  \item An additional MassProperties object, \textit{composite\_properties},
  that provides the same set of properties for the compound object comprising
  the MassBody and everything attached \textbf{to} it in the mass tree (not
  including objects that it is attached to, the hierarchy in the mass tree is
  very important).
\end{enumerate}


\section{Mathematical Formulations}
\subsection {Mass}
The core and composite masses are found at \textit{*\_properties.mass}.
\subsubsection {Core Property}
This value has to be set externally (e.g. by the user).
\subsubsection {Composite Property}
This value is computed trivially,
\begin{equation}
  M_{composite} = M_{core} + \sum_{children} M_{composite,i}
\end{equation}
where $M_{composite,i}$ is the composite mass of each of the children of this
Mass Body.  Clearly, generation of the composite mass is an iterative
procedure, requiring first the calculation of the corresponding value for each
of the children, which require the same for theirs, etc.

\subsection{Center of Mass}
The position of the core and composite centers of mass are found at
\textit{*\_properties.position}; the value is expressed in the structural axes.
\subsubsection {Core property}
This value has to be set externally (e.g. by the user).
\subsubsection {Composite Property}
The position of the composite center of mass is derived from the respective
positions of all of the components of the sub-tree originating with this body,
using classical mechanics:
\begin{equation}
{M_{composite}} \cdot \vec{x}_{composite} = M_{core}  \cdot \vec{x}_{core}  +
\sum_{children}{M_{composite,i} \cdot \vec{y}_{composite,i} }
\label{CoM_cal}
\end{equation}

where $\vec{y}_{composite,i} $ is the position of the respective composite
center of mass for each of the children, expressed with respect to, and in,
the structural axes of this mass body.

\begin{equation}
 \vec{y}_{composite,i} = \vec{x}_i + \relvect T i {this} \left(
 \vec{x}_{composite,i} \right)
\end{equation}

with $\vec{x}_i$ the position of the child's structure point expressed in, and
with respect to the structural axes of this body, and $\relvect T i {this}$ is
the transformation matrix from the structural axes of the child body to the
structural axes of this body.

Again, this is clearly an interative process.


\subsection{Inertia Tensor}
The inertia tensor for the core and composite bodies are found at
\textit{*\_properties.inertia}; the value is referenced to the respective body
axes. The diagonal elements are positive moments of inertia, while the
off-diagonal elements are negative products of inertia.


\subsubsection {Core Property}
This value has to be set externally (e.g. by the user).
\subsubsection {Composite Property}
Computation of the composite body inertia tensor is a multi-step process:
\begin{enumerate}
 \item Compute the inertia tensor for the core-body, referenced to the
 composite-body body-axes, rather than the core-body body-axes.  Since the two
 sets of axes are aligned, we can use the parallel axis theorem:
 \begin{equation}\label{eq:inertia_offset}
  \inertia_{core:comp} = \inertia_{core:core} + M_{core} \begin{bmatrix} y^2 +
  z^2 & -xy & -xz \\-xy & x^2 + z^2 & -yz \\ -xz & -yz & x^2+y^2 \end{bmatrix}
 \end{equation}
 where $x$, $y$, and $z$ represent the position of the core center of mass
 relative to the composite center of mass, expressed in the composite
 body-axes.
 \item For each child, transform a copy of the child's composite-body inertia
 tensor so that it is referenced to this (i.e. the parent) body's
 composite-body body-axes, rather than its own.  This is a multi-step process:
 \begin{enumerate}
  \item Compute the position of the origin of the child's composite-body-axes
  relative to that of this body.
  \item Compute the orientation of the child's composite-body-axes relative to
  that of this body.
  \item Use the orientation data to apply a rotational transformation to the
  child's composite-body inertia tensor such that it references the parent's
  composite-body-axes.  This step is necessary because the composite-body-axes
  for the child body are, in general, not aligned with those for the parent
  body. Consider
  \begin{equation*}
   \tau = \inertia \alpha
  \end{equation*}
  Hence, $\tau_{parent}$ can be expressed as
  \begin{equation*}
   \tau_{parent} = \inertia_{parent} \alpha_{parent}
  \end{equation*}
  and also expressed as a transformation of the same expression in the child
  frame:
  \begin{equation*}
   \tau_{parent} = T_{child \rightarrow parent} \left( \inertia_{child}
   \left( T_{parent \rightarrow child} \left( \alpha_{parent} \right) \right)
   \right)
  \end{equation*}
 Consequently,\begin{equation}
   \inertia_{parent} = T_{child \rightarrow parent}  \inertia_{child}
   T_{parent \rightarrow child}
  \end{equation}
  This term represents the inertia tensor in a set of axes aligned with the
  parent composite-body-axes, with an origin that still matches that of the
  child composite-body-axes.
  \item Evaluate and add the parallel-axis theorem addition term in equation
  \ref{eq:inertia_offset}, such that:
  \begin{equation}\label{eq:inertia_child}
   \inertia_{child:parent} = T_{child \rightarrow parent}
   \inertia_{child:child}   T_{parent \rightarrow child} + M_{core}
   \begin{bmatrix} y^2 + z^2 & -xy & -xz \\-xy & x^2 + z^2 & -yz \\ -xz & -yz
   & x^2+y^2 \end{bmatrix}
  \end{equation}
  where $x$, $y$, and $z$ represent the position of the child composite-body
  center of mass relative to the parent composite-body center of mass,
  expressed in the parent composite-body-axes.  $M_{child}$ is the mass of the
  child composite-body.

 \end{enumerate}
 \item Add to the adjusted parent body inertia tensor (equation
 \ref{eq:inertia_offset}) the resulting inertia tensors for each of the
 children (equation \ref{eq:inertia_child})to give the composite body inertia
 tensor for the parent.\begin{equation}
  \inertia_{comp:comp} = \inertia_{core:comp} + \sum_{children}
  \inertia_{child:parent}
 \end{equation}

 \end{enumerate}




\subsection{Inverse Inertia}
The inverse-inertia tensor for a body is found at
\textit{inverse\_inertia}; the value is referenced to the respective body
axes.The inverse inertia is only needed where torques are going to be applied
to a
MassBody, and then only if the MassBody has a dynamic state to respond to
those torques (i.e. if it is contained in a DynBody).  Furthermore, since torques (and
forces) are only applied to a mass tree in its entirety, the only inverse
inertia that is needed is that of the composite properties of the root body.
Consequently, the inverse inertia is only computed if the MassBody is contained in a
DynBody, if it is at the root of its tree (including single-entity trees), and
then only the composite inertia is inverted.

\subsection{Transformation from Structure to Body}
The transformation from the strucutral-axes to the body-axes is found at
\textit{*\_properties.T\_parent\_this} and at
\textit{*\_properties.Q\_parent\_this} (for a transformation matrix and
quaternion set, respectively).  This value is set at initialization for the
core-properties;

\subsubsection {Core Property}
This value has to be set externally (e.g. by the user) at initialization; it
is fixed for the duration of the simulation.
\subsubsection {Composite Property}
Since both sets of body axes are aligned, the value set for the
core-properties gets copied into the composite-properties at initialization.

\subsection{Attaching Bodies}
The process by which bodies are attached is outlined in the Detailed Design
section (\ref{sec:attaching_bodies})









\section{Detailed Design}\label{sec:detailed_design}
\subsection{API}
Follow this link for the
\href{file:refman.pdf}{\em \ModelDesc\ API}~\cite{api:mass}.

\subsection{Class Overview}

The MassBody is the basic class for a Mass Body.  It comtains (this is not an
exhaustive list):
\begin{itemize}
 \item \textit{core\_properties}, an instance of MassProperties, that
 describes the properties of the MassBody as a single entity.
 \begin{itemize}
  \item A MassProperties class is a MassBasicPoint which also provides data
  elements for mass and the inertia tensor.
  \begin{itemize}
   \item A MassBasicPoint is a MassPointState that provides the linkages
   within the mass tree (to its children, parent, etc.).  A mass tree
   comprises the collection of MassBasicPoint instances that are physically
   connected at some point.
   \begin{itemize}
    \item A MassPointState provides data elements for the position of the
    point, and the orientation of the axes associated with the point, with
    respect to some parent.
   \end{itemize}
   \item A related item, a MassPoint, is simply a MassBasicPoint with a name,
   allowing for it to be found easily.
  \end{itemize}
  \end{itemize}


 \item \textit{composite\_properties}, another instance of MassProperties,
 that describes the properties of the subtree of which the body is the head.
 \item \textit{structure\_point}, an instance of MassBasicPoint (see above).
\end{itemize}

Note that since a MassProperties is a MassBasicPoint, it must have a parent
for evaluation of its position and orientation.  The parent of both instances
of Mass Properties (\textit{core\_properties} and
\textit{composite\_properties}) is the MassBasicPoint
\textit{structure\_point}; the position and orientation are stated relative to
the structural axes.


\subsection{Attaching Bodies}\label{sec:attaching_bodies}
Attaching two bodies together is typically performed with a Body Action.  A
rough outline of the algorithm is presented here, instructions for
implementation are contained in the Body Action document
(\cite{dynenv:DYNBODY}).

The following rules provide the restrictions on which bodies can be commanded
attached to which bodies.
\begin{itemize}
 \item A DynBody can never be a child of a basic
 MassBody.
 \item There can be no circular attachments.
\end{itemize}
IMPORTANT NOTE: Only root bodies may actually attach to another body.  While
it may be legal to command the attachment of body A to body B (subject to the
rules above), the attachment in the mass tree will be represented as the root
of the tree containing body A attaching to body B.

\subsubsection{Attaching using Points}
The easiest implementation of attaching two bodies is to define an attach
point (a MassPoint) on each body, and make those points coincident, with their
z-axes aligned, and their x- and y-axes both anti-aligned.  Because each
MassPoint will have position and orientation defined with respect to its
respective body's structural axes, it is a striaghtforward undertaking to
obtain the orientation and position of the child body's structural axes with
respect to the parent body's structural axes.  Then, the general attachment
method can be implemented, and the parent will have as children its MassPoint
used for attachment and the child structure-point; the child will have as a
child its MassPoint used for attachment.

\subsubsection{Attaching using Offset and Orientation}
This more general form of the attach process requires knowledge of the
orientation and relative position of the child body's structural-axes with
respect to those of the parent body, which may be provided directly or from
the previous method.

If the child body is not the root of its own tree, its root is first found,
then the relative position and orientation of that root body's structure-axes
relative to the structure-axes of the parent body are determined.

With the root state known, the attachment of the root of the child's tree (the
child-root) to the parent can proceed in three steps:
\begin{enumerate}
\item{\bf Validate:} Is this a valid attachment which for the \ModelDesc,
means it follows a tree structure and avoids
invalid circular attachments.
\item{\bf Establish Links:} The child-root MassBody is required to establish
the links to the parent MassBody that define their
relationship in the mass tree.
\item{\bf Update Properties:}  The parent MassBody updates its composite
properties to reflect the addition of the child-root MassBody.
The composite properties of the child-root remain unchanged.
\end{enumerate}

A few examples will help illustrate the concept of attachments and we will
start by supposing that we have three MassBodys called
A, B, and C respectively.

In the first example we decide to attach MassBody C to MassBody B thereby
making C a child of B.  Given that no other attachments exist in
this example this is a valid attachment so we will pass step one(Validate).
In step two(Establish Links) the child will update the
links between the two MassBodys including MassPointLinks and MassBodyLinks.
In step three(Update Properties) the parent will
update its composite properties (mass, inertia, center of mass) to represent
the combined state.

Building on the new mass tree created with the attachment of MassBody C to
MassBody B, we will now attach MassBody B to MassBody A.
MassBody B will then become a child of MassBody A, and MassBody C will remain
a child of MassBody B.  In this attachment process, the validity is checked --
since A is not already attached to B or C this is approved.  MassBody B now
performs the Establish Links step configuring all MassPointLinks and
MassBodyLinks with MassBody A.  Then MassBody A uses the composite properties
of MassBody B to perform the Update Properties step.  It should be noted that
the
composite properties of MassBody B already include the properties of
MassBody C so, following the update, the composite properties of A represent
the entire tree.
Even though the properties of A do contain information for the entire tree,
MassBody A has no knowledge that
MassBody C exists.  MassBody A only knows about MassBody B.

Consider another situation: return to the situation where only MassBody B and
MassBody C are attached and C is a child of B.
Suppose we then want to attach MassBody C to MassBody A.  The \ModelDesc\
recognizes that MassBody C is not a root body, so instead attaches MassBody B
to MassBody A.  The offset and rotation of B with respect to A are set such
that the offset and rotation of C with respect to A are as requested.  Again,
the composite mass properties of A will include those of C through the
composite properties of B.

In both situations, the mass tree looks identical.

\subsection{Detach Overview}
In the detach scenario, the links between the body to be detached and its
parent are severed, and the composite properties of the parent are updated to
reflect that one of its children has been removed.  The composite properties
of the child body are unchanged.
\clearpage
\boilerplateinventory
