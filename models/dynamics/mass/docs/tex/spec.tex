\chapter{Product Specification}\hyperdef{part}{spec}{}\label{ch:spec}
%----------------------------------

\section{Conceptual Design}
This section describes the key concepts of the \ModelDesc and of the 
\erseven Integration module.

\subsection{Initial Value Problem}
An initial value problem comprises 
\begin{itemize}
\item A set of variables that describe some state,
\item A set of initial values for these state variables,
\item A set of ordinary differential equations (ODEs) that describe how state
  changes with respect to an independent variable (typically time), and
\item An interval over which state is to be advanced.
\end{itemize}

Three initial value problems currently arise in JEOD:
\begin{itemize}
\item Propagating the temperature of a radiative surface
  from one time step to the next.
  The temperature of the surface varies with time per the
  Stefan-Boltzmann law.
\item Propagating the translational state (position and velocity) of a
  dynamic body from one time step to the next.
  The translational state varies with time per Newton's second law.
\item Propagating the rotational state (orientation and angular velocity) of a
  dynamic body from one time step to the next.
  The rotational state varies with time per the rotational analog of
  Newton's second law.
\end{itemize}

The three initial value problems mentioned above in general do not have nice
closed-form solutions. Numerical techniques must be used to yield approximations
to the solutions to the problems.
The principal purpose of the \ModelDesc is to provide tools that
numerically solve such initial value problems.
See Shampine\cite{shampine:1994} for a generic discussion of numerical
solution of ordinary differential equations.

\subsection{Integration Techniques}

In the context of the \ModelDesc, an integration technique is a specific
algorithm that approximately propagates state over time per some ODE.
There are many different such techniques, several of which
are provided by the \erseven and by JEOD. They range in
complexity from the very simple Euler techniques to the rather complex
Gauss-Jackson and LSODE integrators. The paragraphs that follow describe
various ways to look at different integration techniques.

\subparagraph{%
Does the technique take multiple steps to achieve the desired end point?}
\leavevmode\newline
All but the very simplest of integration techniques involve multiple steps to
reach the desired end point.
Most instances of the Runge-Kutta family of integration techniques integrate
to intermediate points in the integration interval before making the final
step that necessarily hits the end of the integration interval. For example,
the midpoint method first integrates to the middle of the integration interval
before integrating to the end of the interval, and the classical fourth order
Runge-Kutta technique goes to the midpoint twice before going to the end
of the interval (which it also hits twice).

Predictor-corrector methods also take multiple steps to reach the desired
end point. Each step of a predictor-corrector method advances state to the
end of the integration inteval, but these methods move to the end of the
interval multiple times. The first step uses a predictor algorithm to predict
the value at the end of the interval. The following steps use a different
algorithm to correct that value. Simple predictor-corrector algorithms
apply the correction but once. More advanced techniques repeatedly apply the
corrector until the changes to the corrected state is less than some tolerance.

\subparagraph{%
Does the technique divide the integration interval into subintervals?}
\leavevmode\newline
This concept is distinct from the above.
Integration techniques that do this are adaptive techniques.
An adaptive technique uses some underlying integration technique to propagate
over a subinterval. The adaptive part of the technique tries to find an
optimal length for the subinterval, neither too long nor too short, that
makes the integrator perform close to optimal with regard to accuracy.

\subparagraph{Does the technique use historical data?}
\leavevmode\newline
All but the simplest of predictor-corrector techniques use historical data
to improve the accuracy of the prediction/correction. This improved accuracy
comes at a cost. One issue is that such techniques are not self-starting. Some
other technique must be used to provide the initial set of history data.
The other issue is that discontinuities in the derivatives (e.g., a satellite
subject to radiation pressure coming out of or going into Earth shadow)
can wreak havoc with the historical data. Users of such techniques need
to be aware of the pitfalls as well as the advantages of these techniques.

\subparagraph{How many derivative evaluations are needed?}
\leavevmode\newline
One of the key advantages of predictor-corrector techniques over Runge-Kutta
techniques is that given techniques of the same order, the former tend to need
far fewer evaluations of the derivatives to attain the same level of accuracy
if the function to be integrated is analytic. The fewer derivative calls, the
better, because evaluating derivatives tends to be very expensive
computationally.

\subparagraph{What kind of accuracy can be acheived/expected?}
\leavevmode\newline
This is possibly the most important question of all.
Any numerical integration technique is subject to errors inherent to the
technique itself.
Some techniques are numerically unstable if the time step is too large,
and even where they are stable, truncation errors inevitably result from the
approximations used by the techniques. These truncation errors tend to
decrease as the integration interval decreases, but not indefinitely. At
some point, round-off error that is inherent to the use of finite precision
arithmetic (e.g., double precision variables) overwhelms the truncation error.
For integration intervals smaller than this transition point, decreasing the
size of the integration interval increases the error in the solution.

Characterizing the error behavior of the various integration techniques is
a key aspect of the testing section of this document.

\subsection{Trick Integration} \label{sec:con_des_trick}

The literature on numerical integration of an ODE will typically have the
integrator call some function to calculate the derivative.
Except for the simplest of integrators, there will be
many calls to the derivative function sprinkled throughout the algorithm.

This is not how integration works in a Trick simulation.

In a Trick simulation, derivative class jobs calculate various quantities
that eventually result in state derivatives while integration class jobs
advance state using those already calculated derivatives.
The Trick integration loop calls all of the pertinent derivative class jobs
and then calls all of the pertinent integration class jobs.

This simplifies the development of a Trick-based simulation at the expense of
making things rather difficult for the developers of an integration technique.
Except for the very simplest of integration techniques which call the derivative
function once per integration cycle, the myriad calls to the derivative function
must be implemented as if the integrator was a reentrant coroutine.
The approach taken in all of the \erseven and JEOD state integration
techniques is to emulate this reentrant coroutine behavior via
a finite state machine.

\subsection{ER7 Utilities Integration} \label{sec:con_des_er7}

The \ModelDesc largely stood on its own in previous JEOD releases.
This is no longer the case in \JEODid.
The integration framework and most of the integration techniques have been
migrated to the \erseven package.
This section summarizes the key concepts of the \erseven Integration module.

\subsubsection{Integrable Object}
An integrable object is an object such as a JEOD DynBody that contains
time-dependent state information that needs to be propagated over time in
accordance with some ordinary differential equation.
The integrable object presumably will contain one or more state integrators
(see below) to perform this state propagation.

\subsubsection{State Integrator}
A state integrator is an object that uses a specific integration technique to
propagate state over time. To conform with the Trick integration scheme, the
object's integration function must receive derivatives and finite state as
input arguments.

\subsubsection{Integrator Result}
A state integrator returns an integrator result object to indicate whether
the integrator successfully achieved the target state and to indicate the
amount by which integration time should be advanced.

\subsubsection{Integrator Result Merger}
Oftentimes multiple integrable objects are integrated concurrently.
The integrator results returned by the multiple state integrators must be
merged to form a single merged integrator result. An integrator result merger
object performs these merge operations.

\subsubsection{Integration Group}
An integration group conceptually contains a set of integrable objects that are
to be integrated concurrently. All objects in the group have their integrated
state advanced at the same dynamic step size. On each step within the
integration process, all derivative class jobs associated with the group are
called, followed by calls to each of the integration class jobs associated with
the group.
This means the state integrators used by the objects must be compatible
with one another. On each integration step, the state integrators used by the
integrable objects in the group must advance state to the same point in time.

\subsubsection{Integration Controls}
An integration controls object works with an integration group to manage the
integration process. The group, along with the integrable objects contained by
the group, are indifferent to the specific integration technique used to
propagate state.  Each integration group contains an integration controls object
that directs the integration process at the group level to the target point in
time. The integration controls object manages the finite state machine that
controls the behavior of the state integrators used by the group's integrable
objects. The integration controls defers the calling of the integrable objects
contained with the integration group to the group. The integration result
returned by the group tells the integration controls how to advance the
finite state machine and how to advance time.

\subsubsection{Time}
The \erseven integration module defines time in a very abstract sense.
Integration time in Trick is contained within the Trick Integrator object,
while time in JEOD is a global set of objects. The only common aspect between
these two very disparate concepts is that time is something that needs to be
updated as a part of the integration process.

\subsubsection{Integrator Constructor}
There are two key consistency issues regarding the use of multiple state
integrators within an integration group. On each integration step,
the integration controls' finite state machine directs the state integrators
to advance integrable state to some specific point in time. This means that
state integrators used within an integration group must behave consistently with
regard to the finite state machine managed by the integration controls and
with regard to how time is to be advanced.

This presents a potential issue for JEOD because of the global nature of the
JEOD time objects and because of the use of multiple state integrators within
an integration group. This issue does not arise if the integration controls and
state integrators are consistent with one another.

An integrator constructor creates objects that collectively and consistently
work together to integrate the states of integrable objects over time.
An integrator constructor creates\begin{itemize}
\item State integrator objects that integrable objects can use to
integrate state,
\item Integrator results merger objects that state integrators and integration
groups can use to merge results from multiple integrators, and
\item Integration controls objects that integration groups can use to
direct the integration process.
\end{itemize}
Using an integrator constructor ensures that the state integrator objects,
integrator results merger objects, and integration controls object used within
an integration group are consistent with respect to the finite state machine
and with respect to time.

\subsubsection{Supported Integration Techniques}
\label{sec:con_des_er7_integ_techniques}

The \erseven provides a number of integration techniques. Each integration
technique includes a technique-specific integrator constructor that ensures
consistency of state integrators and integration controls across an integration
group. The paragraphs that follow describe the techniques provided with
the \erseven Integration model.

\paragraph{Euler's Method}
Euler's method is the simplest of integration techniques.
It advances the first order ODE $\dot x(t) = f(x,t)$ via
$x(t+\Delta t) = x(t) + \Delta t \, f(x,t)$.
Euler's method is the basis for all of the other integration techniques supplied
with the \erseven Integration model. Euler's method itself exhibits a global
error that is first order in time. This means Euler's method typically exhibits
very poor accuracy. As a result, its use is not recommended.

See appendix \ref{app:euler} for details.

\paragraph{Symplectic Euler Method}
The symplectic Euler is the simplest of integration techniques
specialized to second order ODE problems.
It advances  the second order ODE
$\dot x(t) = v(t)$, $\dot v(t) = f(x,v,t)$ via
$v(t+\Delta t) = v(t) + \Delta t \, f(x,v,t)$,
$x(t+\Delta t) = x(t) + \Delta t \, v(t+\Delta t)$.
The symplectic Euler method is only marginally better than the basic Euler
method when applied to second order ODE problems.
As is the case with the basic Euler method,
users are advised to avoid using the symplectic Euler method.

See appendix \ref{app:symplectic_euler} for details.

\paragraph{Beeman's Algorithm}
Beeman's algorithm is a predictor-corrector algorithm tailored to
second order ODEs. Its global error is second order in time, making it
significantly better than the Euler techniques described above.

See appendix \ref{app:beeman} for details.

\paragraph{Second Order Nystr\"{o}m-Lear (NL2)}
The second-order Nystr\"{o}m-Lear techhnique is a Runge-Kutta-style technique
tailored to second order ODEs. As the name suggests, it also exhibits a
global error that is second order in time.

See appendix \ref{app:nl2} for details.

\paragraph{Position Verlet}
The position verlet method is a member of the verlet family of integrators,
all of which exhibit a global error that is second order in time.
The verlet algorithms alternate between advancing position and velocity at
half step intervals. The position verlet algorithm advances velocity at the
half step, which means that position verlet only needs one derivative
call per integration step.

See appendix \ref{app:position_verlet} for details.

\paragraph{Heun's Method}
Heun's method can be viewed as either a second order Runge-Kutta integrator
or as a simple predictor-corrector algorithm. It advances state as a full
Euler step using the derivative at the start of the interval
followed by a corrective Euler step using the average of initial and final
derivatives.

See appendix \ref{app:rk2heun} for details.

\paragraph{Midpoint Method}
The midpoint method is another member of the second order Runge-Kutta family
of integrators. It advances state as an Euler step to the midpoint of the
interval followed by a full Euler step using the midpoint derivatives.

See appendix \ref{app:rk2midpoint} for details.

\paragraph{Velocity Verlet}
The velocity verlet method is a member of the verlet family of integrators.
The velocity verlet algorithm advances velocity at step boundaries.
Unlike the position verlet method, the velocity verlet needs two derivative
calls per intergration step because of drag. Also unlike the position verlet
method, the velocity verlet method is self-starting.

See appendix \ref{app:velocity_verlet} for details.

\paragraph{Modified Midpoint (MM4)}
This method is of unknown heritage. The name suggests that it exhibits an
error that is fourth order in time. This is not the case.
This method exists for backward compatibility only. Its use is not recommended.

See appendix \ref{app:mm4} for details.

\paragraph{Fourth Order Adams-Bashforth-Moulton (ABM4)}
The fourth order Adams-Bathforth-Moulton technique is a predictor-corrector
algorithm for first order ODEs. This technique uses the four step
Adams-Bashforth integrator as a predictor and the three step Adams-Moulton
as a corrector. This technique exhibits a global error that is fourth order
in time. It only requires two derivative calls per step, but does so at the
expense of not being self-starting and at the expense of mishandling
discontinuities in the derivatives.

See appendix \ref{app:abm4} for details.

\paragraph{Fourth Order Runge-Kutta}
The classical fourth order Runge-Kutta technique advances state by taking
four steps per integration cycle. The first two steps advance state to the
midpoint of the integration interval while the last two advance state to the
end of the interval. As the name suggests, this technique exhibits a global
error that is fourth order in time.

See appendix \ref{app:rk4} for details.

\paragraph{Fourth Order Runge-Kutta-Gill}
The fourth order Runge-Kutta-Gill method uses the same intermediate steps as
does the classical fourth order Runge-Kutta technique but uses a different
Butcher tableau.

See appendix \ref{app:rkg4} for details.

\paragraph{Fifth Order Runge-Kutta-Fehlberg}
The Runge-Kutta-Fehlberg family of techniques were designed for use in adaptive
integrators, integrators that change the step size to keep the error within
some bound. These style integrators use a pair of Runge-Kutta integrators
to provide different estimates of the integrated state. The fifth order
Runge-Kutta-Fehlberg technique provided with the \erseven Integration model
is not adaptive. It is the higher order member of the fourth and fifth
order techniques used by an adaptive Runge-Kutta-Fehlberg 4/5 integrator.

See appendix \ref{app:rkf45} for details.

\paragraph{Eighth Order Runge-Kutta-Fehlberg}
The eighth order Runge-Kutta-Fehlberg is the higher order member of the seventh
and eighth order techniques used by an adaptive Runge-Kutta-Fehlberg 7/8
integrator. This technique, like the fifth order Runge-Kutta-Fehlberg provided
with the \erseven Integration model, is not adaptive.

See appendix \ref{app:rkf78} for details.

\subsubsection{Supported Initial Value Problems}

The techniques listed above address one or more of the following types of
initial value problems:\begin{itemize}
\item First order ODE,
\item Second order ODE,
\item Second order generalized derivative ODE, and
\item Second order Lie group ODE.
\end{itemize}
These are described in detail in the following paragraphs.

There is an obvious mismatch between the first and second order techniques
in the above list. The reason is simple: Nobody has asked for integrators
that solve the first order ODE equivalents of the second order generalized
derivative and Lie group problems.

\paragraph{First order ODE}
The first order ODE $\dot x(t) = f(x(t),t)$
describes the behavior of the problem to be solved.
The state integrator for a first order ODE solver
receives the vector $x(t)$ and its time derivative $v(t)=f(x(t),t)$
as arguments.

Most of the techniques listed in section~\ref{sec:con_des_er7_integ_techniques}
are first order ODE solvers. Others are specific to second order ODEs. While a
first order ODE solver can be used to solve a second order ODE, the converse
is not true. Those techniques that are specific to second order ODEs provide
a surrogate when asked to provide a solver for a first order ODE.

\paragraph{Second order ODE}
The second order ODE
$\dot x(t) = v(t)$, $\dot v(t) = f(x(t),v(t),t)$
describes the behavior of the problem to be solved.
The state integrator for a second order ODE solver
receives the vectors $x(t)$, $v(t)$, and $a(t)=f(x(t),v(t),t)$ as arguments.

All of the techniques listed in section~\ref{sec:con_des_er7_integ_techniques}
can solve a second order ODE. Some of the listed techniques specifically address
second order ODEs; the remaining techniques are first order ODE solvers that can
easily be adapted to solving a second order ODE.

\paragraph {Second order generalized derivative ODE}
The generalized second order ODE
$\dot x(t) = g(x(t),v(t))$, $\dot v(t) = f(x(t),v(t),t)$
describes the behavior of the problem to be solved.
As is the case with the basic second order ODE solver, the state integrator
for a generalized derivative second order ODE solver
receives the vectors $x(t)$, $v(t)$, and $a(t)=f(x(t),v(t),t)$ as arguments.

The function $g(x(t),v(t))$ maps generalized position and generalized velocity
to the first time derivative of generalized position.
Some integration techniques need an auxiliary function $h(x(t),v(t),a(t))$
that maps generalized position, generalized velocity, and the time derivative
of generalized velocity to the second time derivative of generalized position.
These two functions must be specified when the generalized derivative integrator
is created.

All of the techniques listed in section~\ref{sec:con_des_er7_integ_techniques}
can solve a second order generalized derivative ODE.

\paragraph {Second order Lie group ODE}
The generalized second order ODE
$\dot x(t) = \alpha\, v(t) \bullet x(t)$, $\dot v(t) = f(x(t),v(t),t)$
describes the behavior of the problem to be solved.
As is the case with the basic second order ODE solver, the state integrator
for a generalized derivative second order ODE solver
receives the vectors $x(t)$, $v(t)$, and $a(t)=f(x(t),v(t),t)$ as arguments.

In the above, the generalized position $x(t)$ is an element of a Lie group, the
generalized velocity is an element of a Lie algebra for that Lie group, and
$\alpha$ is a constant scale factor. The product $v(t) \bullet x(t)$ is
performed in the context of the group action of the Lie group.
This multiplication typically is not commutative in these Lie group problems:
$a \bullet b \ne b \bullet a$ in general. Indeed, the Lie bracket
$[a,b] = a \bullet b - b \bullet a$ typically is not zero, and typically plays
a key role in the solutions to the problem.

The second order Lie group problems are obviously a special case of the second
order generalized derivative problems described above. The rationale for these
Lie group solvers is that they pay attention to the geometry of the problem.
If a technique provides a Lie group solver it will inevitably perform much
better than the corresponding generalized derivative solver.

This would be an academic exercise were it not for the fact that rotations in
three dimensional space form a Lie group, the group $SO(3)$. The Lie group
solvers were developed because of the relevance of this group to physics-based
simulations and because the Lie group techniques offer marked improvements in
accuracy over alternative formulations.

The Lie group solvers need two special functions that need to be specified at
construction time. The \emph{expmap} function approximates the Lie group's
exponential map, while the \emph{dexpinv} function approximates the Lie group's
pullback function.

Only some of the techniques listed in
section~\ref{sec:con_des_er7_integ_techniques}
can solve a second order Lie group ODE. The second order generalized derivative
solver should be used in lieu of the Lie group solver in the case of a technique
that does not provide a Lie group solver.

\subsubsection{Integrator Constructor Factory}

The \erseven Integration model provides an integrator constructor factory
that simplifies the specification of an integration technique. The factory
constructs an integrator constructor that corresponds to one of the supported
techniques. All the user needs to specify is an element of the enumeration
of supported integration techniques.

\subsubsection{Support for Generalized Second Order Integrators}

To illustrate how the functions needed by the generalized derivative and Lie
group integrator are to behave, the \erseven Integration module provides
implementations of the four functions needed by a second order ODE left quaternion
representation of orientation.
It is intentional that this is exactly the functionality needed by JEOD.

\subsection{JEOD Integration Model} \label{sec:con_des_jeod}

The JEOD \ModelDesc builds on the functionality provided by the \erseven
Integration model.

\subsubsection{JEOD Integration Group} \label{sec:con_des_jeod_integ_group}

A JEOD integration group extends the concept of an \erseven integration group.
The \erseven concept is rather basic. It has to be so as to accommodate both
the Trick and JEOD concepts of integration. This extension is limited to
core concepts. It does not cover dynamics, as this is the purview of the
JEOD dynamics models. The \hypermodelref{DYNMANAGER} in turn extends the
JEOD integration group to cover dynamics.

\subsubsection{Generalized Second Order ODE}

As mentioned above, a second order Lie group problem is a special case
of a second order generalized derivative problem. Some integration techniques
only provide a second order generalized derivative integrator. The default
in JEOD 3.0 is to use the second order Lie group integrator if available.
If it's not available, the second order generalized derivative solver can always
be used as a backup. The \ModelDesc provides mechanisms for dealing with a
mismatch between requested and available techniques.

\subsubsection{Restartable State Integrator}

C++ does not provide garbage collection, making properly managing resources
a key problem for C++ programs. One approach is to put the burden entirely
on the programmer: make the programmer responsible for allocating and
releasing all needed resources such as allocated memory. A better approach
is to use techniques such as the non-intuitively named RAII
("Resource Allocation Is Initialization") design pattern to remove this burden
from the programmer. This design pattern incorporates resource allocation and
deallocation into the object design.

Because state integrators are allocated by an integrator constructor, this makes
the state integrators used in JEOD subject to the very problem that the RAII
design pattern solves. The \ModelDesc provides class templates that solve the
allocation problem in a generic sense and provides three instances of these
class templates that specifically address the three kinds of
initial value problems found within JEOD.

\subsubsection{Time}

The \erseven Integration model provides a very abstract concept of time.
The JEOD \ModelDesc extends this abstract concept, but only to the point needed
by the model itself. The \hypermodelref{TIME} further extends this JEOD
integration time to form the JEOD time manager.
 
Users can make the JEOD time manager change rate.
This can wreak havoc with predictor-corrector methods unless these integrators
are reset. The \ModelDesc implements a concept of a time change subscriber
so that interested parties can respond to changes in JEOD time. The
JEOD integration group is a time change subscriber.

\subsubsection{JEOD-Specific Integration Techniques}

All of the integration techniques described in
section~\ref{sec:con_des_er7_integ_techniques} can be used in JEOD-based
simulations. JEOD provides two additional techniques aimed at long arc
integration problems. These two techniques are described below.

\paragraph{Gauss-Jackson}
The Gauss-Jackson integrator provided with JEOD 3.0 is a non-adaptive
(fixed step size) predictor-corrector solver for second order ODEs. Users
have control over the order of the technique and tolerance targets of the
corrector algorithm. This technique performs particularly well compared to
other techniques when applied to nearly circular orbits.

See appendix \ref{app:gauss_jackson} for details.

\paragraph{LSODE}
The Livermore Solver for Ordinary Differential Equations (LSODE) uses a number
of integration techniques to solve intial value problems for first order ODEs.
Some of the solvers used by LSODE are adaptive. Users have control over the
solver algorithm, the order technique, and the tolerance targets.
This technique performs particularly well compared to other techniques when
applied to markedly non-circular trajectories.

See appendix \ref{app:lsode} for details.



\section{Mathematical Formulations}
\section{Mathematical Formulation}
\label{sec:mathematics}
This section summarizes key equations used in the implementation of the \ModelDesc. The outline of this section is organized along the lines of the
key concepts described in section~\ref{sec:key_concepts}.
Only those concepts that involve mathematical details are described in
this section.

\subsection{Attach/Detach}\label{sec:math_attach_detach}
The underlying MassBody class addresses the geometry of attachment and
detachment, with BodyRefFrames constructed for MassPoints as necessary. The DynBody class needs to address state as well, and must do this
in a manner consistent with the laws of physics (if possible).
This is not always possible with attachment as the attach model allows
physically impossible attachments to take place:
The attaching bodies instantaneously snap into place upon attachment.
For example, the instantaneous attachment of a MassBody or MassBody derived 
state (which does not have a dynamic state) cannot conserve momentum.
The rationale for instantaneous snap attachment in general is that modeling 
the detailed contact forces and
torques that take place during the docking of two vehicles is beyond the scope
of many orbital simulations. The simulation instead brings the vehicles into
very close proximity and alignment and then magic happens: The vehicles attach.
Linear momentum can always be conserved, even in non-physical attachments.
Angular momentum is more problematic. The model conserves angular momentum
in the frame that is centered at and moving with the post-docking combined
center of mass. This choice is consistent with the laws of physics when the
attachment does not involve a step change in position and orientation.

Conservation of linear momentum determines the velocity of the combined
vehicle. Equating the linear momentum before and after the attachment yields

\begin{equation}
(m_p + m_c) \framevect I v_+ =
  m_p \framevect I v_{p_-} + m_c \framevect I v_{c-}
\end{equation}

where $m_p$ and $m_c$ are the pre-attachment masses of the parent and child
bodies, $\framevect I v_{p_-}$ and $\framevect I v_{c_-}$ are the
pre-attachment velocities of the parent and child bodies in the parent body's
integration frame, and $\framevect I v_+$ is the post-attachment velocity of
the combined center of mass.
Solving for the post-attachment velocity yields

\begin{equation}
\framevect I v_+ =
  \frac{m_p \framevect I v_{p_-} + m_c \framevect I v_{c-}}{m_p + m_c}
\end{equation}

The pre-attachment angular momentum in the non-rotating frame with origin at
the combined center of of mass $\framevect I r_+$ and moving at velocity $\framevect I v+$ results from translation with respect to this origin and
body rotations:

\begin{equation}
\begin{aligned}
\framevect I L_- =\,
  &
  m_p (\framevect I r_{p_-}- \framevect I r_+) \times
      (\framevect I v_{p_-}- \framevect I v_+) +
  {\tmatT I B_p} (\MxV {\inertia_p} {\framevect B \omega_{p_-}})\;+ \\
  &
  m_c (\framevect I r_{c_-}- \framevect I r_+) \times
      (\framevect I v_{c_-}- \framevect I v_+) +
  {\tmatT I B_c} (\MxV {\inertia_c} {\framevect B \omega_{c_-}})
\end{aligned}
\end{equation}

Transforming to the parent body frame,

\begin{equation}
\begin{aligned}
\framevect {B_p} L_- =\,
  & \MxV {\tmat I B_p}
         {\left(
            m_p (\framevect I r_{p_-}- \framevect I r_+) \times
                (\framevect I v_{p_-}- \framevect I v_+)
         \right)}\;+ \\
  & \MxV {\inertia_p} {\framevect B \omega_{p_-}}\;+ \\
  & \MxV {\tmat I B_p}
         {\left(
            m_c (\framevect I r_{c_-}- \framevect I r_+) \times
                (\framevect I v_{c_-}- \framevect I v_+)
         \right)}\;+ \\
  & \MxV {\tmat I B_p}
         {{\tmatT I B_c} (\MxV {\inertia_c} {\framevect B \omega_{c_-}})}
\end{aligned}
\end{equation}

The post-attachment orientation post-attachment angular momentum in this frame results from rotation only:

\begin{equation}
\framevect {B_p} L_+ =
  \MxV {\inertia_+} {\framevect B \omega_{p_+}}
\end{equation}

Equating the pre- and post-angular momenta to conserve angular momentum
and solving for the post-attachment angular velocity yields

\begin{equation}
\framevect B \omega_{p_+} = \MxV{{\inertia_+}^{-1}}{\framevect {B_p} L_-}
\end{equation}

Detachment is much simpler. The only state that needs to change is that
of the parent composite body. The child body's state does not change with
detachment; it simply flies away from the parent with no instantaneous
change in any state element.
The parent body's state is almost correct. The parent similarly flies away
from the child with no instantaneous changes in any of the primitive
states. The composite state is incorrect and needs to be updated.
Any primitive state can be used as the basis for this update.
The mathematics that underlies state update is described in
section~\ref{sec:math_state_prop}.

\subsection{State Initialization}\label{sec:math_state_initialization}
The model participates in the state initialization process by setting state
elements and propagating the set states throughout the DynBody tree.
The mathematics regarding this state propagation is described in
section~\ref{sec:math_state_prop}.

\subsection{Force and Torque Collection}\label{sec:math_collect}

The total effector force acting on a parent body is calculated per the
superposition principle:
\begin{equation}
 \framevect {P_S} F_{\text{eff,tot}} =
 \sum_{\text{vector\ }i} \framevect {P_S} F_{\text{eff,i}} +
 \sum_{\text{child\ }i} \MxV {\tmatT {P_S}{C_S}}
                             {\framevect {C_S} F_{\text{eff,tot}}}
\end{equation}
where $P_S$ is the parent body's structural frame,
the first sum is over the collected effector forces for the parent body,
and the second sum is over the child bodies attached to the parent body.

The total environmental force is calculated similarly while the total
non-transmitted force omits the sum over the child bodies. The total force on a
root body is the sum of the effector, environmental, and non-transmitted forces
on that body.
\begin{equation}
 \framevect {S} F_{\text{tot}} =
 \framevect {S} F_{\text{eff,tot}} +
 \framevect {S} F_{\text{env,tot}} +
 \framevect {S} F_{\text{non-xmit,tot}}
\end{equation}

The equations of motion need the force in the inertial frame:
\begin{equation}
  \framevect I F_{\text{tot}} =
  \MxV{\tmatT I S}
      {\framevect S F_{\text{tot}}} \label{eqn:inertial_force}
\end{equation}

Torque calculations proceed similarly, but with the additional twist that
non-centerline forces produce a torque. The total effector torque acting on
a parent body is
\begin{equation}
 \framevect {P_S} \tau_{\text{eff,tot}} =
 \sum_{\text{vector\ }i} \framevect {P_S} \tau_{\text{eff,i}} +
 \sum_{\text{child\ }i}
   \MxV {\tmatT {P_S}{C_S}}
        {\framevect {C_S} \tau_{\text{eff,tot}}} +
   \framerelvect {P_S} r {P_{cm}} {C_{cm}} \times
     \left(
       \MxV {\tmatT {P_S}{C_S}}
            {\framevect {C_S} F_{\text{eff,tot}}}
     \right)
\end{equation}

The total environmental torque is calculated similarly while the total
non-transmitted torque omits the sum over the child bodies.
The total torque on a root body is the sum of the effector, environmental, and
non-transmitted torques on that body.
\begin{equation}
 \framevect {S} \tau_{\text{tot}} =
 \framevect {S} \tau_{\text{eff,tot}} +
 \framevect {S} \tau_{\text{env,tot}} +
 \framevect {S} \tau_{\text{non-xmit,tot}}
\end{equation}

The equations of motion need the torque in the body frame:
\begin{equation}
  \framevect B \tau_{\text{tot}} =
  \MxV{\tmat S B}
      {\framevect S \tau_{\text{tot}}} \label{eqn:body_torque}
\end{equation}

\subsection{Equations of Motion}\label{sec:math_eom}

Equation~\eqref{eqn:inertial_force} combined with Newton's second law yields
the acceleration due to non-gravitational forces:
\begin{equation}
  \framevect I a_\text{{non-grav,tot}} = \frac 1 m \framevect I F_{\text{tot}}
\end{equation}

The \GRAVITY computes the apparent gravitational acceleration,
the total gravitational acceleration acting on the body
less the gravitational acceleration of the integration frame.
Combining this with the non-gravitational acceleration yields the total
acceleration of the body with respect to the integration frame.

\begin{equation}
  \framerelvect I a I B =
    \framevect I a_\text{{non-grav,tot}} + \framevect I a_\text{{grav,tot}}
  \label{eqn:atot}
\end{equation}

The rotational equations of motion differ conceptually from the above
translational equations of motion in two regards.
That angular momentum is expressed in the body frame adds the complicating
factor of computing derivatives in a rotating frame.
That is offset by lack of gravitation as a specific concern,
which simplifies the the rotational equations of motion a bit.

The rotational analog of Newton's second law
states that the time derivative of the angular momentum vector as expressed in
a non-rotating frame is simply the external torque:

\begin{equation}
  \framevdot I L = \framevect I \tau_{\text{tot}}
\end{equation}

The angular momentum is readily computed in the body frame as the product
of the inertia tensor and the body-referenced angular velocity:

\begin{equation}
  \framevect B L = \MxV {\inertia_B} {\framevect B \omega}
\end{equation}

The time derivative of the body-referenced angular momentum and the
inertial-referenced angular momentum are related by the inertial torque:

\begin{align}
  \framevdot B L &=
    \MxV {\tmat I B}{\framevdot I L}
    - \framerelvect B \omega I B \times\framevect B L \nonumber \\
  &= \framevect B \tau_{\text{tot}}
    - \framerelvect B \omega I B \times\framevect B L
\end{align}

Assuming the inertia tensor is constant (or nearly so) in the body frame,
\begin{align}
  \framevdot B L &=
  \frac{d}{dt}\left(\MxV {\inertia_B} {\framevect B \omega}\right)
  \nonumber \\
  &\approx \MxV {\inertia_B} {\framevdot B \omega}
\end{align}

Combining the above yields the time derivative of the
body-referenced angular momentum:

\begin{equation}
  \framevdot B \omega =
    {\inertia_B}^{-1}
    \left(
      \framevect B \tau_{\text{tot}} -
      \framerelvect B \omega I B \times\framevect B L
    \right)
  \label{eqn:wdotbody}
\end{equation}

\subsection{State Integration}\label{sec:math_state_integ}
This model uses the \hypermodelref{INTEGRATION} to perform the state integration.
See the documentation for that model regarding the mathematics that
underlies numerical integration.

\subsection{State Propagation}\label{sec:math_state_prop}
The initialized state and the integrated state need to be propagated
throughout a DynBody object, including to other DynBody objects attached
to the object in question. The propagation from one frame to another must
be consistent with the local reference frames embodied in the MassBasicPoints
that are associated with the DynBodyFrames.

The model performs state propagation in terms of pairs of reference frames.
The mathematics of the propagation of the state from frame B to frame C
depends on whether the underlying MassBasicPoint is expressed as B to C
or C to B. Different methods address these as forward and reverse propagations.

Forward propagation involves computing the state of frame C with respect to
frame A (the integration frame) given the state of frame B with respect to
frame A and the local transformation from frame B to frame C. Note that this
local transformation is a part of the MassBasicPoint object and hence does
not contain derivative data. This local transformation is assumed to be
constant with respect to time.

\begin{align}
\tmat A C &=
  \MxV {\tmat B C} {\tmat A B}
  && \text{Orientation} \\
\framerelvect C {\omega} A C &=
  \MxV {\tmat B C} {\framerelvect B {\omega} A B}
  && \text{Angular velocity} \\
\framerelvect A r A C &=
  \framerelvect A r A B + \MxV {\tmatT A B} {\framerelvect B r B C}
  && \text{Position} \\
\framerelvect A v A C &=
  \framerelvect A v A B + \MxV {\tmatT A B}
                               {(\framerelvect B {\omega} A B \times
                                 \framerelvect B r B C)}
  && \text{Velocity}
\end{align}

Reverse propagation involves computing the state of frame C with respect to
frame A (the integration frame) given the state of frame B with respect to
frame A and the local transformation from frame C to frame B. In other words,
the local transformation is in the reverse direction of the desired
propagation.

\begin{align}
\tmat A C &=
  \MxV {\tmatT C B} {\tmat A B}
  && \text{Orientation} \\
\framerelvect C {\omega} A C &=
  \MxV {\tmatT C B} {\framerelvect B {\omega} A B}
  && \text{Angular velocity} \\
\framerelvect A r A C &=
  \framerelvect A r A B - \MxV {\tmatT A C} {\framerelvect C r C B}
  && \text{Position} \\
\framerelvect A v A C &=
  \framerelvect A v A B - \MxV {\tmatT A C}
                               {(\framerelvect C {\omega} A C \times
                                 \framerelvect C r C B)}
  && \text{Velocity}
\end{align}

\subsection{Point Acceleration}\label{sec:math_pt_accel}
The following assumes a rigid body connection between the composite center of
mass and a vehicle point. Different mathematics are needed if this assumption is
not valid.

With this assumption, the rotational acceleration at the vehicle point is equal
to the rotational acceleration at the center of mass. The point however has its
own reference frame, and the rotational acceleration needs to be expressed
in this point frame.

\begin{equation}
\framerelvdot {\text{pt}} {\omega} I {\text{pt}} =
\MxV {\tmat B {\text{pt}}}
     {\framerelvdot B {\omega} I B}
\end{equation}

With the rigid body assumption, the total translation acceleration at the point
is the total translation at the center of mass plus the translational
acceleration due to rotational motion. In the rotating composite body frame,
this acceleration is

\begin{equation}
\Delta \framevect B {a_{\text{rot}}} =
  \framerelvect B {\omega} I B \times
  (\framerelvect B {\omega} I B \times \framerelvect B r B {\text{pt}}) +
  \framerelvect B {\omega} I B \times \framerelvect B r B {\text{pt}}
\end{equation}

Transforming to inertial,
\begin{equation}
\Delta \framevect I {a_{\text{rot}}} =
  \MxV {\tmatT I B} {\Delta \framevect B {a_{\text{rot}}}}
\end{equation}

The total acceleration at the point is
\begin{equation}
\framevect I {a_{\text{pt}}} =
  \framevect I a + \Delta \framevect I {a_{\text{rot}}}
  \label{eqn:tot_accel_pt}
\end{equation}

Calculating the non-gravitational acceleration is a bit more complex. The total
acceleration at the point comprises gravitational and non-gravitational
components, as does the total acceleration at the composite center of mass.
\begin{align}
\framevect I {a_{\text{pt}}} &=
  \framevect I {a_{\text{pt,non-grav}}} +
  \framevect I {a_{\text{pt,grav}}}
  \label{eqn:tot_accel_pt_split} \\
\framevect I {a_B} &=
  \framevect I {a_{B,\text{non-grav}}} +
  \framevect I {a_{B,\text{grav}}}
  \label{eqn:tot_accel_B_split}
\end{align}

The gravitational acceleration at the point is that at the center of mass plus a
delta due to the separation between the center of mass and the point. Assuming a
small separation between the center of mass and the point, the gravity gradient
yields an excellent estimate of this delta gravitational acceleration.

\begin{align}
\framevect I {a_{\text{pt,grav}}} &=
  \framevect I {a_{B,\text{grav}}} +
  \Delta \framevect I {a_{pt,\text{grav}}} \\
\Delta \framevect I {a_{\text{pt,grav}}} &\approx
  \MxV {\nabla \framevect I {a_{B,\text{grav}}}}
       {(\MxV {\tmatT I B} {\framerelvect B r B {\text{pt}}})}
  \label{eqn:delta_grav_accel_pt}
\end{align}

Combining equations~\ref{eqn:tot_accel_pt} to~\ref{eqn:delta_grav_accel_pt}
and solving for the non-gravitational acceleration at the point,

\begin{equation}
\framevect I {a_{\text{pt,non-grav}}} =
  \framevect I {a_{B,\text{non-grav}}} +
  \Delta \framevect I {a_{\text{rot}}} -
  \Delta \framevect I {a_{\text{pt,grav}}}
\end{equation}


\section{Detailed Design}\label{sec:detailed_design}
\subsection{API}
Follow this link for the 
\href{file:refman.pdf}{\em \ModelDesc\ API}~\cite{api:mass}.

\subsection{Class Overview}

The MassBody is the basic class for a Mass Body.  It comtains (this is not an 
exhaustive list):
\begin{itemize}
 \item \textit{core\_properties}, an instance of MassProperties, that 
 describes the properties of the MassBody as a single entity.
 \begin{itemize}
  \item A MassProperties class is a MassBasicPoint which also provides data 
  elements for mass and the inertia tensor.
  \begin{itemize}
   \item A MassBasicPoint is a MassPointState that provides the linkages 
   within the mass tree (to its children, parent, etc.).  A mass tree 
   comprises the collection of MassBasicPoint instances that are physically 
   connected at some point.
   \begin{itemize}
    \item A MassPointState provides data elements for the position of the 
    point, and the orientation of the axes associated with the point, with 
    respect to some parent.
   \end{itemize}
   \item A related item, a MassPoint, is simply a MassBasicPoint with a name, 
   allowing for it to be found easily.
  \end{itemize}
  \end{itemize}
 

 \item \textit{composite\_properties}, another instance of MassProperties, 
 that describes the properties of the subtree of which the body is the head.
 \item \textit{structure\_point}, an instance of MassBasicPoint (see above).
\end{itemize}

Note that since a MassProperties is a MassBasicPoint, it must have a parent 
for evaluation of its position and orientation.  The parent of both instances 
of Mass Properties (\textit{core\_properties} and 
\textit{composite\_properties}) is the MassBasicPoint  
\textit{structure\_point}; the position and orientation are stated relative to 
the structural axes.


\subsection{Attaching Bodies}\label{sec:attaching_bodies}
Attaching two bodies together is typically performed with a Body Action.  A 
rough outline of the algorithm is presented here, instructions for 
implementation are contained in the Body Action document 
(\cite{dynenv:DYNBODY}).

The following rules provide the restrictions on which bodies can be commanded 
attached to which bodies.
\begin{itemize}
 \item A DynBody (extension of MassBody) can never be a child of a basic 
 MassBody.
 \item There can be no circular attachments.
\end{itemize}
IMPORTANT NOTE: Only root bodies may actually attach to another body.  While 
it may be legal to command the attachment of body A to body B (subject to the 
rules above), the attachment in the mass tree will be represented as the root 
of the tree containing body A attaching to body B.

\subsubsection{Attaching using Points}
The easiest implementation of attaching two bodies is to define an attach 
point (a MassPoint) on each body, and make those points coincident, with their 
z-axes aligned, and their x- and y-axes both anti-aligned.  Because each 
MassPoint will have position and orientation defined with respect to its 
respective body's structural axes, it is a striaghtforward undertaking to 
obtain the orientation and position of the child body's structural axes with 
respect to the parent body's structural axes.  Then, the general attachment 
method can be implemented, and the parent will have as children its MassPoint 
used for attachment and the child structure-point; the child will have as a 
child its MassPoint used for attachment.

\subsubsection{Attaching using Offset and Orientation}
This more general form of the attach process requires knowledge of the 
orientation and relative position of the child body's structural-axes with 
respect to those of the parent body, which may be provided directly or from 
the previous method.  

If the child body is not the root of its own tree, its root is first found, 
then the relative position and orientation of that root body's structure-axes 
relative to the structure-axes of the parent body are determined.

With the root state known, the attachment of the root of the child's tree (the 
child-root) to the parent can proceed in three steps:
\begin{enumerate}
\item{\bf Validate:} Is this a valid attachment which for the \ModelDesc, 
means it follows a tree structure and avoids 
invalid circular attachments.
\item{\bf Establish Links:} The child-root MassBody is required to establish 
the links to the parent MassBody that define their 
relationship in the mass tree.
\item{\bf Update Properties:}  The parent MassBody updates its composite 
properties to reflect the addition of the child-root MassBody.  
The composite properties of the child-root remain unchanged.
\end{enumerate}
 
A few examples will help illustrate the concept of attachments and we will 
start by supposing that we have three MassBodys called 
A, B, and C respectively.

In the first example we decide to attach MassBody C to MassBody B thereby 
making C a child of B.  Given that no other attachments exist in
this example this is a valid attachment so we will pass step one(Validate).  
In step two(Establish Links) the child will update the 
links between the two MassBodys including MassPointLinks and MassBodyLinks.  
In step three(Update Properties) the parent will 
update its composite properties (mass, inertia, center of mass) to represent 
the combined state.

Building on the new mass tree created with the attachment of MassBody C to 
MassBody B, we will now attach MassBody B to MassBody A. 
MassBody B will then become a child of MassBody A, and MassBody C will remain 
a child of MassBody B.  In this attachment process, the validity is checked -- 
since A is not already attached to B or C this is approved.  MassBody B now 
performs the Establish Links step configuring all MassPointLinks and 
MassBodyLinks with MassBody A.  Then MassBody A uses the composite properties 
of MassBody B to perform the Update Properties step.  It should be noted that 
the
composite properties of MassBody B already include the properties of
MassBody C so, following the update, the composite properties of A represent 
the entire tree.
Even though the properties of A do contain information for the entire tree, 
MassBody A has no knowledge that
MassBody C exists.  MassBody A only knows about MassBody B.

Consider another situation: return to the situation where only MassBody B and 
MassBody C are attached and C is a child of B.  
Suppose we then want to attach MassBody C to MassBody A.  The \ModelDesc\ 
recognizes that MassBody C is not a root body, so instead attaches MassBody B 
to MassBody A.  The offset and rotation of B with respect to A are set such 
that the offset and rotation of C with respect to A are as requested.  Again,
the composite mass properties of A will include those of C through the 
composite properties of B.

In both situations, the mass tree looks identical.

\subsection{Detach Overview}
In the detach scenario, the links between the body to be detached and its 
parent are severed, and the composite properties of the parent are updated to 
reflect that one of its children has been removed.  The composite properties 
of the child body are unchanged.
\clearpage
\boilerplateinventory
