\setcounter{chapter}{0}

%----------------------------------
\chapter{Introduction}\hyperdef{part}{intro}{}\label{ch:intro}
%----------------------------------

\section{Purpose and Objectives of the \relkinDesc}
%%% Incorporate the intro paragraph that used to begin this Chapter here.
%%% This is location of the true introduction where you explain why this
%%% document exists and what it hopes to accomplish.

The purpose of the \relkinDesc\ is to provide the \JEODid\ with a centralized
management interface for the calculation of relative states of multiple points
of interest associated with a particular Dynamics Body, relative to any
desired reference frame.  These points can be anywhere on, or anywhere in
relation to, the Dynamics Body, and the desired reference frame can be any
coordinate frame. All points of interest can be updated in batch fashion, or
individual points can be updated one at a time.  The \relkinDesc\ does no
calculations on its own, but instead serves as a ``manager'' class that
orchestrates a set of user-defined Relative Derived States to accomplish these
capabilities.


\section{Context within JEOD}
The following document is parent to this document:
\begin{itemize}
\item{\href{file:\JEODHOME/docs/JEOD.pdf}
           {\em JSC Engineering Orbital Dynamics}}
                          \cite{dynenv:JEOD}
\end{itemize}

The \relkinDesc\ forms a component of the \ModelClass\ suite of
models within \JEODid. It is located at
models/\ModelClass/rel\_kin.


\section{Document History}
%%% Status of this and only this document.  Any date should be relevant to when
%%% this document was last updated and mention the reason (release, bug fix, etc.)
%%% Mention previous history aka JEOD 1.4-5 heritage in this section.

\begin{tabular}{||l|l|l|l|} \hline
\DocumentChangeHistory
\end{tabular}
\\ \\ \\ %%% forced extra space because Latex was formatting oddly.
The \relkinDesc\ is a model that is heavily revised for JEOD v2.0.x, though it
is partly based on the Relative Kinematics Computations model released as
part of JEOD v1.5.x.  Thus, this document derives lightly from that
model's documentation, entitled JSC Engineering Orbital Dynamics Relative
Kinematics Computations Module, most recently released with JEOD v1.5.2.


\section{Document Organization}
This document is formatted in accordance with the
NASA Software Engineering Requirements Standard~\cite{NASA:SWE}
and is organized into the following chapters:

\begin{description}

\item[Chapter 1: Introduction] -
This introduction contains four sections: purpose and objective,
context within JEOD, document history, and document organization.

\item[Chapter 2: Product Requirements] -
Describes requirements for the \relkinDesc.

\item[Chapter 3: Product Specification] -
Describes the underlying theory, architecture, and design of the
\relkinDesc\ in detail.  It is organized into
four sections: Conceptual Design, Mathematical Formulations, Detailed
Design, and Version Inventory.

\item[Chapter 4:  User's Guide] -
Describes how to use the \relkinDesc\ in a Trick simulation.  It
is broken into three sections to represent the JEOD
defined user types: Analysts or users of simulations (Analysis),
Integrators or developers of simulations (Integration),
and Model Extenders (Extension).

\item[Chapter 5: Verification and Validation] -
Contains verification and validation procedures and
results for the \relkinDesc.

\end{description}



%----------------------------------
\chapter{Product Requirements}\hyperdef{part}{reqt}{}\label{ch:reqt}
%----------------------------------

This chapter identifies the requirements for the \relkinDesc.

\requirement{Top-level Requirement}
\label{reqt:toplevel}
\begin{description}
\item[Requirement:]\ \newline
  This model shall meet the JEOD project requirements specified in
  the \JEODid\
  \hyperref{file:\JEODHOME/docs/JEOD.pdf}{part1}{reqt}{ top-level
  document}.
\item[Rationale:]\ \newline
  This model shall, at a minimum, meet all external and internal requirements
  applied to the \JEODid\ release.
\item[Verification:]\ \newline
     Inspection
\end{description}


\section{Data Requirements}\label{sec:data_reqts}
This section identifies requirements on the data represented by the \relkinDesc.
These as-built requirements are based on the \relkinDesc\ data definition header
file.

\requirement{Relative Kinematics Data Encapsulation}
\label{reqt:relkin_data_encapsulation}
\begin{description}
  \item[Requirement:]\ \newline
    The \relkinDesc\ shall encapsulate the following data to enable its
    proper functionality:
  \subrequirement{Number of Relative States}
    The number of Relative Derived States being managed by the \relkinDesc.
  \subrequirement{List of Relative States}
    A list of pointers to the Relative Derived States being managed.

  \item[Rationale:]\ \newline
    The primary purpose of the \relkinDesc\ is to serve as a consolidated
    management interface for a collection of Relative Derived States; the
    model needs these pieces of information to properly manage the
    collection. This requirement constrains the design of the module data.

  \item[Verification:]\ \newline
    Inspection
\end{description}


\section{Functional Requirements}\label{sec:func_reqts}
This section identifies requirements on the functional
capabilities provided by the \relkinDesc.
These as-built requirements are based on the \relkinDesc\ source files.

\requirement{Add New Relative State To List}
\label{reqt:func_add_relstate}
\begin{description}
  \item[Requirement:]\ \newline
    The \relkinDesc\ shall perform the following actions when commanded to
    add a new Relative Derived State to its list of managed ones:
  \subrequirement{Check For Duplicate Relative State}
    Search the list of currently managed Relative Derived States to ensure
    the candidate is not already being managed.
  \subrequirement{Add Unique Relative State To List}
    If the candidate is found to be unique, then it is added to the list of
    currently managed Relative Derived States.

  \item[Rationale:]\ \newline
    The ability to add new Relative Derived States to the \relkinDesc\ is
    necessary for it to fulfill its purpose as a centralized management
    interface for Relative Derived States.

  \item[Verification:]\ \newline
    Inspection, Test
\end{description}

\requirement{Remove Relative State From List}
\label{reqt:func_remove_relstate}
\begin{description}
  \item[Requirement:]\ \newline
    The \relkinDesc\ shall perform the following actions when commanded to
    remove a Relative Derived State from its list of managed ones:
  \subrequirement{Check For Relative State}
    Search the list of currently managed Relative Derived States to ensure
    the candidate is being managed.
  \subrequirement{Remove Relative State From List}
    If the candidate is found, then it is removed from the list of
    currently managed Relative Derived States.

  \item[Rationale:]\ \newline
    The ability to remove Relative Derived States from the \relkinDesc\ is
    necessary for it to fulfill its purpose as a centralized management
    interface for Relative Derived States.

  \item[Verification:]\ \newline
    Inspection, Test
\end{description}

\requirement{Find Relative State In List}
\label{reqt:func_find_relstate}
\begin{description}
  \item[Requirement:]\ \newline
    The \relkinDesc\ shall find a specified Relative Derived State in its
    list of managed ones when commanded to do so.

  \item[Rationale:]\ \newline
    This capability is necessary both for Relative Derived State list
    maintenance and state updates, the two primary functions of the model.

  \item[Verification:]\ \newline
    Inspection, Test
\end{description}

\requirement{Update Single Relative State}
\label{reqt:func_update_single}
\begin{description}
  \item[Requirement:]\ \newline
    The \relkinDesc\ shall direct the state update of any single Derived
    Relative State in its list of managed ones when commanded to do so.

  \item[Rationale:]\ \newline
    This is one of the core intended capabilities of the model.

  \item[Verification:]\ \newline
    Inspection, Test
\end{description}

\requirement{Update All Relative States}
\label{reqt:func_update_all}
\begin{description}
  \item[Requirement:]\ \newline
    The \relkinDesc\ shall direct the update of all of the Derived Relative
    States in its list of managed ones when commanded to do so.

  \item[Rationale:]\ \newline
    This is one of the core intended capabilities of the model.

  \item[Verification:]\ \newline
    Inspection, Test
\end{description}

%%% Format for the model Requirements is open.  It should include requirements for this model
%%% only and use requirment tags like the one below.
%\requirement{...}
%\label{reqt:...}
%\begin{description}
%  \item[...]\ \newline
%    The documentation for the model shall include
%
%    \subrequirement{}
%    \label{reqt:...}
%      Software requirements specification.
%
%    ...
%
%  \item[title]\ \newline
%    text
%
%  ...
%
%\end{description}



%----------------------------------
\chapter{Product Specification}\hyperdef{part}{spec}{}\label{ch:spec}
%----------------------------------

This chapter defines the conceptual design, the mathematical formulations, and
the detailed design for the \relkinDesc.  It also contains the version inventory
for this release of the \relkinDesc.

\section{Conceptual Design}

The \relkinDesc\ is designed to be the centralized management interface for
groups of Relative Derived States for the \JEODid. These relative state
instances should all be associated with the same Dynamics Body, but can be
tied to any point of interest on, in, or around that body. The relative states
can also be calculated relative to any desired reference frame. All of the
relative states can be updated in batch fashion, or they can be updated one at
a time, or some combination of the two. Note that the \relkinDesc\ does no
calculations on its own, but instead orchestrates the set of Relative Derived
States it is currently managing to accomplish these capabilities.


\section{Mathematical Formulations}

The \relkinDesc\ performs no calculations on its own; it instead orchestrates
instances of Relative Derived States upon command. Thus, the reader should
refer to the Relative Reference Frame State-representation section of the
\JEODid\ Derived State Model documentation \cite{dynenv:DERIVEDSTATE} for
information about the calculations performed via this model.


\section{Detailed Design}

The functionality of the \relkinDesc\ is fully contained within a single class.
This section describes the class in detail.

\subsubsection{Relative Kinematics Class Design}

The {\em relative\_kinematics.hh} file contains the lone \relkinDesc\ class,
which is named ``RelativeKinematics''. RelativeKinematics contains the
following data members that define the list of relative states currently under
management by the \relkinDesc:

\begin{itemize}
\item{relative\_states:} A Standard Template Library (STL) list of pointers to
RelativeDerivedState objects, which are the links to each of the relative states
currently being managed, and

\item{num\_rel\_states:} The integer number of relative states currently being
managed.
\end{itemize}

In addition to the default constructor and the destructor, the
RelativeKinematics class also contains the following member functions.

\paragraph{add\_relstate}

This member function adds a given unique RelativeDerivedState to the list of
ones currently being managed by the \relkinDesc.

\begin{itemize}
\item{Return:} void - no returned value.
\item{In:} RelativeDerivedState \& relstate - address of the
RelativeDerivedState to be added to the \relkinDesc's list.
\end{itemize}

\paragraph{remove\_relstate}

This member function removes a given RelativeDerivedState from the list of
ones currently being managed by the \relkinDesc.

\begin{itemize}
\item{Return:} void - no returned value.
\item{In:} RelativeDerivedState \& relstate - address of the
RelativeDerivedState to be removed from the \relkinDesc's list.
\end{itemize}

\paragraph{find\_relstate}

This member function finds the RelativeDerivedState corresponding to the name
provided in the \relkinDesc's list.

\begin{itemize}
\item{Return:} RelativeDerivedState * - a pointer to the RelativeDerivedState
found; has value ``NULL'' if none were found.
\item{In:} std::string relstate\_name - name of the
RelativeDerivedState to be found in the \relkinDesc's list.
\end{itemize}

\paragraph{update\_single}

This member function orchestrates the update of the single
RelativeDerivedState provided.

\begin{itemize}
\item{Return:} void - no returned value.
\item{In:} std::string relstate\_name - name of the single
RelativeDerivedState to be updated.
\end{itemize}

\paragraph{update\_all}

This member function orchestrates the update of all of the
RelativeDerivedState currently in the \relkinDesc's list.

\begin{itemize}
\item{Return:} void - no returned value
\item{Void:} function takes no inputs.
\end{itemize}

Further information about the design of this model can be found
in the  \href{file:refman.pdf} {\em Reference Manual}
\cite{rel_kinbib:ReferenceManual}.


\section{Version Inventory}

The following enumerates all files that are contained in this version
of the \relkinDesc.

\begin{verbatim}

./docs:
refman.pdf
rel_kin.pdf

./docs/tex:
makefile
rel_kinAbstract.tex
rel_kin.bib
rel_kinChapters.tex
rel_kin.sty
rel_kin.tex

./include:
relative_kinematics.hh

./src:
relative_kinematics.cc

./verif/SIM_RELKIN_VERIF:
S_define
S_overrides.mk

./verif/SIM_RELKIN_VERIF/Log_data:
log_relkin_verif.d

./verif/SIM_RELKIN_VERIF/SET_test/RUN_relkin_test:
input

./verif/SIM_RELKIN_VERIF/SET_test_val/RUN_relkin_test:
log_relkin_verif.trk

\end{verbatim}



%----------------------------------
\chapter{User Guide}\hyperdef{part}{user}{}\label{ch:user}
%----------------------------------
This chapter discusses how to use the \relkinDesc\ in a Trick simulation. Usage
is treated at three levels of detail: Analysis, Integration, and Extension,
each targeted at one of the three main anticipated categories of \JEODid\ users.

The Analysis section of the user guide is intended primarily for users of
pre-existing simulations. It contains an overview of a typical S\_define sim
object that implements the \relkinDesc, but does not discuss how to edit it;
the Analysis section also describes how to modify \relkinDesc\ variables after
the simulation has compiled, such as via the input file. Since the \relkinDesc\
is primarily a manager model, no discussion of variable logging will be
given, as the variables likely to be logged are members of the contained
relative states and thus are covered in other \JEODid\ model documentation.

The Integration section of the user guide is intended for simulation developers.
It describes the necessary configuration of the \relkinDesc\ within an
S\_define file, and the creation of standard run directories.  The Integration
section assumes a thorough understanding of the preceding Analysis section
of the user guide. Where applicable, the user may be directed to selected
portions of the Product Specification (Chapter \ref{ch:spec}).

The Extension section of the user guide is intended primarily for developers
needing to extend the capability of the \relkinDesc.  Such users should have a
thorough understanding of how the model is used in the preceding Integration
section, and of the model specification (described in Chapter \ref{ch:spec}).

Note that the \relkinDesc\ depends completely on the Relative Derived States
model, and any simulation involving implementation of the \relkinDesc\
will thus require the successful setup of one or more RelativeDerivedState
objects (\cite{dynenv:DERIVEDSTATE}) as well. This model dependency will be
discussed as appropriate in the following sections.


\section{Analysis}

The Analysis and the Integration sections will assume, for the purposes of
illustration, S\_define objects of the following form:

\begin{verbatim}
sim_object {

   environment/time:          TimeManager    time_manager;

} time;


sim_object {

   dynamics/dyn_manager:      DynManager     dyn_manager;

} mngr;


sim_object {

   dynamics/dyn_body:         Simple6DofDynBody dyn_body;

} veh;


sim_object {

   dynamics/rel_kin:          RelativeKinematics         relkin;
   dynamics/derived_state:    RelativeDerivedState       cm_relstate;
   dynamics/derived_state:    RelativeDerivedState       sensor1_relstate;
   dynamics/derived_state:    RelativeDerivedState       sensor2_relstate;


   P_DYN (initialization) dynamics/rel_kin:
   relkin.relkin.add_relstate (
      In RelativeDerivedState & new_relstate  = relkin.cm_relstate);

   P_DYN (initialization) dynamics/rel_kin:
   relkin.relkin.add_relstate (
      In RelativeDerivedState & new_relstate  = relkin.sensor1_relstate);

   P_DYN (initialization) dynamics/rel_kin:
   relkin.relkin.add_relstate (
      In RelativeDerivedState & new_relstate  = relkin.sensor2_relstate);


   (DYNAMICS, environment) dynamics/rel_kin:
   relkin.relkin.update_all ( );

   P_ENV Imngr (derivative) dynamics/rel_kin:
   relkin.relkin.update_single (
      In char                 * relstate_name = relkin.cm_relstate.name);

} relkin;
\end{verbatim}

Note that this code is only representative of sim objects necessary for this
discussion, and does not hold a complete implementation. For full
implementation details on the Time model, please see the \JEODid\ Time
Representations Model documentation \cite{dynenv:TIME}; for full
implementation details on the Dynamics Manager model,
please see the \JEODid\ Dynamics Manager Model documentation
\cite{dynenv:DYNMANAGER}; for full implementation details on the Simple
Six-DOF Dynamics Body model, please see the appropriate section of the \JEODid\
Dynamics Body Model documentation \cite{dynenv:DYNBODY}; and for full
implementation details on the Relative Derived States model, please see the
Relative Reference Frame State-representation section of the
\JEODid\ Derived State Model documentation \cite{dynenv:DERIVEDSTATE}.

The input files for the \relkinDesc\ are trivial. There are simply no data
fields unique to the \relkinDesc\ beyond those in the RelativeDerivedState
instances it contains, other than an integer count of the number of currently
contained RelativeDerivedStates. Since this value is automatically incremented
when RelativeDerivedState instances are added to the \relkinDesc's list, there
are no unique settings required in input files for this model. (Note: please
consult the appropriate section of \cite{dynenv:DERIVEDSTATE} for assistance in
implementing the Relative Derived States model, including its input file
settings.) Instead, all capabilities of this model are exercised by declarations
in the simulation's S\_define rather than the manipulation of settings via input
files or similar.


\section{Integration}

This section describes the process of implementing the \relkinDesc\ in a Trick
simulation, and will use the same example S\_define found in the Analysis
section for illustration. Please again note that this code is only
representative of sim objects necessary for this discussion, and does not
describe a complete implementation. For full implementation details on the Time
model, please see the \JEODid\ Time Representations Model documentation
\cite{dynenv:TIME}. For full implementation details on the Dynamics Manager
model, please see the \JEODid\ Dynamics Manager Model documentation
\cite{dynenv:DYNMANAGER}. For full implementation details on the Simple
Six-DOF Dynamics Body model, please see the appropriate section of the \JEODid\
Dynamics Body Model documentation \cite{dynenv:DYNBODY}. For full
implementation details on the Relative Derived States model, please see the
Relative Reference Frame State-representation section of the
\JEODid\ Derived State Model documentation \cite{dynenv:DERIVEDSTATE}.

To successfully integrate the \relkinDesc\ into a Trick simulation, the
S\_define must contain instantiations of the following objects:
\begin{itemize}
\item A TimeManager object,
\item A DynManager object,
\item A DynBody object or one of its derived classes such as Simple6DofDynBody,
\item One or more RelativeDerivedState objects for the \relkinDesc\
to manage, and
\item A RelativeKinematics object.
\end{itemize}

As noted above, the TimeManager, DynManager, DynBody or Simple6DofDynBody,
and RelativeDerivedState classes must be correctly initialized and set up for
successful simulation operation, though discussion of such setup for those
classes is outside of the scope of this document; documentation specific to each
of the applicable models should be consulted for those details.

The main object for the \relkinDesc\ is the RelativeKinematics class, which is
instantiated in the example code above via the line:

\begin{verbatim}
dynamics/rel_kin:          RelativeKinematics         relkin;
\end{verbatim}

After setting up the instantiation of a RelativeKinematics object, the next
step is to add the RelativeDerivedStates that it is to manage to its list.
(Note that they should be properly initialized first, which is outside the
scope of this document; please consult the model-specific documentation for
assistance in implementing this.)  This is accomplished by invoking the
\relkinDesc\ {\em add\_relstate} member function as an initialization class job,
as shown in the example above and reproduced here:

\begin{verbatim}
P_DYN (initialization) dynamics/rel_kin:
relkin.relkin.add_relstate (
   In RelativeDerivedState & new_relstate  = relkin.cm_relstate);

P_DYN (initialization) dynamics/rel_kin:
relkin.relkin.add_relstate (
   In RelativeDerivedState & new_relstate  = relkin.sensor1_relstate);

P_DYN (initialization) dynamics/rel_kin:
relkin.relkin.add_relstate (
   In RelativeDerivedState & new_relstate  = relkin.sensor2_relstate);
\end{verbatim}

This example shows the addition of three separate relative states to the
\relkinDesc's list, each of which must be added individually via a separate
call to the {\em add\_relstate} function.

The final step required to integrate the \relkinDesc\ into a Trick simulation
is to command the model to update the relative states at the desired time
invervals. This is done by invoking either the \relkinDesc\ {\em update\_single}
or {\em update\_all} member functions as desired:

\begin{verbatim}
   (DYNAMICS, environment) dynamics/rel_kin:
   relkin.relkin.update_all ( );

   P_ENV Imngr (derivative) dynamics/rel_kin:
   relkin.relkin.update_single (
      In char                 * relstate_name = relkin.cm_relstate.name);
\end{verbatim}

In this example code snippet, all of the currently managed relative states are
being updated in batch fashion as an environment class job at the ``DYNAMICS''
rate (the definition of which is outside the scope of this document), while
the relative state ``cm\_relstate'' is additionally being updated by itself
as a derivative class job and thus at the so-called derivative rate. Note that
either function can be called at any frequency desired; the choices here are
shown merely for introduction to the model's capabilities.

No further steps are required for integration of the \relkinDesc\ into a Trick
simulation.  As its purpose is primarily to serve as an interface by which the
user can more easily manage multiple RelativeDerivedState objects all at once,
the \relkinDesc\ itself performs no further tasks beyond the ones just
illustrated.


\section{Extension}

Because the \relkinDesc\ was implemented in a very specific way to perform a
very specific interface management function, it is not intended to be extensible.



%----------------------------------
\chapter{Verification and Validation}\hyperdef{part}{ivv}{}\label{ch:ivv}
%----------------------------------

\section{Verification}
%%% code imported from old template structure
%\inspection{<Name of Inspection>}\label{inspect:<label>}
% <description> to satisfy
% requirement \ref{reqt:<label>}.

\inspection{Top-level Inspection}\label{inspect:TLI}
This document structure, the code, and associated files have been inspected,
and together completely satisfy requirement \ref{reqt:toplevel}.

\inspection{Data Requirements Inspection}\label{inspect:data_reqts}
By inspection, the data structures of the \relkinDesc\ completely satisfy
requirement \ref{reqt:relkin_data_encapsulation}.

\inspection{Functional Requirements Inspection}\label{inspect:func_reqts}
By inspection, the as-written function code of the \relkinDesc\
satisfies requirements \ref{reqt:func_add_relstate},
\ref{reqt:func_find_relstate}, \ref{reqt:func_update_single}, and
\ref{reqt:func_update_all}.


\section{Validation}
For the single test case, a Trick simulation was run with an appropriately
configured input file. This run can be found in its own directory in the
SET\_test sub-directory of SIM\_RELKIN\_VERIF.


\test{Relative Kinematics Model Test}\label{test:relkin_test}
\begin{description}
\item[Purpose:]\ \newline
This test case is designed to examine the ability of the \relkinDesc\ to
add new relative states to its internal list, to find relative states that
are already in its internal list, to update individual relative states when
so commanded, and to update all currently managed relative states when
commanded. In doing so, the test case examines all functional aspects of the
\relkinDesc.
\item[Run directory:]\ \newline RUN\_relkin\_test
\item[Requirements:]\ \newline
By passing this test, this model satisfies requirements
\mbox{\ref{reqt:func_add_relstate}}, \mbox{\ref{reqt:func_find_relstate}},
\mbox{\ref{reqt:func_update_single}}, and \mbox{\ref{reqt:func_update_all}}.

\item[Procedure:]\ \newline
Since the model performs no calculations of its own, but instead merely directs
instances of the RelativeDerivedState class to perform their calculations as
appropriate, simple successful execution of a simulation containing both a
properly implemented instance of the \relkinDesc, and properly implemented
instances of the RelativeDerivedState class, is sufficient to fully demonstrate
the capabilities of the model under examination.

\item[Results:]\ \newline
The simulation completed the run successfully. Since calls to all of the
model's member functions are included either directly or indirectly in the test
case (note that function {\em find\_relstate} is exercised by both the
{\em add\_relstate} and {\em update\_single} even though it is not explicitly
called in the S\_define), requirements
\mbox{\ref{reqt:func_add_relstate}}, \mbox{\ref{reqt:func_find_relstate}},
\mbox{\ref{reqt:func_update_single}}, and \mbox{\ref{reqt:func_update_all}}
are demonstrated to be satisfied by this test.
\end{description}

\newpage
\section{Requirements Traceability}\label{sec:traceability}

\begin{longtable}[c]{||p{3in}|p{3in}|}
\caption{Requirements Traceability} \\[6pt]
\hline
{\bf Requirement} & {\bf Inspection and Testing} \\
\hline \hline
\endhead

\ref{reqt:toplevel} - Top-level Requirement &
  Insp.~\ref{inspect:TLI} \\
  \hline

\ref{reqt:relkin_data_encapsulation} - Relative Kinematics Data Encapsulation &
   Insp.~\ref{inspect:data_reqts} \\
\hline

\ref{reqt:func_add_relstate} - Add New Relative State To List &
   Insp.~\ref{inspect:func_reqts} \\
   &Test~\ref{test:relkin_test} \\
\hline

\ref{reqt:func_find_relstate} - Find Relative State In List &
   Insp.~\ref{inspect:func_reqts} \\
   &Test~\ref{test:relkin_test} \\
\hline

\ref{reqt:func_update_single} - Update Single Relative State &
   Insp.~\ref{inspect:func_reqts} \\
   &Test~\ref{test:relkin_test} \\
\hline

\ref{reqt:func_update_all} - Update All Relative States &
   Insp.~\ref{inspect:func_reqts} \\
   &Test~\ref{test:relkin_test} \\
\hline

\end{longtable}
