%%%%%%%%%%%%%%%%%%%%%%%%%%%%%%%%%%%%%%%%%%%%%%%%%%%%%%%%%%%%%%%%%%%%%%%%
%
% Purpose: Abstract for earthlighting
%
%
%%%%%%%%%%%%%%%%%%%%%%%%%%%%%%%%%%%%%%%%%%%%%%%%%%%%%%%%%%%%%%%%%%%%%%%%%

\begin{abstract}

In the analysis of complex dynamic systems, such as the rendezvous of two
spacecraft orbiting the Earth, visual simulation can be an extremelly useful tool.
To give the imagery a sense of realism, the ambient lighting experienced by the
spacecraft must be estimated in some form.
The \earthlightingDesc\ computes the lighting conditions experienced by a vehicle, for a specific time, traveling in low earth orbit.
The \earthlightingDesc\ calculates these effects using the current ephemeris
information for the Sun, Earth  and
Moon and determines if the lighting body is partially or completely
occluded.  It then calculates a lighting fraction parameter using the
occlusion fraction and the current phase of the body.  It also performs
a very crude approximation of Earth-Albedo effects; however, this
calculation should not be used where accurate knowledge of the Earth-Albedo
effects are desired.

\end{abstract}
