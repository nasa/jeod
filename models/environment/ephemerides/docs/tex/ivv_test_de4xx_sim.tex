\test[DE4xx Comparision]{DE4xx Comparison Simulation}
\label{test:de4xx_sim}
\begin{description}
\item[Purpose:] \ \newline
The purposes of this test are to verify that the \ModelDesc
can be employed with the Trick simulation environment and
correctly computes requested ephemerides for various solar system bodies.
\item[Requirements:] \ \newline
By passing this test, the \ModelDesc satisfies
requirements~\tracerefrange{reqt:ephem_items}{reqt:de4xx_ephem_interface}.
\item[Procedure:]\ \newline
Compare the ephemerides generated by the simulation 
with the results of the JPL online tool HORIZONS for various solar 
system bodies and various times.  The HORIZONS tool can be found at URL 
\href{ http://ssd.jpl.nasa.gov/?horizons}
{\em http://ssd.jpl.nasa.gov/?horizons}.
\item[Results:]\ \newline
Each comparison of ephemerides computed by the simulation with ephemerides
computed  by the HORIZONS tool showed a match in position and velocity vectors to
at  least 15 digits.  The results of a sample of the test cases used for
comparison  are shown in table \ref{tab:comp_results}.
NOTE: For some solar system bodies, 
the HORIZONS online tool does not utilize the DE405 ephemerides.  This seemed 
to be the usual case for bodies farther from the Sun than Mars.  In cases where 
the DE405 was not used by the HORIZONS tool the comparison between the model 
and the tool results showed an acceptable (but not exact) match. The ephemeris 
used by the HORIZONS tool in each case is listed in the header of ephemeris 
generated by the tool.
\end{description}


\begin{table}[htp]
\caption{De4xx Simulation Results}
\label{tab:comp_results}
\vspace{1ex}
Ephemerides computed by the \ModelDesc test simulation that 
were compared with results of the JPL HORIZONS tool.  In all cases, the model 
results matched the JPL results to at least 15 digits. The date in the table 
is the Julian Ephemeris Date (JED), which is the Julian Date expressed in the 
Terrestrial Time (TT) scale.  This is the same as Coordinate Time (CT) used by 
the JPL HORIZONS tool.
\begin{center}
\vspace{1ex}
\begin{tabular}{||c c c l l|}
\hline
{\bf Date} & {\bf Reference} & {\bf Target} &
\multicolumn{1}{c}{\bf Position} & \multicolumn{1}{c|}{\bf Velocity} \\
(JED) & {\bf Body} & {\bf Body} &
\multicolumn{1}{c}{(km)} & \multicolumn{1}{c|}{(km/s)} \\
\hline\hline
\rule{0pt}{2.8ex}
2453147.5 & Earth & Moon &
    -1.144514189946983E+04 & -9.671675601865520E-01 \\
&&& $\pminus$3.601512411881938E+05 & -5.528033762838737E-02 \\
&&& $\pminus$1.874270860935389E+05 & $\pminus$3.693039024383601E-02 \\[10pt]

2453147.5 & Earth & Sun &
    $\pminus$7.298448725554791E+07 & -2.562749354399740E+01 \\
&&& $\pminus$1.217548698456285E+08 & $\pminus$1.327280438377108E+01 \\
&&& $\pminus$5.278605001076507E+07 & $\pminus$5.755037832421791E+00 \\[10pt]

2454000.5 & Earth & Venus &
    -2.484673958478145E+08 & -1.437902149252885E+01 \\
&&& $\pminus$4.003218279094194E+07 & -5.695930452876355E+01 \\
&&& $\pminus$2.418466066464748E+07 & -2.428170990877956E+01 \\[10pt]

2451234.0 & Sun & Earth &
    -1.346561079177961E+08 & -1.285186994778481E+01 \\
&&& $\pminus$5.644644430198597E+07 & -2.496341304044990E+01 \\
&&& $\pminus$2.447321438897178E+07 & -1.082367226693504E+01 \\[10pt]

2455013.5 & Sun & Mars &
    $\pminus$1.987606576116480E+08 & -7.414247633852530E+00 \\
&&& $\pminus$6.823251638747539E+07 & $\pminus$2.248902545222841E+01 \\
&&& $\pminus$2.592782174112418E+07 & $\pminus$1.051537808581921E+01
\rule[-1.4ex]{0pt}{0pt} \\
\hline
\end{tabular}
\end{center}
\end{table}
\clearpage