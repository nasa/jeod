\test[Propagated Planet]{Propagated Planet Simulation}
\label{test:sim_prop_planet}
\begin{description:}
\item[Purpose]
The purposes of this test are to:
\begin{itemize}
\item Verify the ability to propagate a planet via JEOD integration,
\item Verify the ability to switch the source of a planet's state 
from an ephemeris model to a propagated planet, and
\item Serve as a testbed for the
\SIMINTERFACE\ multiple integration group capability.
\end{itemize}
\item[Requirements]
By passing this test, the \ModelDesc satisfies
requirements~\traceref{reqt:ephem_items},
\traceref{reqt:ephem_interface},
\traceref{reqt:ephem_manager},
and~\traceref{reqt:prop_planet}.
\item[Procedure]
Compare the planetary states as generated by the simulation
when run in ephemeris model versus those generate when run in propagated mode.
\item[Results]
The comparisons are favorable but far from perfect.
The propagated states degrades from those provided by the
ephemeris model, with differences between propagated and computed
growing as time progresses.
The errors after 150 years of propagation are about the same for RK4 versus
Gauss-Jackson integration.
This suggests that the source of the error is using the
ephemeris model as a point of departure.
The DE ephemerides are optimized for position, not velocity.
They were not optimized for taking some randomly chosen point
as the basis for propagation.
\end{description:}