%%%%%%%%%%%%%%%%%%%%%%%%%%%%%%%%%%%%%%%%%%%%%%%%%%%%%%%%%%%%%%%%%%%%%%%%%%%%%%%%
%
% Purpose: Ephemerides model executive summary
%
% 
%%%%%%%%%%%%%%%%%%%%%%%%%%%%%%%%%%%%%%%%%%%%%%%%%%%%%%%%%%%%%%%%%%%%%%%%%%%%%%%%


\chapter*{Executive Summary}

The JEOD \ModelDesc provides a generic framework for defining ephemeris models.
The model comprises five sub-models:
\begin{description}
\item[Ephemeris Item]
Ephemeris models represent as functions of time the position and velocity of
planetary bodies and barycenters of multiple planetary bodies.
Some models also represent as functions of time the orientations and angular
velocities of planetary bodies.
The Ephemeris Item sub-model defines the classes needed to represent these
planetary states.
\item[Ephemeris Interface]
Any JEOD-based implementation of an ephemeris model needs to function
properly with the rest of JEOD. One key aspect of this is the JEOD concept of a
reference frame tree.  JEOD uses reference frames to describe the states of
objects in space. Reference frames link to other reference frames to form a
reference frame tree. JEOD relies upon the ephemeris models to
define the base of this reference frame tree.
The Ephemeris Interface sub-model prescribes basic behaviors that any
ephemeris model must follow in the form of the EphemerisInterface class.
\item[Ephemeris Manager]
The RefFrameManager class defined in the {\REFFRAMES} provides basic
capabilities regarding the management of a set of reference frames connected
to form a reference frame tree. The Ephemeris Manager sub-model extends the
RefFrameManager, adding capabilities related to managing planets,
ephemeris items, and ephemeris models.
\item[De4xx Ephemerides]
The Jet Propulsion Laboratory has a long history of modeling the behavior of the
solar system in the form of ephemeris models. The most recent members of these
models are the 400/403/405/\ldots series of the JPL Development Ephemerides, or
the DE4xx models for short. The De4xx Ephemerides sub-model provides the data for
the DE405, DE421 and DE440 models and defines the class
De4xxEphemeris, which extends the EphemerisInterface class.
\item[Propagated Planet]
The propagated planet sub-model
provides the ability to have planetary state
computed via some ephemeris model
or propagated by integrating the equations of motion.
This can be particularly useful for objects such as asteroids
for which an ephemeris model is not readily
available.
\end{description}
