%%%%%%%%%%%%%%%%%%%%%%%%%%%%%%%%%%%%%%%%%%%%%%%%%%%%%%%%%%%%%%%%%%%%%%%%%%%%%%%%
%
% Purpose:  abstract for gravity model
%
% 
%
%%%%%%%%%%%%%%%%%%%%%%%%%%%%%%%%%%%%%%%%%%%%%%%%%%%%%%%%%%%%%%%%%%%%%%%%%%%%%%%%

\begin{abstract}
The \ModelDesc\ provides a framework for representation of the gravity effects
of planetary bodies. The basic model is generic and extensible to allow
implementation of any common type of gravity field representation method. The
basic model also provides the simple point-mass gravitational attraction
calculations common to all Newtonian gravity representations to reduce
redundant effort when extending the model.

An extension of the basic model that uses a recursive, stable, and singularity--
free spherical harmonics algorithm to model gravity fields is also provided
as part of JEOD. This algorithm computes the gravitational acceleration acting
on an object due to the gravity of planetary bodies; it can also include the
effects of solid body tides in those calculations if desired. The model also
outputs the gradient of the gravitational acceleration, which can be used for
calculating gravitational torque or other microgravity effects.

This document describes the implementation of the \ModelDesc\ including
the model requirements, specifications, mathematical theory,
and architecture.  A user guide is provided to assist
with implementing the model in simulations.  Finally, the methods and
results of verification and validation of the model are included.
\end{abstract}
