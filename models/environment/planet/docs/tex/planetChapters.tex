\setcounter{chapter}{0}

%----------------------------------
\chapter{Introduction}\hyperdef{part}{intro}{}\label{ch:intro}
%----------------------------------

\section{Purpose and Objectives of the \planetDesc}
%%% Incorporate the intro paragraph that used to begin this Chapter here.
%%% This is location of the true introduction where you explain why this
%%% document exists and what it hopes to accomplish.

Modeling planets is an integral part of the functionality of \JEODid, as
many scenarios likely to be modeled using the package involve
a vehicle or vehicles in orbit about some planetary body. The purpose of
the \planetDesc\ is to provide a single, consolidated location for the
storage of, and connection to, all the related elements that can be
considered to ``define'' a given planet mathematically.
For an ellipsoidal planetary body, its shape constants are stored directly
in the \planetDesc\, along with a pointer to the gravity model for the
same planet.  For all planetary bodies, ellipsoidal or otherwise,
the \planetDesc\ houses the inertial and planet-fixed reference frames
associated with it--though the \planetDesc\ plays no role in updating
or maintaining these two frames.


\section{Context within JEOD}
The following document is parent to this document:
\begin{itemize}
\item{\href{file:\JEODHOME/docs/JEOD.pdf}
           {\em JSC Engineering Orbital Dynamics}}
                          \cite{dynenv:JEOD}
\end{itemize}

The \planetDesc\ forms a component of the \ModelClass\ suite of
models within \JEODid. It is located at
models/\ModelClass/planet.


\section{Document History}
%%% Status of this and only this document.  Any date should be relevant to when
%%% this document was last updated and mention the reason (release, bug fix, etc.)
%%% Mention previous history aka JEOD 1.4-5 heritage in this section.

\begin{tabular}{||l|l|l|l|} \hline
\DocumentChangeHistory
\end{tabular}
\\ \\ \\ %%% forced extra space because Latex was formatting oddly.
The \planetDesc\ is a model that is new for JEOD v2.0.x, though it is partly
derived from the Miscellaneous Transformations and Planetary Functions model
released as part of JEOD v1.5.x.  Thus, this document derives lightly from that
model's documentation, entitled JSC Engineering Orbital Dynamics Miscellaneous
Transformations and Planetary Functions, most recently released with
JEOD v1.5.2.


\section{Document Organization}
This document is formatted in accordance with the
NASA Software Engineering Requirements Standard~\cite{NASA:SWE}
and is organized into the following chapters:

\begin{description}

\item[Chapter 1: Introduction] -
This introduction contains four sections: purpose and objective,
context within JEOD, document history, and document organization.

\item[Chapter 2: Product Requirements] -
Describes requirements for the \planetDesc.

\item[Chapter 3: Product Specification] -
Describes the underlying theory, architecture, and design of the
\planetDesc\ in detail.  It is organized into
four sections: Conceptual Design, Mathematical Formulations, Detailed
Design, and Version Inventory.

\item[Chapter 4:  User's Guide] -
Describes how to use the \planetDesc\ in a Trick simulation.  It
is broken into three sections to represent the JEOD
defined user types: Analysts or users of simulations (Analysis),
Integrators or developers of simulations (Integration),
and Model Extenders (Extension).

\item[Chapter 5: Verification and Validation] -
Contains verification and validation procedures and
results for the \planetDesc.

\end{description}



%----------------------------------
\chapter{Product Requirements}\hyperdef{part}{reqt}{}\label{ch:reqt}
%----------------------------------

This chapter identifies the requirements for the \planetDesc.

\requirement{Top-level Requirement}
\label{reqt:toplevel}
\begin{description}
\item[Requirement:]\ \newline
  This model shall meet the JEOD project requirements specified in
  the \JEODid\
  \hyperref{file:\JEODHOME/docs/JEOD.pdf}{part1}{reqt}{ top-level
  document}.
\item[Rationale:]\ \newline
  This model shall, at a minimum, meet all external and internal requirements
  applied to the \JEODid\ release.
\item[Verification:]\ \newline
     Inspection
\end{description}

\section{Data Requirements}\label{sec:data_reqts}
This section identifies requirements on the data represented by the \planetDesc.
These as-built requirements are based on the \planetDesc\ data definition header
files.

\requirement{Basic Planetary Body Data Encapsulation}
\label{reqt:base_planet_data_encapsulation}
\begin{description}
  \item[Requirement:]\ \newline
    The base \planetDesc\ object shall encapsulate the following
    data for the planetary body it represents:
  \subrequirement{Planet Name}
    The name of the planet being defined.
  \subrequirement{Reference Frames}
    The planet's inertial and planet-fixed reference frames.

  \item[Rationale:]\ \newline
    Providing this basic capability in the \planetDesc\ parent class allows
    its use with all planetary bodies, including irregularly shaped ones.

  \item[Verification:]\ \newline
    Inspection
\end{description}

\requirement{Ellipsoidal Planet Data Encapsulation}
\label{reqt:ellip_planet_data_encapsulation}
\begin{description}
  \item[Requirement:]\ \newline
    For ellipsoidal planetary bodies, \planetDesc\ shall additionally
    encapsulate the following via a provided derived class:
  \subrequirement{Shape Constants}
    The set of constants that define the shape of the planet, consisting of
    the mean equatorial radius, mean polar radius, ellipsoid eccentricity,
    the square of the ellipsoid eccentricity, flattening coefficient, and
    the inverse of the flattening coefficient.
  \subrequirement{Gravity Model Connection}
    A pointer to the planet's gravity model.

  \item[Rationale:]\ \newline
    The primary purpose of the \planetDesc\ is to serve as
    a container for these types of information considered
    to ``define'' a planet's shape for the purposes of \JEODid.

  \item[Verification:]\ \newline
    Inspection
\end{description}


\section{Functional Requirements}\label{sec:func_reqts}
This section identifies requirements on the functional
capabilities provided by the \planetDesc.
These as-built requirements are based on the \planetDesc\ source files.

\requirement{Basic Model Registration}
\label{reqt:func_base_register_model}
\begin{description}
  \item[Requirement:]\ \newline
    The \planetDesc\ shall perform the following actions
    when commanded to register, for all bodies represented:
  \subrequirement{Register with Ephemerides Manager}
    Add itself to the Ephemerides Manager's list of planets, and its
    inertial and planet-fixed frames to the Ephemerides Manager's
    list of reference frames.
  \subrequirement{Connect Reference Frames}
    Connect its two reference frames to each other and to the correct
    reference frame pointers in the corresponding ephemeris object.

  \item[Rationale:]\ \newline
    All of these connections are needed for correct operation of the \planetDesc.

  \item[Verification:]\ \newline
    Inspection, Test
\end{description}

\requirement{Ellipsoidal Planet Model Registration}
\label{reqt:func_ellip_register_model}
\begin{description}
  \item[Requirement:]\ \newline
    The \planetDesc\ shall perform the following additional actions
    when commanded to register, for ellipsoidal bodies:
  \subrequirement{Connect to Gravity Model}
    Find the gravity model corresponding to the same planet, then
    correctly cross-connect itself and the gravity manager together.

  \item[Rationale:]\ \newline
    This additional connection is needed for correct operation of the
    \planetDesc, for ellipsoidal planetary bodies.

  \item[Verification:]\ \newline
    Inspection, Test
\end{description}

\requirement{Initialize Ellipsoidal Planet Constants}
\label{reqt:func_init_ellip_constants}
\begin{description}
  \item[Requirement:]\ \newline
    When commanded to initialize for an ellipsoidal planet, the \planetDesc\
    shall initialize the values of the set of constants that define the
    planetary shape, consisting of: the mean equatorial radius, mean polar
    radius, ellipsoid eccentricity, the square of the ellipsoid eccentricity,
    flattening coefficient, and the inverse of the flattening coefficient.
  \item[Rationale:]\ \newline
    These values must be set for proper operation of any simulation
    that includes the ellipsoidal planet derived class of the \planetDesc.

  \item[Verification:]\ \newline
    Inspection, Test
\end{description}

%%% Format for the model Requirements is open.  It should include requirements for this model
%%% only and use requirment tags like the one below.
%\requirement{...}
%\label{reqt:...}
%\begin{description}
%  \item[...]\ \newline
%    The documentation for the model shall include
%
%    \subrequirement{}
%    \label{reqt:...}
%      Software requirements specification.
%
%    ...
%
%  \item[title]\ \newline
%    text
%
%  ...
%
%\end{description}



%----------------------------------
\chapter{Product Specification}\hyperdef{part}{spec}{}\label{ch:spec}
%----------------------------------

This chapter defines the conceptual design, the mathematical formulations, and
the detailed design for the \planetDesc.  It also contains the version inventory
for this release of the \planetDesc.

\section{Conceptual Design}

The \planetDesc\ is designed to be a container class which serves as the
single consolidated \JEODid\ location for the storage of, or connection
to, all the related elements that mathematically ``define'' a planetary body.
It houses the inertial and planet-fixed reference frames for the planet, though
the \planetDesc\ plays no role in updating them. If the model is representing
an ellipsoidal body, then based on user input, it also calculates and stores in
itself the planetary shape constants for the planet.  Also for ellipsoidal
bodies, the model connects itself to the GravitySource and ephemeris
structures associated with that planet.


\section{Mathematical Formulations}

While the \planetDesc\ contains other information (a pointer to the ephemeris
structure associated with the same planet, as well as the inertial and
planet-fixed reference frames for it), it is only responsible for calculating
something when it is being used to represent an ellipsoidal planetary body. When
this is the case, it calculates the ellipsoidal shape constants found within it.
The techniques used to solve for these values are discussed in this section.

From \cite{ValladoSecond}, the equations used to solve for the other
planetary shape constants based on the ones provided by the user are as follows:

\begin{eqnarray} \label{shape_const_convert}
f &=& \frac{r_{eq} - r_{pol}}{r_{eq}} \\
r_{pol} &=& r_{eq} \times \sqrt{1 - e^2}
\end{eqnarray}

where
\begin{itemize}
\item $f$ is the planet's flattening coefficient,
\item $r_{eq}$ is the planet's mean equatorial radius,
\item $r_{pol}$ is the planet's mean polar radius, and
\item $e$ is the planet's ellipsoid eccentricity.
\end{itemize}

Appropriate permutations of these equations are used depending on which
constants the user provides.  If none are provided, then all constants are
automatically set to zero by the \planetDesc.  More details about how to
specify appropriate combinations of the planetary constants
are given in the User Guide (Chapter \ref{ch:user}).


\section{Detailed Design}

The functionality of the \planetDesc\ is contained within two classes, one
being derived from the other. This section describes both classes in detail.

\subsubsection{BasePlanet Class Design}

The {\em base\_planet.hh} file contains the basic \planetDesc\ class, named
``BasePlanet''.  BasePlanet contains the following parameters common to all
planetary bodies:

\begin{itemize}
\item{name:} The name of the planetary object being defined,

\item{inertial:} The planet-centered J2000 pseudo-inertial reference frame, of
object type EphemerisRefFrame, associated with this object,

\item{alt\_inertial:} A secondary pseudo-inertial reference frame, of
object type EphemerisRefFrame, associated with this object,

\item{pfix:} The planet-centered, planet-fixed Cartesian reference frame, of
object type EphemerisRefFrame, associated with this planet, and

\item{alt\_pfix:} A secondary planet-centered, planet-fixed Cartesian
reference frame, of object type EphemerisRefFrame, associated with this planet.
\end{itemize}

In addition to the default constructor and the destructor,
the BasePlanet class also contains the following member functions.

\paragraph{register\_planet}

This member function registers the planet object and its reference frames
with the Ephemerides Manager.

\begin{itemize}
\item{Return:} void - no returned value.
\item{Inout:} EphemeridesManager \& ephem\_manager - the Ephemerides manager
object for the current simulation.
\end{itemize}

\paragraph{set\_name}

This member function sets the name of the planetary body being represented.

\begin{itemize}
\item{Return:} void - no returned value.
\item{In:} const std::string \& new\_name - new name to be applied
\end{itemize}

\paragraph{set\_alt\_inertial}

This member function sets the fixed transformation from J2000 to alt\_inertial.

\begin{itemize}
\item{Return:} void - no returned value.
\item{Inout:} const double trans[3][3] - the transformation matrix from J2000
to alt\_inertial
\end{itemize}

\paragraph{set\_alt\_inertial}

This member function sets the fixed transformation from J2000 to alt\_inertial
by using the celestial and ecliptic poles.

\begin{itemize}
\item{Return:} void - no returned value.
\item{Inout:} const double cp[3] - the celestial pole
\item{Inout:} const double ep[3] - the ecliptic pole
\end{itemize}

\paragraph{set\_alt\_pfix}

This member function sets the fixed transformation from pfix to alt\_pfix.

\begin{itemize}
\item{Return:} void - no returned value.
\item{Inout:} const double trans[3][3] - the transformation matrix from pfix
to alt\_pfix
\end{itemize}

\paragraph{calculate\_alt\_pfix}

This member function calculates the current transformation from J2000 to
alt\_pfix using the fixed transformation between pfix and alt\_pfix.

\begin{itemize}
\item{Return:} void - no returned value.
\item{Inout:} void - no input value.
\end{itemize}

\subsubsection{Planet Class Design}

The {\em planet.hh} file contains the \planetDesc\ class used to model
ellipsoidal planetary bodies, which is named ``Planet''.  Planet contains the
following parameters that mathematically define the shape of an ellipsoidal
planetary body:

\begin{itemize}
\item{grav\_body:} A pointer to a GravitySource object representing the same
planet being defined, which is linked to the proper GravBody when this class's
register\_model method is correctly invoked,

\item{r\_eq:} The mean equatorial radius for this planet, in meters,

\item{r\_pol:} The mean polar radius for this planet, in meters,

\item{e\_ellipsoid:} The ellipsoid eccentricity (NOT orbit eccentricity) of
the planet being defined,

\item{e\_ellip\_sq:} The square of the planet's ellipsoid eccentricity,

\item{flat\_coeff:} The planet's ellipsoid flattening coefficient, termed the
reciprocal flattening parameter by Vallado, and

\item{flat\_inv:} The inverse of the ellipsoid flattening coefficient.
\end{itemize}

In addition to the default constructor, destructor, and the set\_name function
it inherits from BasePlanet, the Planet class also contains the following
member functions.

\paragraph{register\_model}

This member function invokes the register\_planet function
inherited from the BasePlanet class. In addition to the actions which that
function performs, this function also establishes the connections between the
Planet class and the gravity manager.

\begin{itemize}
\item{Return:} void - no returned value.
\item{Inout:} GravitySource \& grav\_body\_in - GravitySource object corresponding
to the same planetary body.
\item{Inout:} DynManager \& dyn\_manager - the Dynamics manager object for
the current simulation. Note that DynManager is a class that inherits from
EphemeridesManager.
\end{itemize}

\paragraph{initialize}

This member function initializes the internal shape constant member variables
for the ellipsoidal planet being defined.

\begin{itemize}
\item{return:} void - no returned value
\item{void:} function takes no inputs.
\end{itemize}

Further information about the design of this model can be found
in the  \href{file:refman.pdf} {\em Reference Manual}
\cite{planetbib:ReferenceManual}.


\input{planetSpec_Inventory.tex}




%----------------------------------
\chapter{User Guide}\hyperdef{part}{user}{}\label{ch:user}
%----------------------------------
This chapter discusses how to use the \planetDesc\ in a Trick 7
simulation. Usage is treated at three levels of detail: Analysis,
Integration, and Extension, each targeted at one of the three main anticipated
categories of \JEODid\ users.

The Analysis section of the user guide is intended primarily for users of
pre-existing simulations. It contains an overview of a typical S\_define sim
object that implements the Planet class of the \planetDesc, but does not
discuss how to edit it; the Analysis section also describes how to modify
\planetDesc\ variables after the simulation has compiled, such as via the input
file. Since the \planetDesc\ is primarily a container model, no discussion of
variable logging will be given, as the variables likely to be logged are
members of contained objects and thus are covered in other \JEODid\
model documentation.

The Integration section of the user guide is intended for simulation developers.
It describes the necessary configuration of the \planetDesc\ within an
S\_define file, and the creation of standard run directories.  The Integration
section assumes a thorough understanding of the preceding Analysis section
of the user guide. Where applicable, the user may be directed to selected
portions of the Product Specification (Chapter \ref{ch:spec}).

The Extension section of the user guide is intended primarily for developers
needing to extend the capability of the \planetDesc.  Such users should have a
thorough understanding of how the model is used in the preceding Integration
section, and of the model specification (described in Chapter \ref{ch:spec}).

Note that the \planetDesc\ depends heavily on the Ephemerides Manager or
Dynamics Manager model, and the Gravity model; thus any simulation involving
implementation of the \planetDesc\ will also require the successful setup of
either an EphemeridesManager object or a DynamicsManager object (depending
on whether BasePlanet or Planet is being used in the simulation), as well as
some type of a GravitySource object (see \cite{dynenv:GRAVITY}) if using Planet.
 Note that the class DynManager inherits from EphemeridesManager; see
\cite{dynenv:DYNMANAGER} and \cite{dynenv:EPHEMERIDES} respectively for further
information on each of these. See \cite{dynenv:GRAVITY} for further information
on the Gravity model. These model dependencies will be discussed as appropriate
in the following sections.

\section{Analysis}

The Analysis and the Integration sections will assume, for the purposes of
illustration, S\_define objects of the following form:

\begin{verbatim}
sim_object {

   environment/time: TimeManager time_manager;

} time;


sim_object {

   dynamics/dyn_manager:     DynManager             dyn_manager;

} mngr;


sim_object {

   environment/gravity:       SphericalHarmonicsGravitySource   gravity_source
      (environment/gravity/data/earth_GGM02C.d);


   environment/planet:        Planet        planet
      (environment/planet/data/earth.d);


   P_ENV (initialization) environment/planet:
   earth.planet.register_model (
      Inout GravitySource &      grav_source = earth.gravity_source,
      Inout DynManager    &      dyn_manager = mngr.dyn_manager);

   P_BODY (initialization) environment/planet:
   earth.planet.initialize ( );

} earth;
\end{verbatim}

Note that this code is only representative of sim objects necessary for the
discussion of how to use the \planetDesc, and does not hold a
complete implementation. For full implementation details on the Time model,
please see the \JEODid\ Time Representations Model documentation
\cite{dynenv:TIME}; for full implementation details on the Dynamics Manager
model, see the \JEODid\ Dynamics Manager Model documentation
\cite{dynenv:DYNMANAGER}; and for full implementation details on the Gravity
model, please see the \JEODid\ Gravity Model documentation
\cite{dynenv:GRAVITY}.

The input files for using the \planetDesc\ are straight-forward. Basically, the user
is only required to provide a valid name (i.e., matching exactly to the one used
elsewhere in the simulation for the same planet) for the planet being modeled.
Further inputs depend on whether the BasePlanet class or the Planet class is
being used. For BasePlanet, nothing further is required. For Planet, the desired
modeling fidelity affects what the user should provide. For example, setting or
leaving all ellipsoidal planetary shape fields as zeros results in
the model assuming a perfectly spherical planet with a radius and gravitational
constant set equal to the values used in the Gravity Model. If the user desires
a non-spherical planet, then at least one of the following should be specified:
\begin{itemize}
\item Flattening coefficient
\item Inverse flattening coefficient
\item Ellipsoid eccentricity
\item The square of the ellipsoid eccentricity
\item The mean polar radius
\end{itemize}
The model will calculate the remaining values from the first of these
parameters it encounters at runtime based on the listed order.

An example of a non-spherical ellipsoidal planet initialization is the following
set of input file commands, which are based on the assumed S\_define above:

\begin{verbatim}
earth.planet.name = "Earth";
earth.planet.r_eq {km} = 6378.137;
earth.e_ellipsoid = 0.081819221;
\end{verbatim}

Note that this set of example commands details initialization of the
\planetDesc\ as a representation of a non-spherical Earth using the ellipsoid
eccentricity parameter to define the describing set of shape constants.


\section{Integration}

This section describes the process of implementing the \planetDesc\ in a Trick
simulation, and will use the same example S\_define found in the Analysis
section for illustration. Please again note that this code is only
representative of sim objects necessary for this discussion, and does not
describe a complete implementation. For full implementation details on the Time
model, please see the \JEODid\ Time Representations Model documentation
\cite{dynenv:TIME}. For full implementation details on the Dynamics Manager
model, please see the \JEODid\ Dynamics Manager Model documentation
\cite{dynenv:DYNMANAGER}. For full implementation details on the Gravity model,
please see the \JEODid\ Gravity Model documentation \cite{dynenv:GRAVITY}.

To successfully integrate the \planetDesc\ into a Trick simulation, the
S\_define must contain instantiations of the following objects.

If using BasePlanet:
\begin{itemize}
\item A TimeManager object,
\item An EphemeridesManager object, with which the \planetDesc\ must register
itself for proper operation, and
\item A BasePlanet object.
\end{itemize}

If using Planet:
\begin{itemize}
\item A TimeManager object,
\item A DynManager, with which the \planetDesc\ must register itself
for proper operation,
\item A GravitySource object modeling the gravity of the same planet, to which
the \planetDesc\ will link itself at registration with the DynManager, and
\item A Planet object.
\end{itemize}

As noted above, all of the non-\planetDesc\ classes must be correctly
initialized and set up for successful simulation operation, though discussion
of such setup is outside of the scope of this document; documentation specific
to each of the applicable models should be consulted for those details.

To use the \planetDesc\ for modeling ellipsoidal planetary bodies, the main
simulation object is the Planet class, which is instantiated
in the example code above via the lines:

\begin{verbatim}
environment/planet:        Planet        planet
   (environment/planet/data/earth.d);
\end{verbatim}

Note that this code includes a call to the ``earth.d'' default data file as a
way of populating some of the model parameters with initial values. In addition
to this file, there are also default data files shipped with the \JEODid\
\planetDesc\ containing reference parameters for the sun, moon, mars, and
Jupiter for the integrator's convenience, should the need for them arise.

After setting up the instantiation of a Planet object, the next step is to
register the model with the Dynamics Manager for the simulation. This is
done by invoking the \planetDesc\ {\em register\_model} member function as an
initialization class job, as shown in the example above and reproduced here:

\begin{verbatim}
P_ENV (initialization) environment/planet:
earth.planet.register_model (
   Inout GravitySource &      grav_source  = earth.gravity_source,
   Inout DynManager    &      dyn_manager  = mngr.dyn_manager);
\end{verbatim}

Note that this call requires a GravitySource and a DynManager as inputs,
illustrating the previously mentioned dependencies of the \planetDesc\ on other
models. This registration step is the one in which the \planetDesc\ registers
itself with the Dynamics Manager, connects itself with the GravitySource object
associated with the same planet (they must be named {\em identically} or the
linkage will fail and the simulation will terminate), and registers its
internal planet-fixed and inertial reference frames with the DynManager.

The final step required to integrate the \planetDesc\ into a Trick simulation
for modeling ellipsoidal planetary bodies is to initialize the
values of its planet shape constants using the
\planetDesc\ {\em initialize} member function:

\begin{verbatim}
P_BODY (initialization) environment/planet:
earth.planet.initialize ( );
\end{verbatim}

This too is intended to be used as an initialization class job. It reads user
inputs (via default data, input file, modified data, or some combination of the
three) and calculates the shape constants for the modeled planet accordingly.
Further information on providing the \planetDesc\ with inputs can be found in
the Analysis section of this chapter.

No further steps are required for integration of the Planet class of the
\planetDesc\ into a Trick simulation.  As the model's purpose is primarily to
serve as a container object with which other models can interface, the
\planetDesc\ itself performs no further calculations beyond the startup ones
just described.


\section{Extension}

The \planetDesc\ itself contains an example of how to extend the model, since
the Planet class that is part of it derives from and extends the fundamental
BasePlanet class. Thus the extender should use the Planet class as a guide for
creating any new types of planetary body containers that should be needed.



%----------------------------------
\chapter{Verification and Validation}\hyperdef{part}{ivv}{}\label{ch:ivv}
%----------------------------------

\section{Verification}
%%% code imported from old template structure
%\inspection{<Name of Inspection>}\label{inspect:<label>}
% <description> to satisfy
% requirement \ref{reqt:<label>}.

\inspection{Top-level Inspection}\label{inspect:TLI}
This document structure, the code, and associated files have been inspected,
and together completely satisfy requirement \ref{reqt:toplevel}.

\inspection{Data Requirements Inspection}\label{inspect:data_reqts}
By inspection, the data structures of the \planetDesc\ completely satisfy
requirements \ref{reqt:base_planet_data_encapsulation} and
\ref{reqt:ellip_planet_data_encapsulation}.

\inspection{Functional Requirements Inspection}\label{inspect:func_reqts}
By inspection, the as-written function code of the \planetDesc\
satisfies requirements \ref{reqt:func_base_register_model},
\ref{reqt:func_ellip_register_model}, and \ref{reqt:func_init_ellip_constants}.


\section{Validation}
For each test case, a Trick simulation was run with an input file tailored
to that case. Each input file has its own associated run directory, named
appropriately for its test case. These run directories are contained in the
SET\_test sub-directory of SIM\_PLANET\_VERIF.


\test{Initialize From Ellipsoid Eccentricity}\label{test:init_eellip}
\begin{description}
\item[Purpose:]\ \newline
This test case is designed to examine the ability of the \planetDesc\ to
register itself with the Ephemerides Manager, and to test its ability to
initialize the ellipsoidal planet shape constants given input in the form of
ellipsoid eccentricity.
\item[Run directory:]\ \newline RUN\_initplanet\_eellip
\item[Requirements:]\ \newline
By passing this test, this model satisfies requirements
\mbox{\ref{reqt:func_base_register_model}} and
\mbox{\ref{reqt:func_ellip_register_model}}, and partially satisfies
requirement \mbox{\ref{reqt:func_init_ellip_constants}}.

\item[Procedure:]\ \newline
Simple successful execution of a simulation containing a properly implemented
instance of the \planetDesc\ is sufficient to fully demonstrate the
registration capability of the model. Implementing the simulation with an
ellipsoid-specific instance will prove both the base and derived capabilities
for the model, since the derived class inherits, and uses, all the capabilities
of the base class.

To test the ability of the \planetDesc\ to initialize the full \JEODid\ set of
planet shape constants given ellipsoid eccentricity, values for Earth's
ellipsoid eccentricity and mean equatorial radius were obtained from
\cite{ValladoSecond} and used to initialize the model.  The expected values of
the other parameters were then analytically determined and compared against the
resulting \planetDesc\ output to determine its calculation accuracy.

\item[Results:]\ \newline
Table \ref{eellip_init_table} shows the values used for each input parameter,
and both the simulation and analytical results for this test. Note that since
the simulation completed the run successfully, requirements
\mbox{\ref{reqt:func_base_register_model}} and
\mbox{\ref{reqt:func_ellip_register_model}} are demonstrated to be satisfied by
this test.

\begin{table}[ht]
\begin{center}
\begin{tabular}{|c|c|c|c|c|}\hline
 Input & Value & Output & Model Result & Analytical Result \\ \hline
 $e$ & 0.081819221 & $e^2$ & 0.00669438 & 0.00669438 \\ \hline
 $r_{eq}$ & 6378.137 $km$ & $f$ & 0.00335281 & 0.00335281 \\ \hline
   &   & $f^{-1}$ & 298.257 & 298.257 \\ \hline
   &   & $r_{eq}$ & 6378.137 $km$ & 6378.137 $km$ \\ \hline
   &   & $r_{pol}$ & 6356.752 $km$ & 6356.752 $km$ \\ \hline
\end{tabular}
\caption{Results for Ellipsoid Eccentricity Initialization}
\label{eellip_init_table}
\end{center}
\end{table}

As this table shows, the model achieved the same results for the shape
parameters as were obtained analytically for the listed inputs. (Note that the
analytical results were truncated to the same number of significant digits as
returned by the simulation to yield a consistent basis for comparison.) Thus,
the model passed this test, partially satisfying requirement
\mbox{\ref{reqt:func_init_ellip_constants}}.
\end{description}


\test{Initialize From Squared Ellipsoid Eccentricity}\label{test:init_eellipsq}
\begin{description}
\item[Purpose:]\ \newline
This test case is designed to examine the ability of the \planetDesc\ to
initialize the planet shape constants given input in the form of squared
ellipsoid eccentricity.
\item[Run directory:]\ \newline RUN\_initplanet\_eellipsq
\item[Requirements:]\ \newline
By passing this test, this model partially satisfies requirement
\mbox{\ref{reqt:func_init_ellip_constants}}.

\item[Procedure:]\ \newline
To test the ability of the \planetDesc\ to initialize the full \JEODid\ set of
planet shape constants given squared ellipsoid eccentricity, values for Earth's
squared ellipsoid eccentricity and mean equatorial radius were developed from
\cite{ValladoSecond} and used to initialize the model.  The expected values of
the other parameters were then analytically determined and compared against the
resulting \planetDesc\ output to determine its calculation accuracy.

\item[Results:]\ \newline
Table \ref{eellipsq_init_table} shows the values used for each input parameter,
and both the simulation and analytical results for this test.

\begin{table}[ht]
\begin{center}
\begin{tabular}{|c|c|c|c|c|}\hline
 Input & Value & Output & Model Result & Analytical Result \\ \hline
 $e^2$ & 0.006694384925 & $e$ & 0.0818192 & 0.0818192 \\ \hline
 $r_{eq}$ & 6378.137 $km$ & $f$ & 0.00335281 & 0.00335281 \\ \hline
   &   & $f^{-1}$ & 298.257 & 298.257 \\ \hline
   &   & $r_{eq}$ & 6378.137 $km$ & 6378.137 $km$ \\ \hline
   &   & $r_{pol}$ & 6356.752 $km$ & 6356.752 $km$ \\ \hline
\end{tabular}
\caption{Results for Squared Ellipsoid Eccentricity Initialization}
\label{eellipsq_init_table}
\end{center}
\end{table}

As this table shows, the model achieved the same results for the shape
parameters as were obtained analytically for the listed inputs. (Note that the
analytical results were truncated to the same number of significant digits as
returned by the simulation to yield a consistent basis for comparison.) Thus,
the model passed this test, partially satisfying requirement
\mbox{\ref{reqt:func_init_ellip_constants}}.
\end{description}


\test{Initialize From Flattening Coefficient}\label{test:init_flatcoeff}
\begin{description}
\item[Purpose:]\ \newline
This test case is designed to examine the ability of the \planetDesc\ to
initialize the planet shape constants given input in the form of flattening
coefficient.
\item[Run directory:]\ \newline RUN\_initplanet\_flatcoeff
\item[Requirements:]\ \newline
By passing this test, this model partially satisfies requirement
\mbox{\ref{reqt:func_init_ellip_constants}}.

\item[Procedure:]\ \newline
To test the ability of the \planetDesc\ to initialize the full \JEODid\ set of
planet shape constants given flattening coefficient, values for Earth's
flattening coefficient and mean equatorial radius were obtained from
\cite{ValladoSecond} and used to initialize the model.  The expected values of
the other parameters were then analytically determined and compared against the
resulting \planetDesc\ output to determine its calculation accuracy.

\item[Results:]\ \newline
Table \ref{flatcoeff_init_table} shows the values used for each input parameter,
and both the simulation and analytical results for this test.

\begin{table}[ht]
\begin{center}
\begin{tabular}{|c|c|c|c|c|}\hline
 Input & Value & Output & Model Result & Analytical Result \\ \hline
 $f$ & 0.003352813 & $e$ & 0.0818192 & 0.0818192 \\ \hline
 $r_{eq}$ & 6378.137 $km$ & $e^2$ & 0.00669438 & 0.00669438 \\ \hline
   &   & $f^{-1}$ & 298.257 & 298.257 \\ \hline
   &   & $r_{eq}$ & 6378.137 $km$ & 6378.137 $km$ \\ \hline
   &   & $r_{pol}$ & 6356.752 $km$ & 6356.752 $km$ \\ \hline
\end{tabular}
\caption{Results for Flattening Coefficient Initialization}
\label{flatcoeff_init_table}
\end{center}
\end{table}

As this table shows, the model achieved the same results for the shape
parameters as were obtained analytically for the listed inputs. (Note that the
analytical results were truncated to the same number of significant digits as
returned by the simulation to yield a consistent basis for comparison.) Thus,
the model passed this test, partially satisfying requirement
\mbox{\ref{reqt:func_init_ellip_constants}}.
\end{description}


\test{Initialize From Inverse Flattening Coefficient}\label{test:init_invflat}
\begin{description}
\item[Purpose:]\ \newline
This test case is designed to examine the ability of the \planetDesc\ to
initialize the planet shape constants given input in the form of inverse
flattening coefficient.
\item[Run directory:]\ \newline RUN\_initplanet\_invflat
\item[Requirements:]\ \newline
By passing this test, this model partially satisfies requirement
\mbox{\ref{reqt:func_init_ellip_constants}}.

\item[Procedure:]\ \newline
To test the ability of the \planetDesc\ to initialize the full \JEODid\ set of
planet shape constants given inverse flattening coefficient, values for Earth's
inverse flattening coefficient and mean equatorial radius were developed from
\cite{ValladoSecond} and used to initialize the model.  The expected values of
the other parameters were then analytically determined and compared against the
resulting \planetDesc\ output to determine its calculation accuracy.

\item[Results:]\ \newline
Table \ref{invflat_init_table} shows the values used for each input parameter,
and both the simulation and analytical results for this test.

\begin{table}[ht]
\begin{center}
\begin{tabular}{|c|c|c|c|c|}\hline
 Input & Value & Output & Model Result & Analytical Result \\ \hline
 $f^{-1}$ & 298.2570 & $e$ & 0.0818192 & 0.0818192 \\ \hline
 $r_{eq}$ & 6378.137 $km$ & $e^2$ & 0.00669438 & 0.00669438 \\ \hline
   &   & $f$ & 0.00335281 & 0.00335281 \\ \hline
   &   & $r_{eq}$ & 6378.137 $km$ & 6378.137 $km$ \\ \hline
   &   & $r_{pol}$ & 6356.752 $km$ & 6356.752 $km$ \\ \hline
\end{tabular}
\caption{Results for Inverse Flattening Coefficient Initialization}
\label{invflat_init_table}
\end{center}
\end{table}

As this table shows, the model achieved the same results for the shape
parameters as were obtained analytically for the listed inputs. (Note that the
analytical results were truncated to the same number of significant digits as
returned by the simulation to yield a consistent basis for comparison.) Thus,
the model passed this test, partially satisfying requirement
\mbox{\ref{reqt:func_init_ellip_constants}}.
\end{description}


\test{Initialize From Mean Polar Radius}\label{test:init_rpol}
\begin{description}
\item[Purpose:]\ \newline
This test case is designed to examine the ability of the \planetDesc\ to
initialize the planet shape constants given input in the form of mean polar
radius.
\item[Run directory:]\ \newline RUN\_initplanet\_rpol
\item[Requirements:]\ \newline
By passing this test, this model partially satisfies requirement
\mbox{\ref{reqt:func_init_ellip_constants}}.

\item[Procedure:]\ \newline
To test the ability of the \planetDesc\ to initialize the full \JEODid\ set of
planet shape constants given mean polar radius, values for Earth's
mean polar radius and mean equatorial radius were developed from
\cite{ValladoSecond} and used to initialize the model.  The expected values of
the other parameters were then analytically determined and compared against the
resulting \planetDesc\ output to determine its calculation accuracy.

\item[Results:]\ \newline
Table \ref{rpol_init_table} shows the values used for each input parameter,
and both the simulation and analytical results for this test.

\begin{table}[ht]
\begin{center}
\begin{tabular}{|c|c|c|c|c|}\hline
 Input & Value & Output & Model Result & Analytical Result \\ \hline
 $r_{pol}$ & 6356.75160 & $e$ & 0.0818206 & 0.0818206 \\ \hline
 $r_{eq}$ & 6378.137 $km$ & $e^2$ & 0.0066946 & 0.0066946 \\ \hline
   &   & $f$ & 0.00335292 & 0.00335292 \\ \hline
   &   & $f^{-1}$ & 298.247 & 298.247 \\ \hline
   &   & $r_{eq}$ & 6378.137 $km$ & 6378.137 $km$ \\ \hline
\end{tabular}
\caption{Results for Mean Polar Radius Initialization}
\label{rpol_init_table}
\end{center}
\end{table}

As this table shows, the model achieved the same results for the shape
parameters as were obtained analytically for the listed inputs. (Note that the
analytical results were truncated to the same number of significant digits as
returned by the simulation to yield a consistent basis for comparison.) Thus,
the model passed this test, partially satisfying requirement
\mbox{\ref{reqt:func_init_ellip_constants}}.
\end{description}


\test{Initialize As Spherical Planet}\label{test:init_spherical}
\begin{description}
\item[Purpose:]\ \newline
This test case is designed to examine the ability of the \planetDesc\ to
initialize the planet shape constants given only zeros for input.
\item[Run directory:]\ \newline  RUN\_initplanet\_spherical
\item[Requirements:]\ \newline
By passing this test, this model partially satisfies requirement
\mbox{\ref{reqt:func_init_ellip_constants}}.

\item[Procedure:]\ \newline
To test the ability of the \planetDesc\ to initialize the full \JEODid\ set of
planet shape constants given only zeros as input, all shape parameters were
initialized to zeros at the start of the run. The resulting \planetDesc\ output
was then examined to determine whether the model did indeed properly initialize
as a spherical planet with mean equatorial radius equal to the value obtained
from its associated GravitySource object, as was expected.

\item[Results:]\ \newline
Table \ref{spherical_init_table} shows the values used for each input parameter,
and both the simulation and analytical results for this test.

\begin{table}[ht]
\begin{center}
\begin{tabular}{|c|c|c|c|}\hline
 Parameter & Input Value & Model Result & Analytical Result \\ \hline
 $e$ & 0.0 & 0.0 & 0.0 \\ \hline
 $e^2$ & 0.0 & 0.0 & 0.0 \\ \hline
 $f$ & 0.0 & 0.0 & 0.0 \\ \hline
 $f^{-1}$ & 0.0 & 0.0 & 0.0 \\ \hline
 $r_{eq}$ & 0.0 $km$ & 6378.137 $km$ & 6378.137 $km$ \\ \hline
 $r_{pol}$ & 0.0 $km$ & 6378.137 $km$ & 6378.137 $km$ \\ \hline
\end{tabular}
\caption{Results for Spherical Planet Initialization}
\label{spherical_init_table}
\end{center}
\end{table}

As this table shows, the model performed as expected for the given case; it
obtained the mean equatorial radius from the associated GravitySource and used
that value for both $r_{eq}$ and $r_{pol}$, leaving everything else zeros to
result in a representation of a spherical earth. (Note that the analytical
results were truncated to the same number of significant digits as returned by
the simulation to yield a consistent basis for comparison.) Thus, the model
passed this test, partially satisfying requirement
\mbox{\ref{reqt:func_init_ellip_constants}}.
\end{description}

\newpage
\section{Requirements Traceability}\label{sec:traceability}

\begin{longtable}[c]{||p{3in}|p{3in}|}
\caption{Requirements Traceability} \\[6pt]
\hline
{\bf Requirement} & {\bf Inspection and Testing} \\
\hline \hline
\endhead

\ref{reqt:toplevel} - Top-level Requirement &
  Insp.~\ref{inspect:TLI} \\
  \hline

\ref{reqt:base_planet_data_encapsulation} - Basic Data Encapsulation &
   Insp.~\ref{inspect:data_reqts} \\
\hline

\ref{reqt:ellip_planet_data_encapsulation} - Ellipsoidal Data Encapsulation &
   Insp.~\ref{inspect:data_reqts} \\
\hline

\ref{reqt:func_base_register_model} - Basic Model Registration &
   Insp.~\ref{inspect:func_reqts} \\
   &Test~\ref{test:init_eellip} \\
\hline

\ref{reqt:func_ellip_register_model} - Ellipsoidal Model Registration &
   Insp.~\ref{inspect:func_reqts} \\
   &Test~\ref{test:init_eellip} \\
\hline

\ref{reqt:func_init_ellip_constants} - Initialize Ellipsoidal Constants &
   Insp.~\ref{inspect:func_reqts} \\
   &Test~\ref{test:init_eellip} \\
   &Test~\ref{test:init_eellipsq} \\
   &Test~\ref{test:init_flatcoeff} \\
   &Test~\ref{test:init_invflat} \\
   &Test~\ref{test:init_rpol} \\
   &Test~\ref{test:init_spherical} \\
\hline

\end{longtable}

\newpage


%%%%%%%%%%%%%%%%%%%%%%%%%%%%%%%%%%%%%%%%%%%%%%%%%%%%%%%%%%%%%%%%%%%%%%%%%%%%%%%%
\input{planetVV_Metrics.tex}
