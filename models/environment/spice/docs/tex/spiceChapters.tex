\setcounter{chapter}{0}

%----------------------------------
\chapter{Introduction}\hyperdef{part}{intro}{}\label{ch:intro}
%----------------------------------

\section{Model Description}
%%% Incorporate the intro paragraph that used to begin this Chapter here.
%%% This is location of the true introduction where you explain why this 
%%% model exists.
%%% Identify the Model context within JEOD.

% The JEOD \SpiceDesc\ ...
The \SpiceDesc\ is a first step toward replacement of custom JEOD code with
an agency standard library (SPICE). The \SpiceDesc\ provides identical
functionality to the existing JEOD models with the single exception of the
high fidelity JEOD model for Mars orientation. In addition, the \SpiceDesc\
includes high fidelity ephemerides and simple (International Astronomical
Union) orientation models for planets, natural satellites, asteroids, comets,
and many other objects and locations of interest in the solar system. Moreover,
JPL maintains archives of the data files (called kernels) that drive the
ephemerides and orientation models of the \SpiceDesc, which are updated
as higher fidelity models become available. This has the effect of
``future-proofing'' JEOD while removing the requirement to maintain and update
over 60,000 lines of code in the legacy ephemerides and RNP models.



\section{Document History}
%%% Status of this and only this document.  Any date should be relevant to when 
%%% this document was last updated and mention the reason (release, bug fix, etc.)
%%% Mention previous history aka JEOD 1.4-5 heritage in this section.
%%% Mention that JEOD.pdf is the parent document.

\begin{tabular}{||l|l|l|l|} \hline
\DocumentChangeHistory
\end{tabular}

\section{Document Organization}
This document is formatted in accordance with the 
NASA Software Engineering Requirements Standard~\cite{NASA:SWE} 
and is organized into the following chapters:

\begin{description}
%% longer chapter descriptions, more information.

\item[Chapter 1: Introduction] - 
This introduction contains three sections: description of model, document history, and organization.  
The first section provides the introduction to the \SpiceDesc\ and its reason 
for existence.  It also contains a brief description of the interconnections with other models, and 
references to any supporting documents.  The second section displays the history of this document which includes
author, date, and reason for each revision; it also lists the document that is parent to this one.  The final
section contains a description of the how the document is organized.

\item[Chapter 2: Product Requirements] - 
Describes requirements for the \SpiceDesc.

\item[Chapter 3: Product Specification] - 
Describes the underlying theory, architecture, and design of the \SpiceDesc\ in detail.  It is organized into
three sections: Conceptual Design, Mathematical Formulations, and Detailed Design.

\item[Chapter 4: User Guide] - 
Describes how to use the \SpiceDesc\ in a Trick simulation.  It contains information that will be useful to
both users and developers of simulations. The \SpiceDesc\ is not intended to be extended, so there is no
discussion of how to do so.

\item[Chapter 5: Inspections, Tests, and Metrics] -  
The final chapter describes the procedures and results that demonstrate the satisfaction of the
requirements for the \SpiceDesc.

\end{description}


%----------------------------------
\chapter{Product Requirements}\hyperdef{part}{reqt}{}\label{ch:reqt}
%----------------------------------

\section{Project Requirements}
\requirement{Top-level Requirement}
\label{reqt:toplevel}
\begin{description}
\item[Requirement:]\ \newline
  The \SpiceDesc\ shall meet the JEOD project-wide requirements specified in
  the \JEODid\ Top-Level Document.

\item[Rationale:]\ \newline
  This is a project-wide requirement.

\item[Verification:]\ \newline
  Inspection
\end{description}


\requirement{Solar System Body Ephemeris Representation}
\label{reqt:rep_ephem}
\begin{description}
\item[Requirement:]\ \newline
  The \SpiceDesc\ shall provide the capability to use the JPL SPICE library
  to represent the ephemerides of any solar system body for which a
  corresponding data set is provided.

\item[Rationale:]\ \newline
  The existing JEOD DE4xx model is limited in the number of solar system bodies
  it can represent. SPICE provides a standard way of providing the same bodies,
  as well as virtually any other solar system body desired, so long as the data
  is available for it.

\item[Verification:]\ \newline
  Test
\end{description}


\requirement{Solar System Body Orientation Representation}
\label{reqt:rep_rnp}
\begin{description}
\item[Requirement:]\ \newline
  The \SpiceDesc\ shall provide the capability to use the JPL SPICE library
  to represent the orientation of any solar system body for which a
  corresponding data set is provided.

\item[Rationale:]\ \newline
  Existing JEOD models provide only Earth, Moon, and Mars orientation. SPICE
  provides a standard way of providing these plus many others, so long as the
  data is available for it.

\item[Verification:]\ \newline
  Test
\end{description}


%----------------------------------
\chapter{Product Specification}\hyperdef{part}{spec}{}\label{ch:spec}
%----------------------------------

\section{Conceptual Design}
%The \SpiceDesc\ is ...
NASA's Navigation Ancillary Information Facility (NAIF) offers NASA flight
projects and NASA funded researchers an observation geometry information
system called ``SPICE.'' Among other things, SPICE provides ephemerides for
hundreds of natural and man-made objects in the solar system as well as
orientation data for planets and moons which have orientation models.

The JSC Engineering Orbital Dynamics (JEOD) software includes a reference
frame model which relies on the concept of the frames having parents, where
each frame is defined with respect to its parent. The goal then is to use
SPICE to extend JEOD's built-in frame concept to any celestial object of
interest in a seamless way. To this end, it is necessary to derive both the
relational and mathematical components of a JEOD frame from information
provided by SPICE.


\section{Mathematical Formulations}

% The \SpiceDesc\ ...
In order for a given reference frame $B$ (which need not initially be known
to JEOD) to properly mesh with the JEOD reference frame system, it is necessary
to define the following quantities:
\begin{itemize}
\item The parent frame, denoted generically as $A$. This needs to be
a frame that is already part of the JEOD reference frame tree and which is
also known to SPICE. 
\item The position and velocity of the origin of frame $B$ expressed in $A$.
\item T\_parent\_this. This is the $3 \times 3$ transformation from $A$ to
$B$.
\item ang\_vel\_this. This is the angular velocity of $B$ with respect to
$A$, expressed in $B$.
\end{itemize}


\subsection{Nomenclature}
Subscripts will indicate the reference frame to which a quantity is relative
and the coordinate system in which it is expressed. For example,
$\vec x_{B|A:A}$ represents a vector position of frame $B$ with
respect to frame $A$ expressed in frame $A$.


\subsection{Relevant Functions from CSPICE} \label{subsec:functions}
The SPICE kernels and functions provide means of determining the 
information necessary to define $B$ as a JEOD reference frame. This is
accomplished by interfacing with the C-language implementation of SPICE,
known as CSPICE.

\paragraph{Translation}
In order to obtain $\vec x_{B|A:A}$ and $\dot{\vec{x}}_{B|A:A}$,
the position and velocity of the origin of $B$ expressed in $A$, JEOD employs
the CSPICE function spkez\_c. This function provides the position and velocity
of a ``target'' with respect to an ``observer'' in a specified coordinate system
at a given time.  For JEOD, the coordinate system to use is J2000, since JEOD
tracks the origins of all celestial bodies in J2000. The values
$\vec x_{B|A:A}$ and $\dot{\vec{x}}_{B|A:A}$ are returned by
the function in a single $6$-array from which JEOD will then need to extract
them.  The call to spkez\_c by JEOD is:
\begin{verbatim}
spkez_c (int Frame_B_ID,
         double ephemeris_time,
         "J2000",
         "none",
         int Frame_A_ID,
         double state[6],
         double *lt);
\end{verbatim}
The arguments for the function call are explained as follows:
\begin{itemize}
\item The Frame\_x\_ID arguments are the SPICE-defined integer codes for $A$
and $B$. For a description of these codes, see section~\ref{subsec:codes}.
\item The ``ephemeris\_time'' parameter is the time at which the relative state
between $A$ and $B$ is desired, in the Barycentric Dynamic Time (TDB) time scale.
\item Parameter ``J2000'' identifies the coordinate system in which position and
velocity should be expressed.
\item The ``none'' specifies that no aberration correction is to be applied, which is
consistent with the geometric interpretation of the ephemeris understood by JEOD.
\item Parameter ``state'' is the previously mentioned $6$-array from which the
position and velocity will be extracted following the function call.
\item The last parameter is unused by JEOD and thus is populated by a dummy
variable. It is the light travel time.
\end{itemize}
Upon completion of the call to spkez\_c, JEOD then extracts the desired quantities
$\vec{x}_{B|A:A}$ and $\dot{\vec{x}}_{B|A:A}$ from ``state'' by:
\begin{eqnarray}
\label{eq:state}
\vec{x}_{B|A:A} & = & \mbox{state[$i$] for }i = 0 \ldots 2 \\ \nonumber
\dot{\vec{x}}_{B|A:A} & = & \mbox{state[$i$] for }i = 3 \ldots 5
\end{eqnarray}

\paragraph{Orientation and Rotation}
The C implementation of SPICE (CSPICE) includes  the function sxform\_c which
retrieves a $6 \times 6$ matrix mapping position and velocity expressed in a
given frame to its representation in a different frame.  This matrix accounts
for orientation and relative rotation of the frames. (In the context of
ephemeris, this matrix is the mapping between the standard JEOD
pseudo-inertial frame (JEOD\_Planet\_Name.inertial)
centered on a celestial body and its planet-fixed counterpart.)  The sxform\_c
function assumes that both frames share a common origin. The relation
between the frames can be written as:
\begin{equation}
\left [\begin{array}{c}\vec x_B \\ \dot{\vec{x}}_B \end{array}\right ] =
M_{B|A} \left [\begin{array}{c} \vec x_A \\ \dot{\vec{x}}_A \end{array}
\right ]
\label{eq:m6}
\end{equation}
\noindent where
$\left [\begin{array}{c}\vec x_A \\ \dot{\vec{x}}_B \end{array}\right ]$
and
$\left [\begin{array}{c}\vec x_B \\ \dot{\vec{x}}_B \end{array}\right ]$
are position and velocity of a point expressed in frame $A$ and $B$
respectively.

The matrix $M_{B|A}$ can be partitioned into four $3 \times 3$
sub-matrices to facilitate extracting the orientation and angular velocity
between the standard pseudo-inertial and planet-fixed frames of the celestial
body of interest:
\begin{equation}
M_{B|A} = \left [\begin{array}{cc}
T_{B|A} & 0 \\
-\Omega_{B|A:B}T_{B|A} & T_{B|A}
\end{array} \right ]
\label{eq:m33}
\end{equation}
\noindent where $T_{B|A}$ is the $3 \times 3$ transformation from
frame $A$ to frame $B$ (T\_parent\_this in JEOD, because in the JEOD
reference frame tree the planet-fixed frame for a body is child to the
standard pseudo-inertial frame for the same body), and $\Omega_{B|A:B}$
is the skew-symmetric matrix satisfying
\begin{equation}
\Omega_{B|A:B} \vec{x} = \vec{\omega}_{B|A:B} \times \vec{x}
\label{eq:omega_cross_r}
\end{equation}
\noindent where $\vec{\omega}_{B|A:B}$ is the vector angular velocity of $B$ with
respect to $A$ expressed in $B$ (ang\_vel\_this in JEOD).
From ~\ref{eq:m33} it follows that
\begin{equation}
T_{i, j B|A} = M_{i,j} \mbox{ for }i, j = 0\ldots2
\label{eq:T}
\end{equation}
\noindent and
\begin{equation}
\Omega_{B|A:B} = -M_{i,j}T^t_{B|A} \mbox{ for }i = 3\ldots5, j = 0\ldots2
\label{eq:Omega}
\end{equation}
Finally, $\vec{\omega}_{B|A:B}$ is realized as
\begin{equation}
\vec{\omega}_{B|A:B} = \left [\begin{array}{c}
-\Omega_{1,2 B|A:B} \\
\Omega_{0,2 B|A:B} \\
-\Omega_{0,1 B|A:B} \end{array} \right ]
\label{eq:VecOmega}
\end{equation}

The call to sxform\_c by JEOD is:
\begin{verbatim}
sxform_c (char * from,
          char * to,
          double ephemeris_time,
          double trans[6][6]);
\end{verbatim}
The parameter ``from'' should be set to ``J2000'' in order to obtain the
relation between the standard pseudo-inertial frame centered at the
origin of the body of interest and the associated planet-fixed frame.  This is
because in JEOD all standard pseudo-inertial frames are J2000-oriented.  Next,
``to'' is the name of $B$, the planet-fixed frame, which JEOD constructs
consistent with SPICE nomenclature.  The time parameter ``ephemeris\_time'' is
the time at which the rotation state between $A$ and $B$ is desired in the TDB
time scale. Finally, ``trans'' is the previously discussed $6 \times 6$
matrix $M_{B|A}$.


\subsection {Integration with the JEOD Reference Frame System}
Following the calls to spkez\_c and sxform\_c described in the previous
subsections, all information needed to create a new JEOD frame representing
$B$ is now available.  The data members in the new Frame\_B correspond to the
quantities obtained in ~\ref{eq:state}, ~\ref{eq:T} and ~\ref{eq:VecOmega}
as follows:
\begin{eqnarray}
\mbox{Frame\_B.state.trans.position} & = & \vec{x}_{B|A:A} \\ \nonumber
\mbox{Frame\_B.state.trans.velocity} & = & \dot{\vec{x}}_{B|A:A} \\ \nonumber
\mbox{Frame\_B.state.rot.T\_parent\_this} & = & T_{B|A} \\ \nonumber
\mbox{Frame\_B.state.rot.ang\_vel\_this} & = & \vec{\omega}_{B|A:B}
\label{eq:FrameBDefinition}
\end{eqnarray}


\subsection {NAIF Integer ID Codes} \label{subsec:codes}
Every object in the SPICE inventory is assigned an integer identification code.
These codes follow some simple rules which are exploited in order to determine
parent-child relationships in the JEOD reference frame tree.  Specifically, the
solar system and planetary barycenters are numbered $0 \ldots 9$ in order of
distance from the sun. For purposes of this identification system,
the Pluto-Charon barycenter ID is $9$. Each planet is assigned the code $x99$
where $x = 1 \ldots 9$. The Sun is numbered $10$.  The planets Mercury and Venus
are moonless, thus there is no distinction between $1$ and $199$ or $2$ and $299$.
Codes larger than $999$ are reserved for comets and asteroids.  Planetary moons
are ordered from inner to outer, so, for example, the code for Phobos is $401$,
and the code for Deimos is $402$. The code system treats Pluto as a planet, thus
Pluto is $999$ and Charon is $901$. For full documentation,
see \href{https://naif.jpl.nasa.gov/naif/documentation.html}{NAIF Documentation}.

With this information, it is possible to write simple rules to determine the
code of the parent of any solar system object using the object's own code.
\begin{itemize}
\item $Code > 999$ (comet or asteroid): $Parent = 0$ (Solar System Barycenter).
\item $Code = 199$ or $299$ (Mercury or Venus): $Parent = 0$ (Solar System
Barycenter).
\item Otherwise, $Parent = \left\lfloor\frac{Code}{100}\right\rfloor$.
\end{itemize}



\section{Detailed Design}

\subsection{Process Architecture}
The \SpiceDesc\ is an extension of the JEOD ephemeris model framework
and is organized into three primary classes. The first is SpiceEphemeris,
which extends the EphemerisInterface class and thus is the orchestrator
of the entire \SpiceDesc. The second is SpiceEphemPoint, which is a child
class of EphemerisPoint.  There must be a separate SpiceEphemPoint for each
celestial body for which SPICE is to update the translational state in a
simulation. The final class is SpiceEphemOrientation, which extends
EphemerisOrientation. It performs a similar function as SpiceEphemPoint,
but for the rotational state of celestial bodies; one must exist in the
sim for each body that SPICE is to keep updated with respect to rotational
state.

The usual types of methods used for such things as initializing and updating
the model exist for the \SpiceDesc, just as for most JEOD models.  Since
this is a model to interface with SPICE, much of the functionality
involves either creating connections to SPICE or obtaining data from it in
accordance with the setup implemented in the simulation.  Since the
\SpiceDesc\ fits into the existing JEOD ephemeris model framework, the
Dynamics Manager takes care of orchestrating the state updates automatically.


\subsection{Functional Design}
This section describes the functional operation of the methods of the
\SpiceDesc.

The \SpiceDesc\ contains the classes SpiceEphemeris, SpiceEphemPoint, and
SpiceEphemOrientation. While there are several methods for each class, both
new and inherited, this section will focus on only a few key SpiceEphemeris
methods. Discussion of these methods will provide good insight into the inner
workings of the \SpiceDesc.  For an exhaustive treatment of the methods for
all three classes, see the \href{file:refman.pdf}{Reference Manual}\cite{api:spice}.

The SpiceEphemeris class contains the following methods, among others:
\begin{enumerate}

\funcitem{add\_planet\_name}
This is an S\_define or input file method which is used to add a new
SpiceEphemPoint to the \SpiceDesc's list of ones to keep updated.

\funcitem{add\_orientation}
This is an S\_define or input file method which is used to add a new
SpiceEphemOrientation to the \SpiceDesc's list of ones to keep updated.

\funcitem{initialize\_model}
This is an S\_define-level method, which sets the \SpiceDesc\ up
properly.  It has several subroutines and performs critical tasks such as
loading the given SPICE kernels, finding the desired celestial bodies and
orientations within them, figuring out what barycenters need to exist in
the simulation based upon the points and orientations loaded, creating
said barycenters, and making connections to the appropriate ephemeris
point and orientation objects in the SPICE kernels.

\funcitem{ephem\_build\_tree}
This method is called automatically by the Dynamics Manager whenever it
is determined that the reference frame tree needs to be rebuilt.

\funcitem{ephem\_update}
This method is called automatically by the Dynamics Manager and is the
routine that actually performs the state updates for the loaded
SpiceEphemPoints and SpiceEphemOrientations.

\end{enumerate}




%----------------------------------
\chapter{User Guide}\hyperdef{part}{user}{}\label{ch:user}
%----------------------------------

% The \SpiceDesc\ is ...
The User Guide chapters for many JEOD models contain sections for simulation
users, simulation builders, and model extenders.  This chapter will primarily
describe the verification sim provided for the \SpiceDesc\ and how to obtain
SPICE and the data files to use with it.  Thus, most of the content is
intended for simulation users.  However, there is also a short section at the
end of the chapter which describes briefly how to use the \SpiceDesc\ in a
Trick and JEOD simulation more generally, which should be useful for simulation
builders. The \SpiceDesc\ was not designed to be extended, therefore, there is
no section describing extension.

The verification directory for the \SpiceDesc\ includes two simulations: the
\SpiceDesc\ verification sim based on the SPICE ephemeris toolkit, and a
classic JEOD DE4xx-based ephemerides sim. These exercise comparable
capabilities, though it will be demonstrated how to expand SIM\_spice to
provide capabilities not available from the DE4xx model. Both simulations
should basically run ``out-of-the-box'', except that the SPICE simulation
(``SIM\_spice'') requires the C implementation of SPICE library, named CSPICE,
in order to run.


\section {The CSPICE Library}
SIM\_spice requires an external library in order to access the capabilities of
SPICE. In order to run SIM\_spice, it is necessary to download and compile
\href{https://naif.jpl.nasa.gov/naif/toolkit.html}{the SPICE toolkit}, then
define the environment variable ``JEOD\_SPICE\_DIR'' to point to the directory
where it was compiled.

The default build for the SPICE toolkit creates the static library
\begin{verbatim}
${JEOD_SPICE_DIR}/lib/cspice.a.
\end{verbatim}
SIM\_spice includes a file containing make overrides (S\_overrides.mk) which
uses the environment variable \verb|$JEOD_SPICE_DIR| to locate the SPICE library.
Unfortunately, the default build process for SPICE does not prepend the ``lib''
prefix to the name of the library file cspice.a.  To fix this, either rename
the file or provide an appropriately named symbolic link. The shell commands
for each alternative are given below.
\begin{verbatim}
mv ${JEOD_SPICE_DIR}/lib/cspice.a ${JEOD_SPICE_DIR}/lib/libcspice.a

ln -s ${JEOD_SPICE_DIR}/lib/cspice.a ${JEOD_SPICE_DIR}/lib/libcspice.a
\end{verbatim}


\section{Obtaining SPICE kernels}
This section describes the process used to obtain SPICE data files such as
the ones for Sun, Earth, Moon, and Mars used in SIM\_spice.  Two major sources
of SPICE data will be discussed: NAIF and Horizons.


\subsection {NASA's Navigation and Ancillary Information Facility (NAIF)}
\label{subsec:naif}
Data files which are used by the SPICE toolkit are called ``kernels''.  To
obtain many pre-made kernels including those for the Sun and planets, go to
the~\href{http://naif.jpl.nasa.gov/naif}{NAIF homepage}. Follow the links to
Data $\to$ Generic Kernels $\to$ Generic Kernels.  The resulting page should
show a list of directories and a file named ``aareadme.txt''.  At this point
it would be a good idea to read the file, which contains a concise explanation
of the various generic spice kernels.

Kernels for planetary ephemerides which describe translational motion of
astronomical objects are referred generically in SPICE as ``spk''
(SPICE Planetary Kernel) files. There are special high-fidelity versions of
such files which are in binary format and have the extension ``.bsp''.
The spk file that used to be used in SIM\_spice for ephemerides of the Sun and
planets is named ``de421.bsp'' and can be found in the directory
spk/planets/a\_old\_versions. The DE421 ephemeris file was used in order to
provide a consistent comparison with what was available with the classic JEOD
ephemerides model, but any other file containing analogous data could have
been used. The spk file currently used in SIM\_spice is ``de440.bsp'' in order to
come into alignment with other models being used for lunar operations using DE440.

Kernel files for natural satellites of other planets can be found in the
folder spk/satellites. For example, data for Phobos and Deimos, the moons of
Mars, are contained in the file ``mar097.bsp.''  Planets with large moon
systems are broken into separate files; however, since Mars has only two
moons, there is only a single file for the Martian system. The ``readme''
files in that folder can assist in determining which files are needed for
a given object.

Besides de421.bsp, the other files needed for SIM\_spice are the
high-precision orientation models for Earth and Moon.  These can be found in
the~\href{http://naif.jpl.nasa.gov/pub/naif/generic_kernels/pck}
{``pck'' directory} on the ``Generic Kernels'' page.  The specific files
used are ``earth\_000101\_171024\_170805.bpc'' and
``moon\_pa\_de421\_1900-2050.bpc.'' These files provide the high-fidelity
orientation models for Earth and Moon. When using DE440, the files used are
 ``earth\_000101\_240604\_240312.bpc'' and ``moon\_pa\_de440\_200625.bpc.''
The directory contains other versions which may be a better match for some
situations.  The file ``aareadme.txt' contains descriptions of each file which
can aid in determining which best suits the desired application. The files used
with SIM\_spice were chosen because they provide capabilities similar to JEOD's
existing RNP model.

Note that the text kernels
``pck00010.tpc'' and ``moon\_080317.tf'' are also found in this directory,
for use with DE421. For DE440, the files used are ``pck00011.tpc'' and
``moon\_de440\_220930.tf.'' ``moon\_080317.tf '' includes definitions for
standard lunar reference frames while ``pck00010.tpc'' contains
low fidelity (IAU) orientation models for planets and natural satellites in the
solar system. The orientation model for Mars that SPICE offers is one of these
relatively low fidelity IAU models; however, it agrees reasonably well with
the JEOD model for Mars orientation.


\subsection {The Jet Propulsion Laboratory Horizons System}
\label{subsec:horizons}
While the NAIF website includes ephemerides for Sun, Planets, natural
satellites, and barycenters, as well as orientation models for planets and
satellites, it is generally necessary to obtain kernel files for objects
such as comets and asteroids from the JPL Horizons system.  There are both
web ~\href{http://ssd.jpl.nasa.gov/?horizons}{(Horizons main page)} and telnet
interfaces to Horizons; however, high-fidelity binary files are only
available through the telnet interface.  The following example describes the
process for obtaining a high-fidelity ephemeris file for the asteroid Itokawa.

First type the following from the command line:
\begin{verbatim}
telnet ssd.jpl.nasa.gov 6775
\end{verbatim}
The first task is to find the body of interest. In the case of Itokawa, this
is very easy. Just type ``Itokawa'' at the ``Horizons$>$'' prompt. Once Horizons
finds Itokawa, press Return to confirm. The screen will fill with information
about Itokawa, and at the bottom the user will be prompted for input.  Type
``s'' to request an ``spk'' file.  The next step will request an email address.
Enter it and confirm.  Then, Horizons will ask whether a text file is desired.
Since binary is desired instead, respond with ``n''.

Next, the start and stop dates for the file are needed. For instance,
SIM\_spice starts and ends on Jan. 30, 2009, so any range which includes that
day would be fine.

Finally, the system creates a binary spk file and provides instructions on
how/where to retrieve it by anonymous ftp. The name of the file is a somewhat
random string of characters; the user will likely find it helpful to rename
it something more human-friendly after retrieval, such as ``itokawa.bsp''.
Follow the instructions and retrieve the file.


\section{SIM\_spice}
Once SPICE has been downloaded and installed, SIM\_spice is ready to be run.
The simulation directory already contains the SPICE kernels needed to operate the
basic simulation which includes Sun, Earth, Moon, and Mars.  These files were
retrieved using the process described in section~\ref{subsec:naif}.

The standard RUN case delivered with SIM\_spice is named ``RUN\_01''. That run
will create files that log the position and velocity of Sun, Earth, Moon and
Mars in Solar System Barycenter coordinates. The rotational states of Earth,
Moon, and Mars are also logged in their respective files. The exact same
information is logged by SIM\_de4xx and can be used for comparison.

The generalized version of SIM\_spice includes two bodies, Phobos and Itokawa,
which are not part of the JEOD DE4xx model. In order to run the generalized sim,
one merely needs to uncomment a few lines in the input deck and download the
necessary files from NAIF and Horizons.  The appropriate file for Phobos is
named ``mar097.bsp'' and is located in the directory ``spk/satellites'' on
the NAIF Generic Kernels page; see section~\ref{subsec:naif} for further
information on retrieval.  The appropriate file for Itokawa can be retrieved
from the JPL Horizons system; the process is thoroughly described in
section~\ref{subsec:horizons}.

Once those files are retrieved, place them in the data directory of
SIM\_spice. Then open the file SET\_test/RUN\_01/input.py and uncomment
line 19. The additional bodies will now be actively updated upon subsequent
re-runs of SIM\_spice.

(Aside: If one is interested in seeing how the SPICE files are unpacked
and the resulting JEOD reference frame tree is built, then uncomment line 7 of
SET\_test/RUN\_01/input.py. Note that doing so uncorks a spew of other debug
output from JEOD as well, so it might be the kind of thing best left for
debugging purposes.)


\section{Using SPICE in a Trick Simulation}\label{sec:builder}
In order to have planets or other celestial bodies in a Trick simulation,
one should instantiate a SimObject for each object. The typical architecture
includes the following:
\begin{itemize}
\item A SimObject to manage the overall environment.
\item Planetary SimObjects, one for each celestial body, containing Planet
and GravitySource classes to represent the associated celestial body's reference
frames, characteristics, and gravity.
\item ``Default data'' classes to initialize the Planet and GravitySource objects
within each planetary SimObject with the characteristics of the associated
celestial body.
\end{itemize}


\subsection{Planetary SimObjects}\label{subsec:planet.sm}
Planetary SimObjects are the S\_define level classes which encapsulate a planet
or other celestial object. JEOD provides several predefined S\_modules of this
sort for Earth, Moon, Mars, and the Sun; they can be found in
\verb|${JEOD_HOME}/lib/jeod/JEOD_S_modules|.

In addition to the standard JEOD S\_modules for Earth, Moon, Sun, and Mars,
SIM\_spice also includes two custom S\_modules, itokawa\_basic.sm and
phobos\_basic.sm, which are used in the generalized simulation. These
S\_modules were built by inheriting from a generic planet module found
in the directory \verb|${JEOD_HOME}/lib/jeod/JEOD_S_modules/Base| which is
there for that purpose. Thus, itokawa\_basic.sm and phobos\_basic.sm
are excellent prototypes for any new S\_modules that may need to be created
for celestial objects driven by SPICE.


\subsection{Planetary Initialization Classes}\label{subsec:planetary_init}
JEOD provides default data classes to initialize Planet and GravitySource
objects for Earth, Moon, Sun, Mars, and Jupiter. These standard JEOD
initialization classes are found in the ``data'' directories for the planet
and gravity models respectively.  For the purposes of demonstrating the
capabilities of the SPICE model via SIM\_spice, analogous initialization
classes were created for Phobos and Itokawa.  However, these classes are
only for demonstration and should not be used when accurate planet or gravity
models are needed.


\subsection{Working with SPICE Ephemerides}\label{subsec:spice_ephelmerides}
Starting with JEOD version 3.3, the standard JEOD module library includes
an S\_module named environment\_spice.sm which defines an environment
SimObject for the SPICE ephemerides. Thus, including SPICE in a sim is
as simple as adding the line
\begin{verbatim}
#include "JEOD_S_modules/environment_spice.sm"
\end{verbatim}
to an S\_define file.

Unlike the classic DE4xx model, the \SpiceDesc\ has no concept of activation
or deactivation of an individual celestial object once declared.  If a
SPICE ephemeris object exists in a sim, then SPICE assumes that the object
should be active.  So, all that is necessary in order for a sim to contain
celestial bodies driven by the \SpiceDesc\ is to define their S\_modules,
and include environment\_spice.sm as above.

The rest of what is needed is supplied in the input deck. The following
examples assume one is using the standard JEOD S\_module environment\_spice.sm,
which instantiates a SimObject named ``env'' containing the data member
``spice'', which is of type ``SpiceEphemeris'' (i.e., the \SpiceDesc).

First, introduce the kernel files downloaded from NAIF, containing data
for Sun, Earth, Moon, and Mars:
\begin{verbatim}
env.spice.metakernel_filename = "data/kernels_440.tm"
\end{verbatim}
The file kernels\_440.tm is a so-called ``metakernel'' file, meaning it contains
a list of kernels for SPICE to load. This file can be used as a template for
tailoring to one's own needs. The previous DE421 configuration can be accessed
by using kernels\_421.tm instead. For the generalized sim containing Phobos and
Itokawa, please refer to the file data/more\_kernels.tm.

Next, load the planetary ephemerides (see Modified\_data/spice.py):
\begin{verbatim}
env.spice.add_planet_name("Sun")
env.spice.add_planet_name("Earth")
env.spice.add_planet_name("Moon")
env.spice.add_planet_name("Mars")
\end{verbatim}

Finally, request orientation calculations for Earth, Moon, and Mars:
\begin{verbatim}
env.spice.add_orientation("Earth")
env.spice.add_orientation("Moon")
env.spice.add_orientation("Mars")
\end{verbatim}



%----------------------------------
\chapter{Inspections, Tests, and Metrics}\hyperdef{part}{ivv}{}\label{ch:ivv}
%----------------------------------

% This chapter describes...

\section{Inspection}\label{sec:inspect}
This section describes the inspections conducted on the \SpiceDesc\ to examine
its compliance with the inspection requirements levied against it.

\inspection{Top-level Inspection}
\label{inspect:TLI}
This document structure, the code, and associated files have been inspected,
and together satisfy requirement~\traceref{reqt:toplevel}.


\section{Tests}\label{sec:tests}
This section describes the tests conducted to verify and validate
that the \SpiceDesc\ satisfies the requirements levied against it.
All verification and validation test source code, simulations and procedures
are archived in the JEOD directory
{\tt models/environment/spice/verif}.\relax


\test{Ephemeris and Rotation}\label{test:spice_ephem_rot}
\begin{description}

\item[Background]\ \newline
The purpose of this test is to demonstrate the ability of the \SpiceDesc\
to represent both the ephemerides and orientation of any solar system
body.


\item[Test description]\ \newline
This test utilizes two simulations -- one that utilizes the \SpiceDesc\
for ephemeris and rotational state representation of several
solar system bodies, and one that uses the legacy JEOD De4xx ephemeris and
RNP models for analogous calculations for a subset of the same bodies. The
results of the first will be compared against the second for the bodies that
are common to both, in order to build confidence in the correct operation of
the \SpiceDesc\ even for the bodies that are not. 

The SPICE-only simulation contains the solar system bodies Sun, Earth, Moon,
Mars, Itokawa (an asteroid), and Phobos (a moon of Mars). It can provide
ephemeris data for all of them, as well as rotation data for Earth, Moon,
and Mars.

The simulation using legacy JEOD models contains the bodies Sun, Earth,
Moon, and Mars. It provides ephemeris for all of them, along with rotation
for Earth, Moon, and Mars. The legacy JEOD models are unable to provide
ephemeris or rotation information for Itokawa or Phobos, which illustrates
the primary benefit of the new \SpiceDesc: the ability to include many
more solar system bodies in a simulation.


\item[Test directories] {\tt SIM\_spice and SIM\_de4xx}


\item[Success criteria]\ \newline
A comparison of the output of the SPICE-only simulation with the legacy
simulation should show good agreement between the two for both
translational and rotational states of the bodies being compared. Due
to the nature of the data and models being employed, translation
should agree to within approximately machine precision, while rotation
will be less accurate but still agree to near machine precision.


\item[Test results]\ \newline
All output data confirmed expectations.

\item[Applicable Requirements]\ \newline
This test satisfies the requirements~\traceref{reqt:rep_ephem}
and~\traceref{reqt:rep_rnp}.

\end{description}



\newpage
\boilerplatetraceability

\newpage
\boilerplatemetrics
