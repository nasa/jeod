
%%%%%%%%%%%%%%%%%%%%%%%%%%%%%%%%%%%%%%%%%%%%%%%%%%%%%%%%%%%%%%%%%%%%%%%%%%%%%%%%%
%
% Purpose:  requirements for the RadiationPressure model
%
%
%%%%%%%%%%%%%%%%%%%%%%%%%%%%%%%%%%%%%%%%%%%%%%%%%%%%%%%%%%%%%%%%%%%%%%%%%%%%%%%%

% add text here to describe general model requirements
% text is of the form:
%\requirement{REQUIREMENT DESCRIPTION}
%\label{reqt:REQT_DESC_ABBREVIATED}
%\begin{description}
%  \item[Requirement:]\ \newline
%     <Insert description of requirement>
%  \item[Rationale:]\ \newline
%     <Insert description of rationale>
%  \item[Verification:]\ \newline
%     <Insert description of verification (e.g. "Inspection")>
%\end{description}


\requirement{State Encapsulation}
\label{reqt:stateencapsulation}
\begin{description}
  \item[Requirement:]\ \newline
     The \RadiationPressureDesc\ shall encapsulate the positional state of the vehicle and the
     data necessary to calculate the orientation and position of the vehicle
     with respect to the sun, and planetary bodies.

  \item[Rationale:]\ \newline
     The extent to which radiation pressure acts on the vehicle depends on the
     radiation flux and the orientation of the vehicle with respect to the
     radiation flux.  The magnitude of the radiation flux is a function of
     distance from Sun, and potential shadowing effects of intervening bodies.
     The direction of the force resulting form the interaction depends on the
     relative orientation of the surfaces of the vehicle to the flux vector.

  \item[Verification:]\ \newline
     Inspection \vref{inspect:stateencapsulation}
\end{description}


\requirement{Surface Representation}
\label{reqt:surfacerepresentation}
\begin{description}
  \item[Requirement:]\ \newline
     The \RadiationPressureDesc\ shall offer the capability to model the vehicle surface
as a simple approximation, and as a high-fidelity geometric structure comprising multiple materials.

  \item[Rationale:]\ \newline
     A driving design consideration across the development of
	  \JEODid\ is the ability to do
simple tasks simply.  The inclusion of a radiation pressure model to test the significance of
its inclusion must be a simple task.  Conversely, a simple model may not offer the desired
fidelity, particularly in the situation in which a vehicle is made of multiple materials, or where
there is significant diversion from an isothermal surface.  These capabilities must be provided
for maximal utility of the model.

  \item[Verification:]\ \newline
     Test \vref{test:surfacerepresentation}
\end{description}


\requirement{Temperature Structure}
\label{reqt:temperaturestructure}
\begin{description}
  \item[Requirement:]\ \newline
     The \RadiationPressureDesc\ shall incorporate realistic time-dependent
temperature modeling of its surface.

  \item[Rationale:]\ \newline
     Although thermal emission is typically a relatively small effect of radiation
     pressure modeling, the interaction between the vehicle and radiation is a
     primary contributor to the temperature structure of the vehicle, which may be
     desired by other models.  Furthermore, there are situations in which spatial
     temperature variations do play a significant role in the radiation pressure calculation.

  \item[Verification:]\ \newline
     Tests \vref{test:temperature_integration1} and \vref{test:temperature_integration2}.
\end{description}


\requirement{Albedo Pressure Option}
\label{reqt:albedopressure}
\begin{description}
  \item[Requirement:]\ \newline
     The \RadiationPressureDesc\ shall provide the ability to add  -- at a future time --
     a model to simulate the radiation pressure from sources other than the sun.

  \item[Rationale:]\ \newline
     The solar radiation pressure will typically dominate, but secondary pressure (also known as albedo pressure) from planetary objects is necessary for a complete
     description and more accurate modeling.  These other sources will provide
     a flux that will interact with the vehicle in exactly the same way as the
     solar flux, and should be handled by the same routines.

  \item[Verification:]\ \newline
     Inspection \vref{inspect:albedopressure}.
\end{description}

\requirement{Force and Torque}
\label{reqt:forceandtorque}
\begin{description}
  \item[Requirement:]\ \newline
     The data structures shall include storage of the force and torque on the
     vehicle.

  \item[Rationale:]\ \newline
     The force and torque can be collected within Trick -- or within any other
     simulation engine -- from multiple
     interaction models to provide the overall force and torque on the vehicle
     for integration within the dynamics package for purposes of state
     propagation.

  \item[Verification:]\ \newline
     Inspection \vref{inspect:forceandtorque}.
\end{description}

\section{Functional Requirements}\label{sec:func_reqts}




\requirement{Interaction Modeling}
\label{reqt:functional_interaction}
\begin{description}
  \item[Requirement:]\ \newline
    The model shall be capable of modeling the radiation interaction by appropriate
    combination of the effects of photon
    absorption, diffuse photon reflection (rough surface), specular photon
    reflection (smooth surface), and photon emission (thermal emission).
    \subrequirement{Temperature}\ \newline
      The model shall be capable of independently modeling the temperature of each
      thermally independent
      component (facet) of the overall surface representation.
    \subrequirement{Plate Properties}\ \newline
      Each component of the overall surface representation shall be described
      in terms of its material properties to allow
      the combination of these terms to produce the most physically realistic
      description of the surface-photon interaction.
  \item[Rationale:]\ \newline
    Each type of interaction behaves in such a way to produce a different force
    resulting from the same incident flux.  The relative "weighting" of each
    interaction-type is a function of the material properties, which may not be
    constant over an entire vehicle (e.g. the characteristics of solar array panels may be very
    different from those of a thermal shield).  By assigning the material values to each
    facet, these local variations can be modeled and accounted for.
    The temperature of each plate will be particularly influential on the
    photon
    emission, which varies with temperature to the 4th power.  The sides of the
    vehicle in shadow will tend to be significantly cooler than those that are
    in direct solar illumination, and this spatial temperature variation can be
    modeled on a facet-by-facet basis.
  \item[Verification:]\ \newline
    Tests~\vref{test:Fasd}, \vref{test:temperature_integration1}, and \vref{test:temperature_integration2}
\end{description}



\requirement{Temperature Integration}
\label{reqt:functional_temperature}
\begin{description}
  \item[Requirement:]\ \newline
      The model shall be capable of integrating the time-dependent temperature
      structure in order to maintain an energy balance between thermal energy, and
      net radiative power, derived from consideration of absorption and emission
      processes.
	\item[Rationale:]\ \newline
	  The temperature structure should have physical validity.  Its integration must be independent of the overall state integration because of the influence that temperature variability can have on the state integration.
	\item[Verification:]\ \newline
	  Tests \vref{test:temperature_integration1} and \vref{test:temperature_integration2}.
\end{description}

\requirement{Flux Calculation}
\label{reqt:functional_flux}
\begin{description}
  \item[Requirement:]\ \newline
    The model shall be capable of determining the magnitude and direction of
    the solar flux as a function of the current position with respect to the
    sun.

    \subrequirement{Orientation}\ \newline
      The model shall be capable of calculating the orientation of the vehicle
      with respect to the flux vector.

  \item[Rationale:]\ \newline
    For any given vehicle and vehicle-orientation, the force exerted on the
    vehicle is proportional to the magnitude of the incident flux.
    The interaction between the flux and the vehicle depends on
    the respective angles between the vehicle plates and the flux vector.

  \item[Verification:]\ \newline
    Tests \vref{test:eclipse_annular}, \vref{test:eclipse_transverse}, and \vref{test:orbital_long}.
\end{description}


\requirement{Eclipse Calculation}
\label{reqt:functional_eclipse}
\begin{description}
  \item[Requirement:]\ \newline
    The \RadiationPressureDesc\ shall be capable of determining whether the vehicle is in full
    illumination, in the shadow of a third body, or in a position that is
    partially eclipsed (receiving partial illumination).  In the case of a
    partial eclipse, the model shall be capable of calculating an adjustment to
    the background solar flux to account for that part of the solar disk that is
    not visible.
  \item[Rationale:]\ \newline
    The radiation force depends on the flux, which can be affected by
    intervening objects.  To accurately model the flux, the model must have the
    capability to account for those objects.
  \item[Verification:]\ \newline
    Tests \vref{test:eclipse_annular} and \vref{test:eclipse_transverse}.
\end{description}

\requirement{Eclipsing Bodies}
\label{reqt:eclipsingbody}
\begin{description}
   \item[Requirement:]\ \newline
    The \RadiationPressureDesc\ shall have the capability to calculate the shadowing
    effect of any one intervening body, and the option to extend that capability to
    multiple intervening bodies.
  \item[Rationale:]\ \newline
    There are limited situations where more than one body could
    simultaneously provide significant shadowing, so typically only one object need
    be identified at any particular time.  However, for simulations such
    as an Earth-Moon transit scenario, that object may change (from Earth
    to Moon, for example).

      For future development, it is conceivable that it may be desirable to add the capability
      to monitor the eclipsing of non-planetary third bodies (such as an intervening
      vehicle).  An example of this application would be the temperature modeling of components
      of a larger, composite vehicle, such as International Space Station (ISS).
  \item[Verification:]\ \newline
    Inspection \vref{inspect:eclipsingbody}
\end{description}
