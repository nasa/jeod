%%%%%%%%%%%%%%%%%%%%%%%%%%%%%%%%%%%%%%%%%%%%%%%%%%%%%%%%%%%%%%%%%%%%%%%%%%%%%%%%
% ivv_inspect.tex
% Inspections of the Container Model
%
% 
%%%%%%%%%%%%%%%%%%%%%%%%%%%%%%%%%%%%%%%%%%%%%%%%%%%%%%%%%%%%%%%%%%%%%%%%%%%%%%%%

\inspection{Top-level Inspection}
\label{inspect:TLI}

This document structure, the code, and associated files have been inspected.
The \ModelDesc satisfies requirement~\traceref{reqt:toplevel}.

\inspection{Container Inspection}
\label{inspect:STL}

One  purpose of the JEOD container replacements is to transparently
encapsulate the \Cplusplus standard library list, vector, and set containers
in a manner that enables checkpoint/restart.
A design goal was to allow future extensions to
sequence containers such as std::map and
associative containers such as std::map
that are not supported in this release.

Tables 65 and 66 in ISO/IEC 14882:2003\cite{cpp2003} describe functionality
common to all of the container classes. By inspection, the template class
JeodContainer provides all types and methods described in these tables except
for the Allocated-based \verb|reference| and \verb|const_reference| typedefs.
(Support for Allocators is not required in JEOD 2.2.)

Tables 67 and 68 describe the base functionality of the sequence containers.
Table 79 describes the base functionality of the associative containers.
Items in tables 67 and 69 that share a common signature are implemented in
template class JeodContainer. Items in tables 67 and 68 that are common to all
sequence containers are implemented in class template JeodSequenceContainer.
Items in table 69 are implemented in template class JeodAssociativeContainer.

Sections 23.2.2, 23.2.4, and 23.3.3 of the \Cplusplus standard describe the
functionality of list, vector, and set.
The functionality described in those sections is either implemented in one of the
aforementioned base template classes; in the
template classes JeodList (list), JeodVector (vector), and JeodSet (set);
or are omitted because they relate explicitly to Allocator or Compare classes.

The constructors in the classes JeodList, JeodVector, and JeodSet are protected.
The standard requires that the constructors be public.
This is addressed in the template classes
JeodObjectContainer, JeodPrimitiveContainer, and JeodPointerContainer,
which provide the publicly visible constructors as required by the standard
(sans the support for non-default Allocators).

By inspection, the \ModelDesc satisfies
requirement~\traceref{reqt:stl_containers}.

\inspection{Checkpoint/Restart Inspection}
\label{inspect:chkpt}

The primary purpose of the model is to provide a basis for checkpointing and
restarting objects whose content is opaque to simulation engines such as Trick.
The abstract class JeodCheckpointable defines the basis for these operations.
The model provides a simple extension of this class, SimpleCheckpointable,
which can be used to checkpoint and restart files.

The template class
JeodContainer and its derived classes JeodObjectContainer, JeodPointerContainer,
and JeodPrimitiveContainer provide the requisite functionality needed to
make the STL container replacements checkpointable and restartable.

By inspection, the \ModelDesc satisfies
requirement~\traceref{reqt:chkpt}.

