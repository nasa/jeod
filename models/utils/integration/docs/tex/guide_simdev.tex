Instructions are provided for simulation developers working with the Trick13
simulation engine.  \JEODidx is not compatible with versions of Trick older than 13.0.

\subsection{Populating Integration Groups}\label{sec:user_simdev_groups}
The purpose of integration groups is to
allow the simultaneous integration of models of disparate fidelity
requirements within a single simulation.  In most applications, this tool is
not needed, in which case the default behavior  -- in which there is
only one group -- is assumed.  Each group must define its own integrator
following the steps in
Section~\ref{sec:user_simdev_integrator}.

\subsubsection{Single Group Default Settings}
Without the need for multiple groups, the integration command is
straightforward:

\begin{verbatim}
 IntegLoop sim_integ_loop (DYNAMICS) <simulation object list>;
\end{verbatim}

This command comes at the end of the S\_define.  The argument (e.g. DYNAMICS)
is the integration rate (note - this is based on the simulation time, not the
dynamic time; for the difference, see the Time Model
documentation~\cite{dynenv:TIME}).  Following the argument is a
comma-separated list of all of the simulation objects that need integrating.

An equivalent command is:
\begin{verbatim}
 integrate (DYNAMICS) <simulation object list>;
\end{verbatim}


\subsubsection{Multi-Group Settings}

The prefered method for setting multiple groups is to use the standard module,
\textit{JEOD\_S\_modules/integ\_loop.sm}.  In the S\_define file, include this
file:
\begin{verbatim}
 #include "JEOD_S_modules/integ_loop.sm"
\end{verbatim}

Unlike most of the other sim-modules in the JEOD\_S\_modules directory, this
one
does not create a simulation object, it only defines its contents.

Declare an instance of the DynamicsIntegrationGroup somewhere in the
S\_define, for example in a simulation object called \textit{integ\_object}.  Note that this instance is not going to be used directly, so one instance will suffice for as many groups as are desired.
\begin{verbatim}
 ##include "dynamics/dyn_manager/include/dynamics_integration_group.hh"
 ...
 DynamicsIntegrationGroup group;
\end{verbatim}



At the end of the S\_define, create one instance of the loop object defined in
the integ\_loop file for each intended group.

The structure of this argument is as follows:
\begin{verbatim}
JeodIntegLoopSimObject <name of loop-object> (arguments)
\end{verbatim}

The arguments comprise the following list:
\begin{enumerate}
 \item  The integration-rate.  This is often pre-defined, in which case only
 the name is needed.
 \item  The instance of the desired IntegratorConstructor.  For example, to
 use an Euler integration, declare an instance of EulerIntegratorConstructor
 in integ\_object:
 \begin{verbatim}
 EulerIntegratorConstructor euler_constr;
  \end{verbatim}
  and add the argument \textit{integ\_object.euler\_const}.
  \item An instance of the IntegrationGroup class (e.g.
  \textit{integ\_object.group}).

  \item The TimeManager object for the simulation.  There should be only one,
  if using the standard sim-modules, it will be
  \textit{jeod\_time.time\_manager}.
  \item The Dynamics Manager (DynManager) object.  Again, there should be only
  one in the simulation; if using the standard sim-modules, it will be
  \textit{dynamics.dyn\_manager}.

  \item The gravity manager (GravityManager) object.  Again, there should be only
  one in the simulation; if using the standard sim-modules, it will be
  \textit{env.gravity\_model}.
  \item The address of the first simulation object to be integrated in the
  group.  For example, with a simulation-object called vehicle\_1,
  \textit{\&vehicle\_1}.
  \item Repeat the previous step to add the addresses of as many simulation
  objects as go with this group.
  \item Finish the argument with nullptr to indicate that all objects have been
  entered.
\end{enumerate}

Note that an empty group may be created by specifying no simulation objects.

These examples are taken from
\textit{environment/ephemerides/verif/SIM\_prop\_planet\_T10}.
\begin{verbatim}
 JeodIntegLoopSimObject fast_integ_loop (
    FAST_DYNAMICS,
    integ_constructor.selected_constr, integ_constructor.group,
    jeod_time.time_manager, dynamics.dyn_manager, env.gravity_manager,
    &sun, &jupiter, &saturn, nullptr);


JeodIntegLoopSimObject slow_integ_loop (
    SLOW_DYNAMICS,
    integ_constructor.selected_constr, integ_constructor.group,
    jeod_time.time_manager, dynamics.dyn_manager, env.gravity_manager,
    nullptr);
\end{verbatim}


Once created, objects may be moved from one group to another with the command
\begin{verbatim}
<loop_name>.integ_loop.add_sim_object(<name of object>)
\end{verbatim}

For example, from one of the SIM\_prop\_planet input files

(\textit{environment/ephemerides/verif/SIM\_prop\_planet\_T10/SET\_test/RUN\_switch\_integ/input.py}),
\begin{verbatim}
  slow_integ_loop.integ_loop.add_sim_object(saturn);
\end{verbatim}
It is not necessary to remove an object from its previous integration group.
An object can only belong to one group so is removed automatically.


\subsubsection{Caution on Groups with Different Rates}
While having integration groups with different rates is a valuable tool, it
can
lead to some unfortunate consequences if misused.  The user should be aware
that making a direct comparison of logged values from these two groups may
result in inconsistencies.  Each group will be integrated through its complete
tour before the next group starts.  The objects in the group integrated at
lower frequency will sit with the same state while the higher frequency group
catches up.  Looking at relative states between such entities may lead to
inappropriate conclusions.  When the higher rate is not an integer multiple of
the lower rate (e.g. one at 5Hz and one at 7Hz), the times that the two groups
are providing time-consistent data can be few and far-between.

This is particularly important when switching an object from one group to
another.  When the object switches, its current state is maintained even if it
is invalid at the current time.  For example, suppose some vehicle, was
coasting with an integration period of 10 seconds.  At t=126s, the vehicle is
activated, and now needs integrating at the flight software rate, say 50Hz.
Transitioning immediately to the higher rate group would put the vehicle at
t=126s with the state it had at t=120s (its last integration).  From here
(t=126s), that wrong state will be propagated.






\subsection{Setting the Integration Method}
\label{sec:user_simdev_integrator}

In order to use an integrator, it must be available to the simulation engine
at compilation time.  There are two ways by which this can be done:
\begin{enumerate}
 \item inclusion of the integrator from a standard suite, via an enumerated
 list, or
 \item deliberate declaration of the integrator-constructor in the S\_define.
\end{enumerate}

Having the integrator available without deliberate inclusion may be easier to
implement, but this process has the significant drawback that
the code for that integrator must then be compiled as a part of the simulation
whether it is desired or not.  This time-consuming model is suboptimal, and
while it is supported in the current release, it may be deprecated later.  The
recommended practice, therefore, is to instantiate the integrator-constructors
directly, in either the S\_define or the input file.

To provide backward compatibility, the enumerated list of integrators is still
provided, as is the conversion from the Trick enumerated list.  However, there
are no plans to add new integrators to the list.  Consequently, the only way
to access the Gauss-Jackson method is through the recommended practice of
direct declaration.

Furthermore, when using multiple integration groups, the process for defining
which integrator to use with which group requires that the
integrator-constructor be
available for passing as an argument.  In this case, the
recommended method is required.

The pre-loaded integrators are available via an enumerated list.  The
enumeration value can be accessed from either the JEOD list or specified via
the trick simulation interface method using the comparable
Trick list.

There are three methods available for selecting an integrator, as discussed
below.

\subsubsection {Recommended Practice}

Setting the Dynamics Manager \textit{integ\_constructor} value is the priority
method for defining the integrator.  It is this value that is tested first.

There are two ways to do this:
\begin{enumerate}
 \item \textbf{The easy way:} \newline
 \begin{enumerate}
  \item Include the header file in the S\_define, e.g.
   \begin{verbatim}
   ##include "utils/integration/include/rk4_integrator_constructor.hh"
   \end{verbatim}
   \item then instantiate the Integrator Constructor by assignment in the
     input file, e.g.
   \begin{verbatim}
    dynamics.manager_init.integ_constructor = trick.RK4IntegratorConstructor()
   \end{verbatim}
 \end{enumerate}



 \item \textbf{The more traditional way:} \newline
 \begin{enumerate}
  \item Instantiate an instance of the integrator-constructor in the S\_define.
  \item Point the DynManagerInit value integ\_constructor to the instance of
  the integrator-constructor
 \end{enumerate}

\textbf{Example:}

In the S\_define, create a sim-object called \textit{integ\_sim\_object}, that
includes an instance of a RK4IntegratorConstructor called \textit{rk4}.
\begin{verbatim}
##include "utils/integration/include/rk4_integrator_constructor.hh"
class IntegSimObject: public Trick::SimObject {
...
RK4IntegratorConstructor rk4;
...
}
IntegSimObject integ_sim_object;
\end{verbatim}

In the input file, declare the \textit{integ\_constructor} to be this new
constructor
(note - although the \textit{integ\_constructor} is a pointer, Python can
assign it without having to take the address)

\begin{verbatim}
 dynamics.manager_init.integ_constructor = integ_sim_object.rk4
\end{verbatim}

\end{enumerate}


\subsubsection{Allowable non-optimal practice}

If the integ\_constructor pointer is NULL (i.e. has not been set), then the
next-preferred method is to use the \textit{jeod\_integ\_opt} specification.
The allowable targets for this are the enumerated items in
Table~\ref{tab:user_simdev_jeod_integ}, each prefaced with
\verb+trick.Integration.+~.

\textbf{Example:}
\begin{verbatim}
 dynamics.manager_init.jeod_integ_opt = trick.Integration.RK4
\end{verbatim}

\begin{table}[h!]
\centering
\caption{Integration Technique Enumeration}
\label{tab:user_simdev_jeod_integ}
\vspace{1ex}
\begin{tabular}{||l|c|c|}
\hline
\bf{Integrator} & {\bf Enumeration} & {\bf Trick Equivalent}
\\ \hline \hline
Euler & 1 & Euler \\
Symplectic Euler &2&Euler\_Cromer\\
Beeman &3&N/A\\
RK2 &4&Runge\_Kutta\_2\\
RK4 &5&Runge\_Kutta\_4\\
ABM4 &6&ABM\_Method\\
Gauss-Jackson &N/A&N/A \\
\hline
\end{tabular}
\end{table}

\subsubsection{Deprecated Practice}

It is still possible to declare the method using the Trick enumerated list,
and that practice is still widely used, although it is no longer supprted.
This
method is only followed when both the \textit{integ\_constructor} and
\textit{jeod\_integ\_opt} are not set.  It is
included here to provide simulation developers with information on how to
interpret this command if seen in other simulations.

The \textit{sim\_integ\_opt} value in the dynamics manager initializer can be
set to an element in the Trick integration list.  Note that only those
elements in the Trick integration list that appear in
Table~\ref{tab:user_simdev_jeod_integ} are allowable entries for JEOD
integration.  Each entry must be prefaced with \verb+trick.sim_services+.

\textbf{Example:}
\begin{verbatim}
 dynamics.manager_init.sim_integ_opt = trick.sim_services.Runge_Kutta_4
\end{verbatim}

The four examples above will all produce the same result.


