
\subsection{JEOD Models Used by the \ModelDesc}
The \ModelDesc uses JEOD models to manage memory and to update time.

Several model classes allocate memory and deallocate
memory using the \hypermodelref{MEMORY}.
The IntegratorConstructorFactory creates an instance of a class
that derives from the IntegratorConstructor class.
Each integrator constructor creates instances of state and time integrators.
State integrators allocate memory for internal storage.
The internal storage allocated by the state integrators is freed
when the state integrators are destroyed. The freeing of the instances
of integrator constructors, state integrators, and time integrators
is the responsibility of the objects that created these instances
rather than the \ModelDesc.

The time integrators update the \hypermodelref{TIME}
by invoking the \verb+TimeManager::update+ method with the interpolated
simulation time at the end of each intermediate step.

\subsection{Use of the \ModelDesc in JEOD}
Two JEOD models use the \ModelDesc to integrate vehicle states over time:
\begin{itemize}
\item The \hypermodelref{DYNMANAGER}
uses one of the \ModelDesc integrator constructors to create a time propagator
and to create state integrators for each dynamic body in the simulation.
The time integrator is owned by the dynamics manager while the
state integrators are owned by the dynamic bodies.
In the event that the dynamics manager is lacking a provided integrator 
constructor, the dynamics manager uses the
\ModelDesc's integrator constructor factory to create an integrator constructor
based on the integration option in a Trick integration structure.

Once the simulation is running, the dynamics manager propagates time using
the time integrator and directs each root dynamic body to propagate its state.
As the time and state integrators were created by a single integrator
constructor, this approach ensures consistent state integration
and time propagation across a simulation.
\item The \hypermodelref{DYNBODY}
uses the integrator constructor supplied by the dynamics manager to create rotational and/or translational state propagators. While the simulation is
running, a dynamic body uses its constructed state propagators to propagate
the body's rotational and translational states.
\end{itemize}


\subsection{Interactions with Trick}
The \ModelDesc does not interact directly with Trick. In JEOD, that
is the job of the Dynamics Manager Model. The \ModelDesc does
provide interfaces that support these Trick/JEOD integration interactions.

Trick uses INTEGRATOR structures and outputs from integration
class jobs to drive the integration process. The \verb+first_step_deriv+
flag in the INTEGRATOR structure determines whether the derivative
functions are to be called on the first step of each integration cycle.
Each IntegratorConstructor defines a \verb+first_step_second_derivs_flag+
method to support the setting of this flag. Trick deems the integration
cycle to be complete for a given integrator when the integrator returns
a step number of zero. The Integration Controls supports this interface by 
having the \verb+integrate+ methods return
the step number.  The step number in integration controls is advanced only when 
the integration group (and thereby all of the two-state integrators contained 
therein) and the time integrator indicate that they have met their targets.

\subsection{Interaction Requirements on the Dynamics Manager Model}
The simulation's dynamics manager propagates the contents of all integration 
groups (time and all dynamic bodies within that group) registered with the 
dynamics manager.
Time must be propagated within the integration cycle as some derivative
functions such as non-spherical gravity depend on time.
The \ModelDesc explicitly provides the ability to meet this need.
The Dynamics Manager must use this provided capability to have the
integrated vehicle states best reflect reality.


\subsection{Interaction Requirements on the Dynamic Body Model}
The dynamic bodies in a simulation use the \ModelDesc to propagate states
over time. The \ModelDesc provides a generic capability to integrate generalized
position and velocity. The user of a state integrator must properly tailor
its use of the integrator to the problem at hand. This tailoring includes
initialization and operation. The integrator must be constructed properly
at initialization time and must be fed with properly constructed state
and derivative vectors during operations.
