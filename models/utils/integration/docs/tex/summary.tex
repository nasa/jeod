%%%%%%%%%%%%%%%%%%%%%%%%%%%%%%%%%%%%%%%%%%%%%%%%%%%%%%%%%%%%%%%%%%%%%%%%%%%%%%%%
%
% Purpose: Integration model executive summary
%
%
%%%%%%%%%%%%%%%%%%%%%%%%%%%%%%%%%%%%%%%%%%%%%%%%%%%%%%%%%%%%%%%%%%%%%%%%%%%%%%%%

\chapter*{Executive Summary}

The \ModelDesc forms a component of the utilities suite of
models within \JEODid. It is located at
models/utils/integration.

Propagating the evolution of a vehicle's translational and/or rotational
state over the course of a simulation is an essential part of every
space-based Trick simulation. The underlying equations of motion for this
state propagation yield second order initial value problems.
While analytic solutions do exist for a limited set of such problems,
the complex and unpredictable nature of the forces and torques acting on a
space vehicle precludes the use of analytic methods for a generic solution
to these state propagation problems. Numerical integration techniques
must be used to solve the problem.

Performing this numerical integration is the primary goal of this model.
Several such numerical integration techniques exist. Deciding which technique
is best-suited for the problem at hand is a problem-dependent tradeoff
between accuracy and computational cost. The \ModelDesc thus provides a
variety of numerical integration techniques.

\section*{Purpose}
The \ModelDesc serves primarily as an interface to the standard integration
techniques provided by the \erseven. The \ModelDesc also provides two special
purpose integration techniques -- Gauss-Jackson and the Livermore Solver for
Ordinary differential equations (LSODE). These techniques are intended to
propagate translational state over long time steps as might be
desired for interplanetary vehicles or other situations for which the
environmental forces are well known.

LSODE is an adaptive technique which determines its optimal step size.
Gauss-Jackson is a conventional fixed step size technique which
bootstraps itself from a step consistent with familiar methods such as RK4 to
a far larger step by means of a doubling strategy. Both LSODE and
Gauss-jackson require some level of configuration, however,
default settings are provided which will suffice in most cases.
