\setcounter{chapter}{0}

%----------------------------------
\chapter{Introduction}\hyperdef{part}{intro}{}\label{ch:intro}
%----------------------------------


\section{Model Description}
%%% Incorporate the intro paragraph that used to begin this Chapter here.
%%% This is location of the true introduction where you explain why this 
%%% model exists.
%%% Identify the Model context within JEOD.

This documentation describes the design and testing of matrix and vector
math routines in the JSC Engineering Orbital 
Dynamics (JEOD) \mathDesc.  These routines 
are derived from well-known rules and operators of linear algebra.

Included in this documentation are validation 
cases that describe tests done on the algorithms to verify that 
they are working correctly and computing the correct values for given 
input data.

There is also a User Guide which describes how to incorporate the above 
mentioned routines as part of a Trick simulation.

The parent document to this model document is the
JEOD Top Level Document~\cite{dynenv:JEOD}.

\section{Document History}
%%% Status of this and only this document.  Any date should be relevant to when 
%%% this document was last updated and mention the reason (release, bug fix, etc.)
%%% Mention previous history aka JEOD 1.4-5 heritage in this section.
%%% Mention that JEOD.pdf is the parent document.

\begin{tabular}{||l|l|l|l|} \hline
\DocumentChangeHistory
\end{tabular}

\section{Document Organization}
This document is formatted in accordance with the 
NASA Software Engineering Requirements Standard~\cite{NASA:SWE} 
and is organized into the following chapters:

\begin{description}
%% longer chapter descriptions, more information.

\item[Chapter 1: Introduction] - 
This introduction contains three sections: description of model, document history, and organization.  
The first section provides the introduction to the \mathDesc\ and its reason 
for existence.  It contains a brief description of the interconnections with other models, and 
references to any supporting documents. It also lists the document that is parent to this one.
The second section displays the history of this document which includes
author, date, and reason for each revision.  The final
section contains a description of the how the document is organized.

\item[Chapter 2: Product Requirements] - 
Describes requirements for the \mathDesc.

\item[Chapter 3: Product Specification] - 
Describes the underlying theory, architecture, and design of the \mathDesc\ in detail.  It is organized in 
three sections: Conceptual Design, Mathematical Formulations, and Detailed Design.

\item[Chapter 4: User Guide] - 
Describes how to use the \mathDesc\ in a Trick simulation.  It is broken into three sections to represent the JEOD 
defined user types: Analysts or users of simulations (Analysis), Integrators or developers of simulations (Integration),
and Model Extenders (Extension).

\item[Chapter 5: Verification and Validation] -  
Contains \mathDesc\ verification and validation procedures and results.

\end{description}

%%%%%%%%%%%%%%%%%%%%%%%%%%%%%%%%%%%%%%%%%%%%%%%%%%%%%%%%%%%%%%%%%%%%%%%%%%%%%%%%
%%%%%%%%%%%%%%%%%%%%%%%%%%%%%%%%%%%%%%%%%%%%%%%%%%%%%%%%%%%%%%%%%%%%%%%%%%%%%%%%
%%%%%%%%%%%%%%%%%%%%%%%%%%%%%%%%%%%%%%%%%%%%%%%%%%%%%%%%%%%%%%%%%%%%%%%%%%%%%%%%
%----------------------------------
\chapter{Product Requirements}\hyperdef{part}{reqt}{}\label{ch:reqt}
%----------------------------------
The \mathDesc\ shall meet the JEOD project requirements specified in the 
JEOD Top Level Document~\cite{dynenv:JEOD}.

\requirement{Matrix Math Operations}
\label{reqt:matrix_math_operations}
\begin{description}
\item[Requirement:]\ \newline
The \mathDesc\ shall be capable of performing common matrix operations on 
3x3 matrices. The operations shall include:

   \subrequirement{}
   Matrix Nullification

   \subrequirement{}
   Skew Symmetric Cross Product

   \subrequirement{}
   Matrix Outer Product

   \subrequirement{}
   Matrix Multiplication

   \subrequirement{}
   Matrix Addition

   \subrequirement{}
   Matrix Subtraction

   \subrequirement{}
   Matrix Transposition

   \subrequirement{}
   Matrix Inversion

   \subrequirement{}
   Various combinations of the operations listed above that are commonly
   used in aerospace simulations.

\item[Rationale:]\ \newline
3x3 matrices are very common in aerospace simulations.  There are also many
common required matrix math operations. For ease of programming
and speed of computation, a pre-established ``library" of matrix math functions
is necessary.

\item[Verification:]\ \newline
Test

\end{description}



\requirement{Vector Math Operations}
\label{reqt:vector_math_operations}
\begin{description}
\item[Requirement:]\ \newline
The \mathDesc\ shall be capable of performing common vector operations on 
3x1 (or 1x3) vectors. The operations shall include:

   \subrequirement{}
   Vector Normalization

   \subrequirement{}
   Vector Cross Product

   \subrequirement{}
   Vector Inner Product (Dot Product)

   \subrequirement{}
   Vector Magnitude

   \subrequirement{}
   Vector Addition

   \subrequirement{}
   Vector Subtraction

   \subrequirement{}
   Various combinations of the operations listed above that are commonly
   used in aerospace simulations.

\item[Rationale:]\ \newline
3x1 vectors are very common in aerospace simulations.  There are also many
common required vector math operations. For ease of programming
and speed of computation, a pre-established ``library" of vector math functions
is necessary.

\item[Verification:]\ \newline
Test

\end{description}

%%% Format for the model Requirements is open.  It should include requirements for this model 
%%% only and use requirment tags like the one below.
%\requirement{...}
%\label{reqt:...}
%\begin{description}
%  \item[...]\ \newline
%    The documentation for the model shall include
%
%    \subrequirement{}
%    \label{reqt:...}
%      Software requirements specification.
%      
%    ...
%   
%  \item[title]\ \newline
%    text
%
%  ...
%
%\end{description}


%%%%%%%%%%%%%%%%%%%%%%%%%%%%%%%%%%%%%%%%%%%%%%%%%%%%%%%%%%%%%%%%%%%%%%%%%%%%%%%%
%%%%%%%%%%%%%%%%%%%%%%%%%%%%%%%%%%%%%%%%%%%%%%%%%%%%%%%%%%%%%%%%%%%%%%%%%%%%%%%%
%%%%%%%%%%%%%%%%%%%%%%%%%%%%%%%%%%%%%%%%%%%%%%%%%%%%%%%%%%%%%%%%%%%%%%%%%%%%%%%%
%----------------------------------
\chapter{Product Specification}\hyperdef{part}{spec}{}\label{ch:spec}
%----------------------------------

\section{Conceptual Design}
The \mathDesc\ is a collection of functions that peform common matrix and vector
operations for aerospace simulations.  The functions are based on well-known
rules and operations of linear algebra.

%%%%%%%%%%%%%%%%%%%%%%%%%%%%%%%%%%%%%%%%%%%%%%%%%%%%%%%%%%%%%%%%%%%%%%%%%%%%%%%%
\section{Mathematical Formulations}
The mathematical operations (or functions) used in the \mathDesc\ are based 
on common matrix and vector operations.  The rules for these operations can 
be found in a variety of math texts (e.g., Kreyszig ~\cite{kreyszig1997}).

%%%%%%%%%%%%%%%%%%%%%%%%%%%%%%%%%%%%%%%%%%%%%%%%%%%%%%%%%%%%%%%%%%%%%%%%%%%%%%%%
\section{Detailed Design}
The \mathDesc\ functions are designed to be called by other JEOD functions.  They
are not intended to be called from the S\_define level.


%%%%%%%%%%%%%%%%%%%%%%%%%%%%%%%%%%%%%%%%%%%%%%%%%%%%%%%%%%%%%%%%%%%%%%%%%%%%%%%%
%%%%%%%%%%%%%%%%%%%%%%%%%%%%%%%%%%%%%%%%%%%%%%%%%%%%%%%%%%%%%%%%%%%%%%%%%%%%%%%%
%%%%%%%%%%%%%%%%%%%%%%%%%%%%%%%%%%%%%%%%%%%%%%%%%%%%%%%%%%%%%%%%%%%%%%%%%%%%%%%%
%----------------------------------
\chapter{User Guide}\hyperdef{part}{user}{}\label{ch:user}
%----------------------------------
Three types of JEOD users are described in the model documents.  However, the only
type of user who can manipulate the \mathDesc\ are model Extenders.  Therefore,
Analysts and Integrators are not addressed in this particular user guide.

%%%%%%%%%%%%%%%%%%%%%%%%%%%%%%%%%%%%%%%%%%%%%%%%%%%%%%%%%%%%%%%%%%%%%%%%%%%%%%%%
\section{Analysis}
(The \mathDesc\ functions are not intended for use at the simulation analyst/user level.
Therefore, this section is not applicable.)

%%%%%%%%%%%%%%%%%%%%%%%%%%%%%%%%%%%%%%%%%%%%%%%%%%%%%%%%%%%%%%%%%%%%%%%%%%%%%%%%
\section{Integration}
(The \mathDesc\ functions are not intended for use at the simulation integrator level.
Therefore, this section is not applicable.)

%%%%%%%%%%%%%%%%%%%%%%%%%%%%%%%%%%%%%%%%%%%%%%%%%%%%%%%%%%%%%%%%%%%%%%%%%%%%%%%%
\section{Extension}

\subsection{Matrix Functions}
In order to use the \mathDesc\ matrix functions, the following line must be included in the
file of the calling function:
\begin{verbatim}
#include "utils/math/include/matrix3x3.hh"
\end{verbatim}

An example call to the matrix function {\it copy} is:
\begin{verbatim}
Matrix3x3::copy(subject_body->composite_properties.inertia, inertia);
\end{verbatim}

The matrix functions available in the \mathDesc\ are listed below.  The equivalent math
expression is shown along with a brief explanation, if necessary. Overloading has been used
to repeat some function names. Note that all matrices are assumed to be 3x3. \newline\newline



\begin{description}
  \item[initialize, $A \leftarrow null$ (all elements of 3x3 $null$ matrix are zero)]\ \newline

  \item[identity, $A\leftarrow I$ ($I$ is the 3x3 identity matrix)]\ \newline

  \item[cross\_matrix, $A\leftarrow \tilde{v} $]\ (returns skew symmetric cross product matrix from input vector $\bar{f}$)\newline
    \begin{equation}\nonumber
    A = \left[
    \begin{array}{rrr}
     0   & -f_3 &  f_2 \\
     f_3 &  0   & -f_1 \\
    -f_2 &  f_1 &  0   \\
    \end{array}\right]
    \end{equation}

  \item[outer\_product, $A\leftarrow \bar{f}\otimes\bar{g}$]\ (returns outer product of two input vectors $\bar{f}$ and$\bar{g}$)\newline
    \begin{equation}\nonumber
    A = \left[
    \begin{array}{rrr}
     f_1g_1 & f_1g_2 & f_1g_3 \\
     f_2g_1 & f_2g_2 & f_2g_3 \\
     f_3g_1 & f_3g_2 & f_3g_3 \\
    \end{array}\right]
    \end{equation}

  \item[negate (in place), $A\leftarrow(-A)$]\ \newline

  \item[negate, $A\leftarrow (-B)$]\ \newline

  \item[transpose (in place), $A\leftarrow A^T$]\ \newline

  \item[transpose, $A\leftarrow B^T$]\ \newline

  \item[scale (in place), $A\leftarrow sA$ ($s$ is a scalar)]\ \newline

  \item[scale, $A\leftarrow sB$ ($s$ is a scalar)]\ \newline

  \item[incr (in place), $A\leftarrow A + C$ ($C$ is a 3x3 matrix of constants)]\ \newline

  \item[decr (in place), $A\leftarrow A - C$ ($C$ is a 3x3 matrix of constants)]\ \newline

  \item[copy, $A \leftarrow B$]\ \newline

  \item[add, $A \leftarrow B+C$]\ \newline

  \item[subtract, $A \leftarrow B-C$]\ \newline

  \item[product, $A \leftarrow BC$]\ \newline

  \item[product\_left\_transpose, $A \leftarrow B^TC$]\ \newline

  \item[product\_right\_transpose, $A \leftarrow BC^T$]\ \newline

  \item[product\_transpose\_transpose, $A \leftarrow B^T C^T$]\ \newline

  \item[transform\_matrix, $A \leftarrow BCB^T$]\ (i.e., {\it similarity transformation})\newline

  \item[transpose\_transform\_matrix, $A \leftarrow B^T CB$]\ \newline

  \item[invert, $A \leftarrow B^{-1}$]\ \newline

  \item[invert\_symmetric, $A \leftarrow B^{-1}$ ($B$ is symmetric)]\ \newline

  \item[print]\ (Prints matrix to standard output.)\newline

\end{description}

\newpage

%%%%%%%%%%%%%%%%%%%%%%%%%%%%%%%%%%%%%%%%
\subsection{Vector Functions}
In order to use the \mathDesc\ vector functions, the following line must be included in the
file of the calling function:
\begin{verbatim}
#include "utils/math/include/vector3.hh"
\end{verbatim}

An example call to the vector function {\it transform\_transpose} is:
\begin{verbatim}
Vector3::transform_transpose (subject_body->composite_body.state.rot.T_parent_this,torque);
\end{verbatim}

The vector functions available in the \mathDesc\ are listed below.  The equivalent math
expression is shown along with a brief explanation, if necessary. Overloading has been used
to repeat some function names. Note that all vectors are assumed to be of dimension three. \newline\newline


\begin{description}
  \item[initialize, $\bar{A}\leftarrow null$ (all elements of $null$ vector are zero)]\ \newline

  \item[unit, $\bar{A}\leftarrow \hat{B}$]\ (returns 1.0 for only one user selected vector element, the
   other two elements are returned equal to 0.0) \newline

  \item[fill]\ (returns matrix with each element equal to user defined input scalar) \newline

  \item[zero\_small]\ (return 0.0 for each element that has an absolute value less than a user specified limit)\newline

  \item[copy, $\bar{A} \leftarrow \bar{B}$]\ \newline

  \item[dot, $A \leftarrow B \cdot C$]\ (vector dot product) \newline

  \item[vmagsq, $A \leftarrow |B|^2$]\ (returns a scalar)\newline

  \item[vmag, $A \leftarrow |B|$]\ (returns a scalar)\newline

  \item[normalize (in place), $\hat{A} \leftarrow \bar{A}/|\bar{A}|$]\ \newline

  \item[normalize, $\hat{A} \leftarrow \bar{B}/|\bar{B}|$]\ \newline

  \item[scale (in place), $\bar{A} \leftarrow b\bar{A}$]\ \newline

  \item[scale, $\bar{A} \leftarrow s\bar{B}$]\ \newline

  \item[negate (in place), $\bar{A} \leftarrow -\bar{A}$]\ \newline

  \item[negate, $\bar{A} \leftarrow  -\bar{B}$]\ \newline

  \item[transform (in place), $\bar{A} \leftarrow T\bar{A}$ ($T$ is 3x3 transformation matrix)]\ \newline

  \item[transform, $\bar{A} \leftarrow T\bar{B}$ ($T$ is 3x3 transformation matrix)]\ \newline

  \item[transform\_transpose (in place),  $\bar{A} \leftarrow T^T\bar{A}$ ($T$ is 3x3 transformation matrix)]\ \newline

  \item[transform\_transpose,  $\bar{A} \leftarrow T^T\bar{B}$ ($T$ is 3x3 transformation matrix)]\ \newline

  \item[incr, $\bar{A} \leftarrow \bar{A}+\bar{C}$]\ \newline

  \item[incr, $\bar{A} \leftarrow \bar{A}+\bar{C_1}+\bar{C_2}$]\ \newline

  \item[decr, $\bar{A} \leftarrow \bar{A}-\bar{C}$]\ \newline

  \item[decr, $\bar{A} \leftarrow \bar{A}-\bar{C}+\bar{D}$]\ \newline

  \item[sum, $\bar{A} \leftarrow \bar{B}+\bar{C}$]\ \newline

  \item[sum, $\bar{A} \leftarrow \bar{B}+\bar{C}+\bar{D}$]\ \newline

  \item[diff, $\bar{A} \leftarrow \bar{B}-\bar{C}$]\ \newline

  \item[cross, $\bar{A} \leftarrow \bar{B}\times\bar{C}$]\ \newline

  \item[scale\_incr, $\bar{A} \leftarrow \bar{A}+s\bar{B}$]\ \newline

  \item[scale\_decr, $\bar{A} \leftarrow \bar{A}-s\bar{B}$]\ \newline

  \item[cross\_incr, $\bar{A} \leftarrow \bar{A} + \bar{B}\times\bar{C}$]\ \newline

  \item[cross\_decr, $\bar{A} \leftarrow \bar{A} - \bar{B}\times\bar{C}$]\ \newline

  \item[transform\_incr, $A\leftarrow\bar{A}+T\bar{B}$ ($T$ is 3x3 transformation matrix)]\ \newline

  \item[transform\_decr, $A\leftarrow\bar{A}-T\bar{B}$ ($T$ is 3x3 transformation matrix)]\ \newline

  \item[transform\_transpose\_incr, $\bar{A}\leftarrow\bar{A}+T^T\bar{B}$ ($T$ is 3x3 transformation matrix)]\ \newline

  \item[transform\_transpose\_decr, $\bar{A}\leftarrow\bar{A}-T^T\bar{B}$ ($T$ is 3x3 transformation matrix)]\ \newline

\end{description}


\newpage

%%%%%%%%%%%%%%%%%%%%%%%%%%%%%%%%%%%%%%%%%%%%%%%%%%%%%%%%%%%%%%%%%%%%%%%%%%%%%%%%
%%%%%%%%%%%%%%%%%%%%%%%%%%%%%%%%%%%%%%%%%%%%%%%%%%%%%%%%%%%%%%%%%%%%%%%%%%%%%%%%
%%%%%%%%%%%%%%%%%%%%%%%%%%%%%%%%%%%%%%%%%%%%%%%%%%%%%%%%%%%%%%%%%%%%%%%%%%%%%%%%
%----------------------------------
\chapter{Verification and Validation}\hyperdef{part}{ivv}{}\label{ch:ivv}
%----------------------------------

\section{Verification}\label{sec:verif}
(No verification tests were performed on the \mathDesc\ functions.)

%%% code imported from old template structure
%\inspection{<Name of Inspection>}\label{inspect:<label>}
% <description> to satisfy  
% requirement \ref{reqt:<label>}.




%made up point mass model for planet and spacecraft. Show the I matrix.  Show the list of 
%test point locations, the torques and the test results.

%show the equation for point mass toque.  


%%%%%%%%%%%%%%%%%%%%%%%%%%%%%%%%%%%%%%%%%%%%%%%%%%%%%%%%%%%%%%%%%%%%%%%%%%%%%%%%
\section{Validation}\label{sec:valid}
%%% code imported from old template structure
%\test{<Title>}\label{test:<label>}
%\begin{description}
%\item[Purpose:] \ \newline
%<description>
%\item[Requirements:] \ \newline
%By passing this test, the universal time module 
%partially satisfies requirement~\ref{reqt:<label1>} and 
%completely satisfies requirement~\ref{reqt:<label2>}.
%\item[Procedure:]\ \newline
%<procedure>
%\item[Results:]\ \newline
%<results>
%\end{description}


%%%%%%%%%%%%%%%%%%%%%%%%%%%%%%%%%%%%%%%%%%%%%%%%%%%%%%%%%%%%%%%%%%%%%%%%%%%%%%%%
\test{Matrix Math}\label{test:matrix_math}

To validate the \mathDesc\ matrix functions, 3x3 test matrices were
built and sent to the various functions. The results were compared to
the same operation(s) performed in Matlab.  

The test matrices used were:
\begin{equation}\nonumber
A = \left[
\begin{array}{rrr}
 7 &  8 &  2 \\
-1 &  5 & -3 \\
 6 &  4 & -7 \\
\end{array}\right]
\end{equation}

\begin{equation}\nonumber
B = \left[
\begin{array}{rrr}
 5 & -3 &  6 \\
-3 &  7 &  4 \\
 6 &  4 & -8 \\
\end{array}\right]
\end{equation}

\begin{equation}\nonumber
C = \left[
\begin{array}{rrr}
 1 & 2 & 3 \\
 4 & 5 & 6 \\
 7 & 8 & 9 \\
\end{array}\right]
\end{equation}

\begin{equation}\nonumber
f = \left[
\begin{array}{r}
 2 \\
-3 \\
 8 \\
\end{array}\right]
\end{equation}

\begin{equation}\nonumber
g = \left[
\begin{array}{r}
  7 \\
  2 \\
 -1 \\
\end{array}\right]
\end{equation}

\begin{equation}\nonumber
s = 3
\end{equation}



{\bf Results}
\begin{description}
  \item[initialize, $A \leftarrow null$]\ \newline
   \begin{equation}\nonumber
   A = \left[
   \begin{array}{rrr}
    0 & 0 & 0 \\
    0 & 0 & 0 \\
    0 & 0 & 0 \\
   \end{array}\right]
   \end{equation}

  \item[identity, $A\leftarrow I$]\ \newline
   \begin{equation}\nonumber
   A = \left[
   \begin{array}{rrr}
    1 & 0 & 0 \\
    0 & 1 & 0 \\
    0 & 0 & 1 \\
   \end{array}\right]
   \end{equation}

  \item[cross\_matrix, $A\leftarrow \tilde{f} $]\ \newline
    \begin{equation}\nonumber
    A = \left[
    \begin{array}{rrr}
     0   & -8 & -3 \\
     8   &  0 & -2 \\
     3   &  2 &  0 \\
    \end{array}\right]
    \end{equation}

  \item[outer\_product, $A\leftarrow \bar{f}\otimes\bar{g}$]\ \newline
    \begin{equation}\nonumber
    A = \left[
    \begin{array}{rrr}
     14 &  4  & -2 \\
    -21 & -6  &  3 \\
     56 &  16 & -8 \\
    \end{array}\right]
    \end{equation}

  \item[negate (in place), $A\leftarrow(-A)$]\ \newline
    \begin{equation}\nonumber
    A = \left[
    \begin{array}{rrr}
    -7 & -8  & -2 \\
     1 & -5  &  3 \\
    -6 & -4  &  7 \\
    \end{array}\right]
    \end{equation}

  \item[negate, $A\leftarrow (-B)$]\ \newline
    \begin{equation}\nonumber
    A = \left[
    \begin{array}{rrr}
    -5 &  3  & -6 \\
     3 & -7  & -4 \\
    -6 & -4  &  8 \\
    \end{array}\right]
    \end{equation}

  \item[transpose (in place), $A\leftarrow A^T$]\ \newline
    \begin{equation}\nonumber
    A = \left[
    \begin{array}{rrr}
     7 & -1  &  6 \\
     8 &  5  &  4 \\
     2 & -3  & -7 \\
    \end{array}\right]
    \end{equation}

  \item[transpose, $A\leftarrow B^T$]\ \newline
    \begin{equation}\nonumber
    A = \left[
    \begin{array}{rrr}
     5 & -3  &  6 \\
    -3 &  7  &  4 \\
     6 &  4  & -8 \\
    \end{array}\right]
    \end{equation}

  \item[scale (in place), $A\leftarrow sA$ ($s$ is a scalar)]\ \newline
    \begin{equation}\nonumber
    A = \left[
    \begin{array}{rrr}
     21 &  24 &  6  \\
    -3  &  15 & -9  \\
     18 &  12 & -21 \\
    \end{array}\right]
    \end{equation}

  \item[scale, $A\leftarrow sB$ ($s$ is a scalar)]\ \newline
    \begin{equation}\nonumber
    A = \left[
    \begin{array}{rrr}
     15 &  -9 &  18 \\
    -9  &  21 &  12 \\
     18 &  12 & -24 \\
    \end{array}\right]
    \end{equation}

  \item[incr (in place), $A\leftarrow A + C$]\ \newline
    \begin{equation}\nonumber
    A = \left[
    \begin{array}{rrr}
     8  &  10 &  5 \\
     3  &  10 &  3 \\
     13 &  12 &  2 \\
    \end{array}\right]
    \end{equation}

  \item[decr (in place), $A\leftarrow A - C$]\ \newline
    \begin{equation}\nonumber
    A = \left[
    \begin{array}{rrr}
     6  &   6  &   -1 \\
    -5  &   0  &   -9 \\
    -1  &  -4  &  -16 \\
    \end{array}\right]
    \end{equation}

  \item[copy, $A \leftarrow B$]\ \newline
    \begin{equation}\nonumber
    A = \left[
    \begin{array}{rrr}
     5  &  -3  &  6 \\
    -3  &   7  &  4 \\
     6  &   4  & -8 \\
    \end{array}\right]
    \end{equation}

  \item[add, $A \leftarrow B+C$]\ \newline
    \begin{equation}\nonumber
    A = \left[
    \begin{array}{rrr}
     6  &  -1  &  9 \\
     1  &  12  & 10 \\
    13  &  12  &  1 \\
    \end{array}\right]
    \end{equation}

  \item[subtract, $A \leftarrow B-C$]\ \newline
    \begin{equation}\nonumber
    A = \left[
    \begin{array}{rrr}
     4  &  -5  &   3 \\
    -7  &   2  &  -2 \\
    -1  &  -4  & -17 \\
    \end{array}\right]
    \end{equation}

  \item[product, $A \leftarrow BC$]\ \newline
    \begin{equation}\nonumber
    A = \left[
    \begin{array}{rrr}
     35  &  43  &  51 \\
     53  &  61  &  69 \\
    -34  & -32  & -30 \\
    \end{array}\right]
    \end{equation}

  \item[product\_left\_transpose, $A \leftarrow B^TC$]\ \newline
    \begin{equation}\nonumber
    A = \left[
    \begin{array}{rrr}
     35  &  43  &  51 \\
     53  &  61  &  69 \\
    -34  & -32  & -30 \\
    \end{array}\right]
    \end{equation}

  \item[product\_right\_transpose, $A \leftarrow BC^T$]\ \newline
    \begin{equation}\nonumber
    A = \left[
    \begin{array}{rrr}
     17  &  41  &  65 \\
     23  &  47  &  71 \\
    -10  &  -4  &   2 \\
    \end{array}\right]
    \end{equation}

  \item[product\_transpose\_transpose, $A \leftarrow B^T C^T$]\ \newline
    \begin{equation}\nonumber
    A = \left[
    \begin{array}{rrr}
     17  &  41  &  65 \\
     23  &  47  &  71 \\
    -10  &  -4  &   2 \\
    \end{array}\right]
    \end{equation}

  \item[transform\_matrix, $A \leftarrow BCB^T$]\ \newline
    \begin{equation}\nonumber
    A = \left[
    \begin{array}{rrr}
     352  &  400  &  -26 \\
     496  &  544  &   10 \\
    -254  & -242  &  -92 \\
    \end{array}\right]
    \end{equation}

  \item[transpose\_transform\_matrix, $A \leftarrow B^T CB$]\ \newline
    \begin{equation}\nonumber
    A = \left[
    \begin{array}{rrr}
     352  &  400  &  -26 \\
     496  &  544  &   10 \\
    -254  & -242  &  -92 \\
    \end{array}\right]
    \end{equation}

  \item[invert, $A \leftarrow B^{-1}$]\ \newline
    \begin{equation}\nonumber
    A = \left[
    \begin{array}{rrr}
     0.1053  &  0.0000  &  0.0789 \\
     0.0000  &  0.1111  &  0.0556 \\
     0.0789  &  0.0556  & -0.0380 \\
    \end{array}\right]
    \end{equation}

  \item[invert\_symmetric, $A \leftarrow B^{-1}$ ($B$ is symmetric)]\ \newline
    \begin{equation}\nonumber
    A = \left[
    \begin{array}{rrr}
     0.1053  &  0.0000  &  0.0789 \\
     0.0000  &  0.1111  &  0.0556 \\
     0.0789  &  0.0556  & -0.0380 \\
    \end{array}\right]
    \end{equation}

\end{description}


%%%%%%%%%%%%%%%%%%%%%%%%%%%%%%%%%%%%%%%%
\test{Vector Math}\label{test:vector_math}
To validate the \mathDesc\ vector functions, 3x1 test vectors were
built and sent to the various functions. The results were compared to
the same operation(s) performed in Matlab.  

The test vectors used were:
\begin{equation}\nonumber
\bar{A} = \left[
\begin{array}{r}
  2 \\
 -3 \\
  8 \\
\end{array}\right]
\end{equation}

\begin{equation}\nonumber
\bar{B} = \left[
\begin{array}{r}
  7 \\
  2 \\
 -1 \\
\end{array}\right]
\end{equation}

\begin{equation}\nonumber
\bar{C} = \left[
\begin{array}{r}
  5 \\
 -5 \\
  7 \\
\end{array}\right]
\end{equation}

\begin{equation}\nonumber
\bar{D} = \left[
\begin{array}{r}
  4 \\
  3 \\
 -1 \\
\end{array}\right]
\end{equation}

\begin{equation}\nonumber
T = \left[
\begin{array}{rrr}
  0.819152 & 0.573576 & 0.000000 \\
 -0.573576 & 0.819152 & 0.000000 \\
  0.000000 & 0.000000 & 1.000000 \\
\end{array}\right]
\end{equation}

\begin{equation}\nonumber
s =3
\end{equation}


{\bf Results}

\begin{description}
  \item[initialize, $\bar{A}\leftarrow null$]\ \newline
   \begin{equation}\nonumber
   \bar{A} = \left[
   \begin{array}{r}
     0 \\
     0 \\
     0 \\
   \end{array}\right]
   \end{equation}

  \item[unit, $\bar{A}\leftarrow i$]\ \newline
   \begin{equation}\nonumber
   \bar{A} = \left[
   \begin{array}{r}
     1 \\
     0 \\
     0 \\
   \end{array}\right]_{i=1}
   \end{equation}

  \item[fill]\ \newline
   \begin{equation}\nonumber
   \bar{A} = \left[
   \begin{array}{r}
     3 \\
     3 \\
     3 \\
   \end{array}\right]_{3}
   \end{equation}

  \item[zero\_small]\ \newline
   \begin{equation}\nonumber
   \bar{A} = \left[
   \begin{array}{r}
     0 \\
     0 \\
     0 \\
   \end{array}\right]
   \end{equation}

  \item[copy, $\bar{A} \leftarrow \bar{B}$]\ \newline
   \begin{equation}\nonumber
   \bar{A} = \left[
   \begin{array}{r}
     7 \\
     2 \\
    -1 \\
   \end{array}\right]
   \end{equation}

  \item[dot, $A \leftarrow B \cdot C$]\ (vector dot product) \newline
   \begin{equation}\nonumber
   A = 18
   \end{equation}

  \item[vmagsq, $A \leftarrow |B|^2$]\ (returns a scalar)\newline
   \begin{equation}\nonumber
   A = 54
   \end{equation}

  \item[vmag, $A \leftarrow |B|$]\ (returns a scalar)\newline
   \begin{equation}\nonumber
   A = 7.3485
   \end{equation}

  \item[normalize (in place), $\hat{A} \leftarrow \bar{A}/|\bar{A}|$]\ \newline
   \begin{equation}\nonumber
   \bar{A} = \left[
   \begin{array}{r}
     0.2279 \\
    -0.3419 \\
     0.9117 \\
   \end{array}\right]
   \end{equation}

  \item[normalize, $\hat{A} \leftarrow \bar{B}/|\bar{B}|$]\ \newline
   \begin{equation}\nonumber
   \bar{A} = \left[
   \begin{array}{r}
     0.9526 \\
     0.2722 \\
    -0.1361 \\
   \end{array}\right]
   \end{equation}

  \item[scale (in place), $\bar{A} \leftarrow s\bar{A}$]\ \newline
   \begin{equation}\nonumber
   \bar{A} = \left[
   \begin{array}{r}
      6  \\
     -9  \\
      24 \\
   \end{array}\right]
   \end{equation}

  \item[scale, $\bar{A} \leftarrow s\bar{B}$]\ \newline
   \begin{equation}\nonumber
   \bar{A} = \left[
   \begin{array}{r}
      21  \\
       6  \\
      -3  \\
   \end{array}\right]
   \end{equation}

  \item[negate (in place), $\bar{A} \leftarrow -\bar{A}$]\ \newline
   \begin{equation}\nonumber
   \bar{A} = \left[
   \begin{array}{r}
      -2  \\
       3  \\
      -8  \\
   \end{array}\right]
   \end{equation}

  \item[negate, $\bar{A} \leftarrow  -\bar{B}$]\ \newline
   \begin{equation}\nonumber
   \bar{A} = \left[
   \begin{array}{r}
      -7  \\
      -2  \\
       1  \\
   \end{array}\right]
   \end{equation}

  \item[transform (in place), $\bar{A} \leftarrow T\bar{A}$]\ \newline
   \begin{equation}\nonumber
   \bar{A} = \left[
   \begin{array}{r}
      -0.0824  \\
      -3.6046  \\
       8.0000  \\
   \end{array}\right]
   \end{equation}

  \item[transform, $\bar{A} \leftarrow T\bar{B}$]\ \newline
   \begin{equation}\nonumber
   \bar{A} = \left[
   \begin{array}{r}
       6.8812  \\
      -2.3767  \\
      -1.0000  \\
   \end{array}\right]
   \end{equation}

  \item[transform\_transpose (in place),  $\bar{A} \leftarrow T^T\bar{A}$]\ \newline
   \begin{equation}\nonumber
   \bar{A} = \left[
   \begin{array}{r}
       3.3590  \\
      -1.3102  \\
       8.0000  \\
   \end{array}\right]
   \end{equation}

  \item[transform\_transpose,  $\bar{A} \leftarrow T^T\bar{B}$]\ \newline
   \begin{equation}\nonumber
   \bar{A} = \left[
   \begin{array}{r}
       4.5869  \\
       5.6533  \\
      -1.0000  \\
   \end{array}\right]
   \end{equation}

  \item[incr, $\bar{A} \leftarrow \bar{A}+\bar{C}$]\ \newline
   \begin{equation}\nonumber
   \bar{A} = \left[
   \begin{array}{r}
       7  \\
      -8  \\
       15 \\
   \end{array}\right]
   \end{equation}

  \item[incr, $\bar{A} \leftarrow \bar{A}+\bar{C}+\bar{D}$]\ \newline
   \begin{equation}\nonumber
   \bar{A} = \left[
   \begin{array}{r}
      11  \\
      -5  \\
      14 \\
   \end{array}\right]
   \end{equation}

  \item[decr, $\bar{A} \leftarrow \bar{A}-\bar{C}$]\ \newline
   \begin{equation}\nonumber
   \bar{A} = \left[
   \begin{array}{r}
      -3  \\
       2  \\
       1 \\
   \end{array}\right]
   \end{equation}

  \item[decr, $\bar{A} \leftarrow \bar{A}-\bar{C}-\bar{D}$]\ \newline
   \begin{equation}\nonumber
   \bar{A} = \left[
   \begin{array}{r}
      -7  \\
      -1  \\
       2  \\
   \end{array}\right]
   \end{equation}

  \item[sum, $\bar{A} \leftarrow \bar{B}+\bar{C}$]\ \newline
   \begin{equation}\nonumber
   \bar{A} = \left[
   \begin{array}{r}
      12  \\
      -3  \\
       6 \\
   \end{array}\right]
   \end{equation}

  \item[sum, $\bar{A} \leftarrow \bar{B}+\bar{C}+\bar{D}$]\ \newline
   \begin{equation}\nonumber
   \bar{A} = \left[
   \begin{array}{r}
      16  \\
       0  \\
       5 \\
   \end{array}\right]
   \end{equation}

  \item[diff, $\bar{A} \leftarrow \bar{B}-\bar{C}$]\ \newline
   \begin{equation}\nonumber
   \bar{A} = \left[
   \begin{array}{r}
       2  \\
       7  \\
      -8 \\
   \end{array}\right]
   \end{equation}

  \item[cross, $\bar{A} \leftarrow \bar{B}\times\bar{C}$]\ \newline
   \begin{equation}\nonumber
   \bar{A} = \left[
   \begin{array}{r}
       9  \\
     -54  \\
     -45 \\
   \end{array}\right]
   \end{equation}

  \item[scale\_incr, $\bar{A} \leftarrow \bar{A}+s\bar{B}$]\ \newline
   \begin{equation}\nonumber
   \bar{A} = \left[
   \begin{array}{r}
      23  \\
       3  \\
       5  \\
   \end{array}\right]
   \end{equation}

  \item[scale\_decr, $\bar{A} \leftarrow \bar{A}-s\bar{B}$]\ \newline
   \begin{equation}\nonumber
   \bar{A} = \left[
   \begin{array}{r}
     -19  \\
      -9  \\
      11  \\
   \end{array}\right]
   \end{equation}

  \item[cross\_incr, $\bar{A} \leftarrow \bar{A} + \bar{B}\times\bar{C}$]\ \newline
   \begin{equation}\nonumber
   \bar{A} = \left[
   \begin{array}{r}
      11  \\
     -57  \\
     -37  \\
   \end{array}\right]
   \end{equation}

  \item[cross\_decr, $\bar{A} \leftarrow \bar{A} - \bar{B}\times\bar{C}$]\ \newline
   \begin{equation}\nonumber
   \bar{A} = \left[
   \begin{array}{r}
      -7  \\
      51  \\
      53  \\
   \end{array}\right]
   \end{equation}

  \item[transform\_incr, $A\leftarrow\bar{A}+T\bar{B}$]\ \newline
   \begin{equation}\nonumber
   \bar{A} = \left[
   \begin{array}{r}
      8.8812 \\
     -5.3767 \\
      7.0000 \\
   \end{array}\right]
   \end{equation}

  \item[transform\_decr, $A\leftarrow\bar{A}-T\bar{B}$]\ \newline
   \begin{equation}\nonumber
   \bar{A} = \left[
   \begin{array}{r}
     -4.8812 \\
     -0.6233 \\
      9.0000 \\
   \end{array}\right]
   \end{equation}

  \item[transform\_transpose\_incr, $\bar{A}\leftarrow\bar{A}+T^T\bar{B}$]\ \newline
   \begin{equation}\nonumber
   \bar{A} = \left[
   \begin{array}{r}
      6.5869 \\
      2.6533 \\
      7.0000 \\
   \end{array}\right]
   \end{equation}

  \item[transform\_transpose\_decr, $\bar{A}\leftarrow\bar{A}-T^T\bar{B}$]\ \newline
   \begin{equation}\nonumber
   \bar{A} = \left[
   \begin{array}{r}
     -2.5869 \\
     -8.6533 \\
      9.0000 \\
   \end{array}\right]
   \end{equation}

\end{description}


%%%%%%%%%%%%%%%%%%%%%%%%%%%%%%%%%%%%%%%%%%%%%%%%%%%%%%%%%%%%%%%%%%%%%%%%%%%%%%%%


\section{Requirements Traceability}\label{sec:traceability}
The table below cross-references each requirement of the 
\mathDesc\ 
%(as defined in \href{file:GRAVITYReqt.pdf}
%{\em \GRAVITY\ Product Requirements}\cite{dynenv:GRAVITYReqt})
to a corresponding verification and/or validation test
described in sections \ref{sec:verif} and \ref{sec:valid} of this document. 

\label{tab:reqt_ivv_xref}
\begin{longtable}[c]{||p{3.5in}|p{3.5in}|}
\caption{Requirements Traceability} \\[6pt]
\hline
{\bf Requirement} & {\bf Inspection or Test} \\ \hline \hline
\endfirsthead
\hline
\endfoot
\caption[]{Requirements Traceability (continued)} \\[6pt]
\hline
{\bf Requirement} & {\bf Test} \\ \hline \hline
\endhead

\ref{reqt:matrix_math_operations} - Matrix Math Operations &
  Test \ref{test:matrix_math} - Matrix Math \\ \hline

\ref{reqt:vector_math_operations} - Vector Math Operations &
  Test \ref{test:vector_math} - Vector Math \\ \hline

\end{longtable}








