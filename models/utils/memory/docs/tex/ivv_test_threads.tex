\test{Threads}\label{test:threads}
\begin{description}
\item[Test Directory]
{\tt models/utils/memory/verif/unit\_tests/threads}

\item[Test Description] This unit test exercises the
\ModelDesc in a multithreaded environment. The program spawns twelve threads.
Each thread operates at a specific rate and
allocates and deletes multiple chunks memory of a specific type.
The collisions that would result from the memory operations performed by
these threads would result in the model becoming corrupted were it not
for the thread-safe features of the model.

To run the test, enter the test directory and type the command {\tt make}.
The test driver exercises the test program four times:
\begin{itemize}
\item Test \#1 tests nominal behavior. Each thread allocates and deallocates
memory. Eventually all memory allocated by a thread is deallocated. When all
threads have finished there should be no extant allocated memory.

\item Test \#2 tests the ability to detect simple fence post errors, the
most common type of memory error. Each thread writes just outside of one
piece of allocated memory. The \ModelDesc should detect and report these
as corrupted memory.

\item Test \#3 tests the ability to detect leaks. Each thread fails to release
one piece of allocated memory. The \ModelDesc should detect these as leaks.

\item Test \#4 combines tests \#2 and \#3. The \ModelDesc should detect and
report both the corrupted memory errors and the leaks.
\end{itemize}

\item[Success Criteria]
Test criteria are embedded in the test program. The output for each test
must indicate that the test passes.

\item[Test Results] Each of the four test passes its test criteria.

\item[Applicable Requirements]
This test demonstrates the partial or complete satisfaction of the 
following requirements:
\begin{itemize}
\item \ref{reqt:alloc}. Memory is allocated and initialized as expected.
\item \ref{reqt:free}. Memory is destructed and freed as expected.
\item \ref{reqt:base_class_pointer_free}. Freeing from a base class pointer
  works as expected.
\item \ref{reqt:threads}. The model performs in a thread-safe manner.
\item \ref{reqt:tracking}. Memory allocations are tracked as evidenced
  by properly reporting leaks.
\item \ref{reqt:reporting}. Leaks and corrupted memory are reported
  upon request.
\end{itemize}

\end{description}
