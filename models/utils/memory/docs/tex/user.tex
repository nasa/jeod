%%%%%%%%%%%%%%%%%%%%%%%%%%%%%%%%%%%%%%%%%%%%%%%%%%%%%%%%%%%%%%%%%%%%%%%%%%%%%%%%
% user.tex
% User guide for the Memory Model
%
%
%%%%%%%%%%%%%%%%%%%%%%%%%%%%%%%%%%%%%%%%%%%%%%%%%%%%%%%%%%%%%%%%%%%%%%%%%%%%%%%%

%----------------------------------
\chapter{User Guide}\hyperdef{part}{user}{}
\label{ch:overview:user}
%----------------------------------

This chapter describes of the use of the \ModelDesc.

\section{Analysis}
The \ModelDesc is not an analytic tool. It can however be used to identify
memory usage problems by defining {\tt JEOD\_MEMORY\_DEBUG} and
{\tt JEOD\_MEMORY\_GUARD} in the {\tt TRICK]\_CFLAGS} environment variable.
The following settings are available:
\begin{itemize}
\item {\tt -DJEOD\_MEMORY\_DEBUG=0} (default setting).
  The \ModelDesc reports only serious errors in this mode.
\item {\tt -DJEOD\_MEMORY\_DEBUG=1}. Level 0 plus summary.
  This setting makes the model generate a summary report at shutdown time.
\item {\tt -DJEOD\_MEMORY\_DEBUG=2}. Reserved.
\item {\tt -DJEOD\_MEMORY\_DEBUG=3}. Blow-by-blow account of all transactions.
\item {\tt -DJEOD\_MEMORY\_GUARD=0} (default setting).
  The model does not guard against memory overwrites.
\item {\tt -DJEOD\_MEMORY\_GUARD=1}
  If both debugging and guarding are enabled, the model guards against simple
  memory overwrites by allocating 24 extra bytes around each allocation.
  Any detected modifications to these guard bytes are reported as errors.
\end{itemize}

\section{Integration}
Any JEOD-based simulation must contain an object of a class that derives from
the JeodSimulationInterface class. A compliant version of a class that derives
from JeodSimulationInterface (e.g., the JeodTrickSimInterface class recommended
for use in all Trick-based simulations that use JEOD)
must contain a JeodMemoryManager object.
Following the instructions for use of the \hypermodelref{SIMINTERFACE}.

\section{Programmatic Use of the Model}
Model developers are free to use the externally-usable macros in their own
models.

\subsection{Primitive Types}
To allocate/deallocate primitive types, or pointers to primitive types,
all that is needed is to include the model header file and use the appropriate
macros. For example, suppose the class {\tt MyModel} has a {\tt double **}
member data element named {\tt double\_arr}. It is to be a $2\times N$ array,
where $N$ is a to-be-supplied parameter. Displayed below are the
{\tt initialize} and {\tt cleanup} methods for this class.


\begin{verbatim}
// Jeod includes
#include "utils/memory/jeod_alloc.hh"

// Model includes
#include "../my_model.hh"

void
MyModel::initialize (
   unsigned int nelem_in)  // In: -- Number of elements
{
   // Allocate double_arr
   nelem = nelem_in;
   double_arr    = JEOD_ALLOC_PRIM_ARRAY (2, double *);
   double_arr[0] = JEOD_ALLOC_PRIM_ARRAY (nelem, double);
   double_arr[1] = JEOD_ALLOC_PRIM_ARRAY (nelem, double);

   // Populate double_arr (not shown)
}

void
MyModel::cleanup ()
{
   JEOD_DELETE_ARRAY (double_arr[0]);
   JEOD_DELETE_ARRAY (double_arr[1]);
   JEOD_DELETE_ARRAY (double_arr);
}
\end{verbatim}


\subsection{Structured Types}
Using the model with structured types in a Trick setting requires a
bit more work. The user must declare as an external reference the
automatically-generated {\tt ATTRIBUTES} array for the types.
Suppose that a pointer to a {\tt Foo} object is added as a data
member to {\tt MyModel} class. The {\tt Foo} object is to be constructed
with a non-default constructor that takes a reference to the MyModel object
as an argument.

\begin{verbatim}
// Trick includes
#include "sim_services/include/attributes.h"
extern ATTRIBUTES * attrFoo;

// Jeod includes
#include "utils/memory/jeod_alloc.hh"

// Model includes
#include "../my_model.hh"

void
MyModel::initialize (
   unsigned int nelem_in)  // In: -- Number of elements
{
   // Allocate double_arr
   nelem = nelem_in;
   double_arr    = JEOD_ALLOC_PRIM_ARRAY (2, double *);
   double_arr[0] = JEOD_ALLOC_PRIM_ARRAY (nelem, double);
   double_arr[1] = JEOD_ALLOC_PRIM_ARRAY (nelem, double);

   // Populate double_arr (not shown)

   // Allocate and construct the Foo object.
   foo = JEOD_ALLOC_CLASS_OBJECT (Foo, (*this));
}

void
MyModel::cleanup ()
{
   JEOD_DELETE_OBJECT (foo);
   JEOD_DELETE_ARRAY (double_arr[0]);
   JEOD_DELETE_ARRAY (double_arr[1]);
   JEOD_DELETE_ARRAY (double_arr);
}
\end{verbatim}

\subsection{Macro Descriptions}

\subsubsection{Macro {\tt JEOD\_ALLOC\_CLASS\_MULTI\_POINTER\_ARRAY(nelem,type,asters)}}
\begin{description}
\item[Purpose]
Allocate an array of {\tt nelem} multi-level pointers to the specified
{\tt type}. The {\tt asters} are asterisks that specify the pointer level.
The allocated memory is initialized via {\tt new}.
\item[Returns]
Allocated array of specified type.
\item[Arguments] \ \\
\begin{tabular}{@{}ll}
{\tt nelem} &  Size of the array. \\
{\tt type} &  The underlying type, which must be a structured type. \\
{\tt asters} &  A bunch of asterisks. \\
\end{tabular}
\item[Example]
Allocate two pointers-to-pointers to the class {\tt Foo}.
Note that this does not allocate either the {\tt Foo} objects or pointers to
the {\tt Foo} objects.
\begin{verbatim}
Foo *** foo_array = JEOD_ALLOC_CLASS_MULTI_POINTER_ARRAY(2,Foo,**);
\end{verbatim}
\end{description}

\subsubsection{Macro {\tt JEOD\_ALLOC\_CLASS\_POINTER\_ARRAY(nelem,type)}}
\begin{description}
\item[Purpose]
Allocate an array of {\tt nelem} pointers to the specified {\tt type}.
The allocated memory is initialized via {\tt new}.
\item[Returns] Allocated array of specified type.
\item[Arguments] \ \\
\begin{tabular}{@{}ll}
{\tt nelem} &  Size of the array. \\
{\tt type} &  The underlying type, which must be a structured type.
\end{tabular}
\item[Example]
Allocate an array of two pointers to the class {\tt Foo}.
Note that this does not allocate the {\tt Foo} objects themselves.\begin{verbatim}
Foo ** foo_array = JEOD_ALLOC_CLASS_POINTER_ARRAY(2,Foo);
\end{verbatim}
\end{description}


\subsubsection{Macro {\tt JEOD\_ALLOC\_CLASS\_ARRAY(nelem,type)}}
\begin{description}
\item[Purpose]
Allocate an array of {\tt nelem} instances of the specified structured
{\tt type}. The default constructor is invoked to initialize each
allocated object.
\item[Returns] Allocated array of specified type.
\item[Arguments] \ \\
\begin{tabular}{@{}ll}
{\tt nelem} &  Size of the array. \\
{\tt type} &  The underlying type, which must be a structured type.
\end{tabular}
\item[Example]
Allocate an array containing two objects to the class {\tt Foo}.
\begin{verbatim}
Foo ** foo_array = JEOD_ALLOC_CLASS_ARRAY(2,Foo);
\end{verbatim}
\end{description}

\subsubsection{Macro {\tt JEOD\_ALLOC\_PRIM\_ARRAY(nelem,type)}}
\begin{description}
\item[Purpose]
Allocate {\tt nelem} elements of the specified primitive {\tt type}.
The allocated array is zero-filled.
\item[Returns] Allocated array of specified type.
\item[Arguments] \ \\
\begin{tabular}{@{}ll}
{\tt nelem} &  Size of the array. \\
{\tt type} &  The underlying type, which must be a C++ primitive type.
\end{tabular}
\item[Example]
Allocate an array of two doubles.
\begin{verbatim}
double * double_array = JEOD_ALLOC_PRIM_ARRAY(2,double);
\end{verbatim}
\end{description}

\subsubsection{Macro {\tt JEOD\_ALLOC\_CLASS\_OBJECT(type,constr)}}
\begin{description}
\item[Purpose]
Allocate one instance of the specified class. The supplied
constructor arguments, {\tt constr}, are used as arguments to
{\tt new}.
The default constructor will be invoked if the {\tt constr}
argument is the empty list; a non-default constructor will be
invoked for a non-empty list.
\item[Returns] Pointer to allocated object.
\item[Arguments] \ \\
\begin{tabular}{@{}ll}
{\tt type} &  The underlying type, which must be a structured type. \\
{\tt constr} &  Constructor arguments, enclosed in parentheses.
\end{tabular}
\item[Example]
Allocate a new object of type Foo, invoking the
\verb|Foo::Foo(bar,baz)| constructor.
\begin{verbatim}
Foo * foo = JEOD_ALLOC_CLASS_OBJECT(Foo,(bar,baz));
\end{verbatim}
\end{description}


\subsubsection{Macro {\tt JEOD\_ALLOC\_PRIM\_OBJECT(type, initial)}}
\begin{description}
\item[Purpose]
Allocate one instance of the specified {\tt type}.
The object is initialized with the supplied {\tt initial} value.
\item[Returns] Pointer to allocated primitive.
\item[Arguments] \ \\
\begin{tabular}{@{}ll}
{\tt type} &  The underlying type, which must be a C++ primitive type. \\
{\tt initial} &  Initial value.
\end{tabular}
\item[Example]
Allocate a double and initialize it to $\pi$.
\begin{verbatim}
double * foo = JEOD_ALLOC_PRIM_OBJECT(double, 3.14159265358979323846);
\end{verbatim}
\end{description}

\subsubsection{Macro {\tt JEOD\_IS\_ALLOCATED (ptr)}}
\begin{description}
\item[Purpose]
Determine if {\tt ptr} was allocated by some
\verb|JEOD_ALLOC_xxx_ARRAY| macro.
\item[Returns]
true if {\tt ptr} was allocated by the model, false otherwise.
\item[Arguments] \ \\
\begin{tabular}{@{}ll}
{\tt ptr} &  Memory to be checked.
\end{tabular}
\item[Example]
Check that memory was allocated by JEOD before freeing it.
\begin{verbatim}
if (JEOD_IS_ALLOCATED (name)) {
   JEOD_DELETE_ARRAY(name);
}
\end{verbatim}
\end{description}

\subsubsection{Macro {\tt JEOD\_DELETE\_ARRAY (ptr)}}
\begin{description}
\item[Purpose]
Free memory at {\tt ptr} that was earlier allocated with some
{\tt JEOD\_ALLOC\_xxx\_ARRAY} macro.
\item[Arguments] \ \\
\begin{tabular}{@{}ll}
{\tt ptr} &  Memory to be released.
\end{tabular}
\item[Example]
Allocate a chunk of memory and then free it.
\begin{verbatim}
Foo * foo1 = JEOD_ALLOC_CLASS_ARRAY(2,Foo);
...
JEOD_DELETE_ARRAY(foo1);
\end{verbatim}
\end{description}

\subsubsection{Macro {\tt JEOD\_DELETE\_OBJECT (ptr)}}
\begin{description}
\item[Purpose]
Free memory at {\tt ptr} that was earlier allocated with some
{\tt JEOD\_ALLOC\_xxx\_OBJECT} macro.
\item[Returns] N/A
\item[Arguments] \ \\
\begin{tabular}{@{}ll}
{\tt ptr} &  Memory to be released.
\end{tabular}
\item[Example]
Allocate a chunk of memory and then free it.
\begin{verbatim}
Foo * foo1 = JEOD_ALLOC_CLASS_OBJECT(Foo,());
...
JEOD_DELETE_OBJECT(foo1);
\end{verbatim}
\end{description}

\section{Extension}
The \ModelDesc is not designed for extensibility.