The User Guide is composed of three sections. The Analysis section is 
intended primarily for users of pre-existing simulations.  
The Integration section of the user guide is intended for simulation developers.  
The Extension section of the user guide is intended primarily for developers 
needing to extend the capability of the \MessageHandlerDesc.  Such users 
should have a thorough understanding of how the model is used in the preceding 
Integration section, and of the model 
specification (described in Chapter \ref{ch:spec}).

\section{Analysis}
The primary purpose of the \MessageHandlerDesc\ is to provide analysts
(i.e., sim users) useful information regarding the status of a sim
while it is running. Some messages are related to the 
intentional termination of a sim. Other messages may occur without
sim termination. Sim users can use the information conveyed by the
\MessageHandlerDesc\ to help ensure their sim is running
correctly or to report possible problems in the JEOD models to 
the JEOD development team. When reporting a message be sure
to capture the entire message.  This will aid with the
determination and correction of the problem.

An example of a message from the JEOD Gravity Model is:
\begin{verbatim}
SIMULATION TERMINATED IN
  PROCESS: 1
  JOB/ROUTINE: 2/gravity_source.cc line 551
DIAGNOSTIC:
Error environment/gravity ord_exceeds_deg_error:
Gravity field order (80) is greater than gravity field degree (70).
\end{verbatim}
In this case, the user requested a gravity order which is larger
than the gravity degree. 

The analyst has three inputs to control the types of messages displayed
and the information related to those messages:

{\bf suppression\_level} - Default value: 10 (warnings and non-fatal errors).
All messages have an associated severity level, with increasingly positive
values indicating warnings of decreasing severity. Fatal errors have a
negative severity level. Messages whose severity equals or exceeds the
value of the global message handler's suppression\_level are suppressed.
Note that fatal errors cannot be suppressed.

{\bf suppress\_id} - Default value: false.
This flag indicates whether the message ID is printed for unsuppressed
messages. The ID is not printed if this flag is set to true.
The message ID is always printed for errors.

{\bf suppress\_location} - Default value: false.
This flag indicates whether the message source file and line number
are printed for unsuppressed messages. The location is not printed if this
flag is set to true.
The message location is always printed for errors.

The first Trick sim\_object created should contain an instance of 
JeodSimulationInterface, which
will automatically construct a singleton MessageHandler for the sim.
Here is a typical definition of such a sim\_object.
\begin{verbatim}
/* ===================== JEOD System Simulation Object ===================== **/
/* ========================================================================= **/
sim_object { /* jeod_sys
---------------------------------------------------*/
   /*=========================================================================
    =  This is the JEOD executive model, this model should be basically
    =  the same for all JEOD-based Trick applications.
=
    =========================================================================*/
   /*==================
    = DATA Structures =
    ==================*/
   utils/sim_interface/include: JeodTrickSimInterface sim_interface;
} jeod_sys;
\end{verbatim}


An example of how these three controls might be set in an input file is:
\begin{verbatim}
jeod_sys.sim_interface.message_handler.suppression_level = MessageHandler::Warning;
jeod_sys.sim_interface.message_handler.suppress_id       = true;
jeod_sys.sim_interface.message_handler.suppress_location = false;
\end{verbatim}
In this example case, only errors will be reported; warnings and other
messages with a higher severity level will be suppressed.
For any message that is not suppressed, the ID number will be suppressed and location
of any messages will be displayed.

%%%%%%%%%%%%%%%%%%%%%%%%%%%%%
\section{Integration}
The use of the \MessageHandlerDesc\ by sim integrators is essentially the
same as for sim analysts. The severity level of suppressed messages
can be changed to suit the needs of the sim integrator. See previous section
for more information.

%%%%%%%%%%%%%%%%%%%%%%%%%%%%%
\section{Extension}
Sim developers can add messages to a model. The messages are normally defined
in the model class in the {\bf include} directory.  For example,
messages for the JEOD Gravity Model are defined in {\bf models/environment/gravity/include}
in the file gravity\_messages.hh as shown below:
\begin{verbatim}
   // Errors
   static char const * max_deg_error; /* --
      Error issued when requested gravity field degree exceeds maximum */

   static char const * max_ord_error; /* --
      Error issued when requested gravity field order exceeds maximum */

   static char const * ord_exceeds_deg_error; /* --
      Error issued when requested gravity field order is greater than degree */


   // Warnings
   static char const * radial_distance_warning; /* --
      Warning issued when radial distance is less than equatorial radius of
      gravity model */

\end{verbatim}

The same messages must also be added to a corresponding file in the 
{\bf src} directory, such as:
\begin{verbatim}
// Errors
char const * GravityMessages::max_deg_error =
    PATH " max_deg_error";

char const * GravityMessages::max_ord_error =
    PATH " max_ord_error";

char const * GravityMessages::ord_exceeds_deg_error =
    PATH " ord_exceeds_deg_error";

// Warnings
char const * GravityMessages::radial_distance_warning =
    PATH " radial_distance_warning";
\end{verbatim}

Finally, the code that actually triggers the message must be included
in the proper source code file.  For the Gravity Model example, the
message trigger code appears in the gravity\_body.cc file.  An
example of a {\bf failure} and a {\bf warning} trigger are:
\begin{verbatim}
      // check if maximum order to be used for computations is greater than
      // maximum degree
      if (controls.order > controls.degree) {
         MessageHandler::fail (
         __FILE__, __LINE__, GravityMessages::ord_exceeds_deg_error,
         "Gravity field order (%i) is greater than gravity field degree (%i).\n",
         controls.order,controls.degree);
         return;

      }

      // check if radial distance is less than equatorial radius
      if (r_mag < radius) {
         MessageHandler::warn (
            __FILE__, __LINE__, GravityMessages::radial_distance_warning,
            "Radial distance (%E) is less than equatorial radius of gravity model (%E).\n",
            r_mag,radius);
      }

\end{verbatim}

%----------------------------------
