%%%%%%%%%%%%%%%%%%%%%%%%%%%%%%%%%%%%%%%%%%%%%%%%%%%%%%%%%%%%%%%%%%%%%%%%%%%%%%%%
%
% ivv.tex
%
% Inspection, verification, and validation of the <model name>
%
% Usage instructions:
% This file should comprise a
% - The \chapter command, which must not be changed.
% - A number of \section command. You can change the labels but do not
%   change the names, and do not add any sections of your own.
% - Text to fill in the contents of the sections (which you must provide).
%
% Feel free to:
% - Change the labels on the \section commands.
% - Use your own \subsection commands and below. Structure is good.
% - Use figures and tables.
% - Split the file into parts if it gets too big.
%
%%%%%%%%%%%%%%%%%%%%%%%%%%%%%%%%%%%%%%%%%%%%%%%%%%%%%%%%%%%%%%%%%%%%%%%%%%%%%%%%

\chapter{Inspection, Verification, and Validation}
\hyperdef{part}{ivv}{}\label{ch:ivv}

\section{Inspection}\label{sec:inspect}
This section describes the inspections of the \MODELTITLE.

% \inspection{Top-level Inspection}
\label{inspect:TLI}
This document structure, the code, and associated files have been inspected, and together satisfy requirement~\ref{reqt:toplevel}.

\inspection{Design Inspection}
\label{inspect:design}
Table~\ref{tab:design_inspection} summarizes the key elements of the
implementation of the \ModelDesc that satisfy requirements levied on the model.
By inspection, the \ModelDesc satisfies
requirements~\ref{reqt:mass_requirements} to~\ref{reqt:attach_detach}.

\begin{longtable}{||l @{\hspace{4pt}} p{1.38in} |p{3.95in}|}
\caption{Design Inspection}
\label{tab:design_inspection} \\[6pt]
\hline
\multicolumn{2}{||l|}{\bf Requirement} & \bf{Satisfaction}
\\ \hline\hline
\endfirsthead

\caption[]{Design Inspection (continued from previous page)} \\[6pt]
\hline
\multicolumn{2}{||l|}{\bf Requirement} & \bf{Satisfaction}
\\ \hline\hline
\endhead

\hline \multicolumn{3}{r}{{Continued on next page}} \\
\endfoot

\hline
\endlastfoot

\ref{reqt:mass_requirements} & Mass &
  The class DynBody utilizes functionality from the MassBody class through 
  friendship. The friendship is not inheritable, thereby granting DynBody
  objects full access MassBody functionality while conserving core 
  behavior for user-derived classes.
\tabularnewline[4pt]
\ref{reqt:integ_frame} & Integration Frame &
  The DynBody class data member \verb+integ_frame+ points to the integration
  frame for a particular DynBody object. This member is set at initialization
  time by name. Upon attachment as a child to another DynBody object, the
  integration frame for the child bodies are set to that of the root body.
  The integration frame can be changed dynamically via the
  \verb+switch_integration_frames()+ member function.
\tabularnewline[4pt]
\ref{reqt:state_representation} & State Representation &
  The class BodyRefFrame derives from the RefFrame class.
  The inheritance is public, thereby granting external users of a DynBody
  object full access to the public RefFrame functionality of the BodyRefFrame
  objects contained in a DynBody object.
  A DynBody object contains three BodyRefFrame objects that represent the
  structure frame, core body frame, and the composite body frame.
  In addition to these three basic frames, a DynBody object contains an
  STL list of BodyRefFrame pointers that represent additional points
  of interest associated with the DynBody object.
\tabularnewline[4pt]
\ref{reqt:staged_initialization} & Staged Initialization &
  The BodyRefFrame \verb+initialized_items+ data member indicates which
  elements of a BodyRefFrame object's state have been set.
  The \ModelDesc methods described in section~\ref{sec:detailed_state_prop}
  ensure that a BodyRefFrame object's \verb+initialized_items+ properly
  reflect which of the BodyRefFrame  object's state elements have been set.
\tabularnewline[4pt]
\ref{reqt:eom} & Equations of Motion &
  This requirement has four sub-requirements that specify the treatment of
  forces, torques, translational acceleration, and rotational acceleration.
  Sections~\ref{sec:detailed_force_torque} to~\ref{sec:detailed_eom}
  describe the treatment of forces and torques in the \ModelDesc and
  how these lead to the development of the equations of motion.
\tabularnewline[4pt]
\ref{reqt:state_integ_prop} &
  \raggedright State Integration \\ and Propagation &
  As described in section~\ref{sec:detailed_state_integ}, the DynBody
  class provides the ability to integrate the composite body frame associated
  with the root body of a composite body or of an isolated (unconnected) body.
  The integrated state is propagated throughout a DynBody tree as described
  in section~\ref{sec:detailed_state_prop}.
\tabularnewline[4pt]
\ref{reqt:vehicle_points} & Vehicle Points &
  The DynBody class overrides the MassBody \verb+add_mass_point()+ method.
  This override creates a BodyRefFrame that corresponds to the MassPoint,
  thereby making these registered points of interest have a corresponding
  state. State is propagated to all reference frames associated with a DynBody
  object, including these vehicle points. The DynBody method
  \verb+compute_vehicle_point_derivatives()+ computes accelerations
   for a specific vehicle point as required.
\tabularnewline[4pt]
\ref{reqt:attach_detach} & Attach/Detach &
  As described in section ~\ref{sec:detailed_attach_detach}, the DynBody
  class overrides MassBody attach/detach member functions to provide the
  the required capability to attach and detach DynBody objects and to do
  so in a manner consistent with the laws of physics.
\tabularnewline[4pt]
\end{longtable}

\inspection{Mathematical Formulation}
\label{inspect:math}
The algorithmic implementations of the methods that provide the functionality of
the \ModelDesc follow the mathematics described in
section~\ref{sec:mathematics}.

By inspection, the \ModelDesc satisfies
requirements~\ref{reqt:eom}, \ref{reqt:vehicle_points}
and~\ref{reqt:state_integ_prop}.


\newpage
\section{Test}
This section describes various tests conducted to demonstrate
that the \MODELTITLEx satisfies the requirements levied against it.
The tests described in this section
are archived in the JEOD directory FIXME.


% \test{Simulation Interface Simulation}
\label{test:local_verif}
\begin{description}
\item[Background]
This test is an extremely simple simulation which
creates a Trick sim\_object containing 
an instance of a JeodTrickSimInterface to be tested.
One other Trick sim\_object creates a test object
with a single scheduled job.  The 
test object calls JeodTrickSimInterface::get\_job\_cycle() as
a scheduled job, thus the output can be compared
to the cycle time of the scheduled job.

The test object also includes private fields which are initialized by the 
Trick input processor and are logged by Trick.

\item[Test directory] {\tt verif} \\
This is a standard verification directory containing
the simulation directory  {\tt SIM\_sim\_interface}
along with src and include directories for the test class code.

\item[Success criteria]
The simulation includes initialization and logging of private fields 
which also reflect the output of 
JeodTrickSimInterface::get\_job\_cycle(). The logged data should be
identical to the 
reference data in the SET\_test\_val directory.

\item[Test results]
Passed.

\item[Applicable requirements]
This test demonstrates the satisfaction of
requirements \traceref{reqt:hidden_data_visibility},
\traceref{reqt:sim_engine_interface},
and \traceref{reqt:job_cycle}.
\end{description}


\test{Container Simulation}
\label{test:container_model_sim}
\begin{description}
\item[Background]
This simulation, located in the \CONTAINER\ verification directory,
exercises the checkpoint/restart capabilities of the model.
For a complete description of this test, see the
\hypermodelrefinside{CONTAINER}{part}{ivv} for details.
\item[Test Directory]
\verb|models/utils/container/verif/SIM_container_T10|
\item[Test Results]
Passed.
\item[Applicable Requirements]
This test demonstrates the satisfaction of
requirements \traceref{reqt:allocated_data_visibility},
\traceref{reqt:sim_engine_interface},
\traceref{reqt:checkpoint_restart},
and \traceref{reqt:addr_name_xlate}.
\end{description}


\test{Memory Simulation}
\label{test:memory_model_sim}
\begin{description}
\item[Background]
This simulation, located in the \MEMORY\ verification directory,
tests the ability of the \MEMORY\ to allocate and deallocate
memory. This in turn tests the ability of this model to
make those allocations visible to the simulation engine.
 For a complete description of this test, see the
\hypermodelrefinside{MEMORY}{part}{ivv} for details.
\item[Test Directory]
\verb|models/utils/memory/verif/SIM_memory_T10|
\item[Test Results]
Passed.
\item[Applicable Requirements]
This test demonstrates the satisfaction of
requirements
\traceref{reqt:allocated_data_visibility},
\traceref{reqt:sim_engine_interface},
and \traceref{reqt:checkpoint_restart}.
\end{description}


\test{Message Handler Simulation}
\label{test:message_model_sim}
\begin{description}
\item[Background]
This simulation, located in the \MESSAGE\ verification directory,
exercises the Trick-based implementation of the abstract
class MessageHandler.
For a complete description of this test, see the
\hypermodelrefinside{MESSAGE}{part}{ivv} for details.
\item[Test Directory]
\verb|models/utils/message/verif/SIM_message_handler_verif_T10|
\item[Test Results]
Passed.
\item[Applicable Requirements]
This test demonstrates the satisfaction of
requirements \traceref{reqt:sim_engine_interface}
and \traceref{reqt:trick_message_handler}.
\end{description}


\test{Propagated Planet Simulation}
\label{test:sim_prop_planet}
\begin{description}
\item[Background]
This simulation, located in the \EPHEMERIDES\ verification directory,
tests the ability of the \EPHEMERIDES\ to propagate a planet as an
alternative to using an ephemeris model.
The Trick 10 version was constructed to serve as a test case
of the multiple integration group capability provided by this model. 
 For a complete description of this test, see the
\hypermodelrefinside{EPHEMERIDES}{part}{ivv} for details.
\item[Test Directory]
\verb|models/environment/ephemerides/verif/SIM_prop_planet_T10|
\item[Test Results]
Passed.
\item[Applicable Requirements]
This test demonstrates the satisfaction of
requirements \traceref{reqt:allocated_data_visibility},
\traceref{reqt:sim_engine_interface},
\traceref{reqt:integ_interface},
\traceref{reqt:checkpoint_restart},
\traceref{reqt:addr_name_xlate},
and \traceref{reqt:multiple_integ_groups}.
\end{description}



\newpage
\section{Metrics}

Table~\ref{tab:coarse_metrics} presents coarse metrics on the source
files that comprise the model.
\input{coarse_metrics}

Table~\ref{tab:metrix_metrics} presents the extended cyclomatic complexity (ECC)
of the methods defined in the model.
\input{metrix_metrics}


\newpage
\section{Requirements Traceability}
This section is intentionally left blank for this release.
%Table~\ref{tab:reqt_ivv_xref} summarizes the inspections and tests
that demonstrate the satisfaction of the requirements levied on the model.

\begin{table}[hbp!]
%\begin{longtable}{||l @{\hspace{4pt}} l|l @{\hspace{2pt}} l @{\hspace{4pt}} l|}
\centering
\caption{Requirements Traceability}
\label{tab:reqt_ivv_xref}
\vspace{1ex}
% \\[6pt]
%\vspace{1ex}
\begin{tabular}{||l @{\hspace{4pt}} l|l @{\hspace{2pt}} l @{\hspace{4pt}} l|}
\hline
\multicolumn{2}{||l|}{\bf Requirement} &
\multicolumn{3}{l|}{\bf Inspection or test} \\ \hline\hline
\ref{reqt:toplevel} & Project Requirements &
     Insp. & \ref{inspect:TLI}     & Top-level Inspection
\tabularnewline[4pt]
\ref{reqt:mass_requirements}          & Mass &
     Insp. & \ref{inspect:design}     & Design Inspection \\
  && Test  & \ref{test:state_init}    & State Initialization \\
  && Test  & \ref{test:attach_detach} & Attach/Detach \\
  && Test  & \ref{test:force_torque}  & Force/Torque
\tabularnewline[4pt]
\ref{reqt:integ_frame}                & Integration Frame &
     Insp. & \ref{inspect:design}     & Design Inspection \\
  && Test  & \ref{test:state_init}    & State Initialization \\
  && Test  & \ref{test:frame_switch}  & Frame Switch
\tabularnewline[4pt]
\ref{reqt:state_representation}       & State Representation &
     Insp. & \ref{inspect:design}     & Design Inspection \\
  && Test  & \ref{test:state_init}    & State Initialization \\
  && Test  & \ref{test:frame_switch}  & Frame Switch
\tabularnewline[4pt]
\ref{reqt:staged_initialization}      & Staged Initialization &
     Insp. & \ref{inspect:design}     & Design Inspection \\
  && Test  & \ref{test:state_init}    & State Initialization
\tabularnewline[4pt]
\ref{reqt:eom}                        & Equations of Motion &
     Insp. & \ref{inspect:design}     & Design Inspection \\
  && Insp. & \ref{inspect:math}       & Mathematical Formulation \\
  && Test  & \ref{test:attach_detach} & Attach/Detach \\
  && Test  & \ref{test:force_torque}  & Force/Torque \\
  && Test  & \ref{test:frame_switch}  & Frame Switch \\
  && Test  & \ref{test:struct_integ}  & Structure Integration
\tabularnewline[4pt]
\ref{reqt:state_integ_prop}           & State Integration &
     Insp. & \ref{inspect:design}     & Design Inspection \\
& and Propagation
   & Insp. & \ref{inspect:math}       & Mathematical Formulation \\
  && Test  & \ref{test:attach_detach} & Attach/Detach \\
  && Test  & \ref{test:force_torque}  & Force/Torque \\
  && Test  & \ref{test:frame_switch}  & Frame Switch \\
  && Test  & \ref{test:struct_integ}  & Structure Integration
\tabularnewline[4pt]
\ref{reqt:vehicle_points}             & Vehicle Points &
     Insp. & \ref{inspect:design}     & Design Inspection \\
  && Test  & \ref{test:attach_detach} & Attach/Detach \\
  && Test  & \ref{test:force_torque}  & Force/Torque \\
  && Test  & \ref{test:frame_switch}  & Frame Switch
\tabularnewline[4pt]
\ref{reqt:attach_detach}              & Attach/Detach &
     Insp. & \ref{inspect:design}     & Design Inspection \\
  && Insp. & \ref{inspect:math}       & Mathematical Formulation \\
  && Test  & \ref{test:attach_detach} & Attach/Detach \\
  && Test  & \ref{test:force_torque}  & Force/Torque
\tabularnewline[4pt]
\hline
\end{tabular}
\end{table}
%\end{longtable}

