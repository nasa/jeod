\setcounter{chapter}{0}

%----------------------------------
\chapter{Introduction}\hyperdef{part}{intro}{}\label{ch:intro}
%----------------------------------


\section{Model Description}
%%% Incorporate the intro paragraph that used to begin this Chapter here.
%%% This is location of the true introduction where you explain why this 
%%% model exists.
%%% Identify the Model context within JEOD.
The \NamedItemDesc\ is basically a wrapper class for a set of static methods which construct a name as a dot-conjoined string.
These functions return a newly-allocated string, and the caller is
responsible for freeing the allocated memory via JEOD\_DELETE.

There is also a static method which, given a prefix and a dot-conjoined name, will find the part of the name
that follows the prefix. If the name is of the form "prefix.suffix", the 
suffix function returns a pointer to "suffix". If the name does not start with
"prefix." the function returns the input name. 

Most of these methods are used throughout the JEOD code.  Creating dot-conjoined strings in their own allocated memory is an extremely common task best handled by a central utility such as the NamedItem class.
There are two additional public static methods -- vconstruct\_name and 
va\_construct\_name which are only used presently by the construct\_name methods of the NamedItem class.  

\section{Document History}
%%% Status of this and only this document.  Any date should be relevant to when 
%%% this document was last updated and mention the reason (release, bug fix, etc.)
%%% Mention previous history aka JEOD 1.4-5 heritage in this section.
%%% Mention that JEOD.pdf is the parent document.

\begin{tabular}{||l|l|l|l|} \hline
\DocumentChangeHistory
\end{tabular}

\section{Document Organization}
This document is formatted in accordance with the 
NASA Software Engineering Requirements Standard~\cite{NASA:SWE} 
and is organized into the following chapters:

\begin{description}
%% longer chapter descriptions, more information.

\item[Chapter 1: Introduction] - 
This introduction contains three sections: description of model, document history, and organization.  
The first section provides the introduction to the \NamedItemDesc\ and its reason 
for existence. 
The second section displays the history of this document which includes
author, date, and reason for each revision.
The final
section contains a description of how the document is organized.

\item[Chapter 2: Product Requirements] - 
Describes requirements for the \NamedItemDesc.

\item[Chapter 3: Product Specification] - 
Describes the underlying theory, architecture, and design of the \NamedItemDesc\ in detail.  It is organized in 
three sections: Conceptual Design, Mathematical Formulations, and Detailed Design.

\item[Chapter 4: User Guide] - 
Describes how to use the \NamedItemDesc\ in a Trick simulation.  It is broken into three sections to represent the JEOD 
defined user types: Analysts or users of simulations (Analysis), Integrators or developers of simulations (Integration),
and Model Extenders (Extension).

\item[Chapter 5: Verification and Validation] -  
Contains \NamedItemDesc\ verification and validation procedures and results.

\end{description}

%----------------------------------
\chapter{Product Requirements}\hyperdef{part}{reqt}{}\label{ch:reqt}
%----------------------------------
This model shall meet the JEOD project requirements specified in the 
\hyperref{file:\JEODHOME/docs/JEOD.pdf}{part1}{reqt}{JEOD} top-level document.

%%% Format for the model Requirements is open.  It should include requirements for this model 
%%% only and use requirment tags like the one below.
%\requirement{...}
%\label{reqt:...}
%\begin{description}
%  \item[...]\ \newline
%    The documentation for the model shall include
%
%    \subrequirement{}
%    \label{reqt:...}
%      Software requirements specification.
%      
%    ...
%   
%  \item[title]\ \newline
%    text
%
%  ...
%
%\end{description}
\requirement{String Construction}
\label{reqt:SC}
\begin{description}
  \item[Requirement:]\ \newline
The \NamedItemDesc\ shall offer static functions capable of accepting one or more 
character strings, computing space requirements for a dot-conjoined concatenation
of the input strings, allocating the required space (including the null character terminating the string), and returning the newly created dot-conjoined string.
  \item[Rationale:]\ \newline
This is the function of the construct\_name methods

  \item[Verification:]\ \newline
    Inspection
\end{description}
\requirement{Suffix Determination}
\label{reqt:SD}
\begin{description}
\item[Requirement:]\ \newline
The \NamedItemDesc\ shall offer a static method capable of finding the part of a dot-conjoined name 
             following a given prefix. For names of the form "prefix.suffix",
             this function shall return a pointer to "suffix". The function shall return 
             the input name if the name does not start with "prefix.".
\item[Rationale:]\ \newline
Determination of a suffix in this manner supports JEOD naming and message conventions.
  \item[Verification:]\ \newline
    Inspection
\end{description}
%----------------------------------
\chapter{Product Specification}\hyperdef{part}{spec}{}\label{ch:spec}
%----------------------------------
\section{Conceptual Design}
The \NamedItemDesc\ is a single class with no instance fields or methods.  It's content 
consists of $7$ construct\_name static methods 
which construct dot-conjoined names from $1$ to $7$ input string arguments; two generic methods 
for constructing dot-conjoined strings from a variable number of arguments (null terminated); 
and a method for determining a suffix of a dot-conjoined string given an input prefix.  These
methods are basic C++ functions which serve their intended purpose and play no role in the object design of JEOD.
\section{File Inventory}
\begin{longtable}{c}
Files Comprising the \NamedItemDesc\  \\
\endfirsthead
Continued from previous page \\
\endhead
Continued on next page \\
\endfoot
\caption{File Inventory}
\endlastfoot
named\_item: \\
docs \\
include \\
src \\
 \\
named\_item/docs: \\
NamedItem.pdf \\
tex \\
 \\
named\_item/docs/tex: \\
makefile \\
NamedItemAbstract.tex \\
NamedItem.bib \\
NamedItemChapters.tex \\
NamedItemReq.tex \\
NamedItemSpec.tex \\
NamedItem.sty \\
NamedItem.tex \\
 \\
named\_item/include: \\
named\_item.hh \\
 \\
named\_item/src: \\
named\_item.cc \\
\end{longtable}


%----------------------------------
\chapter{User Guide}\hyperdef{part}{user}{}\label{ch:user}
%----------------------------------
The \NamedItemDesc\ is never referenced in any JEOD simulations, thus
the Analyst and Integrator sections of this chapter are omitted.
Readers wishing to use the \NamedItemDesc\ in their extensions of JEOD classes
are directed to the API reference.
%----------------------------------
\chapter{Verification and Validation}\hyperdef{part}{ivv}{}\label{ch:ivv}
%----------------------------------

\section{Verification}
%%% code imported from old template structure
%\inspection{<Name of Inspection>}\label{inspect:<label>}
% <description> to satisfy  
% requirement \ref{reqt:<label>}.
Verification of requirements \ref{reqt:SC} and \ref{reqt:SD} is performed
by code inspection.
\section{Validation}
%%% code imported from old template structure
%\test{<Title>}\label{test:<label>}
%\begin{description}
%\item[Purpose:] \ \newline
%<description>
%\item[Requirements:] \ \newline
%By passing this test, the universal time module 
%partially satisfies requirement~\ref{reqt:<label1>} and 
%completely satisfies requirement~\ref{reqt:<label2>}.
%\item[Procedure:]\ \newline
%<procedure>
%\item[Results:]\ \newline
%<results>
%\end{description}
There are no validation cases specifically designed for
the \NamedItemDesc; however, the functionality is thoroughly exercised 
throughout the JEOD software package.  This usage constitutes
an implicit validation of nominal performance.
