\test{Euler Angles Test}\label{test:euler}
\begin{description}
\item[Background]
The purpose of this test is to determine whether the three static
methods that convert to and from Euler angles are implemented correctly.
The transformations from an Euler angle sequence to a matrix or a quaternion
are fairly simple code. These methods involve little branching and have little if any numerical sensitivity.

As with the eigen rotation test, the true challenge is to verify the correctness
of the conversion from transformation matrices to Euler angles. The algorithm is
rather complex, with an extended cyclomatic complexity of 14. Making matters
worse, each of the twelve Euler rotation sequences potentially has a problem with
gimbal lock. Matrices that correspond to gimbal lock / near-gimbal lock
conditions need to be tested carefully.

\item[Test description]
This test centers on the middle angle $\theta$ of the Euler rotation sequence.
This is the element that determines whether gimbal lock is present.
The test covers three intervals for this angle: Angles well-removed
from gimbal lock positions, pure gimbal lock, and near-gimbal lock.

For each $\theta$ value, the test sweeps both the first angle $\phi$
and last angle $\psi$ over a range of 360 degrees.
Each of the twelve Euler sequences is tested for each $(\phi, \theta, \psi)$
triple. Each test involves
\begin{enumerate}
\item Using the model conversions from Euler angles to transformation quaternions
and to transformation matrices to compute the quaternion and transformation
matrix that corresponds to the given Euler sequence and angles.
\item Using the model conversion from transformation matrix to Euler angles
to recompute the Euler angles from the transformation matrix step one above.
\item For aerodynamics sequences only,
computing the transformation matrix using the Trick math library and
comparing the result to the matrix computed in step one.
\item Computing the error between the quaternion and matrix computed in step one.
\item Computing a quaternion from the Euler angles computed in step two and
comparing that to the quaternion computed in step one.
\end{enumerate}


\item[Test directory] {\tt verif/unit\_tests/euler\_angles} \\
This is a standard unit test directory with two configuration managed items,
main.cc and makefile. Simply type {\tt make} to build the test article and run
the test in default mode. This default mode summarizes the results of the test.
To see individual output, issue the command {\tt ./test\_program -verbose} after
having made the test article.

\item[Success criteria]
A common threshold of $10^{-15}$ radians angular error (numerical error) is
used for all tests. Over 1.5 million cases are evaluated during the course of
the test, and each one of these must pass for the test to pass as a whole.

\item[Test results]
The test passes.

\item[Relevant requirements]
This test demonstrates the satisfaction of requirement~\ref{reqt:conversions}
sub-requirements~\ref{reqt:euler_to_mat} to~\ref{reqt:euler_to_quat}
and partially demonstrates the satisfaction of
requirements~\ref{reqt:representations} and~\ref{reqt:euler_angles}.

\end{description}
