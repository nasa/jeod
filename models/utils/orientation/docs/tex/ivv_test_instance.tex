\test{Instance Test}\label{test:instance}
\begin{description}
\item[Background]
The two unit tests described above address the static conversion methods provided
by the model. These conversion methods are secondary to the primary purpose of
the model, which is to serve as an input mechanism by which simulation users can
specify an orientation in any of the supported forms. The model accomplishes
this by making the Orientation class instantiable.
The purpose of this test is to test the completeness and correctness of the
treatment of Orientation objects.

The model provides three different schemes for setting an Orientation object:
By construction, through setters, and by direct assignment to member data.
The model provides two schemes for accessing an Orientation object:
Through getters and by direct access to member data. The direct access to
member data means that the model must provide means to ensure that an
accessible representation is consistent with input. This leads to yet another
set of member functions that compute some desired product.

This test tests various combinations of assignment techniques paired with access
techniques, with each of the twelve different Euler sequences counted as
a separate technique. While the two previous unit tests used numerical overkill
to demonstrate correctness, this test uses a combinatoric overkill approach.

\item[Test description]
This test assumes that all of the direct conversions depicted in
figure~\ref{fig:representations} operate correctly. Ignoring the multiplicity
of representation schemes, this test uses but one orientation as the basis for
testing, a rotation of 120 degrees about the line $x=y=z$. As this orientation
is well-removed from all gimbal lock positions, the errors that result from
converting from one form to another should be small numerical errors.

The following pairings of assignment and access techniques are tested:
\begin{itemize}
\item Setters and getters.
\item Constructors and getters.
\item Direct assignments and direct access.
\end{itemize}

\item[Test directory] {\tt verif/unit\_tests/instance} \\
This is a standard unit test directory with two configuration managed items,
main.cc and makefile. Simply type {\tt make} to build the test article and run
the test in default mode. This default mode summarizes the results of the test.
To see individual output, issue the command {\tt ./test\_program -verbose} after
having made the test article.

\item[Success criteria]
Having a known correct response means that the error in any retrieved
representation can be readily calculated. The test passes if each of the
retrieved representations is within some threshold of the known value.
The selected orientation is well-removed from all gimbal lock positions.
The errors that result from converting from any one form to another should be
very small numerical errors. The test uses a threshold of $10^{-15}$ radians to
detect errors.

\item[Test results]
The test exhaustively tests all combinations of inputs and outputs. All outputs
are numerically close to the expected correct value. The test passes.

\item[Relevant requirements]
This test demonstrates the satisfaction of
requirements~\ref{reqt:representations} and~\ref{reqt:data_access}
and partially demonstrates the satisfaction of
requirement~\ref{reqt:euler_angles}.

\end{description}
