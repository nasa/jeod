%%%%%%%%%%%%%%%%%%%%%%%%%%%%%%%%%%%%%%%%%%%%%%%%%%%%%%%%%%%%%%%%%%%%%%%%%%%%%%%%
% User Guide
%%%%%%%%%%%%%%%%%%%%%%%%%%%%%%%%%%%%%%%%%%%%%%%%%%%%%%%%%%%%%%%%%%%%%%%%%%%%%%%%

\chapter{User Guide}\hyperdef{part}{user}{}\label{ch:user}

This chapter describes how to use the \ModelDesc from the 
perspective of a simulation user, a simulation developer,
and a model developer.

\section{Analysis}
One use of the \ModelDesc is to embed an Orientation object as a
data member in a class used to initialize some object.
For example, the \hypermodelref{MASS}, which uses the \ModelDesc to enable a user
to specify the orientation of a MassBody object's body reference frame
with respect to the MassBody object's structural frame.
Another example is in the \hypermodelref{BODYACTION}. This model uses
the \ModelDesc to enable a user to specify the initial orientation of a DynBody
object with respect to some other reference frame. The sample code below focus on
this latter usage.

To specify an orientation via the \ModelDesc, the user must set the Orientation
object's \verb|data_source| data member to identify which of the four supported
representations is being used and must populate the data member(s) appropriate
to that selected representation. To specify an orientation via a
\begin{itemize}
\item Transformation matrix:
  \begin{itemize}
  \item Set the \verb|data_source| data member to
    \verb|Orientation::InputMatrix| and
  \item Populate the \verb|trans| data member with the desired matrix.
  \end{itemize}
\item Left transformation quaternion:
  \begin{itemize}
  \item Set the \verb|data_source| data member to
    \verb|Orientation::InputQuaternion| and
  \item Populate the \verb|quat| data member with the desired quaternion.
  \end{itemize}
\item Eigen rotation:
  \begin{itemize}
  \item Set the \verb|data_source| data member to
    \verb|Orientation::InputEigenRotation| and
  \item Populate the \verb|eigen_angle| and \verb|eigen_axis| data members
    with the desired rotation angle about the desired rotation axis.
  \end{itemize}
\item Euler rotation sequence:
  \begin{itemize}
  \item Set the \verb|data_source| data member to
    \verb|Orientation::InputEulerRotation| and
  \item Populate the \verb|euler_sequence| and \verb|euler_angles| data members
    with the desired Euler rotation sequence about the desired rotation axis.
  \end{itemize}
\end{itemize}

Examples of using a transformation matrix and an Euler sequence are shown below.

Transformation matrix:
\begin{verbatim}
VEH_OBJ.rot_init.orientation.data_source = Orientation::InputMatrix;
VEH_OBJ.rot_init.orientation.trans[0][0] =
    0.1412307175854331, -0.9892782736897275,  0.03718039289432346;
VEH_OBJ.rot_init.orientation.trans[1][0] =
   -0.7726919750981249, -0.1336331223242702, -0.6205556383087864;
VEH_OBJ.rot_init.orientation.trans[2][0] =
    0.6188707425862546,  0.0589125268795973, -0.7832804849780176;
\end{verbatim}

Euler sequence:
\begin{verbatim}
VEH_OBJ.lvlh_init.orientation.data_source    = Orientation::InputEulerRotation;
VEH_OBJ.lvlh_init.orientation.euler_sequence = Orientation::Roll_Pitch_Yaw;
VEH_OBJ.lvlh_init.orientation.euler_angles[0] {d} = 0.0, 85.0, 1.0;
\end{verbatim}


\section{Integration}
The \ModelDesc is not intended to be used directly at the \Sdefine level.
Orientation objects are almost inevitably embedded as data members of some other
class. That said, this does not mean that a simulation integrator has no
responsibility with respect to the \ModelDesc. For example, the standard set of
input files for a multi-body simulation typical specify how the bodies in the
simulation attach to one another. These attachments must be physically correct,
and ensuring this often is an integration-level task.


\section{Embedding}
As mentioned above, various JEOD models embed an Orientation object as a data
member of some class. External model developers are free to do so as well.
The user populates the Orientation object; methods of the containing class query
the Orientation object to extract the desired orientation data from the object. 
Model developers who wish to use the model in this way should read the
Orientation API.

\section{Extension}
The Orientation class is extensible. The VectorOrientation class defined in the
\INTEGRATION\ verification code is an example of such an extension.