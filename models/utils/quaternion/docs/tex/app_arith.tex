\section{Quaternion Arithmetic}\label{sec:app_arith}

\subsection{Basic Quaternion Operations}

This section develops the basic arithmetic operations on quaternions.
In this section,
$\quat Q$, $\quat{Q}_1$, and $\quat{Q}_2$
are quaternions defined with components
\begin{align*}
\quat Q    &= q_s + q_i i + q_j j + q_k k
            = \quatsv {q_s} {\vect{q_v}} \\
\quat{Q}_1 &= q_{1_s} + q_{1_i} i + q_{1_j} j + q_{1_k} k 
            = \quatsv{q_{1_s}}{\vect{q_{1_v}}} \\
\quat{Q}_2 &= q_{2_s} + q_{2_i} i + q_{2_j} j + q_{2_k} k
            = \quatsv{q_{2_s}}{\vect{q_{2_v}}}
\end{align*}

Quaternion addition, scaling a quaternion by a real, and the quaternion norm
are defined as isomorphisms to the corresponding operations on a $4-$vector:
\begin{definition}[Quaternion addition]\label{def:quat_sum}
Given quaternions $\quat{Q}_1$ and $\quat{Q}_2$,
the quaternion sum $\quat{Q}_1 + \quat{Q}_2$ is defined as
\begin{equation}
{\quat{Q}_1} + {\quat{Q}_2} \equiv
         q_{1_s} + q_{2_s} 
    + (q_{2_i}  + q_{2_i}))\,i
    + (q_{2_j}  + q_{2_j}))\,j
    + (q_{2_k}  + q_{2_k}))\,k
    = \quatsv{q1_s+q2_s} {\vect{q_{1_v}}+\vect{q_{2_v}}}
  \label{eqn:quat_sum_def}
\end{equation}
\end{definition}

\begin{definition}[Quaternion scaling]\label{def:quat_scale}
Given a quaternion $\quat Q$ and a real number $s$,
the product of $\quat Q$ with $s$ is defined as
\begin{equation}
s \quat Q \equiv s q_s + s q_i i + s q_j j + s q_k k
  = \quatsv {s q_s} {s \vect{q_v}}
\label{eqn:quat_prod_scalar}
\end{equation}
\end{definition}

\begin{definition}[Quaternion norm]\label{def:quat_norm}
Given a quaternion $\quat Q$,
the quaternion norm of $\quat Q$ is defined as
\begin{equation}
\norm{\quat Q} \equiv \sqrt{q_s^2 + q_i^2 + q_j^2 + q_k^2}
  = \sqrt{q_s^2 + \vect{q_{v}} \!\cdot \vect{q_{v}}}
\label{eqn:quat_norm_def}
\end{equation}
\end{definition}

The quaternion conjugate is defined analogously to the complex conjugate:
\begin{definition}[Quaternion conjugate]\label{def:quat_conj}
Given a quaternion $\quat Q$,
the quaternion conjugate of $\quat Q$,
denoted herein as either $\conjop \quat Q$ or $\quatconj Q$,
is defined as
\begin{equation}
\conjop{\quat Q} \equiv
\quatconj Q \equiv q_s - q_i i - q_j j - q_k k
  = \quatsv {q_s} {- \vect{q_v}}
\label{eqn:quat_conj_def}
\end{equation}
\end{definition}

The definition of the quaternion product follows from
Hamilton's Fundamental Formula of Quaternion Algebra,
equation~\eqref{eqn:quat_fundamental_formula}:
\begin{definition}[Quaternion multiplication]\label{def:quat_prod}
Given quaternions $\quat{Q}_1$ and $\quat{Q}_2$,
the quaternion product $\QxQ{\quat{Q}_1}{\quat{Q}_2}$ is defined as
\begin{equation}
\begin{split}
\QxQ{\quat{Q}_1}{\quat{Q}_2} \equiv&\;
        q_{1_s} q_{2_s} -
        (q_{1_i} q_{2_i} + q_{1_j} q_{2_j} + q_{1_k} q_{2_k}) \\
    &+ (q_{1_s} q_{2_i}  + q_{2_s} q_{1_i} +
        (q_{1_j} q_{2_k} - q_{1_k} q_{2_j}))\,i \\
    &+ (q_{1_s} q_{2_j}  + q_{2_s} q_{1_j} +
        (q_{1_k} q_{2_i} - q_{1_i} q_{2_k}))\,j \\
    &+ (q_{1_s} q_{2_k}  + q_{2_s} q_{1_k} +
        (q_{1_i} q_{2_j} - q_{1_j} q_{2_i}))\,k
  \end{split}  \label{eqn:quat_prod_explicit}
\end{equation}
or in scalar $+$ vector form,
\begin{equation}
\QxQ{\quat{Q}_1}{\quat{Q}_2} = 
  \quatsv
    {q_{1_s} q_{2_s} - \vect{q_{1_v}} \!\cdot \vect{q_{2_v}}}
    {q_{1_s} \vect{q_{2_v}} +
     q_{2_s} \vect{q_{1_v}} +
     \vect{q_{1_v}} \!\times \vect{q_{2_v}}}
  \label{eqn:quat_prod_short}
\end{equation}
\end{definition}

Note that definitions~\ref{def:quat_scale} and~\ref{def:quat_prod} are
consistent: The product of a quaternion of the form
$\quatsv {q_s}{\vect{q_v}}$ and a real quaternion
$\quatsv {s}{\vect 0}$ is
\begin{equation*}
  \QxQ{\quatsv {q_s}{\vect{q_v}}}{\quatsv {s}{\vect 0}}
  = \QxQ{\quatsv {s}{\vect 0}}{\quatsv {q_s}{\vect{q_v}}}
  = \quatsv {s q_s}{s \vect {q_v}} = s \quat Q
\end{equation*}

\begin{corollary}\label{thm:quat_quat_conj_prod}
The product of a quaternion and its conjugate is the square of
the norm of the quaternion.
\end{corollary}

\begin{proof}
The product of a quaternion $\quat Q = \quatsv {q_s}{\vect{q_v}}$
and its conjugate is
\begin{align*}
  \QxQ{\quat Q}{\quatconj Q} &=
  \QxQ{\quatsv {q_s}{\vect{q_v}}}{\quatsv {q_s}{- \vect{q_v}}}  \\
  &= \quatsv {q_s^2 + \vect{q_v} \cdot \vect{q_v}}{\vect 0}
\end{align*}
\end{proof}


\subsection{Commutativity, Associativity, and Distributivity}

\begin{theorem}\label{thm:quat_sum_and_scale_commute}
Quaternion addition and scaling a quaternion by a real scalar
are commutative operations.
\end{theorem}
\begin{proof}
The quaternion sum and the product of a quaternion and a real are defined as
isomorphisms to the corresponding commutative $4-$vector operations.
\end{proof}

Note that the product of two quaternions does not in general commute.
This follows directly from the Fundamental Formula of Quaternion Algebra:
$ij = k, ji = -k$.

\begin{theorem}\label{thm:quat_prod_assoc}
Quaternion multiplication is associative:
\begin{equation*}
  \QxQ{(\QxQ{\quat{Q}_1}{\quat{Q}_2})}{\quat{Q}_3} =
    \QxQ{\quat{Q}_1}{(\QxQ{\quat{Q}_2}{\quat{Q}_3})}
\end{equation*}
\end{theorem}
\begin{proof}
This can be easily proven to be the case for three quaternions.
Extending the theorem to a product of any number of quaternions and to any
grouping of parentheses in the product is accomplished by induction.
\end{proof}

\begin{theorem}\label{thm:quat_prod_distrib}
Quaternion multiplication distributes over quaternion addition:
\begin{equation}
  \QxQ{\quat Q _1}{\left({\quat Q _2 + \quat Q _3}\right)} =
    \QxQ{\quat Q _1}{\quat Q _2} + {\quat Q _1}{\quat Q _3}
\end{equation}
\end{theorem}
This can be easily proven true for any three arbitrary quaternions
$\quat Q _1$, $\quat Q _2$, and $\quat Q _3$ by applying the
definitions of quaternion addition and multiplication.

\subsection{Quaternion Identity, Inverse, and Division}

\begin{theorem}\label{thm:quat_identity}
The real unit quaternion is the identity element for quaternion multiplication.
\end{theorem}
\begin{proof}
The left and right products of any quaternion
$\quatsv {q_s}{\vect{q_v}}$ and the real unit quaternion $\quatsv 1 {\vect 0}$
reproduce the quaternion $\quat Q$:
\begin{equation*}
  \QxQ{\quatsv {q_s}{\vect{q_v}}}{\quatsv 1 {\vect 0}} =
  \QxQ{\quatsv 1 {\vect 0}}{\quatsv {q_s}{\vect{q_v}}} =
  \quatsv {q_s}{\vect{q_v}}
\end{equation*}
The real unit quaternion is thus both a left- and right- identity element.
Since quaternion multiplication is associative,
the multiplicative identity is unique.
The real unit quaternion is thus the unique identity element
for quaternion multiplication.
\end{proof}

\begin{theorem}\label{thm:quat_inverse}
The inverse $\quat Q ^ {-1} $ of a non-zero quaternion $\quat Q$ is
\begin{equation}
\quat Q ^ {-1} = \frac{1}{\quat Q}
  \equiv \frac{1}{\norm{\quat Q}^2} \quatconj Q
  = \frac{1}{\QxQ {\quat Q} {\quatconj Q}} \quatconj Q
\label{eqn:quat_inverse_def}
\end{equation}
\end{theorem}
\begin{proof}
\begin{align*}
\intertext{Let}
  \quat Q &= \quatsv {q_s}{\vect{q_v}} \\
  \quat Q_2 &= \frac{1}{\QxQ {\quat Q} {\quatconj Q}} \quatconj Q \\
\intertext{then}
  \QxQ {\quat Q}{\quat Q_2}
    &= \frac{1}{\QxQ {\quat Q} {\quatconj Q}}
      \quatsv {q_s^2 + \vect{q_v} \cdot \vect{q_v}} {\vect 0} \\
    &= \quatsv 1 {\vect 0}
\end{align*}
Thus the stated quaternion is a right multiplicative inverse.
Since quaternion multiplication is associative,
the stated quaternion is also a left multiplicative inverse and is unique.
\end{proof}

\begin{corollary}\label{thm:quat_unit_inverse_is_conjugate}
The multiplicative inverse of a unit quaternion is the quaternion conjugate.
\end{corollary}
\begin{proof}
This follows directly from the definitions of the quaternion multiplicative
inverse and the quaternion conjugate.
\end{proof}

Quaternion division is defined in terms of multiplication with the quaternion
inverse.
Since quaternion multiplication is not commutative,
pre-multiplying and post-multiplying a quaternion
with the inverse of another quaternion yield different results.
This observation leads to the definition of a left and right division operator:
\begin{definition}[Quaternion division]\label{def:quat_div}
\begin{align}
  \quat Q_1 \setminus \quat Q_2 &\equiv \QxQ{{\quat Q_2}^{-1}} {\quat Q_1} \\
  \quat Q_1 / \quat Q_2 &\equiv \QxQ{\quat Q_1} {{\quat Q_2}^{-1}}
\end{align}
\end{definition}

\subsection{Quaternion Decomposition}\label{sec:app_decomp}

This section develops an alternative representation scheme for quaternions.
\begin{definition}\label{def:quat_decomp}
A non-zero quaternion represented in terms of
its magnitude $s\in\mathbb R$,
a unit vector $\vhat u\in\mathbb R^3$,
and a rotation angle $\theta\in\mathbb R$,
\begin{equation}
  \quat Q = s\quatsv{\cos\theta}{\sin\theta\vhat u}\label{eqn:quat_decomp_def}
\end{equation}
is a \emph{decomposition} of the quaternion.
\end{definition}

\begin{corollary}\label{thm:quat_decomp_multivalued}
The decomposition of a quaternion not unique.
\end{corollary}
\begin{proof}
if $s\quatsv{\cos\theta\phantom{\vhat u}}{\sin\theta \vhat u} = \quat Q$
then so does
$s\quatsv{\cos(\theta+2n\pi)\phantom{\vhat u}}{\sin(\theta+2n\pi) \vhat u}$
for all integer values of $n$.
\end{proof}

\begin{corollary}\label{thm:quat_real_unit_decomp_indeterminate}
The decomposition of the real unit quaternion $\quatsv 1{\vect 0}$ is
a trivial rotation ($\theta=0$) about an indeterminate axis.
\end{corollary}
\begin{proof}
Applying equation~\eqref{eqn:quat_decomp_def} with $s=1$ and $\theta=0$
generates the real unit quaternion regardless of the value of $\vhat u$.
\end{proof}

\begin{corollary}\label{thm:quat_negation_rotates_by_pi}
Given a quaternion $\quat Q$ with decomposition
$\quat Q = s\quatsv{\cos\theta}{\sin\theta \vhat u}$,
a decomposition of the additive inverse of $\quat Q$ is
$-\quat Q = s\quatsv{\cos(\theta+\pi)}{\sin(\theta+\pi) \vhat u}$,
\end{corollary}
\begin{proof}
Expanding $\quatsv{\cos(\theta+\pi)}{\sin(\theta+\pi) \vhat u}$
results in $-\quat Q$.
\end{proof}

\begin{algorithm}[Unit quaternion decomposition]\label{alg:quat_unit_decomp}
Let $\quat Q = \quatsv{q_s}{\vect{q_v}}$ be a unit quaternion
(${q_s}^2 + \vect{q_v} \cdot \vect{q_v} = 1$) with non-zero vector part. Then
\begin{align}
  \theta &=
    \arctan(\norm {\vect{q_v}}, q_s) \label{eqn:quat_unit_decomp_theta} \\
  \vhat u &=
    \frac {\vect{q_v}}{\norm {\vect{q_v}}} \label{eqn:quat_unit_decomp_uhat}
\end{align}
is a \emph{decomposition} of the unit quaternion.
\paragraph{Note}
The two-argument inverse tangent function in
equation~\eqref{eqn:quat_unit_decomp_theta}
computes the inverse tangent of $y/x$, with the signs of $y$ and $x$
dictating the quadrant of the result in which the result is placed.
The C-library function \texttt{atan2} approximates $\arctan(y,x)$.
The single-argument inverse cosine and inverse sine functions
can be used as an alternative to \texttt{atan2}.
However, care must be taken to use the function that produces
better accuracy and to place the angle in the correct quadrant:
\begin{equation}
  \theta =
    \begin{cases}
      \pi - \arcsin(\norm {\vect{q_v}}) &
        \text{if %
              $\phantom{-\frac{\surd 2}2 \leq\;}
               q_s < -\frac{\surd 2}2$} \\
      \arccos(q_s) &
         \text{if %
               $-\frac{\surd 2}2 \leq
                q_s \leq \phantom{-}\frac{\surd 2}2$} \\
      \arcsin(\norm {\vect{q_v}}) &
        \text{if %
              $\phantom{-\frac{\surd 2}2 \leq\;}
               q_s > \phantom{-}\frac{\surd 2}2$}
    \end{cases}\label{eqn:quat_unit_decomp_theta_alt}
\end{equation}
Which alternative (equation~\eqref{eqn:quat_unit_decomp_theta} or
~\eqref{eqn:quat_unit_decomp_theta_alt}) is faster and/or more accurate
is machine-dependent.
\end{algorithm}
\begin{proof}
\begin{align*}
  \vect{q_v} &= \norm{\vect{q_v}} \frac{\vect{q_v}}{\norm{\vect{q_v}}} \\[5pt]
  &= \sqrt{1-{q_s}^2}\frac{\vect{q_v}}{\norm{\vect{q_v}}}
  \quad \text{since $\norm{\vect{q_v}}^2 = 1-{q_s}^2$} \\
\intertext{thus}
  \cos\theta &= q_s \\
  \sin\theta &= \sqrt{1-{q_s}^2} = \norm{\vect q_v} \\
\intertext{Applying equation~\eqref{eqn:quat_decomp_def} with $s=1$,}
  \quatsv{\cos\theta}{\sin\theta\vhat u}
    &= \quatsv{q_s}{\norm{\vect q_v} \frac{\vect{q_v}}{\norm {\vect{q_v}}}} \\
    &= \quat Q
\end{align*}
\end{proof}

\begin{corollary}\label{thm:quat_unit_decomp}
A decomposition exists for all unit quaternions.
\end{corollary}
\begin{proof}
Corollaries~\ref{thm:quat_real_unit_decomp_indeterminate}
and~\ref{thm:quat_negation_rotates_by_pi}
provide decompositions for the two unit quaternions with zero vector parts.
Since $\arctan(y,x)$ is defined for all finite values of $x$ and $y$,
algorithm ~\ref{alg:quat_unit_decomp}
provides a decomposition for all unit quaternions with non-zero vector parts.
\end{proof}

\begin{corollary}\label{thm:quat_decomp}
A decomposition exists for all quaternions.
\end{corollary}
\begin{proof}
The zero quaternion has an indeterminate decomposition: $s=0$ and
$\theta$ and $\vhat u$ are indeterminate.
Represent a non-zero quaternion $\quat Q$ as
$\norm{\quat Q} \frac{\quat Q}{\norm{\quat Q}}$.
The unit quaternion $\frac{\quat Q}{\norm{\quat Q}}$ has some decomposition
$(s=1, \theta, \vhat u)$ by corollary~\ref{thm:quat_unit_decomp}.
The decomposition $(s=\norm{\quat Q}, \theta, \vhat u)$ generates the
original quaternion $\quat Q$.
\end{proof}

\begin{corollary}\label{thm:quat_decomp_inverse}
The multiplicative inverse of a quaternion $\quat Q$ is
\begin{equation}
  \quat Q^{-1} =
    \frac{1}{s} \quatsv {\cos \theta}{- \sin \theta \vhat u}
  \label{eqn:quat_decomp_inverse}
\end{equation}
where $s$, $\theta$, and $\vhat u$ are a decomposition of the
quaternion $\quat Q$.
\end{corollary}
\begin{proof}
Multiplying the decomposition of $\quat Q$ and the inverse as expressed in
equation~\eqref{eqn:quat_decomp_inverse} yields the real unit quaternion.
\end{proof}

\subsection{Quaternion Exponentiation}\label{sec:app_exponentiation}

Raising a non-zero quaternion $\quat Q$ to a real integer power
is defined in terms of iterated multiplication,
analogous to the definition for real and complex numbers:
\begin{definition}[Integer Power of a Quaternion]\label{def:quat_int_power}
\begin{equation}
\quat Q^n \equiv
\begin{cases}
  \prod_{r=1}^n \quat Q
    & \text{if $n>0$} \\[5pt]
  \quatsv 1 {\vect 0}
    & \text{if $n =0$} \\[12pt]
  \left(\quat Q^{-1}\right)^n = \prod_{r=1}^n \quat Q^{-1}
    & \text{if $n<0$}
\end{cases}\label{eqn:quat_int_power_def}
\end{equation}
\end{definition}
Since quaternion multiplication is associative
(theorem~\ref{thm:quat_prod_assoc}),
the products in equation~\eqref{eqn:quat_int_power_def} are well-defined.

\begin{theorem}\label{thm:quat_rat_power}
The result of raising a quaternion $\quat Q$ to a rational power $r$ is
\begin{equation}
  \quat Q^r =
    s^r \quatsv {\cos r\theta}{\sin r\theta \vhat u}
  \label{eqn:quat_rat_power}
\end{equation}
where $s$, $\theta$, and $\vhat u$ are a decomposition of the
quaternion $\quat Q$.
\end{theorem}
\begin{proof}
The theorem is clearly true for $r=0$,
since equations~\eqref{eqn:quat_int_power_def} and~\eqref{eqn:quat_rat_power}
yields the real unit quaternion for $r=0$.
Assuming the theorem is true for some integer $r \neq 0$, then
\begin{align*}
  \quat Q^{r+1} &= \QxQ{\quat Q^r}{\quat Q} \\
  &=
    \QxQ
      {s^r \quatsv {\cos r\theta}{\sin r\theta \vhat u}}
      {s \quatsv {\cos \theta}{\sin \theta \vhat u}} \\
    &= s^{r+1}
    \quatsv
      {\cos\left(\left(r+1\right)\theta\right)}
      {\sin\left(\left(r+1\right)\theta\right) \vhat u}
\intertext{and}
  \quat Q^{r-1} &= \QxQ{\quat Q^r}{\quat Q^{-1}} \\
  &=
    \QxQ
      {s^r \quatsv {\cos r\theta}{\sin r\theta \vhat u}}
      {s^{-1} \quatsv {\cos \theta}{-\sin \theta \vhat u}} \\
    &= s^{r-1}
    \quatsv
      {\cos\left(\left(r-1\right)\theta\right)}
      {\sin\left(\left(r-1\right)\theta\right) \vhat u}
\end{align*}
By induction, the theorem is true for all integers.

Now consider the $n^{th}$ root of $\quat Q$
as defined by equation~\eqref{eqn:quat_rat_power}. Raising this to
the integral $n^{th}$ power yields
\begin{align*}
  \left(s^{1/n}
    \quatsv {\cos \frac \theta n}{\sin \frac \theta n \vhat u}\right)^n
  &=
  \left(s^{1/n}\right)^n
    \quatsv {\cos n \frac \theta n}{\sin n \frac \theta n \vhat u} \\
  &= \quat Q
\end{align*}
The theorem has now been proven for all integers and
all rationals of the form $1/n$.
It is therefore valid for any number of the form $p/q$,
where $p$ and $q$ are integers.
\end{proof}

By continuity, equation~\eqref{eqn:quat_rat_power} is extended to cover
all the real numbers. 
\begin{definition}[Real power of a quaternion]\label{def:quat_exponentiation}
The result of raising a quaternion $\quat Q$ to a real power $a$ is
\begin{equation}
  \quat Q^a =
    s^a \quatsv {\cos a\theta}{\sin a\theta \vhat u}
  \label{eqn:quat_power}
\end{equation}
where $s$, $\theta$, and $\vhat u$ are a decomposition of the
quaternion $\quat Q$.
\end{definition}


%\subsection{Matrix Form of Quaternion Multiplication}\label{sec:app_matrix_mulop}
%
%Just as the vector cross product $\vect a \times \vect b$ can be written in matrix form as
%\begin{align}
%  \vect a \times \vect b &= \MxV{\boldsymbol{\mathop{Sk}}(\vect a)} {\vect b} \\
%\intertext{where $\boldsymbol{\mathop{Sk}} (\vect a)$ is the skew-symmetric matrix generated from %$\vect a$:}
%  \boldsymbol{\mathop{Sk}} (\vect a) & =
%    \begin{bmatrix} 0 & -a_z & a_y \\ a_z & 0 & -a_x \\ -a_y & a_x & 0\end{bmatrix}
%\end{align}
%
%The quaternion product can similarly be expressed in matrix form:
%\begin{align}
%  \QxQ{\quat Q_a} {\quat Q_b} &= \MxV{\mat{M}_{\quat{Q}}(\quat Q_a)} {\quat Q_b}
%\intertext{where $\mat{M}_{\quat{Q}}(\quat Q_a)$ is the $4x4$ skew-symmetric matrix generated from %$\quat Q_a$:}
%  \mat{M}_{\quat{Q}}(\quat Q) & =
%    \begin{bmatrix}
%      q_s &                     -q_x &                    -q_y &                     -q_z \\
%      q_x & \phantom{-}q_s &                    -q_z & \phantom{-}q_y \\
%      q_y & \phantom{-}q_z & \phantom{-}q_s &                    -q_x \\
%      q_z &                     -q_y & \phantom{-}q_x & \phantom{-}q_s\end{bmatrix} \label{eqn:quat_MQdef}
%\end{align}
 
