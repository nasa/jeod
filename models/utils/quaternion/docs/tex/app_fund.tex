 \section{Quaternion Fundamentals}\label{sec:app_fund}
  
 The quaternions are an extension of the complex numbers
 first invented by William Rowan Hamilton with three distinct square roots of $-1$, which are
 typically denoted as $i$, $j$, and $k$.

\subsection{Fundamental Formula}

The quaternion imaginary units obey Hamilton's Fundamental Formula of Quaternion Algebra,

\begin{equation}
i^2 = j^2 = k^2 = ijk = -1 \label{eqn:quat_fundamental_formula}
\end{equation}

An immediate consequence of this definition is that multiplication of quaternion imaginary units is not commutative:
\begin{subequations}
\begin{align}
ij &= k = -ji \\
jk &= i = -kj \\
ki &= j = -ik
\end{align}
\end{subequations}


\subsection{Quaternion Representation}\label{sec:app_rep}

Just as  a complex number can be represented as a sum of real and imaginary parts, a quaternion can also be written as a linear combination of real and imaginary quaternion parts:
\begin{align}
\quat Q &= q_s + q_i i + q_j j + q_k k \label{eqn:quat_rep_explicit} \\
\intertext{or more compactly as the four-vector}
\quat Q &= \begin{bmatrix}q_s \\ q_i \\ q_j \\ q_k \end{bmatrix} \label{eqn:quat_rep_4vector}\\
\intertext{or even more compactly by representing the imaginary quaternion part as a vector:}
\quat Q &=\quatsv {q_s} {\vect{q_v}} \label{eqn:quat_rep_scalar_vector}\\
\intertext{where $\vect{q_v}$ comprises the imaginary quaternion part of $\quat Q$:}
\vect{q_v} &= \begin{bmatrix} q_i \\ q_j \\ q_k \end{bmatrix} \label{eqn:quat_vector_part}
\end{align}

JEOD represents quaternions in the scalar $+$ vector form.
This representation scheme is used in this document.

\subsection{Special Cases}\label{sec:app_scases}

Quaternions of particular interest are:

\begin{itemize}
\item
The quaternion whose scalar and vector parts are identically zero
is the \emph{zero quaternion}.
\item
A quaternion whose vector part is identically zero
is a \emph{pure real quaternion}.
\item
A quaternion whose scalar part is identically zero
is a \emph{pure imaginary quaternion}.
\item
A quaternion of the form $\quat Q = \quatsv {q_s} {\vect{q_v}}$ with
${q_s}^2 +  \vect{q_v} \cdot \vect{q_v} = 1$
is a \emph{unit quaternion}.
\item
The quaternion $\quatsv 1 {\vect{0}}$
is the \emph{real unit quaternion}.
\end{itemize}
 
