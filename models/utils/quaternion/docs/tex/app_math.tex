%%%%%%%%%%%%%%%%%%%%%%%%%%%%%%%%%%%%%%%%%%%%%%%%%%%%%%%%%%%%%%%%%%%%%%%%%
%
% Purpose: Quaternion math description
%
% 
%%%%%%%%%%%%%%%%%%%%%%%%%%%%%%%%%%%%%%%%%%%%%%%%%%%%%%%%%%%%%%%%%%%%%%%%%


\chapter{Quaternion Mathematics}\label{sec:app_math}

The JEOD uses unit quaternions\cite{kuipers:2002} as one of several means
for representing rotations and transformations in three space.
This section provides an overview on the use of quaternions in the
JEOD.  The contents of this section are:
\begin{itemize}
\item Section~\ref{sec:app_nomen} defines the nomenclature used in this
chapter.
\item Section~\ref{sec:app_fund} establishes the fundamentals of
quaternions.
\item Section~\ref{sec:app_arith} develops elementary quaternion
arithmetic.
\item Section~\ref{sec:app_exp} describes the extension of the exponential
and logarithm
to the quaternions.
\item Section~\ref{sec:app_rot_trans} describes how quaternions are used to
represent transformations and rotations.
\item Section~\ref{sec:app_time_deriv} develops the time derivative of a
transformation quaternion.
\item Sections~\ref{sec:app_att_prop} and~\ref{sec:app_rate_prop}
describes techniques for propagating vehicle attitude and attitude rate
using quaternions.
\end{itemize}

%%% Nomenclature
\pagebreak

\section{Nomenclature}\label{sec:app_nomen}

\providecommand{\acchdr}[1]{%
\multicolumn{4}{l}{\rule{0pt}{3ex}\parbox[c]{0.6\textwidth}{#1}} \\
\cline{2-4} \rule{1.5em}{0pt} & {\bf Description} & {\bf Nomenclature} & {\bf Example}\\ \cline{2-4} \cline{2-4}}

\begin{table*}[h]
\begin{tabular}{l||l|l|l|}
\acchdr{{\bf{Display style for vectors, matrices, and quaternions}}}
\rule{0pt}{3ex} & Scalar &
  in plain math font & $\omega$ \\
\rule{0pt}{1.5ex} & Vector &
  in bold math font& $\vect \omega$ \\
\rule{0pt}{1.5ex} & Matrix &
  in bold math font& $\mat T$ \\
\rule{0pt}{1.5ex} & Quaternion&
  in caligraphy font, uppercase & $\quat Q$ \\
\cline{2-4}
\acchdr{{\bf{Vector Adornments}}}
\rule{0pt}{3ex} & Vector from $a$ to $b$ &
 Arrow-separated subscript & $\relvect x a b$ \\
\rule{0pt}{3ex} & Vector from origin to $b$ &
  Subscript on right & $\absvect x b$ \\
\rule{0pt}{3ex} & Vector in frame $A$ &
  Subscript on right & $\framevect A x$ \\
\rule{0pt}{3ex} & Vector time derivative in frame $A$&
  Frame to right of dot & $\framedot A {\vect x}$ \\
\cline{2-4}
\acchdr{{\bf{Matrix Adornments}}}
\rule{0pt}{3ex} & Transformation from $A$ to $B$ &
  Arrow-separated subscript & $\tmat A B$ \\
\rule{0pt}{3ex} & Matrix product &
  No operator & $\MxM{\tmat B C}{\tmat A B}$ \\
\rule{0pt}{3ex} & Matrix transpose &
  Superscript $\top$ & $\matT T$ \\
\cline{2-4}
\acchdr{{\bf{Quaternion Adornments}}}
\rule{0pt}{3ex} & Transformation from $A$ to $B$ &
  Arrow-separated subscript & $\tquat A B$ \\
\rule{0pt}{3ex} & Quaternion product &
  No operator & $\QxQ{\tquat B C}{\tquat A B}$ \\
\rule{0pt}{3ex} & Quaternion conjugate &
  Superscript $\star$ & $\quatconj Q$ \\
\rule{0pt}{3ex} & Quaternion components &
  Scalar $+$ vector $or$ four-vector&
  $\quatsv{q_s}{\vect{q_v}} \;or\; \bmatrix q_s \\ q_x \\ q_y \\ q_z\endbmatrix$ \\
\cline{2-4}
\end{tabular}
\end{table*}

\pagebreak
 


%%% Fundamentals
 \section{Quaternion Fundamentals}\label{sec:app_fund}
  
 The quaternions are an extension of the complex numbers
 first invented by William Rowan Hamilton with three distinct square roots of $-1$, which are
 typically denoted as $i$, $j$, and $k$.

\subsection{Fundamental Formula}

The quaternion imaginary units obey Hamilton's Fundamental Formula of Quaternion Algebra,

\begin{equation}
i^2 = j^2 = k^2 = ijk = -1 \label{eqn:quat_fundamental_formula}
\end{equation}

An immediate consequence of this definition is that multiplication of quaternion imaginary units is not commutative:
\begin{subequations}
\begin{align}
ij &= k = -ji \\
jk &= i = -kj \\
ki &= j = -ik
\end{align}
\end{subequations}


\subsection{Quaternion Representation}\label{sec:app_rep}

Just as  a complex number can be represented as a sum of real and imaginary parts, a quaternion can also be written as a linear combination of real and imaginary quaternion parts:
\begin{align}
\quat Q &= q_s + q_i i + q_j j + q_k k \label{eqn:quat_rep_explicit} \\
\intertext{or more compactly as the four-vector}
\quat Q &= \begin{bmatrix}q_s \\ q_i \\ q_j \\ q_k \end{bmatrix} \label{eqn:quat_rep_4vector}\\
\intertext{or even more compactly by representing the imaginary quaternion part as a vector:}
\quat Q &=\quatsv {q_s} {\vect{q_v}} \label{eqn:quat_rep_scalar_vector}\\
\intertext{where $\vect{q_v}$ comprises the imaginary quaternion part of $\quat Q$:}
\vect{q_v} &= \begin{bmatrix} q_i \\ q_j \\ q_k \end{bmatrix} \label{eqn:quat_vector_part}
\end{align}

JEOD represents quaternions in the scalar $+$ vector form.
This representation scheme is used in this document.

\subsection{Special Cases}\label{sec:app_scases}

Quaternions of particular interest are:

\begin{itemize}
\item
The quaternion whose scalar and vector parts are identically zero
is the \emph{zero quaternion}.
\item
A quaternion whose vector part is identically zero
is a \emph{pure real quaternion}.
\item
A quaternion whose scalar part is identically zero
is a \emph{pure imaginary quaternion}.
\item
A quaternion of the form $\quat Q = \quatsv {q_s} {\vect{q_v}}$ with
${q_s}^2 +  \vect{q_v} \cdot \vect{q_v} = 1$
is a \emph{unit quaternion}.
\item
The quaternion $\quatsv 1 {\vect{0}}$
is the \emph{real unit quaternion}.
\end{itemize}
 


%%% Arithmetic
\section{Quaternion Arithmetic}\label{sec:app_arith}

\subsection{Basic Quaternion Operations}

This section develops the basic arithmetic operations on quaternions.
In this section,
$\quat Q$, $\quat{Q}_1$, and $\quat{Q}_2$
are quaternions defined with components
\begin{align*}
\quat Q    &= q_s + q_i i + q_j j + q_k k
            = \quatsv {q_s} {\vect{q_v}} \\
\quat{Q}_1 &= q_{1_s} + q_{1_i} i + q_{1_j} j + q_{1_k} k 
            = \quatsv{q_{1_s}}{\vect{q_{1_v}}} \\
\quat{Q}_2 &= q_{2_s} + q_{2_i} i + q_{2_j} j + q_{2_k} k
            = \quatsv{q_{2_s}}{\vect{q_{2_v}}}
\end{align*}

Quaternion addition, scaling a quaternion by a real, and the quaternion norm
are defined as isomorphisms to the corresponding operations on a $4-$vector:
\begin{definition}[Quaternion addition]\label{def:quat_sum}
Given quaternions $\quat{Q}_1$ and $\quat{Q}_2$,
the quaternion sum $\quat{Q}_1 + \quat{Q}_2$ is defined as
\begin{equation}
{\quat{Q}_1} + {\quat{Q}_2} \equiv
         q_{1_s} + q_{2_s} 
    + (q_{2_i}  + q_{2_i}))\,i
    + (q_{2_j}  + q_{2_j}))\,j
    + (q_{2_k}  + q_{2_k}))\,k
    = \quatsv{q1_s+q2_s} {\vect{q_{1_v}}+\vect{q_{2_v}}}
  \label{eqn:quat_sum_def}
\end{equation}
\end{definition}

\begin{definition}[Quaternion scaling]\label{def:quat_scale}
Given a quaternion $\quat Q$ and a real number $s$,
the product of $\quat Q$ with $s$ is defined as
\begin{equation}
s \quat Q \equiv s q_s + s q_i i + s q_j j + s q_k k
  = \quatsv {s q_s} {s \vect{q_v}}
\label{eqn:quat_prod_scalar}
\end{equation}
\end{definition}

\begin{definition}[Quaternion norm]\label{def:quat_norm}
Given a quaternion $\quat Q$,
the quaternion norm of $\quat Q$ is defined as
\begin{equation}
\norm{\quat Q} \equiv \sqrt{q_s^2 + q_i^2 + q_j^2 + q_k^2}
  = \sqrt{q_s^2 + \vect{q_{v}} \!\cdot \vect{q_{v}}}
\label{eqn:quat_norm_def}
\end{equation}
\end{definition}

The quaternion conjugate is defined analogously to the complex conjugate:
\begin{definition}[Quaternion conjugate]\label{def:quat_conj}
Given a quaternion $\quat Q$,
the quaternion conjugate of $\quat Q$,
denoted herein as either $\conjop \quat Q$ or $\quatconj Q$,
is defined as
\begin{equation}
\conjop{\quat Q} \equiv
\quatconj Q \equiv q_s - q_i i - q_j j - q_k k
  = \quatsv {q_s} {- \vect{q_v}}
\label{eqn:quat_conj_def}
\end{equation}
\end{definition}

The definition of the quaternion product follows from
Hamilton's Fundamental Formula of Quaternion Algebra,
equation~\eqref{eqn:quat_fundamental_formula}:
\begin{definition}[Quaternion multiplication]\label{def:quat_prod}
Given quaternions $\quat{Q}_1$ and $\quat{Q}_2$,
the quaternion product $\QxQ{\quat{Q}_1}{\quat{Q}_2}$ is defined as
\begin{equation}
\begin{split}
\QxQ{\quat{Q}_1}{\quat{Q}_2} \equiv&\;
        q_{1_s} q_{2_s} -
        (q_{1_i} q_{2_i} + q_{1_j} q_{2_j} + q_{1_k} q_{2_k}) \\
    &+ (q_{1_s} q_{2_i}  + q_{2_s} q_{1_i} +
        (q_{1_j} q_{2_k} - q_{1_k} q_{2_j}))\,i \\
    &+ (q_{1_s} q_{2_j}  + q_{2_s} q_{1_j} +
        (q_{1_k} q_{2_i} - q_{1_i} q_{2_k}))\,j \\
    &+ (q_{1_s} q_{2_k}  + q_{2_s} q_{1_k} +
        (q_{1_i} q_{2_j} - q_{1_j} q_{2_i}))\,k
  \end{split}  \label{eqn:quat_prod_explicit}
\end{equation}
or in scalar $+$ vector form,
\begin{equation}
\QxQ{\quat{Q}_1}{\quat{Q}_2} = 
  \quatsv
    {q_{1_s} q_{2_s} - \vect{q_{1_v}} \!\cdot \vect{q_{2_v}}}
    {q_{1_s} \vect{q_{2_v}} +
     q_{2_s} \vect{q_{1_v}} +
     \vect{q_{1_v}} \!\times \vect{q_{2_v}}}
  \label{eqn:quat_prod_short}
\end{equation}
\end{definition}

Note that definitions~\ref{def:quat_scale} and~\ref{def:quat_prod} are
consistent: The product of a quaternion of the form
$\quatsv {q_s}{\vect{q_v}}$ and a real quaternion
$\quatsv {s}{\vect 0}$ is
\begin{equation*}
  \QxQ{\quatsv {q_s}{\vect{q_v}}}{\quatsv {s}{\vect 0}}
  = \QxQ{\quatsv {s}{\vect 0}}{\quatsv {q_s}{\vect{q_v}}}
  = \quatsv {s q_s}{s \vect {q_v}} = s \quat Q
\end{equation*}

\begin{corollary}\label{thm:quat_quat_conj_prod}
The product of a quaternion and its conjugate is the square of
the norm of the quaternion.
\end{corollary}

\begin{proof}
The product of a quaternion $\quat Q = \quatsv {q_s}{\vect{q_v}}$
and its conjugate is
\begin{align*}
  \QxQ{\quat Q}{\quatconj Q} &=
  \QxQ{\quatsv {q_s}{\vect{q_v}}}{\quatsv {q_s}{- \vect{q_v}}}  \\
  &= \quatsv {q_s^2 + \vect{q_v} \cdot \vect{q_v}}{\vect 0}
\end{align*}
\end{proof}


\subsection{Commutativity, Associativity, and Distributivity}

\begin{theorem}\label{thm:quat_sum_and_scale_commute}
Quaternion addition and scaling a quaternion by a real scalar
are commutative operations.
\end{theorem}
\begin{proof}
The quaternion sum and the product of a quaternion and a real are defined as
isomorphisms to the corresponding commutative $4-$vector operations.
\end{proof}

Note that the product of two quaternions does not in general commute.
This follows directly from the Fundamental Formula of Quaternion Algebra:
$ij = k, ji = -k$.

\begin{theorem}\label{thm:quat_prod_assoc}
Quaternion multiplication is associative:
\begin{equation*}
  \QxQ{(\QxQ{\quat{Q}_1}{\quat{Q}_2})}{\quat{Q}_3} =
    \QxQ{\quat{Q}_1}{(\QxQ{\quat{Q}_2}{\quat{Q}_3})}
\end{equation*}
\end{theorem}
\begin{proof}
This can be easily proven to be the case for three quaternions.
Extending the theorem to a product of any number of quaternions and to any
grouping of parentheses in the product is accomplished by induction.
\end{proof}

\begin{theorem}\label{thm:quat_prod_distrib}
Quaternion multiplication distributes over quaternion addition:
\begin{equation}
  \QxQ{\quat Q _1}{\left({\quat Q _2 + \quat Q _3}\right)} =
    \QxQ{\quat Q _1}{\quat Q _2} + {\quat Q _1}{\quat Q _3}
\end{equation}
\end{theorem}
This can be easily proven true for any three arbitrary quaternions
$\quat Q _1$, $\quat Q _2$, and $\quat Q _3$ by applying the
definitions of quaternion addition and multiplication.

\subsection{Quaternion Identity, Inverse, and Division}

\begin{theorem}\label{thm:quat_identity}
The real unit quaternion is the identity element for quaternion multiplication.
\end{theorem}
\begin{proof}
The left and right products of any quaternion
$\quatsv {q_s}{\vect{q_v}}$ and the real unit quaternion $\quatsv 1 {\vect 0}$
reproduce the quaternion $\quat Q$:
\begin{equation*}
  \QxQ{\quatsv {q_s}{\vect{q_v}}}{\quatsv 1 {\vect 0}} =
  \QxQ{\quatsv 1 {\vect 0}}{\quatsv {q_s}{\vect{q_v}}} =
  \quatsv {q_s}{\vect{q_v}}
\end{equation*}
The real unit quaternion is thus both a left- and right- identity element.
Since quaternion multiplication is associative,
the multiplicative identity is unique.
The real unit quaternion is thus the unique identity element
for quaternion multiplication.
\end{proof}

\begin{theorem}\label{thm:quat_inverse}
The inverse $\quat Q ^ {-1} $ of a non-zero quaternion $\quat Q$ is
\begin{equation}
\quat Q ^ {-1} = \frac{1}{\quat Q}
  \equiv \frac{1}{\norm{\quat Q}^2} \quatconj Q
  = \frac{1}{\QxQ {\quat Q} {\quatconj Q}} \quatconj Q
\label{eqn:quat_inverse_def}
\end{equation}
\end{theorem}
\begin{proof}
\begin{align*}
\intertext{Let}
  \quat Q &= \quatsv {q_s}{\vect{q_v}} \\
  \quat Q_2 &= \frac{1}{\QxQ {\quat Q} {\quatconj Q}} \quatconj Q \\
\intertext{then}
  \QxQ {\quat Q}{\quat Q_2}
    &= \frac{1}{\QxQ {\quat Q} {\quatconj Q}}
      \quatsv {q_s^2 + \vect{q_v} \cdot \vect{q_v}} {\vect 0} \\
    &= \quatsv 1 {\vect 0}
\end{align*}
Thus the stated quaternion is a right multiplicative inverse.
Since quaternion multiplication is associative,
the stated quaternion is also a left multiplicative inverse and is unique.
\end{proof}

\begin{corollary}\label{thm:quat_unit_inverse_is_conjugate}
The multiplicative inverse of a unit quaternion is the quaternion conjugate.
\end{corollary}
\begin{proof}
This follows directly from the definitions of the quaternion multiplicative
inverse and the quaternion conjugate.
\end{proof}

Quaternion division is defined in terms of multiplication with the quaternion
inverse.
Since quaternion multiplication is not commutative,
pre-multiplying and post-multiplying a quaternion
with the inverse of another quaternion yield different results.
This observation leads to the definition of a left and right division operator:
\begin{definition}[Quaternion division]\label{def:quat_div}
\begin{align}
  \quat Q_1 \setminus \quat Q_2 &\equiv \QxQ{{\quat Q_2}^{-1}} {\quat Q_1} \\
  \quat Q_1 / \quat Q_2 &\equiv \QxQ{\quat Q_1} {{\quat Q_2}^{-1}}
\end{align}
\end{definition}

\subsection{Quaternion Decomposition}\label{sec:app_decomp}

This section develops an alternative representation scheme for quaternions.
\begin{definition}\label{def:quat_decomp}
A non-zero quaternion represented in terms of
its magnitude $s\in\mathbb R$,
a unit vector $\vhat u\in\mathbb R^3$,
and a rotation angle $\theta\in\mathbb R$,
\begin{equation}
  \quat Q = s\quatsv{\cos\theta}{\sin\theta\vhat u}\label{eqn:quat_decomp_def}
\end{equation}
is a \emph{decomposition} of the quaternion.
\end{definition}

\begin{corollary}\label{thm:quat_decomp_multivalued}
The decomposition of a quaternion not unique.
\end{corollary}
\begin{proof}
if $s\quatsv{\cos\theta\phantom{\vhat u}}{\sin\theta \vhat u} = \quat Q$
then so does
$s\quatsv{\cos(\theta+2n\pi)\phantom{\vhat u}}{\sin(\theta+2n\pi) \vhat u}$
for all integer values of $n$.
\end{proof}

\begin{corollary}\label{thm:quat_real_unit_decomp_indeterminate}
The decomposition of the real unit quaternion $\quatsv 1{\vect 0}$ is
a trivial rotation ($\theta=0$) about an indeterminate axis.
\end{corollary}
\begin{proof}
Applying equation~\eqref{eqn:quat_decomp_def} with $s=1$ and $\theta=0$
generates the real unit quaternion regardless of the value of $\vhat u$.
\end{proof}

\begin{corollary}\label{thm:quat_negation_rotates_by_pi}
Given a quaternion $\quat Q$ with decomposition
$\quat Q = s\quatsv{\cos\theta}{\sin\theta \vhat u}$,
a decomposition of the additive inverse of $\quat Q$ is
$-\quat Q = s\quatsv{\cos(\theta+\pi)}{\sin(\theta+\pi) \vhat u}$,
\end{corollary}
\begin{proof}
Expanding $\quatsv{\cos(\theta+\pi)}{\sin(\theta+\pi) \vhat u}$
results in $-\quat Q$.
\end{proof}

\begin{algorithm}[Unit quaternion decomposition]\label{alg:quat_unit_decomp}
Let $\quat Q = \quatsv{q_s}{\vect{q_v}}$ be a unit quaternion
(${q_s}^2 + \vect{q_v} \cdot \vect{q_v} = 1$) with non-zero vector part. Then
\begin{align}
  \theta &=
    \arctan(\norm {\vect{q_v}}, q_s) \label{eqn:quat_unit_decomp_theta} \\
  \vhat u &=
    \frac {\vect{q_v}}{\norm {\vect{q_v}}} \label{eqn:quat_unit_decomp_uhat}
\end{align}
is a \emph{decomposition} of the unit quaternion.
\paragraph{Note}
The two-argument inverse tangent function in
equation~\eqref{eqn:quat_unit_decomp_theta}
computes the inverse tangent of $y/x$, with the signs of $y$ and $x$
dictating the quadrant of the result in which the result is placed.
The C-library function \texttt{atan2} approximates $\arctan(y,x)$.
The single-argument inverse cosine and inverse sine functions
can be used as an alternative to \texttt{atan2}.
However, care must be taken to use the function that produces
better accuracy and to place the angle in the correct quadrant:
\begin{equation}
  \theta =
    \begin{cases}
      \pi - \arcsin(\norm {\vect{q_v}}) &
        \text{if %
              $\phantom{-\frac{\surd 2}2 \leq\;}
               q_s < -\frac{\surd 2}2$} \\
      \arccos(q_s) &
         \text{if %
               $-\frac{\surd 2}2 \leq
                q_s \leq \phantom{-}\frac{\surd 2}2$} \\
      \arcsin(\norm {\vect{q_v}}) &
        \text{if %
              $\phantom{-\frac{\surd 2}2 \leq\;}
               q_s > \phantom{-}\frac{\surd 2}2$}
    \end{cases}\label{eqn:quat_unit_decomp_theta_alt}
\end{equation}
Which alternative (equation~\eqref{eqn:quat_unit_decomp_theta} or
~\eqref{eqn:quat_unit_decomp_theta_alt}) is faster and/or more accurate
is machine-dependent.
\end{algorithm}
\begin{proof}
\begin{align*}
  \vect{q_v} &= \norm{\vect{q_v}} \frac{\vect{q_v}}{\norm{\vect{q_v}}} \\[5pt]
  &= \sqrt{1-{q_s}^2}\frac{\vect{q_v}}{\norm{\vect{q_v}}}
  \quad \text{since $\norm{\vect{q_v}}^2 = 1-{q_s}^2$} \\
\intertext{thus}
  \cos\theta &= q_s \\
  \sin\theta &= \sqrt{1-{q_s}^2} = \norm{\vect q_v} \\
\intertext{Applying equation~\eqref{eqn:quat_decomp_def} with $s=1$,}
  \quatsv{\cos\theta}{\sin\theta\vhat u}
    &= \quatsv{q_s}{\norm{\vect q_v} \frac{\vect{q_v}}{\norm {\vect{q_v}}}} \\
    &= \quat Q
\end{align*}
\end{proof}

\begin{corollary}\label{thm:quat_unit_decomp}
A decomposition exists for all unit quaternions.
\end{corollary}
\begin{proof}
Corollaries~\ref{thm:quat_real_unit_decomp_indeterminate}
and~\ref{thm:quat_negation_rotates_by_pi}
provide decompositions for the two unit quaternions with zero vector parts.
Since $\arctan(y,x)$ is defined for all finite values of $x$ and $y$,
algorithm ~\ref{alg:quat_unit_decomp}
provides a decomposition for all unit quaternions with non-zero vector parts.
\end{proof}

\begin{corollary}\label{thm:quat_decomp}
A decomposition exists for all quaternions.
\end{corollary}
\begin{proof}
The zero quaternion has an indeterminate decomposition: $s=0$ and
$\theta$ and $\vhat u$ are indeterminate.
Represent a non-zero quaternion $\quat Q$ as
$\norm{\quat Q} \frac{\quat Q}{\norm{\quat Q}}$.
The unit quaternion $\frac{\quat Q}{\norm{\quat Q}}$ has some decomposition
$(s=1, \theta, \vhat u)$ by corollary~\ref{thm:quat_unit_decomp}.
The decomposition $(s=\norm{\quat Q}, \theta, \vhat u)$ generates the
original quaternion $\quat Q$.
\end{proof}

\begin{corollary}\label{thm:quat_decomp_inverse}
The multiplicative inverse of a quaternion $\quat Q$ is
\begin{equation}
  \quat Q^{-1} =
    \frac{1}{s} \quatsv {\cos \theta}{- \sin \theta \vhat u}
  \label{eqn:quat_decomp_inverse}
\end{equation}
where $s$, $\theta$, and $\vhat u$ are a decomposition of the
quaternion $\quat Q$.
\end{corollary}
\begin{proof}
Multiplying the decomposition of $\quat Q$ and the inverse as expressed in
equation~\eqref{eqn:quat_decomp_inverse} yields the real unit quaternion.
\end{proof}

\subsection{Quaternion Exponentiation}\label{sec:app_exponentiation}

Raising a non-zero quaternion $\quat Q$ to a real integer power
is defined in terms of iterated multiplication,
analogous to the definition for real and complex numbers:
\begin{definition}[Integer Power of a Quaternion]\label{def:quat_int_power}
\begin{equation}
\quat Q^n \equiv
\begin{cases}
  \prod_{r=1}^n \quat Q
    & \text{if $n>0$} \\[5pt]
  \quatsv 1 {\vect 0}
    & \text{if $n =0$} \\[12pt]
  \left(\quat Q^{-1}\right)^n = \prod_{r=1}^n \quat Q^{-1}
    & \text{if $n<0$}
\end{cases}\label{eqn:quat_int_power_def}
\end{equation}
\end{definition}
Since quaternion multiplication is associative
(theorem~\ref{thm:quat_prod_assoc}),
the products in equation~\eqref{eqn:quat_int_power_def} are well-defined.

\begin{theorem}\label{thm:quat_rat_power}
The result of raising a quaternion $\quat Q$ to a rational power $r$ is
\begin{equation}
  \quat Q^r =
    s^r \quatsv {\cos r\theta}{\sin r\theta \vhat u}
  \label{eqn:quat_rat_power}
\end{equation}
where $s$, $\theta$, and $\vhat u$ are a decomposition of the
quaternion $\quat Q$.
\end{theorem}
\begin{proof}
The theorem is clearly true for $r=0$,
since equations~\eqref{eqn:quat_int_power_def} and~\eqref{eqn:quat_rat_power}
yields the real unit quaternion for $r=0$.
Assuming the theorem is true for some integer $r \neq 0$, then
\begin{align*}
  \quat Q^{r+1} &= \QxQ{\quat Q^r}{\quat Q} \\
  &=
    \QxQ
      {s^r \quatsv {\cos r\theta}{\sin r\theta \vhat u}}
      {s \quatsv {\cos \theta}{\sin \theta \vhat u}} \\
    &= s^{r+1}
    \quatsv
      {\cos\left(\left(r+1\right)\theta\right)}
      {\sin\left(\left(r+1\right)\theta\right) \vhat u}
\intertext{and}
  \quat Q^{r-1} &= \QxQ{\quat Q^r}{\quat Q^{-1}} \\
  &=
    \QxQ
      {s^r \quatsv {\cos r\theta}{\sin r\theta \vhat u}}
      {s^{-1} \quatsv {\cos \theta}{-\sin \theta \vhat u}} \\
    &= s^{r-1}
    \quatsv
      {\cos\left(\left(r-1\right)\theta\right)}
      {\sin\left(\left(r-1\right)\theta\right) \vhat u}
\end{align*}
By induction, the theorem is true for all integers.

Now consider the $n^{th}$ root of $\quat Q$
as defined by equation~\eqref{eqn:quat_rat_power}. Raising this to
the integral $n^{th}$ power yields
\begin{align*}
  \left(s^{1/n}
    \quatsv {\cos \frac \theta n}{\sin \frac \theta n \vhat u}\right)^n
  &=
  \left(s^{1/n}\right)^n
    \quatsv {\cos n \frac \theta n}{\sin n \frac \theta n \vhat u} \\
  &= \quat Q
\end{align*}
The theorem has now been proven for all integers and
all rationals of the form $1/n$.
It is therefore valid for any number of the form $p/q$,
where $p$ and $q$ are integers.
\end{proof}

By continuity, equation~\eqref{eqn:quat_rat_power} is extended to cover
all the real numbers. 
\begin{definition}[Real power of a quaternion]\label{def:quat_exponentiation}
The result of raising a quaternion $\quat Q$ to a real power $a$ is
\begin{equation}
  \quat Q^a =
    s^a \quatsv {\cos a\theta}{\sin a\theta \vhat u}
  \label{eqn:quat_power}
\end{equation}
where $s$, $\theta$, and $\vhat u$ are a decomposition of the
quaternion $\quat Q$.
\end{definition}


%\subsection{Matrix Form of Quaternion Multiplication}\label{sec:app_matrix_mulop}
%
%Just as the vector cross product $\vect a \times \vect b$ can be written in matrix form as
%\begin{align}
%  \vect a \times \vect b &= \MxV{\boldsymbol{\mathop{Sk}}(\vect a)} {\vect b} \\
%\intertext{where $\boldsymbol{\mathop{Sk}} (\vect a)$ is the skew-symmetric matrix generated from %$\vect a$:}
%  \boldsymbol{\mathop{Sk}} (\vect a) & =
%    \begin{bmatrix} 0 & -a_z & a_y \\ a_z & 0 & -a_x \\ -a_y & a_x & 0\end{bmatrix}
%\end{align}
%
%The quaternion product can similarly be expressed in matrix form:
%\begin{align}
%  \QxQ{\quat Q_a} {\quat Q_b} &= \MxV{\mat{M}_{\quat{Q}}(\quat Q_a)} {\quat Q_b}
%\intertext{where $\mat{M}_{\quat{Q}}(\quat Q_a)$ is the $4x4$ skew-symmetric matrix generated from %$\quat Q_a$:}
%  \mat{M}_{\quat{Q}}(\quat Q) & =
%    \begin{bmatrix}
%      q_s &                     -q_x &                    -q_y &                     -q_z \\
%      q_x & \phantom{-}q_s &                    -q_z & \phantom{-}q_y \\
%      q_y & \phantom{-}q_z & \phantom{-}q_s &                    -q_x \\
%      q_z &                     -q_y & \phantom{-}q_x & \phantom{-}q_s\end{bmatrix} \label{eqn:quat_MQdef}
%\end{align}
 


%%% Exponential
\section{Quaternion Exponential}\label{sec:app_exp}

This section develops the exponential of a quaternion.
The exponential function is the most important function in mathematics.
It is defined, for every complex number $z$,
by $\exp z = \sum_{n=0}^\infty \frac{z^n}{n!}$.
Extending this definition to the quaternions,
\begin{definition}[Quaternion Exponential]\label{def:quat_exp}
The quaternion exponential function is defined as
\begin{equation}
\exp \quat Q \equiv \sum_{n=0}^\infty \frac{\quat Q^n}{n!}\label{eqn:quat_expq}
\end{equation}
\end{definition}

The complex exponential maps the pure imaginary numbers to the unit circle via
Euler's Equation, $e^{i\theta} = \cos \theta + i \sin \theta$.
The following theorem demonstrates that the quaternion exponential
has a similar mapping function when applied to pure imaginary quaternions.

\begin{theorem}\label{thm:quat_exp_imQ}
Let $\quat Q$ be a pure imaginary quaternion of the form
$\quatsv 0 {\theta \vhat u}$. \\
Then
\begin{equation}
  \exp \quat Q = \quatsv {\cos \theta} {\sin \theta \vhat u}\label{eqn:quat_exp_imQ_Euler}
\end{equation}
\end{theorem}
\begin{proof}
\begin{align*}
\intertext{The quaternion as defined above has decomposition}
  \quat Q &\equiv \quatsv 0 {\theta \vhat u}
          = \theta \quatsv {\cos \frac \pi 2} {\sin \pi 2 \vhat u} \\
\intertext{By equation~\eqref{eqn:quat_rat_power},}
  \quat Q ^{2n} &= \theta^{2n}
                   \quatsv {\cos ((2n) \frac \pi 2)}
                           {\sin ((2n) \frac\pi 2) \vhat u} \\
                &= (-1)^n \theta^{2n} \quatsv 1 {\vect 0} \\
  \quat Q ^{2n+1} &= \theta^{2n+1}
                     \quatsv {\cos ((2n+1) \frac \pi 2)}
                             {\sin ((2n+1) \frac\pi 2) \vhat u} \\
                &= (-1)^n \theta^{2n+1}  \quatsv 0 {\vhat u} \\
\intertext{Partitioning the sum equation~\eqref{def:quat_exp}
           into even and odd powers of $n$ yields}
\exp \quat Q &= \sum_{n=0}^\infty \frac{\quat Q^{2n}}{(2n)!} +
                \sum_{n=0}^\infty \frac{\quat Q^{2n+1}}{(2n+1)!} \\
\intertext{and thus}
  \exp\left.\quatsv 0 {\theta \vhat u} \right. &=
    \left(\sum_{n=0}^\infty \frac {(-1)^n} {(2n)!} \theta^{2n} \right)
      \quatsv 1 {\vect 0} +
    \left(\sum_{n=0}^\infty \frac {(-1)^n} {(2n+1)!} \theta^{2n+1} \right)
      \quatsv 0 {\vhat u} \\
  &= \quatsv {\cos \theta} {\sin \theta \vhat u}
\end{align*}
\end{proof}

\begin{corollary}\label{thm:quat_exp}
Let $\quat Q$ be a quaternion of the form $\quatsv {q_s} {q_v \vhat u}$. \\
Then
\begin{align}
  \exp \quat Q &=
    \exp q_s \quatsv {\cos q_v} {\sin q_v \vhat u}
  \label{eqn:quat_exp_general} \\
\intertext{where}
  \exp q_s\;&\text{is the real exponential function of the real scalar $q_s$}
  \nonumber
\end{align}
\end{corollary}
\begin{proof}
Since the product of a quaternion and a real quaternion commutes,
\begin{align*}
  \quat Q^{n} &=
    \sum_{r=0}^n
      \frac{n!}{r!(n-r)!}
      \quatsv {q_s} {\vect 0}^r \; \quatsv {0} {q_v \vhat u}^{n-r} \\
\intertext{and thus}
  \exp \quat Q &=
    \sum_{n=0}^\infty \sum_{r=0}^n
      \frac1{r!(n-r)!}
      \quatsv {q_s} {\vect 0}^r \; \quatsv {0} {q_v \vhat u}^{n-r} \\
\intertext{Collecting like powers of $\quatsv {q_s} {\vect 0}$ %
           and setting $n-r\equiv s$,}
  \exp \quat Q &=
      \left(\sum_{r=0}^\infty \frac1{r!} \quatsv {q_s} {\vect 0} ^r\right)
      \left(\sum_{s=0}^\infty \frac1{s!} \quatsv {0} {q_v \vhat u} ^s\right) \\
    &= \exp q_s \quatsv {\cos q_v} {\sin q_v \vhat u}
\end{align*}
\end{proof}

The quaternion logarithm function is defined as the inverse of the
quaternion exponential:
\begin{definition}[Quaternion logarithm]\label{def:quat_log}
\begin{equation*}
  \quat Q_2 = \log \quat Q \iff \exp \quat Q_2 = \quat Q
\end{equation*}
\end{definition}
Note that the quaternion logarithm of the zero quaternion is undefined
since there is no quaternion $\quat Q$ such that $\exp \quat Q = \vect 0$.

\begin{theorem}\label{thm:quat_log}
The quaternion logarithm of a quaternion $\quat Q$ is
\begin{equation}
  \log \quat Q = \quatsv{\log s}{\theta \vhat u} \label{eqn:quat_log_def}
\end{equation}
where $s$, $\theta$, and $\vhat u$ are a decomposition of the quaternion $\quat Q$.
\end{theorem}
\begin{proof}
Exponentiating the quaternion $\quatsv{\log s}{\theta \vhat u}$ yields,
per equation~\eqref{eqn:quat_exp_general},
$s \quatsv {\cos \theta} {\sin \theta \vhat u} = \quat Q$.
\end{proof}


\begin{corollary}\label{thm:quat_unit_log}
The quaternion logarithm of a unit quaternion $\quat Q$ is
\begin{equation}
  \log \quat Q = \quatsv{0}{\theta \vhat u} \label{eqn:quat_unit_log_def}
\end{equation}
where $\theta$, and $\vhat u$ are a decomposition of the unit quaternion $\quat Q$.
\end{corollary}
\begin{proof}
This follows directly from theorem~\ref{thm:quat_log}. The scalar $s$ in the
decomposition of a unit quaternion is $1$, and $\log 1=0$.
\end{proof}
 


%%% Transformations
\section{Rotation and Transformation}\label{sec:app_rot_trans}

One reason quaternions are so useful is their ability
to represent rotations and transformations.
This section develops these capabilities.
While quaternions do have other applications beyond their ability to compactly
represent rotations and transformations, the JEOD uses
quaternions solely for this purpose.

Two possible conventions exist when describing rotations:
rotation of an object relative to fixed axes and rotation of the axes
relative to a fixed object.
In this discussion, the former will be referred to
as a \emph{rotation} and the latter, a \emph{transformation}.
The two concepts are closely related.
For example, the \emph{rotation} matrix that rotates a vector
about the $\vhat u$ axis by an angle of $\theta$
is the transpose of the \emph{transformation} matrix that transforms
a vector to the frame whose axes are rotated about the $\vhat u$ axis
by an angle of $\theta$ relative to the original frame.

\subsection{Single-Axis Rotations}\label{sec:app_single_axis}

\begin{theorem}[Single-Axis Rotation]\label{thm:quat_single_axis_rot}
Rotating a $3-$vector $\vect x$
by an angle $\theta$
about an axis directed along the $\vhat u$ unit vector results in
\begin{equation}
  \vect x^\prime =
  \cos\theta\, \vect x +
  (1-\cos\theta) (\vhat u \cdot \vect x) \vhat u +
  \sin\theta\, \vhat u \times \vect x
\label{eqn:quat_single_axis_rot}
\end{equation}
where $\vect x^\prime$ is the result of the rotation.
\end{theorem}
\begin{proof}
The rotation does not affect the component of $\vect x$ along the rotation axis.
Rotating a vector $\vect x$ directed solely along the rotation axis
$\vect x = x \vhat u$ has no effect: $\vect x^\prime = \vect x$,
which agrees with the theorem
since $\vhat u \times \vect x = 0$ in this case.

Assuming $\vect x$ has some non-zero component normal to the rotation axis,
represent $\vect x$ as
\begin{align*}
  \vect x &= \vect x_u + \vect x_v
\intertext{where}
  \vect x_u &= (\vect x \cdot \vhat u) \vhat u \\
  \vect x_v &= \vect x - (\vect x \cdot \vhat u) \vhat u \\
\intertext{Note that $\vect x_v$ is normal to $\vhat u$ by construction %
           and is non-zero by assumption. Let}
  \vhat v &= \frac{\vect x_v}{\norm{\vect x_v}} \\
  \vhat w &= \vhat u \times \vhat v\text{, completing a RHS.} \\
\intertext{Rotating $\vect x_v$ by an angle $\theta$ in the $vw$ plane %
           results in}
  \vect x_v^\prime &=
    \norm{\vect x_v}(\cos\theta \, \vhat v + \sin\theta \, \vhat w) \\
  &=
    \cos\theta\, (\vect x - (\vect x \cdot \vhat u) \vhat u) +
    \sin\theta\, \vhat u \times \vect x_v  \\
\intertext{The rotation does not affect $\vect x_u$ and thus}
  \vect x^\prime &= \vect x_u + \vect x_v^\prime \\
  &=
    (\vect x \cdot \vhat u) \vhat u +
    \cos\theta\, (\vect x - (\vect x \cdot \vhat u) \vhat u) +
    \sin\theta\, \vhat u \times \vect x_v  \\
   &=
     \cos\theta\, \vect x
     + (1-\cos\theta)(\vect x \cdot \vhat u) \vhat u
     + \sin\theta\, \vhat u \times \vect x
\end{align*}
\end{proof} 


\begin{theorem}[Single-Axis Transformation]\label{thm:quat_single_axis_trans}
Given a frame $B$
whose axes are rotated
about the $\vhat u$ axis by an angle of $\theta$ relative to frame $A$,
a three-vector $\framevect A x$ expressed in frame $A$
transforms to frame $B$ via
\begin{equation}
  \framevect B x =
  \cos\theta\, \framevect A x
      + (1-\cos\theta) (\vhat u \cdot \framevect A x) \vhat u
      - \sin\theta\, \vhat u \times \framevect A x
\label{eqn:quat_single_axis_trans}
\end{equation}
\end{theorem}
\begin{proof}
This theorem  follows immediately from theorem~\ref{thm:quat_single_axis_rot}
by recognizing that
each axis of frame $B$ is the corresponding axis of frame $A$ rotated
according to theorem~\ref{thm:quat_single_axis_rot}
and applying the scalar triple product rule.
\end{proof} 


\subsection{Rotation and Transformation Quaternions}

\emph{Any} quaternion can be used to represent a rotation or transformation
of a $3-$vector via one of the two forms
\begin{align*}
  \QxVxQ{\quat Q}{\vect x}{\quat Q^{-1}} && \text{or} &&
  \QxVxQ{\quat Q^{-1}}{\vect x}{\quat Q} &\qquad\qquad
\end{align*}
A \emph{unit} quaternion can be used to represent a rotation or transformation
of a $3-$vector via one of the two forms
\begin{align*}
  \QxVxQ{\quat Q}{\vect x}{\quatconj Q} && \text{or} &&
  \QxVxQ{\quatconj Q}{\vect x}{\quat Q} &\qquad\qquad
\end{align*}
That such forms do indeed rotate or transform a $3-$vector
will be developed later in this section.

The four forms differ in the use of the quaternion inverse versus the conjugate
and the placement of the quaternion and its inverse or conjugate relative
to the vector to be rotated or transformed:
\begin{itemize}
\item The inverse is used in the first pair of forms,
while the conjugate in the second pair.
Since the inverse of a unit quaternion is the conjugate,
the second pair of forms is
a specialization of the first pair for the unit quaternions.
\item The quaternion is placed to the \emph{left}
of the vector in the first triple product of each pair of forms.
Quaternion used for rotation or transformation based on these forms
are thus called \emph{left} rotation or transformation quaternions.
\item The quaternion is placed to the \emph{right}
of the vector in the second triple product of each pair of forms.
Quaternion used for rotation or transformation based on these forms
are thus called \emph{right} rotation or transformation quaternions.

\end{itemize}

There is no loss and much to be gained by eliminating the first pair of forms
from consideration---\emph{i.e.}, to restrict rotation/transformation
quaternions to the unit quaternions.
Since multiplying a quaternion by a real scalar commutes,
pre- and post-multiplying either of the first pair of rotation/transformation
forms by a real scalar
and its multiplicative inverse will not affect the outcome of form.
Given a general quaternion that achieves a desired rotation or transformation,
the unit quaternion formed by scaling the original quaternion by the inverse of
its magnitude thus achieves the same rotation or transformation
as does the original quaternion. There is no loss of expressiveness in
retricting rotation/transformation quaternions to the unit quaternions.
At the same time, there is a considerable gain
in making this restriction.
The most obvious gain is in the rotation/transformation
forms themselves. Finding a quaternion's inverse involves finding its conjugate
and then performing extra calculations. The restriction bypasses those extra
calculations.
More importantly, operations that involve the rotation/transformation quaternion
but not its inverse are sensitive to scaling and typically take on a simpler
form. For example, the logarithm and the time derivative of a unit quaternion
are pure imaginary quaternions. For these reasons, unit quaternions
are used almost exclusively to represent rotations and transformations.

There is no outstanding reason, however, to prefer left versus right
rotation/transformation quaternions. Making an \emph{a-priori} choice is
beneficial in the sense that doing so simplifies the software and reduces
the chances of an error in interpretation.
However, when one receives a quaternion from an external organization,
care must be taken in interpreting that quaternion. One must also take care
in the interpretation of the elements of some externally-generated quaternion,
as the choice of placing the scalar part first or last in a $4-$vector
representation is arbitrary.

\subsubsection{Left Rotation and Transformation Quaternions}

\begin{algorithm}\label{thm:quat_quat_left_rot}
The left rotation unit quaternion
\begin{align}
  \quat Q_{rot} &= \quatrrot{\theta}{\vhat u} \label{eqn:quat_left_rot_def} \\
\intertext{rotates a three-vector $\vect x$ about the $\vhat u$ axis %
           by an angle of $\theta$:}
  \quatsv 0 {\vect x^\prime} &=
    \QxVxQ{\quat Q_{rot}}{\vect x}{\quatconj Q_{rot}}
  \label{eqn:quat_left_rot_QxVxQ}
\end{align}
where $\vect x^\prime$ is the result of the rotation.
\end{algorithm}
\begin{proof}
To demonstrate that this quaternion does indeed perform the specified rotation,
expanding equation~\eqref{eqn:quat_left_rot_QxVxQ} results in
\begin{align*}
  \QxVxQ{\quat Q_{rot}}{\vect x}{\quatconj Q_{rot}} &=
   \QxVxQ{\quatrrot{\theta}{\vhat u}}{\vect x}{\quattrot{\theta}{\vhat u}} \\
   &=
   \quatsv
     0
     {\sin^2\frac\theta2 \vhat u \cdot \vect x \vhat u
      + \cos^2\frac\theta2 \vect x
      + 2 \cos\frac\theta2 \sin\frac\theta2 \vhat u \times \vect x
      - \sin^2\frac\theta2 \left(\vhat u \times \vect x\right)\times \vhat u} \\
\intertext{Using the half-angle formulae}
  \sin^2\frac\theta2 &= \frac 1 2 (1-\cos\theta) \\
  \cos^2\frac\theta2 &= \frac 1 2 (1+\cos\theta) \\
\intertext{and the vector triple product identity}
  \left(\vhat u \times \vect x\right) \times \vhat u
    &= \vect x - \left(\vect x \cdot \vhat u\right) \vhat u \\
\intertext{the quaternion product equation~\eqref{eqn:quat_left_rot_QxVxQ} %
           simplifies to}
   \QxVxQ{\quat Q_{rot}}{\vect x}{\quatconj Q_{rot}} &=
   \quatsv
     0
     {\cos\theta \vect x +
      (1-\cos\theta) (\vhat u \cdot \vect x) \vhat u +
      \sin\theta \vhat u \times \vect x}
\end{align*}
\end{proof}

Just as rotation and transformation matrices are related via matrix
transposition, rotation and transformation quaternions are related via
quaternion conjugation:
\begin{theorem}\label{thm:quat_left_trans}
The left transformation unit quaternion
\begin{align}
  \tquat A B &= \quattrot{\theta}{\vhat u}\label{eqn:quat_left_trans_def} \\
\intertext{transforms a three-vector $\framevect A x$ expressed in frame $A$ %
           to frame $B$ whose axes are rotated %
           about the $\vhat u$ axis by an angle of $\theta$ relative to the %
           original frame $A$ via:}
  \quatsv 0 {\framevect B x}
    &= \QxVxQ{\tquat A B}{\framevect A x}{\tquatconj A B}
  \label{eqn:quat_left_trans_QxVxQ}
 \end{align}
\end{theorem}
\begin{proof}
Expanding and simplifying equation~\eqref{eqn:quat_left_trans_QxVxQ} results in
\begin{align*}
  \QxVxQ{\tquat A B}{\framevect A x}{\tquatconj A B} &=
   \quatsv
     0
     {\cos\theta \framevect A x
      + (1-\cos\theta) (\vhat u \cdot \framevect A x) \vhat u
      - \sin\theta \vhat u \times \framevect A x}
\end{align*}
\end{proof}

Consider a left transformation unit quaternion as
defined in equation~\eqref{eqn:quat_left_trans_def}.
\begin{theorem}\label{thm:quat_left_trans_log}
The quaternion logarithm of a left transformation unit quaternion
\begin{align}
  \quat Q &= \quattrot{\theta}{\vhat u} \nonumber \\
\intertext{is}
  \log \quat Q &=
    \quatsv 0 {-\frac 1 2 \theta {\vhat u}}\label{eqn:quat_left_trans_log}
\end{align}
\end{theorem}
\begin{proof}
The theorem follows immediately from theorem~\ref{thm:quat_log}.
\end{proof}


\subsubsection{Right Rotation and Transformation Quaternions}

One unfortunate aspect of quaternions is that they magnify the confusion
regarding \emph{rotation} and \emph{transformation}.
The conjugate of a \emph{left} rotation or transformation quaternion can also
be used as the basis for rotation or transformation.
For example, the \emph{right rotation unit quaternion}
\begin{align*}
  \quat Q_{rot,\text{right}} &= \quattrot{\theta}{\vhat u} \\
\intertext{\emph{rotates} a three-vector $\vect x$ about the
           $\vhat u$ axis by an angle of $\theta$ via:}
  \quatsv 0 {\vect x^\prime} &=
    \QxVxQ{\quatconj Q_{rot,\text{right}}}{\vect x}{\quat Q_{rot,\text{right}}}
\end{align*}
which yields the same value for $\vect x^\prime$
as does equation~\eqref{eqn:quat_left_rot_QxVxQ}.

The \ModelDesc uses left transformation unit quaternions
because they chain in the same manner
as do transformation matrices and because the dynamics
package is concerned with transformations, not rotations.


\subsubsection{Chains of Transformations and Rotations}\label{sec:app_chains}

Transformation matrices chain from right to left:
given a pair transformations $\tmat A B$ and $\tmat B C$
from frame A to frame B  and
from frame B to frame C, the transformation from
frame A to frame C is
\begin{equation}
\tmat A C = \MxM{\tmat B C}{\tmat A B} \label{eqn:quat_tmat_chain}
\end{equation}

Left transformation quaternions similarly chain from right to left:
\begin{theorem}\label{thm:quat_ltquat_chain}
\begin{equation}
\tquat A C = \QxQ{\tquat B C}{\tquat A B} \label{eqn:quat_ttquat_chain}
\end{equation}
\end{theorem}
\begin{proof}
A vector $\framevect A x$ transforms from frame $A$ to frame $B$
and from frame $B$ to $C$ via equation~\eqref{eqn:quat_left_trans_QxVxQ}:
\begin{align*}
  \quatsv 0 {\framevect B x} &=
    \QxVxQ{\tquat A B}{\framevect A x}{\tquatconj A B} \\
  \quatsv 0 {\framevect C x} &=
     \QxVxQ{\tquat B C}{\framevect B x}{\tquatconj B C} \\
     &= \QxQxQ
        {\tquat B C}
        {\left(\QxVxQ{\tquat A B}{\framevect A x}{\tquatconj A B}\right)}
        {\tquatconj B C} \\
\intertext{Since quaternion multiplication is associative,}
  \quatsv 0 {\framevect C x}
   &= \QxQxQ
   {\left(\QxQ{\tquat B C}{\tquat A B}\right)}
   {\quatsv 0 {\framevect A x}}
   {\left(\QxQ{\tquatconj A B}{\tquatconj B C}\right)} \\
   &= \QxQxQ
   {\left(\QxQ{\tquat B C}{\tquat A B}\right)}
   {\quatsv 0 {\framevect A x}}
   {\quatconjlr{\QxQ{\tquat B C}{\tquat A B}}}
\intertext{The direct transformation from frame $A$ to frame $C$ is}
  \quatsv 0 {\framevect C x} &=
    \QxVxQ{\tquat A C}{\framevect A x}{\tquatconj A C} \\
\intertext{from which}
  \tquat A C &= \QxQ{\tquat B C}{\tquat A B}
\end{align*}
\end{proof}

Rotation matrices chain from left to right:
Given a pair of rotation matrices $\mat{R}_{rot_1}$ and $\mat{R}_{rot_2}$,
the rotation matrix that represents performing rotation $rot_2$ after
performing rotation $rot_1$ is
\begin{align}
  \mat{R}_{rot_{1+2}} &=
    \MxM{\mat{R}_{rot_1}}{\mat{R}_{rot_2}} \label{eqn:quat_rmat_chain} \\
  \intertext{Left rotation quaternions still chain from right to left:}
  \quat Q_{rot_{1+2}} &=
    \QxQ{\quat Q_{rot_2}}{\quat Q_{rot_1}}
  \label{eqn:quat_rquat_chain}
\end{align}

\subsection{Quaternions and Alternate Representation Schemes}

\subsubsection{Single-Axis Rotations}

Quaternions are closely related to a single axis rotation.
Given a rotation about some axis $\vhat u$ by an angle $\theta$
that describes the relative orientation of two reference frames,
one merely need apply equation~\eqref{eqn:quat_left_trans_def}
to form the corresponding left transformation unit quaternion.

The next theorem establishes a constructive technique for determining
the single axis rotation given a left transformation unit quaternion.
\begin{theorem}\label{thm:quat_left_trans_to_rot}
Given a left transformation unit quaternion $\quat Q$
and its decomposition into a unit vector $\vhat u$ and angle $\theta^\prime$
per theorem~\ref{thm:quat_unit_decomp},
The single axis rotation corresponding
to $\quat Q$ is a rotation of an angle $\theta = -2 \theta^\prime$
about the $\vhat u$ axis.
\end{theorem}
\begin{proof}
Forming the left transformation unit quaternion from the single axis rotation
$\theta$ about $\vhat u$ per equation~\eqref{eqn:quat_left_trans_def} yields
the original left transformation unit quaternion $\quat Q$.
\end{proof}

\subsubsection{Transformation Matrices}\label{sec:app_to_mat}

\begin{theorem}\label{thm:quat_single_to_tmat}
Given a single axis rotation
about the $\vhat u$ axis by an angle of $\theta$
from reference frame $A$ to reference frame $B$,
the $i,j^{th}$element of the transformation matrix
$\tmat A B$ from frame $A$ to frame $B$ is
\begin{align}
  {\tmat A B}_{ij} &=
    \cos\theta\;
    \delta_{ij} + (1-\cos\theta) u_i u_j + \sin\theta \sum_k \epsilon_{ijk} u_k
\label{eqn:quat_single_to_tmat} \\
\intertext{where}
  \delta_{ij}\;&\text{is} \; \text{the Kronecker delta} \nonumber\\
  \epsilon_{ijk}\;&\text{is} \; \text{the permutation symbol:} \nonumber\\
  \epsilon_{ijk} &= \begin{cases}
    {\phantom{-}1} &
           \text{if $i$, $j$, and $k$ are an even permutation of $(1,2,3)$}, \\
    {-1} & \text{if $i$, $j$, and $k$ are an odd permutation of $(1,2,3)$}, \\
    {\phantom{-}0} & \text{otherwise ($i=j$, $i=k$, or $j=k$)}. \\
  \end{cases}\nonumber
\end{align}
\end{theorem}
\begin{proof}
The $i^{th}$ row of $\tmat A B$ contains the transpose of the unit vector
$\vhat e_i$ rotated about the $\vhat u$ axis by an angle of $\theta$.
(The unit vector $\vhat e_i$ contains a one in row $i$ and zeros elsewhere:
$\vhat e_{i_j} = \delta_{ij}$).
Applying equation~\eqref{eqn:quat_single_axis_rot} to $\vhat e_i$
and representing the cross product of two
vectors $a$ and $b$ as $(a\times b)_j = \sum_i \sum_k \epsilon_{ijk} a_k b_i$
results in equation~\eqref{eqn:quat_single_to_tmat}.
\end{proof}
 
 \begin{theorem}\label{thm:quat_quat_to_tmat}
 Given a left transformation unit quaternion
from reference frame $A$ to reference frame $B$
 $\tquat A B = \quatsv {q_s}{\vect{q_v}}$,
the transformation matrix corresponding to $\tquat A B$ is
\begin{equation}
 {\tmat A B}_{ij} = (2 q_s^{\phantom{s}2} - 1) \delta_{ij}
   + 2 (q_{v_i} q_{v_j} - \sum_k \epsilon_{ijk} q_s q_{v_k})
\label{eqn:quat_quat_to_tmat}
\end{equation}
\end{theorem}
\begin{proof}
By using the half-angle formulae
\begin{align*}
  \cos\theta  &= 2 \cos^2\frac\theta2 - 1 = 1 - 2\sin^2\frac\theta2 \\
  \sin\theta   &= 2 \cos\frac\theta2 \sin\frac\theta2 \\
\intertext{equation~\eqref{eqn:quat_single_to_tmat} becomes}
  {\tmat A B}_{ij} &=
    (2 \cos^2\frac\theta2 - 1) \delta_{ij}
    + 2\sin^2\frac\theta2 u_i u_j
    + 2 \cos\frac\theta2 \sin\frac\theta2 \sum_k \epsilon_{ijk} u_k \\
\intertext{Per equation~\eqref{eqn:quat_left_trans_def},}
  q_s &= \cos\frac\theta2 \\
  \vect{q_v} &= -\sin\frac\theta2{\vhat u}
\end{align*}
by which equation~\eqref{eqn:quat_single_to_tmat} further reduces to 
equation~\eqref{eqn:quat_quat_to_tmat}.
\end{proof}

\begin{theorem}\label{thm:quat_tmat_to_quat}
Given a transformation matrix
from reference frame $A$ to reference frame $B$ $\tmat A B$
the left transformation unit quaternion $\tquat A B$ with
scalar and vector parts $q_s$ and $\vect{q_v}$
corresponding to $\tmat A B$ is given by four methods
labeled $q_s$ and $q_{v_i}, i\in(0,1,2)$. \\
\begin{align}
\intertext{Defining}
  tr &\equiv \operatorname{tr}(\tmat A B) \\
  t_i &\equiv {\tmat A B}_{ii} - ({\tmat A B}_{jj} + {\tmat A B}_{kk}) \\
  d_k &\equiv {\tmat A B}_{ji} - {\tmat A B}_{ij} \quad (\epsilon_{ijk} = 1) \\
  s_{ij} &\equiv {\tmat A B}_{ji} + {\tmat A B}_{ij} \quad(i \neq j)
\end{align}
Method $q_s$ is
\begin{equation}
\label{eqn:tmat_to_quat_meth_qs}
\left. %{
\begin{aligned}
  f_1 &= \sqrt{tr+1}\qquad \\
  f_2 &= \frac 1{2 f_1} \\
  q_s &= \frac 1 2 f_1 \\
  q_{v_i} &= d_i f_2 \\
  q_{v_j} &= d_j f_2 \\
  q_{v_k} &= d_k f_2
\end{aligned}\right\}
\end{equation}
Methods $q_{v_i}, i\in(0,1,2)$ are
\begin{equation}
\label{eqn:tmat_to_quat_meth_qvi}
\left. %{
\begin{aligned}
  f_1 &= \sqrt{t_i+1}\qquad\\
  f_2 &= \frac 1{2 f_1} \\
  q_{v_i} &= \frac 1 2 f_1 \\
  q_{v_j} &= s_{ij} f_2 \\
  q_{v_k} &= s_{ik} f_2 \\
  q_s &= d_i f_2
\end{aligned}\right\}
\end{equation}
\end{theorem}
\begin{proof}
By equation~\eqref{eqn:quat_quat_to_tmat},
the terms $tr$, $t_i$, $d_k$ and $s_{ij}$ are
\begin{align}
  tr     &= \operatorname{tr}(\tmat A B) =  4 q_s^{\phantom{s}2} - 1 \\
  t_i    &= {\tmat A B}_{ii} - ({\tmat A B}_{jj} + {\tmat A B}_{kk})
          = 4 q_{v_i}^2 - 1 \\
  d_k    &= {\tmat A B}_{ji} - {\tmat A B}_{ij} \  (\epsilon_{ijk} = 1)
          = 4 q_s q_{v_k} \\
  s_{ij} &= {\tmat A B}_{ji} + {\tmat A B}_{ij}  \  (i \neq j)
          = 4 q_{v_i} q_{v_j}
\end{align}
from which equations~\eqref{eqn:tmat_to_quat_meth_qs}
and~\eqref{eqn:tmat_to_quat_meth_qvi}
are readily derived.
\end{proof}

\subsubsection{Euler Angles}\label{sec:app_to_euler}

Euler proved in 1776 that any arbitrary rotation or transformation can be
described by only three parameters. An Euler sequence comprises a sequence of
rotations by various angles about various axes in a specified order.
For example, a yaw-pitch-roll transformation sequence involves a transformation
about the $z-$axis followed by a second transformation about the transformed
$y-$axis followed by a final rotation about the doubly-transformed $x-$axis.

\begin{theorem}\label{thm:quat_rot_to_quat_general}
Given an Euler sequence comprising:
\begin{itemize}
\item A rotation through an angle $\theta_1$ counterclockwise
about the $\vhat u_1$ axis followed by
\item A rotation through an angle $\theta_2$ counterclockwise
about the $\vhat u_2$ axis followed by
\item A rotation through an angle $\theta_3$ counterclockwise
about the $\vhat u_3$ axis
\end{itemize}
that represents the transformation from
from reference frame $A$ to reference frame $B$,
the corresponding left transformation unit quaternion
$\quat{Q}_{123}
  (\theta_{1}, \vhat u_1; \theta_{2}, \vhat u_2; \theta_{3}, \vhat u_3)$ is
\begin{align}
  \quat{Q}_{123}
    (\theta_{1}, \vhat u_1; \theta_{2}, \vhat u_2; \theta_{3}, \vhat u_3) &=
    \QxQxQ{\quat{Q}(\theta_3, \vhat u_3)}
          {\quat{Q}(\theta_2, \vhat u_2)}
          {\quat{Q}(\theta_1, \vhat u_1)}
    \label{eqn:quat_euler_to_quat_general} \\
  \intertext{where}
  \quat{Q}(\theta_1, \vhat u_1) &\equiv \quattrot{\theta_1}{\vhat u_1} \\
  \quat{Q}(\theta_2, \vhat u_2) &\equiv \quattrot{\theta_2}{\vhat u_2} \\
  \quat{Q}(\theta_3, \vhat u_3) &\equiv \quattrot{\theta_3}{\vhat u_3}
\end{align}
\end{theorem}
\begin{proof}
The left transformation quaternions corresponding to the individual rotations,
$\quat{Q}(\theta_1, \vhat u_1)$,
$\quat{Q}(\theta_2, \vhat u_2)$, and
$\quat{Q}(\theta_3, \vhat u_3)$,
follow from equation~\eqref{eqn:quat_left_trans_def}.
The product follows from equation~\eqref{eqn:quat_ttquat_chain}.
\end{proof}

The Trick package is only concerned with Euler sequences that use some
permutation of rotations about the $x$, $y$, and $z$ axes.
(Trick does not cover the standard astronaumical $\phi-\theta-\psi$ sequence.)
With this restriction, equation~\eqref{eqn:quat_euler_to_quat_general} can be
reduced to two cases:
one if the unit vectors $(\vhat u_1, \vhat u_2, \vhat u_3)$
form an even permutation of $(x, y, z)$
and another if the unit vectors $(\vhat u_1, \vhat u_2, \vhat u_3)$
form an odd permutation of $(x, y, z)$.

\begin{theorem}\label{thm:quat_rot_to_quat_even}
Given an Euler sequence
$\theta_1$ about the $\vhat u_1$ axis,
$\theta_2$ about the $\vhat u_2$ axis,
$\theta_3$ about the $\vhat u_3$ axis,
where $(\vhat u_1, \vhat u_2, \vhat u_3)$ is an
even permutation of $(\vhat x, \vhat y, \vhat z)$,
that represents the transformation from
from reference frame $A$ to reference frame $B$,
the corresponding left transformation unit quaternion
$\quat{Q}_{123}
  (\theta_{1}, \vhat u_1; \theta_{2}, \vhat u_2; \theta_{3}, \vhat u_3)$
is
\begin{equation}
  \quat{Q}_{123}
    (\theta_{1}, \vhat u_1; \theta_{2}, \vhat u_2; \theta_{3}, \vhat u_3) =
       \quatsv[5pt]
        {\cos\frac{\theta_3}2 \cos\frac{\theta_2}2 \cos\frac{\theta_1}2
          - \sin\frac{\theta_3}2 \sin\frac{\theta_2}2 \sin\frac{\theta_1}2}
        {
          - \begin{pmatrix}
          \phantom{+}
            (\cos\frac{\theta_3}2 \cos\frac{\theta_2}2 \sin\frac{\theta_1}2
            + \sin\frac{\theta_3}2 \sin\frac{\theta_2}2 \cos\frac{\theta_1}2)
            \vhat u_1 \\[5pt]
          +
            (\cos\frac{\theta_3}2 \sin\frac{\theta_2}2 \cos\frac{\theta_1}2
            - \sin\frac{\theta_3}2 \cos\frac{\theta_2}2 \sin\frac{\theta_1}2)
            \vhat u_2 \\[5pt]
          +
           (\cos\frac{\theta_3}2 \sin\frac{\theta_2}2 \sin\frac{\theta_1}2
            + \sin\frac{\theta_3}2 \cos\frac{\theta_2}2 \cos\frac{\theta_1}2)
           \vhat u_3
           \end{pmatrix}}
   \label{eqn:quat_rot_to_quat_even}
\end{equation}
\end{theorem}
\begin{proof}
This follows directly by expanding
equation~\eqref{eqn:quat_euler_to_quat_general}
for the special case of a set of orthogonal unit vectors
$\vhat u_1$, $\vhat u_2$, $\vhat u_3$
for which $\vhat u_1 \times \vhat u_2 = \vhat u_3$.
\end{proof}

\begin{theorem}\label{thm:quat_rot_to_quat_odd}
Given an Euler sequence
$\theta_1$ about the $\vhat u_1$ axis,
$\theta_2$ about the $\vhat u_2$ axis,
$\theta_3$ about the $\vhat u_3$ axis,
where $(\vhat u_1, \vhat u_2, \vhat u_3)$ is an
odd permutation of $(\vhat x, \vhat y, \vhat z)$,
that represents the transformation from
from reference frame $A$ to reference frame $B$,
the corresponding left transformation unit quaternion
$\quat{Q}_{123}
  (\theta_{1}, \vhat u_1; \theta_{2}, \vhat u_2; \theta_{3}, \vhat u_3)$
is
\begin{equation}
  \quat{Q}_{123}
    (\theta_{1}, \vhat u_1; \theta_{2}, \vhat u_2; \theta_{3}, \vhat u_3) =
       \quatsv[5pt]
        {\cos\frac{\theta_3}2 \cos\frac{\theta_2}2 \cos\frac{\theta_1}2
          + \sin\frac{\theta_3}2 \sin\frac{\theta_2}2 \sin\frac{\theta_1}2}
        {
          - \begin{pmatrix}
          \phantom{+}
            (\cos\frac{\theta_3}2 \cos\frac{\theta_2}2 \sin\frac{\theta_1}2
            - \sin\frac{\theta_3}2 \sin\frac{\theta_2}2 \cos\frac{\theta_1}2)
            \vhat u_1 \\[5pt]
          +
            (\cos\frac{\theta_3}2 \sin\frac{\theta_2}2 \cos\frac{\theta_1}2
            + \sin\frac{\theta_3}2 \cos\frac{\theta_2}2 \sin\frac{\theta_1}2)
            \vhat u_2 \\[5pt]
          -
           (\cos\frac{\theta_3}2 \sin\frac{\theta_2}2 \sin\frac{\theta_1}2
            - \sin\frac{\theta_3}2 \cos\frac{\theta_2}2 \cos\frac{\theta_1}2)
            \vhat u_3
           \end{pmatrix}}
   \label{eqn:quat_rot_to_quat_odd}
\end{equation}
\end{theorem}
\begin{proof}
This follows directly by expanding
equation~\eqref{eqn:quat_euler_to_quat_general}
for the special case of a set of orthogonal unit vectors
$\vhat u_1$, $\vhat u_2$, $\vhat u_3$
for which $\vhat u_1 \times \vhat u_2 = - \vhat u_3$.
\end{proof}

With a bit of trigonemtric manipulation,
the inverse operation of determining the Euler angles given a quaternion
and a rotation sequence follow from theorems~\ref{thm:quat_rot_to_quat_even}
and~\ref{thm:quat_rot_to_quat_odd}.

\begin{theorem}\label{thm:quat_quat_to_rot_even}
Given an even permutation $(\vhat u_1, \vhat u_2, \vhat u_3)$ of
$(\vhat x, \vhat y, \vhat z)$
and a quaternion \\
$\quat Q = \quatsv {q_s}
                   {q_{v_1} \vhat u_1 +
                    q_{v_2} \vhat u_2 +
                    q_{v_3} \vhat u_3}$,
the Euler sequence
$\theta_1$ about the $\vhat u_1$ axis,
$\theta_2$ about the $\vhat u_2$ axis,
$\theta_3$ about the $\vhat u_3$ axis is given by

\begin{align}
  \theta_2 &= \arcsin(-2(q_s q_{v_2} - q_{v_1} q_{v_3}))
    \label{eqn:quat_quat_to_rot_even_theta_2} \\
  \theta_1 &= \arctan(-2(q_s q_{v_1} + q_{v_2} q_{v_3}),
                     ({q_s}^2-q_{v_2}^2)-(q_{v_1}^2-q_{v_3}^2))
    \label{eqn:quat_quat_to_rot_even_theta_1} \\
  \theta_3 &= \arctan(-2(q_s q_{v_3} + q_{v_2} q_{v_1}),
                     ({q_s}^2-q_{v_2}^2)-(q_{v_3}^2-q_{v_1}^2))
    \label{eqn:quat_quat_to_rot_even_theta_3} \\
\intertext{Equations~\eqref{eqn:quat_quat_to_rot_even_theta_1} %
           and~\eqref{eqn:quat_quat_to_rot_even_theta_3} are valid %
           only if $\abs{\sin\theta_2} \neq 1$. %
           $\abs{\sin\theta_2} = 1$ represents a singularity, %
           in which case $\theta_1$ and $\theta_3$ are related via}
  \theta_1 + \theta_3 \sin\theta_2 &=
      2 \arctan(-q_{v_3} \sin\theta_2, q_{v_2} \sin\theta_2) =
      2 \arctan(-q_{v_1},q_s)
    \label{eqn:quat_quat_to_rot_even_singular}
\end{align}
\end{theorem}
\begin{proof}
Th solution for $\theta_2$ and the non-singular solutions for
$\theta_1$ and $\theta_3$ follow by replacing the arguments of the inverse
trigonometric functions in equations~\eqref{eqn:quat_quat_to_rot_even_theta_2}
to~\eqref{eqn:quat_quat_to_rot_even_theta_3} into
equation~\eqref{eqn:quat_rot_to_quat_even}.
The singular solutions follow by replacing $\theta_2 = \pm \frac \pi 2$ into
equation~\eqref{eqn:quat_rot_to_quat_even}.
\end{proof}

\begin{theorem}\label{thm:quat_quat_to_rot_odd}
Given an odd permutation $(\vhat u_1, \vhat u_2, \vhat u_3)$ of
$(\vhat x, \vhat y, \vhat z)$
and a quaternion \\
$\quat Q = \quatsv {q_s}
                   {q_{v_1} \vhat u_1 +
                    q_{v_2} \vhat u_2 +
                    q_{v_3} \vhat u_3}$,
the Euler sequence
$\theta_1$ about the $\vhat u_1$ axis,
$\theta_2$ about the $\vhat u_2$ axis,
$\theta_3$ about the $\vhat u_3$ axis is given by

\begin{align}
  \theta_2 &= \arcsin(-2(q_s q_{v_2} + q_{v_1} q_{v_3}))
    \label{eqn:quat_quat_to_rot_odd_theta_2} \\
  \theta_1 &= \arctan(-2(q_s q_{v_1} - q_{v_2} q_{v_3}),
                     ({q_s}^2-q_{v_2}^2)-(q_{v_1}^2-q_{v_3}^2))
    \label{eqn:quat_quat_to_rot_odd_theta_1} \\
  \theta_3 &= \arctan(-2(q_s q_{v_3} - q_{v_2} q_{v_1}),
                     ({q_s}^2-q_{v_2}^2)-(q_{v_3}^2-q_{v_1}^2))
    \label{eqn:quat_quat_to_rot_odd_theta_3} \\
\intertext{Equations~\eqref{eqn:quat_quat_to_rot_odd_theta_1} %
           and~\eqref{eqn:quat_quat_to_rot_odd_theta_3} are valid %
           only if $\abs{\sin\theta_2} \neq 1$.%
           $\abs{\sin\theta_2} = 1$ represents a singularity, %
           in which case $\theta_1$ and $\theta_3$ are related via}
  \theta_1 - \theta_3 \sin\theta_2 &=
      2 \arctan(q_{v_3} \sin\theta_2,-q_{v_2} \sin\theta_2) =
      2 \arctan(-q_{v_1},q_s)
    \label{eqn:quat_quat_to_rot_odd_singular}
\end{align}
\end{theorem}
\begin{proof}
Th solution for $\theta_2$ and the non-singular solutions for
$\theta_1$ and $\theta_3$ follow by replacing the arguments of the inverse
trigonometric functions in equations~\eqref{eqn:quat_quat_to_rot_odd_theta_2}
to~\eqref{eqn:quat_quat_to_rot_odd_theta_3} into
equation~\eqref{eqn:quat_rot_to_quat_odd}.
The singular solutions follow by replacing $\theta_2 = \pm \frac \pi 2$ into
equation~\eqref{eqn:quat_rot_to_quat_odd}.
\end{proof}
 

\subsection{Comparing and Averaging Transformation Quaternions}

Suppose $\vect x_1$ and $\vect x_2$ are two vectors
that represent the some comparable quantity
such as the position of a spacecraft at a specific point in time as computed
by two different methods
or the position of a spacecraft at two different points in time.
Various analyses are frequently based on the
difference between the two vectors,
$\Delta \vect x \equiv \vect x_2 - \vect x_1$.

Now suppose $\quat Q_1$ and $\quat Q_2$ are two left transformation
quaternions that that represent some comparable transformation.
The additive difference between the two transformation quaternions has no
physical meaning and thus is not useful for analysis.
The normative mathematics of the transformation quaternions is
multiplication rather than addition.
A multiplicative rather than additive scheme is needed to
determine the "distance" between two quaternions.
\begin{definition}[Quaternion difference]\label{def:quat_err}
Given quaternions $\quat{Q}_1$ and $\quat{Q}_2$,
the difference between the two quaternions is defined as
\begin{equation}
  \Delta \quat Q \equiv \ 
                 \operatorname{acute} (\QxQ{\quat{Q}_2}{\quat{Q}_1^\conj})
  \label{eqn:quat_err_def}
\end{equation}
where \emph{acute} denotes that all components of the product
$\QxQ{\quat{Q}_2}{\quat{Q}_1^\conj}$
are to be negated when the scalar part of the product is negative.
\end{definition}

The magnitude of an error vector is often a more significant indicator
than is the direction of the error when analyzing errors in vectors.
Similarly, the single axis rotation angle computed from the quaternion
difference is often a more significant indicator than the direction of the
rotation when analyzing errors in transformation quaternions.

The difference between two vectors is also useful for computing
a weighted average:
$\bar{\vect x} = w  \Delta \vect x + \vect x_1$ where $w$ is some
weight factor between 0 and 1.
The arithmetic mean results when $w=1/2$.

Similarly, the weighted mean of two quaternions is
 \begin{definition}[Quaternion weighted mean]\label{def:quat_wmean}
Given quaternions $\quat{Q}_1$ and $\quat{Q}_2$,
the weighted mean of the two quaternions is defined as
\begin{equation}
  \bar{\quat Q} = \QxQ{(\Delta \quat Q)^w}{\quat{Q}_1}\label{eqn:quat_wmean_def}
\end{equation}
where $w$ is a weight factor between 0 and 1 and
$\Delta \quat Q$ is computed via equation~\eqref{eqn:quat_err_def}.
\end{definition}

\begin{align*}
\intertext{Let}
  \Delta \quat Q &= \quatsv{\Delta q_s}{\vect {\Delta q_v}}
\intertext{if $\abs{\Delta q_s} \approx 1$ then $(\Delta \quat Q)^w$ %
           can be approximated using the small angle approximations} \\
  \cos(w\theta) &\approx 1 + w^2(\cos\theta-1) \\
  \sin(w\theta) &\approx w\sin\theta \\
  (\Delta \quat Q)^w &\approx
    \quatsv{1 + w^2(\Delta q_s-1)}{w \vect {\Delta q_v}}
\end{align*}



%%% Time Derivative
\section{Quaternion Time Derivative}\label{sec:app_time_deriv}

This section develops the time derivative of a left transformation quaternion.

The time derivative of a vector $\vect x$ is observer-dependent.
The relation between the time derivative of $\vect x$ as observed in an inertial
frame $I$ and the time derivative of $\vect x$ as observed in a frame $B$ rotating at
a rate $\omega$ with respect to the inertial frame\cite{Goldstein} is\begin{align}
\framedot{I}{\vect x} &= \framedot{B}{\vect x} + \vect \omega \times \vect x
  \label{eqn:quat_xdot_in_frame} \\
\intertext{where}
  \framedot{F}{\vect x}\;&\text{is}
    \; \text{the time derivative of $\vect x$ as observed in frame $F$} \nonumber\\
\end{align}

This can be expressed in quaternion form as
\begin{align}
\QBI & \QxQ
    {\left(
      \frac{d}{dt}
      \left(
        \QxVxQ
          {\QBIconj}
          {\framevect B x}
          {\QBI}
      \right)
    \right)}
    {\QBIconj}
  = \quatsv0{\framevdot B x}
    + \quatsv0{\framerelvect B \omega I B \times \framevect B x}
\label{eqn:quat_xdot_quat_form_1} \\
\intertext{where}
  \QBI\;&\text{is}
    \; \text{the left transformation quaternion from frame $I$ to frame $B$} \nonumber\\
  \framerelvect B \omega I B\;&\text{is}
    \; \text{the angular velocity of frame $B$ with respect to frame $I$, expressed in frame $B$}
      \nonumber\\
  \framevect B x\;&\text{is}
    \; \text{an arbitrary vector $\vect x$, expressed in frame $B$} \nonumber
\end{align}

The derivative in the left-hand side of equation~\eqref{eqn:quat_xdot_quat_form_1} expands to
\begin{equation}
\begin{split}
\text{\makebox[3.0em]{}}&\text{\makebox[-3.0em]{}}
 \frac{d}{dt} \left(\QxVxQ{\QBIconj}{\framevect B x}{\QBI}\right) =\\
   & \QxVxQ{\QBIconjdot}{\framevect B x}{\QBI}%\\
  \;+\; \QxVxQ{\QBIconj}{\framevdot B x}{\QBI} %\\
  \;-\; \quatconjlr{\QxVxQ{\QBIconjdot}{\framevect B x}{\QBI}}
\end{split}
\end{equation}

Applying the above to the left-hand side of equation~\eqref{eqn:quat_xdot_quat_form_1} and simplifying yields
\begin{equation}
\begin{split}
\text{\makebox[3.0em]{}}&\text{\makebox[-3.0em]{}}
\QxQxQ
  {\QBI}
  {\left(\frac{d}{dt} \left(\QxVxQ{\QBIconj}{\framevect B x}{\QBI} \right) \right)}
  {\QBIconj} = \\
& \QxQ{\QBI}{\QxV{\QBIconjdot}{\framevect B x}}
   - \quatconjlr{\QxQ{\QBI}{\QxV{\QBIconjdot}{\framevect B x}}}
  +  \quatsv0{\framevdot B x}
\end{split}\label{eqn:quat_xdot_quat_form_lhs}
\end{equation}

Using
\begin{equation}
\quatsv0{\framerelvect B \omega I B \times \framevect B x} =
\frac{1}{2}\left(\left(\QxQ{\quatsv0{\framerelvect B \omega I B}}{\quatsv0{\framevect B x}}\right)-
\quatconjlr{\QxQ{\quatsv0{\framerelvect B \omega I B}}{\quatsv0{\framevect B x}}}\right)
\end{equation}

The right-hand side of equation~\eqref{eqn:quat_xdot_quat_form_1} expands to
\begin{equation}
\begin{split}
\text{\makebox[3.0em]{}}&\text{\makebox[-3.0em]{}}
\frac{d}{dt} \quatsv0{\framevect B x}
    + \quatsv0{\framerelvect B \omega I B \times \framevect B x} = \\
   & \frac{d}{dt} \quatsv0{\framevect B x}
  + \frac{1}{2} \left(\left(\QxQ{\quatsv0{\framerelvect B \omega I B}}{\quatsv0{\framevect B x}}\right)
  - \quatconjlr{\QxQ{\quatsv0{\framerelvect B \omega I B}}{\quatsv0{\framevect B x}}}\right)
\end{split}\label{eqn:quat_xdot_quat_form_rhs}
\end{equation}

Equating
equations~\eqref{eqn:quat_xdot_quat_form_lhs}
and~\eqref{eqn:quat_xdot_quat_form_rhs}
and eliminating the common term $\quatsv0{\framevdot B x}$ yields
\begin{equation}
\begin{split}
& \QxQxQ{\QBI}{\QBIconjdot}{\quatsv0{\framevect B x}}
   - \quatconjlr{\QxQxQ{\QBI}{\QBIconjdot}{\quatsv0{\framevect B x}}} = \\
\qquad &  \left(\frac{1}{2} \QxQ{\quatsv0{\framerelvect B \omega I B}}{\quatsv0{\framevect B x}}\right)
  - \quatconjlr{\frac{1}{2} \QxQ{\quatsv0{\framerelvect B \omega I B}}{\quatsv0{\framevect B x}}}
\end{split} \label{eqn:quat_xdot_quat_form_2}
\end{equation}

Since equation \eqref{eqn:quat_xdot_quat_form_2} must be satisfied for \emph{any} vector $\vect x$,
\begin{equation}
\QxQ{\QBI}{\QBIconjdot} - \quatconjlr{\QxQ{\QBI}{\QBIconjdot}} = 
\left(\frac{1}{2} \quatsv0{\framerelvect B \omega I B}\right)
  - \quatconjlr{\frac{1}{2}\quatsv0{\framerelvect B \omega I B}}
  \label{eqn:quat_xdot_quat_form_3}
\end{equation}

Note that equation ~\eqref{eqn:quat_xdot_quat_form_3} is of the form
\begin{equation*}
\quat{Q}_a - \quatconj{Q}_a = \quat{Q}_b - \quatconj{Q}_b
\end{equation*}
Such a form requires that the vector parts of $\quat{Q}_a$ and $\quat{Q}_b$ be equal;
the scalar parts are unconstrained.
Since the scalar parts of $\QBI \QBIconjdot$ and $\quatsv0{\framerelvect B \omega I B}$
are zero, equation~\eqref{eqn:quat_xdot_quat_form_3} reduces to
\begin{align}
  \QxQ{\QBI}{\QBIconjdot} &= \frac{1}{2} \quatsv0{\framerelvect B \omega I B} \\
\intertext{or}
  \QBIdot &=  \QxQ{\quatsv0{-\frac{1}{2} {\framerelvect B \omega I B}}}{\QBI}
  \label{eqn:quat_qdot}
\end{align}

The \ModelDesc uses equation~\eqref{eqn:quat_qdot} to form the
derivative of the inertial-to-body left transformation quaternion $\QBI$.

Solving equation~\eqref{eqn:quat_qdot} for the body rate yields
\begin{align}
  \quatsv0{\framerelvect B \omega I B} &= 2 \QxQ{\QBI}{\QBIconjdot} 
  \label{eqn:quat_body_rate_from_qbidot}
\end{align}

Differentiating equation~\eqref{eqn:quat_qdot} again yields the second derivative
of the inertial-to-body left transformation quaternion $\QBI$:
\begin{align}
  \QBIdotdot &=
     \QxQ{\quatsv0{-\frac{1}{2} {\framerelvdot B \omega I B}}}{\QBI} +
     \QxQ{\quatsv0{-\frac{1}{2} {\framerelvect B \omega I B}}}{\QBIdot} \\
    &= 
     \QxQ
       {\quatsv{-\frac{1}{4} \norm{\framerelvect B \omega I B}^2}{-\frac{1}{2} {\framerelvdot B \omega I B}}}
     {\QBI}
    \label{eqn:quat_qdot_deriv}
\end{align} 


%%% Propagation
\section{Attitude Propagation}\label{sec:app_att_prop}


Suppose the body rate vector $\framerelvect B \omega I B$ is constant.
The quaternion time derivative equation~\eqref{eqn:quat_qdot} then takes on the form
\begin{equation}
  \dot{x} = -A x
\end{equation}
When x is a scalar or a vector and A is a scalar or matrix,
equations of this form have a solution
\begin{equation}
  x(t0+dt) = \exp(-A dt) x(t0)
\end{equation}
Due to the developments in preceding sections, the
same form of solution applies when A is a pure imaginary quaternion
and x is a quaternion, and thus the constant body rate solution
to equation~\eqref{eqn:quat_qdot} is
\begin{align}
  \QBI(t+dt) &= \exp\left(\quatsv 0 {-\frac{1}{2} {\framerelvect B \omega I B} dt}\right) \QBI(t0) \nonumber \\
  &= \quattrot {\omega dt}{\vhat \omega}\QBI(t0) \label{eqn:quat_prop_transcendental} \\
\intertext{where}
  \omega &\equiv \norm {\framerelvect B \omega I B} \nonumber \\
  \vect \omega &\equiv \frac {\framerelvect B \omega I B} \omega \nonumber
\end{align}

For a sufficiently small time step $dt$, the body rate will be approximately constant.
Equation~\eqref{eqn:quat_prop_transcendental} can thus be used as the basis for
quaternion propagation. However, since this equation involves the use of
transcendental functions, applying this equation directly over small time steps would be
computationally prohibitive.

A simple approach to avoiding transcendental functions is to make the
first order small angle assumptions
\begin{align}
  \cos(\frac{\omega dt} 2) &\approx 1 \\
  \sin(\frac{\omega dt} 2) &\approx \frac{\omega dt} 2 \\
\intertext{in which case equation~\eqref{eqn:quat_prop_transcendental} becomes}
  \QBI(t+dt) &= \quatsv 1 {- \frac 1 2  {\framerelvect B \omega I B} dt} \QBI(t0) \nonumber \\
  &= \QBI(t0) + \QBIdot dt \label{eqn:quat_prop_first_order}
\end{align}
A standard numerical integrator can be used to propagate a quaternion via
equation~\eqref{eqn:quat_prop_first_order}.
The first order small angle assumptions thus lead to a simple and very appealing propagator.

\section{Body Rate Propagation}\label{sec:app_rate_prop}

For a sufficiently small time step $dt$, the body rate will be approximately constant.
However, the body rate for most spacecraft will rarely be constant for any extended period of time.
The body rate must be propagated along with the attitude.

The rotational analog of Newton's Second Law is \cite{Goldstein}
\begin{align}
  \framerelvdot I L I B &= \frameabsvect I {\tau} {ext}
    \label{eqn:quat_L_dot_inertial} \\
\intertext{where}
\framerelvdot I L I B\;&\text{is the body's angular momentum vector and} \nonumber \\
\frameabsvect I {\tau} {ext}\;&\text{is the external torque acting on the body.} \nonumber \\
\intertext{The angular momentum is related to the angular velocity via}
  \relvect L I B &= \MxV{\mat I}{\relvect \omega I B}
    \label{eqn:quat_angular_momentum_general}\\
\intertext{where}
\mat I\;&\text{is the body's inertia tensor.} \nonumber
\end{align}
Note that equation~\eqref{eqn:quat_L_dot_inertial} is valid in an inertial frame only
while equation~\eqref{eqn:quat_angular_momentum_general} is valid in any reference frame
(but $\vect L$, $\mat I$, and $\vect \omega$ must all represented in the same reference frame).

Applying equation~\eqref{eqn:quat_xdot_in_frame} to equations~\eqref{eqn:quat_L_dot_inertial}
and transforming to the body frame
yields the body-frame rotational equations of motion
\begin{equation}
  \MxV{\framerelvdot B \omega I B} =
  \MxV
    {{\framemat B I}^{-1}}
    {\left(
      \frameabsvect B {\tau} {ext}  +
      \left(\MxV{\framemat B I}{\framerelvect B \omega I B}\right) \times \framerelvect B \omega I B
    \right)}
    \label{eqn:quat_omega_dot_EOM}
\end{equation}

