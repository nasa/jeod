\section{Attitude Propagation}\label{sec:app_att_prop}


Suppose the body rate vector $\framerelvect B \omega I B$ is constant.
The quaternion time derivative equation~\eqref{eqn:quat_qdot} then takes on the form
\begin{equation}
  \dot{x} = -A x
\end{equation}
When x is a scalar or a vector and A is a scalar or matrix,
equations of this form have a solution
\begin{equation}
  x(t0+dt) = \exp(-A dt) x(t0)
\end{equation}
Due to the developments in preceding sections, the
same form of solution applies when A is a pure imaginary quaternion
and x is a quaternion, and thus the constant body rate solution
to equation~\eqref{eqn:quat_qdot} is
\begin{align}
  \QBI(t+dt) &= \exp\left(\quatsv 0 {-\frac{1}{2} {\framerelvect B \omega I B} dt}\right) \QBI(t0) \nonumber \\
  &= \quattrot {\omega dt}{\vhat \omega}\QBI(t0) \label{eqn:quat_prop_transcendental} \\
\intertext{where}
  \omega &\equiv \norm {\framerelvect B \omega I B} \nonumber \\
  \vect \omega &\equiv \frac {\framerelvect B \omega I B} \omega \nonumber
\end{align}

For a sufficiently small time step $dt$, the body rate will be approximately constant.
Equation~\eqref{eqn:quat_prop_transcendental} can thus be used as the basis for
quaternion propagation. However, since this equation involves the use of
transcendental functions, applying this equation directly over small time steps would be
computationally prohibitive.

A simple approach to avoiding transcendental functions is to make the
first order small angle assumptions
\begin{align}
  \cos(\frac{\omega dt} 2) &\approx 1 \\
  \sin(\frac{\omega dt} 2) &\approx \frac{\omega dt} 2 \\
\intertext{in which case equation~\eqref{eqn:quat_prop_transcendental} becomes}
  \QBI(t+dt) &= \quatsv 1 {- \frac 1 2  {\framerelvect B \omega I B} dt} \QBI(t0) \nonumber \\
  &= \QBI(t0) + \QBIdot dt \label{eqn:quat_prop_first_order}
\end{align}
A standard numerical integrator can be used to propagate a quaternion via
equation~\eqref{eqn:quat_prop_first_order}.
The first order small angle assumptions thus lead to a simple and very appealing propagator.

\section{Body Rate Propagation}\label{sec:app_rate_prop}

For a sufficiently small time step $dt$, the body rate will be approximately constant.
However, the body rate for most spacecraft will rarely be constant for any extended period of time.
The body rate must be propagated along with the attitude.

The rotational analog of Newton's Second Law is \cite{Goldstein}
\begin{align}
  \framerelvdot I L I B &= \frameabsvect I {\tau} {ext}
    \label{eqn:quat_L_dot_inertial} \\
\intertext{where}
\framerelvdot I L I B\;&\text{is the body's angular momentum vector and} \nonumber \\
\frameabsvect I {\tau} {ext}\;&\text{is the external torque acting on the body.} \nonumber \\
\intertext{The angular momentum is related to the angular velocity via}
  \relvect L I B &= \MxV{\mat I}{\relvect \omega I B}
    \label{eqn:quat_angular_momentum_general}\\
\intertext{where}
\mat I\;&\text{is the body's inertia tensor.} \nonumber
\end{align}
Note that equation~\eqref{eqn:quat_L_dot_inertial} is valid in an inertial frame only
while equation~\eqref{eqn:quat_angular_momentum_general} is valid in any reference frame
(but $\vect L$, $\mat I$, and $\vect \omega$ must all represented in the same reference frame).

Applying equation~\eqref{eqn:quat_xdot_in_frame} to equations~\eqref{eqn:quat_L_dot_inertial}
and transforming to the body frame
yields the body-frame rotational equations of motion
\begin{equation}
  \MxV{\framerelvdot B \omega I B} =
  \MxV
    {{\framemat B I}^{-1}}
    {\left(
      \frameabsvect B {\tau} {ext}  +
      \left(\MxV{\framemat B I}{\framerelvect B \omega I B}\right) \times \framerelvect B \omega I B
    \right)}
    \label{eqn:quat_omega_dot_EOM}
\end{equation}
