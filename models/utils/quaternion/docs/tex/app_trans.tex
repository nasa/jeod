\section{Rotation and Transformation}\label{sec:app_rot_trans}

One reason quaternions are so useful is their ability
to represent rotations and transformations.
This section develops these capabilities.
While quaternions do have other applications beyond their ability to compactly
represent rotations and transformations, the JEOD uses
quaternions solely for this purpose.

Two possible conventions exist when describing rotations:
rotation of an object relative to fixed axes and rotation of the axes
relative to a fixed object.
In this discussion, the former will be referred to
as a \emph{rotation} and the latter, a \emph{transformation}.
The two concepts are closely related.
For example, the \emph{rotation} matrix that rotates a vector
about the $\vhat u$ axis by an angle of $\theta$
is the transpose of the \emph{transformation} matrix that transforms
a vector to the frame whose axes are rotated about the $\vhat u$ axis
by an angle of $\theta$ relative to the original frame.

\subsection{Single-Axis Rotations}\label{sec:app_single_axis}

\begin{theorem}[Single-Axis Rotation]\label{thm:quat_single_axis_rot}
Rotating a $3-$vector $\vect x$
by an angle $\theta$
about an axis directed along the $\vhat u$ unit vector results in
\begin{equation}
  \vect x^\prime =
  \cos\theta\, \vect x +
  (1-\cos\theta) (\vhat u \cdot \vect x) \vhat u +
  \sin\theta\, \vhat u \times \vect x
\label{eqn:quat_single_axis_rot}
\end{equation}
where $\vect x^\prime$ is the result of the rotation.
\end{theorem}
\begin{proof}
The rotation does not affect the component of $\vect x$ along the rotation axis.
Rotating a vector $\vect x$ directed solely along the rotation axis
$\vect x = x \vhat u$ has no effect: $\vect x^\prime = \vect x$,
which agrees with the theorem
since $\vhat u \times \vect x = 0$ in this case.

Assuming $\vect x$ has some non-zero component normal to the rotation axis,
represent $\vect x$ as
\begin{align*}
  \vect x &= \vect x_u + \vect x_v
\intertext{where}
  \vect x_u &= (\vect x \cdot \vhat u) \vhat u \\
  \vect x_v &= \vect x - (\vect x \cdot \vhat u) \vhat u \\
\intertext{Note that $\vect x_v$ is normal to $\vhat u$ by construction %
           and is non-zero by assumption. Let}
  \vhat v &= \frac{\vect x_v}{\norm{\vect x_v}} \\
  \vhat w &= \vhat u \times \vhat v\text{, completing a RHS.} \\
\intertext{Rotating $\vect x_v$ by an angle $\theta$ in the $vw$ plane %
           results in}
  \vect x_v^\prime &=
    \norm{\vect x_v}(\cos\theta \, \vhat v + \sin\theta \, \vhat w) \\
  &=
    \cos\theta\, (\vect x - (\vect x \cdot \vhat u) \vhat u) +
    \sin\theta\, \vhat u \times \vect x_v  \\
\intertext{The rotation does not affect $\vect x_u$ and thus}
  \vect x^\prime &= \vect x_u + \vect x_v^\prime \\
  &=
    (\vect x \cdot \vhat u) \vhat u +
    \cos\theta\, (\vect x - (\vect x \cdot \vhat u) \vhat u) +
    \sin\theta\, \vhat u \times \vect x_v  \\
   &=
     \cos\theta\, \vect x
     + (1-\cos\theta)(\vect x \cdot \vhat u) \vhat u
     + \sin\theta\, \vhat u \times \vect x
\end{align*}
\end{proof} 


\begin{theorem}[Single-Axis Transformation]\label{thm:quat_single_axis_trans}
Given a frame $B$
whose axes are rotated
about the $\vhat u$ axis by an angle of $\theta$ relative to frame $A$,
a three-vector $\framevect A x$ expressed in frame $A$
transforms to frame $B$ via
\begin{equation}
  \framevect B x =
  \cos\theta\, \framevect A x
      + (1-\cos\theta) (\vhat u \cdot \framevect A x) \vhat u
      - \sin\theta\, \vhat u \times \framevect A x
\label{eqn:quat_single_axis_trans}
\end{equation}
\end{theorem}
\begin{proof}
This theorem  follows immediately from theorem~\ref{thm:quat_single_axis_rot}
by recognizing that
each axis of frame $B$ is the corresponding axis of frame $A$ rotated
according to theorem~\ref{thm:quat_single_axis_rot}
and applying the scalar triple product rule.
\end{proof} 


\subsection{Rotation and Transformation Quaternions}

\emph{Any} quaternion can be used to represent a rotation or transformation
of a $3-$vector via one of the two forms
\begin{align*}
  \QxVxQ{\quat Q}{\vect x}{\quat Q^{-1}} && \text{or} &&
  \QxVxQ{\quat Q^{-1}}{\vect x}{\quat Q} &\qquad\qquad
\end{align*}
A \emph{unit} quaternion can be used to represent a rotation or transformation
of a $3-$vector via one of the two forms
\begin{align*}
  \QxVxQ{\quat Q}{\vect x}{\quatconj Q} && \text{or} &&
  \QxVxQ{\quatconj Q}{\vect x}{\quat Q} &\qquad\qquad
\end{align*}
That such forms do indeed rotate or transform a $3-$vector
will be developed later in this section.

The four forms differ in the use of the quaternion inverse versus the conjugate
and the placement of the quaternion and its inverse or conjugate relative
to the vector to be rotated or transformed:
\begin{itemize}
\item The inverse is used in the first pair of forms,
while the conjugate in the second pair.
Since the inverse of a unit quaternion is the conjugate,
the second pair of forms is
a specialization of the first pair for the unit quaternions.
\item The quaternion is placed to the \emph{left}
of the vector in the first triple product of each pair of forms.
Quaternion used for rotation or transformation based on these forms
are thus called \emph{left} rotation or transformation quaternions.
\item The quaternion is placed to the \emph{right}
of the vector in the second triple product of each pair of forms.
Quaternion used for rotation or transformation based on these forms
are thus called \emph{right} rotation or transformation quaternions.

\end{itemize}

There is no loss and much to be gained by eliminating the first pair of forms
from consideration---\emph{i.e.}, to restrict rotation/transformation
quaternions to the unit quaternions.
Since multiplying a quaternion by a real scalar commutes,
pre- and post-multiplying either of the first pair of rotation/transformation
forms by a real scalar
and its multiplicative inverse will not affect the outcome of form.
Given a general quaternion that achieves a desired rotation or transformation,
the unit quaternion formed by scaling the original quaternion by the inverse of
its magnitude thus achieves the same rotation or transformation
as does the original quaternion. There is no loss of expressiveness in
retricting rotation/transformation quaternions to the unit quaternions.
At the same time, there is a considerable gain
in making this restriction.
The most obvious gain is in the rotation/transformation
forms themselves. Finding a quaternion's inverse involves finding its conjugate
and then performing extra calculations. The restriction bypasses those extra
calculations.
More importantly, operations that involve the rotation/transformation quaternion
but not its inverse are sensitive to scaling and typically take on a simpler
form. For example, the logarithm and the time derivative of a unit quaternion
are pure imaginary quaternions. For these reasons, unit quaternions
are used almost exclusively to represent rotations and transformations.

There is no outstanding reason, however, to prefer left versus right
rotation/transformation quaternions. Making an \emph{a-priori} choice is
beneficial in the sense that doing so simplifies the software and reduces
the chances of an error in interpretation.
However, when one receives a quaternion from an external organization,
care must be taken in interpreting that quaternion. One must also take care
in the interpretation of the elements of some externally-generated quaternion,
as the choice of placing the scalar part first or last in a $4-$vector
representation is arbitrary.

\subsubsection{Left Rotation and Transformation Quaternions}

\begin{algorithm}\label{thm:quat_quat_left_rot}
The left rotation unit quaternion
\begin{align}
  \quat Q_{rot} &= \quatrrot{\theta}{\vhat u} \label{eqn:quat_left_rot_def} \\
\intertext{rotates a three-vector $\vect x$ about the $\vhat u$ axis %
           by an angle of $\theta$:}
  \quatsv 0 {\vect x^\prime} &=
    \QxVxQ{\quat Q_{rot}}{\vect x}{\quatconj Q_{rot}}
  \label{eqn:quat_left_rot_QxVxQ}
\end{align}
where $\vect x^\prime$ is the result of the rotation.
\end{algorithm}
\begin{proof}
To demonstrate that this quaternion does indeed perform the specified rotation,
expanding equation~\eqref{eqn:quat_left_rot_QxVxQ} results in
\begin{align*}
  \QxVxQ{\quat Q_{rot}}{\vect x}{\quatconj Q_{rot}} &=
   \QxVxQ{\quatrrot{\theta}{\vhat u}}{\vect x}{\quattrot{\theta}{\vhat u}} \\
   &=
   \quatsv
     0
     {\sin^2\frac\theta2 \vhat u \cdot \vect x \vhat u
      + \cos^2\frac\theta2 \vect x
      + 2 \cos\frac\theta2 \sin\frac\theta2 \vhat u \times \vect x
      - \sin^2\frac\theta2 \left(\vhat u \times \vect x\right)\times \vhat u} \\
\intertext{Using the half-angle formulae}
  \sin^2\frac\theta2 &= \frac 1 2 (1-\cos\theta) \\
  \cos^2\frac\theta2 &= \frac 1 2 (1+\cos\theta) \\
\intertext{and the vector triple product identity}
  \left(\vhat u \times \vect x\right) \times \vhat u
    &= \vect x - \left(\vect x \cdot \vhat u\right) \vhat u \\
\intertext{the quaternion product equation~\eqref{eqn:quat_left_rot_QxVxQ} %
           simplifies to}
   \QxVxQ{\quat Q_{rot}}{\vect x}{\quatconj Q_{rot}} &=
   \quatsv
     0
     {\cos\theta \vect x +
      (1-\cos\theta) (\vhat u \cdot \vect x) \vhat u +
      \sin\theta \vhat u \times \vect x}
\end{align*}
\end{proof}

Just as rotation and transformation matrices are related via matrix
transposition, rotation and transformation quaternions are related via
quaternion conjugation:
\begin{theorem}\label{thm:quat_left_trans}
The left transformation unit quaternion
\begin{align}
  \tquat A B &= \quattrot{\theta}{\vhat u}\label{eqn:quat_left_trans_def} \\
\intertext{transforms a three-vector $\framevect A x$ expressed in frame $A$ %
           to frame $B$ whose axes are rotated %
           about the $\vhat u$ axis by an angle of $\theta$ relative to the %
           original frame $A$ via:}
  \quatsv 0 {\framevect B x}
    &= \QxVxQ{\tquat A B}{\framevect A x}{\tquatconj A B}
  \label{eqn:quat_left_trans_QxVxQ}
 \end{align}
\end{theorem}
\begin{proof}
Expanding and simplifying equation~\eqref{eqn:quat_left_trans_QxVxQ} results in
\begin{align*}
  \QxVxQ{\tquat A B}{\framevect A x}{\tquatconj A B} &=
   \quatsv
     0
     {\cos\theta \framevect A x
      + (1-\cos\theta) (\vhat u \cdot \framevect A x) \vhat u
      - \sin\theta \vhat u \times \framevect A x}
\end{align*}
\end{proof}

Consider a left transformation unit quaternion as
defined in equation~\eqref{eqn:quat_left_trans_def}.
\begin{theorem}\label{thm:quat_left_trans_log}
The quaternion logarithm of a left transformation unit quaternion
\begin{align}
  \quat Q &= \quattrot{\theta}{\vhat u} \nonumber \\
\intertext{is}
  \log \quat Q &=
    \quatsv 0 {-\frac 1 2 \theta {\vhat u}}\label{eqn:quat_left_trans_log}
\end{align}
\end{theorem}
\begin{proof}
The theorem follows immediately from theorem~\ref{thm:quat_log}.
\end{proof}


\subsubsection{Right Rotation and Transformation Quaternions}

One unfortunate aspect of quaternions is that they magnify the confusion
regarding \emph{rotation} and \emph{transformation}.
The conjugate of a \emph{left} rotation or transformation quaternion can also
be used as the basis for rotation or transformation.
For example, the \emph{right rotation unit quaternion}
\begin{align*}
  \quat Q_{rot,\text{right}} &= \quattrot{\theta}{\vhat u} \\
\intertext{\emph{rotates} a three-vector $\vect x$ about the
           $\vhat u$ axis by an angle of $\theta$ via:}
  \quatsv 0 {\vect x^\prime} &=
    \QxVxQ{\quatconj Q_{rot,\text{right}}}{\vect x}{\quat Q_{rot,\text{right}}}
\end{align*}
which yields the same value for $\vect x^\prime$
as does equation~\eqref{eqn:quat_left_rot_QxVxQ}.

The \ModelDesc uses left transformation unit quaternions
because they chain in the same manner
as do transformation matrices and because the dynamics
package is concerned with transformations, not rotations.


\subsubsection{Chains of Transformations and Rotations}\label{sec:app_chains}

Transformation matrices chain from right to left:
given a pair transformations $\tmat A B$ and $\tmat B C$
from frame A to frame B  and
from frame B to frame C, the transformation from
frame A to frame C is
\begin{equation}
\tmat A C = \MxM{\tmat B C}{\tmat A B} \label{eqn:quat_tmat_chain}
\end{equation}

Left transformation quaternions similarly chain from right to left:
\begin{theorem}\label{thm:quat_ltquat_chain}
\begin{equation}
\tquat A C = \QxQ{\tquat B C}{\tquat A B} \label{eqn:quat_ttquat_chain}
\end{equation}
\end{theorem}
\begin{proof}
A vector $\framevect A x$ transforms from frame $A$ to frame $B$
and from frame $B$ to $C$ via equation~\eqref{eqn:quat_left_trans_QxVxQ}:
\begin{align*}
  \quatsv 0 {\framevect B x} &=
    \QxVxQ{\tquat A B}{\framevect A x}{\tquatconj A B} \\
  \quatsv 0 {\framevect C x} &=
     \QxVxQ{\tquat B C}{\framevect B x}{\tquatconj B C} \\
     &= \QxQxQ
        {\tquat B C}
        {\left(\QxVxQ{\tquat A B}{\framevect A x}{\tquatconj A B}\right)}
        {\tquatconj B C} \\
\intertext{Since quaternion multiplication is associative,}
  \quatsv 0 {\framevect C x}
   &= \QxQxQ
   {\left(\QxQ{\tquat B C}{\tquat A B}\right)}
   {\quatsv 0 {\framevect A x}}
   {\left(\QxQ{\tquatconj A B}{\tquatconj B C}\right)} \\
   &= \QxQxQ
   {\left(\QxQ{\tquat B C}{\tquat A B}\right)}
   {\quatsv 0 {\framevect A x}}
   {\quatconjlr{\QxQ{\tquat B C}{\tquat A B}}}
\intertext{The direct transformation from frame $A$ to frame $C$ is}
  \quatsv 0 {\framevect C x} &=
    \QxVxQ{\tquat A C}{\framevect A x}{\tquatconj A C} \\
\intertext{from which}
  \tquat A C &= \QxQ{\tquat B C}{\tquat A B}
\end{align*}
\end{proof}

Rotation matrices chain from left to right:
Given a pair of rotation matrices $\mat{R}_{rot_1}$ and $\mat{R}_{rot_2}$,
the rotation matrix that represents performing rotation $rot_2$ after
performing rotation $rot_1$ is
\begin{align}
  \mat{R}_{rot_{1+2}} &=
    \MxM{\mat{R}_{rot_1}}{\mat{R}_{rot_2}} \label{eqn:quat_rmat_chain} \\
  \intertext{Left rotation quaternions still chain from right to left:}
  \quat Q_{rot_{1+2}} &=
    \QxQ{\quat Q_{rot_2}}{\quat Q_{rot_1}}
  \label{eqn:quat_rquat_chain}
\end{align}

\subsection{Quaternions and Alternate Representation Schemes}

\subsubsection{Single-Axis Rotations}

Quaternions are closely related to a single axis rotation.
Given a rotation about some axis $\vhat u$ by an angle $\theta$
that describes the relative orientation of two reference frames,
one merely need apply equation~\eqref{eqn:quat_left_trans_def}
to form the corresponding left transformation unit quaternion.

The next theorem establishes a constructive technique for determining
the single axis rotation given a left transformation unit quaternion.
\begin{theorem}\label{thm:quat_left_trans_to_rot}
Given a left transformation unit quaternion $\quat Q$
and its decomposition into a unit vector $\vhat u$ and angle $\theta^\prime$
per theorem~\ref{thm:quat_unit_decomp},
The single axis rotation corresponding
to $\quat Q$ is a rotation of an angle $\theta = -2 \theta^\prime$
about the $\vhat u$ axis.
\end{theorem}
\begin{proof}
Forming the left transformation unit quaternion from the single axis rotation
$\theta$ about $\vhat u$ per equation~\eqref{eqn:quat_left_trans_def} yields
the original left transformation unit quaternion $\quat Q$.
\end{proof}

\subsubsection{Transformation Matrices}\label{sec:app_to_mat}

\begin{theorem}\label{thm:quat_single_to_tmat}
Given a single axis rotation
about the $\vhat u$ axis by an angle of $\theta$
from reference frame $A$ to reference frame $B$,
the $i,j^{th}$element of the transformation matrix
$\tmat A B$ from frame $A$ to frame $B$ is
\begin{align}
  {\tmat A B}_{ij} &=
    \cos\theta\;
    \delta_{ij} + (1-\cos\theta) u_i u_j + \sin\theta \sum_k \epsilon_{ijk} u_k
\label{eqn:quat_single_to_tmat} \\
\intertext{where}
  \delta_{ij}\;&\text{is} \; \text{the Kronecker delta} \nonumber\\
  \epsilon_{ijk}\;&\text{is} \; \text{the permutation symbol:} \nonumber\\
  \epsilon_{ijk} &= \begin{cases}
    {\phantom{-}1} &
           \text{if $i$, $j$, and $k$ are an even permutation of $(1,2,3)$}, \\
    {-1} & \text{if $i$, $j$, and $k$ are an odd permutation of $(1,2,3)$}, \\
    {\phantom{-}0} & \text{otherwise ($i=j$, $i=k$, or $j=k$)}. \\
  \end{cases}\nonumber
\end{align}
\end{theorem}
\begin{proof}
The $i^{th}$ row of $\tmat A B$ contains the transpose of the unit vector
$\vhat e_i$ rotated about the $\vhat u$ axis by an angle of $\theta$.
(The unit vector $\vhat e_i$ contains a one in row $i$ and zeros elsewhere:
$\vhat e_{i_j} = \delta_{ij}$).
Applying equation~\eqref{eqn:quat_single_axis_rot} to $\vhat e_i$
and representing the cross product of two
vectors $a$ and $b$ as $(a\times b)_j = \sum_i \sum_k \epsilon_{ijk} a_k b_i$
results in equation~\eqref{eqn:quat_single_to_tmat}.
\end{proof}
 
 \begin{theorem}\label{thm:quat_quat_to_tmat}
 Given a left transformation unit quaternion
from reference frame $A$ to reference frame $B$
 $\tquat A B = \quatsv {q_s}{\vect{q_v}}$,
the transformation matrix corresponding to $\tquat A B$ is
\begin{equation}
 {\tmat A B}_{ij} = (2 q_s^{\phantom{s}2} - 1) \delta_{ij}
   + 2 (q_{v_i} q_{v_j} - \sum_k \epsilon_{ijk} q_s q_{v_k})
\label{eqn:quat_quat_to_tmat}
\end{equation}
\end{theorem}
\begin{proof}
By using the half-angle formulae
\begin{align*}
  \cos\theta  &= 2 \cos^2\frac\theta2 - 1 = 1 - 2\sin^2\frac\theta2 \\
  \sin\theta   &= 2 \cos\frac\theta2 \sin\frac\theta2 \\
\intertext{equation~\eqref{eqn:quat_single_to_tmat} becomes}
  {\tmat A B}_{ij} &=
    (2 \cos^2\frac\theta2 - 1) \delta_{ij}
    + 2\sin^2\frac\theta2 u_i u_j
    + 2 \cos\frac\theta2 \sin\frac\theta2 \sum_k \epsilon_{ijk} u_k \\
\intertext{Per equation~\eqref{eqn:quat_left_trans_def},}
  q_s &= \cos\frac\theta2 \\
  \vect{q_v} &= -\sin\frac\theta2{\vhat u}
\end{align*}
by which equation~\eqref{eqn:quat_single_to_tmat} further reduces to 
equation~\eqref{eqn:quat_quat_to_tmat}.
\end{proof}

\begin{theorem}\label{thm:quat_tmat_to_quat}
Given a transformation matrix
from reference frame $A$ to reference frame $B$ $\tmat A B$
the left transformation unit quaternion $\tquat A B$ with
scalar and vector parts $q_s$ and $\vect{q_v}$
corresponding to $\tmat A B$ is given by four methods
labeled $q_s$ and $q_{v_i}, i\in(0,1,2)$. \\
\begin{align}
\intertext{Defining}
  tr &\equiv \operatorname{tr}(\tmat A B) \\
  t_i &\equiv {\tmat A B}_{ii} - ({\tmat A B}_{jj} + {\tmat A B}_{kk}) \\
  d_k &\equiv {\tmat A B}_{ji} - {\tmat A B}_{ij} \quad (\epsilon_{ijk} = 1) \\
  s_{ij} &\equiv {\tmat A B}_{ji} + {\tmat A B}_{ij} \quad(i \neq j)
\end{align}
Method $q_s$ is
\begin{equation}
\label{eqn:tmat_to_quat_meth_qs}
\left. %{
\begin{aligned}
  f_1 &= \sqrt{tr+1}\qquad \\
  f_2 &= \frac 1{2 f_1} \\
  q_s &= \frac 1 2 f_1 \\
  q_{v_i} &= d_i f_2 \\
  q_{v_j} &= d_j f_2 \\
  q_{v_k} &= d_k f_2
\end{aligned}\right\}
\end{equation}
Methods $q_{v_i}, i\in(0,1,2)$ are
\begin{equation}
\label{eqn:tmat_to_quat_meth_qvi}
\left. %{
\begin{aligned}
  f_1 &= \sqrt{t_i+1}\qquad\\
  f_2 &= \frac 1{2 f_1} \\
  q_{v_i} &= \frac 1 2 f_1 \\
  q_{v_j} &= s_{ij} f_2 \\
  q_{v_k} &= s_{ik} f_2 \\
  q_s &= d_i f_2
\end{aligned}\right\}
\end{equation}
\end{theorem}
\begin{proof}
By equation~\eqref{eqn:quat_quat_to_tmat},
the terms $tr$, $t_i$, $d_k$ and $s_{ij}$ are
\begin{align}
  tr     &= \operatorname{tr}(\tmat A B) =  4 q_s^{\phantom{s}2} - 1 \\
  t_i    &= {\tmat A B}_{ii} - ({\tmat A B}_{jj} + {\tmat A B}_{kk})
          = 4 q_{v_i}^2 - 1 \\
  d_k    &= {\tmat A B}_{ji} - {\tmat A B}_{ij} \  (\epsilon_{ijk} = 1)
          = 4 q_s q_{v_k} \\
  s_{ij} &= {\tmat A B}_{ji} + {\tmat A B}_{ij}  \  (i \neq j)
          = 4 q_{v_i} q_{v_j}
\end{align}
from which equations~\eqref{eqn:tmat_to_quat_meth_qs}
and~\eqref{eqn:tmat_to_quat_meth_qvi}
are readily derived.
\end{proof}

\subsubsection{Euler Angles}\label{sec:app_to_euler}

Euler proved in 1776 that any arbitrary rotation or transformation can be
described by only three parameters. An Euler sequence comprises a sequence of
rotations by various angles about various axes in a specified order.
For example, a yaw-pitch-roll transformation sequence involves a transformation
about the $z-$axis followed by a second transformation about the transformed
$y-$axis followed by a final rotation about the doubly-transformed $x-$axis.

\begin{theorem}\label{thm:quat_rot_to_quat_general}
Given an Euler sequence comprising:
\begin{itemize}
\item A rotation through an angle $\theta_1$ counterclockwise
about the $\vhat u_1$ axis followed by
\item A rotation through an angle $\theta_2$ counterclockwise
about the $\vhat u_2$ axis followed by
\item A rotation through an angle $\theta_3$ counterclockwise
about the $\vhat u_3$ axis
\end{itemize}
that represents the transformation from
from reference frame $A$ to reference frame $B$,
the corresponding left transformation unit quaternion
$\quat{Q}_{123}
  (\theta_{1}, \vhat u_1; \theta_{2}, \vhat u_2; \theta_{3}, \vhat u_3)$ is
\begin{align}
  \quat{Q}_{123}
    (\theta_{1}, \vhat u_1; \theta_{2}, \vhat u_2; \theta_{3}, \vhat u_3) &=
    \QxQxQ{\quat{Q}(\theta_3, \vhat u_3)}
          {\quat{Q}(\theta_2, \vhat u_2)}
          {\quat{Q}(\theta_1, \vhat u_1)}
    \label{eqn:quat_euler_to_quat_general} \\
  \intertext{where}
  \quat{Q}(\theta_1, \vhat u_1) &\equiv \quattrot{\theta_1}{\vhat u_1} \\
  \quat{Q}(\theta_2, \vhat u_2) &\equiv \quattrot{\theta_2}{\vhat u_2} \\
  \quat{Q}(\theta_3, \vhat u_3) &\equiv \quattrot{\theta_3}{\vhat u_3}
\end{align}
\end{theorem}
\begin{proof}
The left transformation quaternions corresponding to the individual rotations,
$\quat{Q}(\theta_1, \vhat u_1)$,
$\quat{Q}(\theta_2, \vhat u_2)$, and
$\quat{Q}(\theta_3, \vhat u_3)$,
follow from equation~\eqref{eqn:quat_left_trans_def}.
The product follows from equation~\eqref{eqn:quat_ttquat_chain}.
\end{proof}

The Trick package is only concerned with Euler sequences that use some
permutation of rotations about the $x$, $y$, and $z$ axes.
(Trick does not cover the standard astronaumical $\phi-\theta-\psi$ sequence.)
With this restriction, equation~\eqref{eqn:quat_euler_to_quat_general} can be
reduced to two cases:
one if the unit vectors $(\vhat u_1, \vhat u_2, \vhat u_3)$
form an even permutation of $(x, y, z)$
and another if the unit vectors $(\vhat u_1, \vhat u_2, \vhat u_3)$
form an odd permutation of $(x, y, z)$.

\begin{theorem}\label{thm:quat_rot_to_quat_even}
Given an Euler sequence
$\theta_1$ about the $\vhat u_1$ axis,
$\theta_2$ about the $\vhat u_2$ axis,
$\theta_3$ about the $\vhat u_3$ axis,
where $(\vhat u_1, \vhat u_2, \vhat u_3)$ is an
even permutation of $(\vhat x, \vhat y, \vhat z)$,
that represents the transformation from
from reference frame $A$ to reference frame $B$,
the corresponding left transformation unit quaternion
$\quat{Q}_{123}
  (\theta_{1}, \vhat u_1; \theta_{2}, \vhat u_2; \theta_{3}, \vhat u_3)$
is
\begin{equation}
  \quat{Q}_{123}
    (\theta_{1}, \vhat u_1; \theta_{2}, \vhat u_2; \theta_{3}, \vhat u_3) =
       \quatsv[5pt]
        {\cos\frac{\theta_3}2 \cos\frac{\theta_2}2 \cos\frac{\theta_1}2
          - \sin\frac{\theta_3}2 \sin\frac{\theta_2}2 \sin\frac{\theta_1}2}
        {
          - \begin{pmatrix}
          \phantom{+}
            (\cos\frac{\theta_3}2 \cos\frac{\theta_2}2 \sin\frac{\theta_1}2
            + \sin\frac{\theta_3}2 \sin\frac{\theta_2}2 \cos\frac{\theta_1}2)
            \vhat u_1 \\[5pt]
          +
            (\cos\frac{\theta_3}2 \sin\frac{\theta_2}2 \cos\frac{\theta_1}2
            - \sin\frac{\theta_3}2 \cos\frac{\theta_2}2 \sin\frac{\theta_1}2)
            \vhat u_2 \\[5pt]
          +
           (\cos\frac{\theta_3}2 \sin\frac{\theta_2}2 \sin\frac{\theta_1}2
            + \sin\frac{\theta_3}2 \cos\frac{\theta_2}2 \cos\frac{\theta_1}2)
           \vhat u_3
           \end{pmatrix}}
   \label{eqn:quat_rot_to_quat_even}
\end{equation}
\end{theorem}
\begin{proof}
This follows directly by expanding
equation~\eqref{eqn:quat_euler_to_quat_general}
for the special case of a set of orthogonal unit vectors
$\vhat u_1$, $\vhat u_2$, $\vhat u_3$
for which $\vhat u_1 \times \vhat u_2 = \vhat u_3$.
\end{proof}

\begin{theorem}\label{thm:quat_rot_to_quat_odd}
Given an Euler sequence
$\theta_1$ about the $\vhat u_1$ axis,
$\theta_2$ about the $\vhat u_2$ axis,
$\theta_3$ about the $\vhat u_3$ axis,
where $(\vhat u_1, \vhat u_2, \vhat u_3)$ is an
odd permutation of $(\vhat x, \vhat y, \vhat z)$,
that represents the transformation from
from reference frame $A$ to reference frame $B$,
the corresponding left transformation unit quaternion
$\quat{Q}_{123}
  (\theta_{1}, \vhat u_1; \theta_{2}, \vhat u_2; \theta_{3}, \vhat u_3)$
is
\begin{equation}
  \quat{Q}_{123}
    (\theta_{1}, \vhat u_1; \theta_{2}, \vhat u_2; \theta_{3}, \vhat u_3) =
       \quatsv[5pt]
        {\cos\frac{\theta_3}2 \cos\frac{\theta_2}2 \cos\frac{\theta_1}2
          + \sin\frac{\theta_3}2 \sin\frac{\theta_2}2 \sin\frac{\theta_1}2}
        {
          - \begin{pmatrix}
          \phantom{+}
            (\cos\frac{\theta_3}2 \cos\frac{\theta_2}2 \sin\frac{\theta_1}2
            - \sin\frac{\theta_3}2 \sin\frac{\theta_2}2 \cos\frac{\theta_1}2)
            \vhat u_1 \\[5pt]
          +
            (\cos\frac{\theta_3}2 \sin\frac{\theta_2}2 \cos\frac{\theta_1}2
            + \sin\frac{\theta_3}2 \cos\frac{\theta_2}2 \sin\frac{\theta_1}2)
            \vhat u_2 \\[5pt]
          -
           (\cos\frac{\theta_3}2 \sin\frac{\theta_2}2 \sin\frac{\theta_1}2
            - \sin\frac{\theta_3}2 \cos\frac{\theta_2}2 \cos\frac{\theta_1}2)
            \vhat u_3
           \end{pmatrix}}
   \label{eqn:quat_rot_to_quat_odd}
\end{equation}
\end{theorem}
\begin{proof}
This follows directly by expanding
equation~\eqref{eqn:quat_euler_to_quat_general}
for the special case of a set of orthogonal unit vectors
$\vhat u_1$, $\vhat u_2$, $\vhat u_3$
for which $\vhat u_1 \times \vhat u_2 = - \vhat u_3$.
\end{proof}

With a bit of trigonemtric manipulation,
the inverse operation of determining the Euler angles given a quaternion
and a rotation sequence follow from theorems~\ref{thm:quat_rot_to_quat_even}
and~\ref{thm:quat_rot_to_quat_odd}.

\begin{theorem}\label{thm:quat_quat_to_rot_even}
Given an even permutation $(\vhat u_1, \vhat u_2, \vhat u_3)$ of
$(\vhat x, \vhat y, \vhat z)$
and a quaternion \\
$\quat Q = \quatsv {q_s}
                   {q_{v_1} \vhat u_1 +
                    q_{v_2} \vhat u_2 +
                    q_{v_3} \vhat u_3}$,
the Euler sequence
$\theta_1$ about the $\vhat u_1$ axis,
$\theta_2$ about the $\vhat u_2$ axis,
$\theta_3$ about the $\vhat u_3$ axis is given by

\begin{align}
  \theta_2 &= \arcsin(-2(q_s q_{v_2} - q_{v_1} q_{v_3}))
    \label{eqn:quat_quat_to_rot_even_theta_2} \\
  \theta_1 &= \arctan(-2(q_s q_{v_1} + q_{v_2} q_{v_3}),
                     ({q_s}^2-q_{v_2}^2)-(q_{v_1}^2-q_{v_3}^2))
    \label{eqn:quat_quat_to_rot_even_theta_1} \\
  \theta_3 &= \arctan(-2(q_s q_{v_3} + q_{v_2} q_{v_1}),
                     ({q_s}^2-q_{v_2}^2)-(q_{v_3}^2-q_{v_1}^2))
    \label{eqn:quat_quat_to_rot_even_theta_3} \\
\intertext{Equations~\eqref{eqn:quat_quat_to_rot_even_theta_1} %
           and~\eqref{eqn:quat_quat_to_rot_even_theta_3} are valid %
           only if $\abs{\sin\theta_2} \neq 1$. %
           $\abs{\sin\theta_2} = 1$ represents a singularity, %
           in which case $\theta_1$ and $\theta_3$ are related via}
  \theta_1 + \theta_3 \sin\theta_2 &=
      2 \arctan(-q_{v_3} \sin\theta_2, q_{v_2} \sin\theta_2) =
      2 \arctan(-q_{v_1},q_s)
    \label{eqn:quat_quat_to_rot_even_singular}
\end{align}
\end{theorem}
\begin{proof}
Th solution for $\theta_2$ and the non-singular solutions for
$\theta_1$ and $\theta_3$ follow by replacing the arguments of the inverse
trigonometric functions in equations~\eqref{eqn:quat_quat_to_rot_even_theta_2}
to~\eqref{eqn:quat_quat_to_rot_even_theta_3} into
equation~\eqref{eqn:quat_rot_to_quat_even}.
The singular solutions follow by replacing $\theta_2 = \pm \frac \pi 2$ into
equation~\eqref{eqn:quat_rot_to_quat_even}.
\end{proof}

\begin{theorem}\label{thm:quat_quat_to_rot_odd}
Given an odd permutation $(\vhat u_1, \vhat u_2, \vhat u_3)$ of
$(\vhat x, \vhat y, \vhat z)$
and a quaternion \\
$\quat Q = \quatsv {q_s}
                   {q_{v_1} \vhat u_1 +
                    q_{v_2} \vhat u_2 +
                    q_{v_3} \vhat u_3}$,
the Euler sequence
$\theta_1$ about the $\vhat u_1$ axis,
$\theta_2$ about the $\vhat u_2$ axis,
$\theta_3$ about the $\vhat u_3$ axis is given by

\begin{align}
  \theta_2 &= \arcsin(-2(q_s q_{v_2} + q_{v_1} q_{v_3}))
    \label{eqn:quat_quat_to_rot_odd_theta_2} \\
  \theta_1 &= \arctan(-2(q_s q_{v_1} - q_{v_2} q_{v_3}),
                     ({q_s}^2-q_{v_2}^2)-(q_{v_1}^2-q_{v_3}^2))
    \label{eqn:quat_quat_to_rot_odd_theta_1} \\
  \theta_3 &= \arctan(-2(q_s q_{v_3} - q_{v_2} q_{v_1}),
                     ({q_s}^2-q_{v_2}^2)-(q_{v_3}^2-q_{v_1}^2))
    \label{eqn:quat_quat_to_rot_odd_theta_3} \\
\intertext{Equations~\eqref{eqn:quat_quat_to_rot_odd_theta_1} %
           and~\eqref{eqn:quat_quat_to_rot_odd_theta_3} are valid %
           only if $\abs{\sin\theta_2} \neq 1$.%
           $\abs{\sin\theta_2} = 1$ represents a singularity, %
           in which case $\theta_1$ and $\theta_3$ are related via}
  \theta_1 - \theta_3 \sin\theta_2 &=
      2 \arctan(q_{v_3} \sin\theta_2,-q_{v_2} \sin\theta_2) =
      2 \arctan(-q_{v_1},q_s)
    \label{eqn:quat_quat_to_rot_odd_singular}
\end{align}
\end{theorem}
\begin{proof}
Th solution for $\theta_2$ and the non-singular solutions for
$\theta_1$ and $\theta_3$ follow by replacing the arguments of the inverse
trigonometric functions in equations~\eqref{eqn:quat_quat_to_rot_odd_theta_2}
to~\eqref{eqn:quat_quat_to_rot_odd_theta_3} into
equation~\eqref{eqn:quat_rot_to_quat_odd}.
The singular solutions follow by replacing $\theta_2 = \pm \frac \pi 2$ into
equation~\eqref{eqn:quat_rot_to_quat_odd}.
\end{proof}
 

\subsection{Comparing and Averaging Transformation Quaternions}

Suppose $\vect x_1$ and $\vect x_2$ are two vectors
that represent the some comparable quantity
such as the position of a spacecraft at a specific point in time as computed
by two different methods
or the position of a spacecraft at two different points in time.
Various analyses are frequently based on the
difference between the two vectors,
$\Delta \vect x \equiv \vect x_2 - \vect x_1$.

Now suppose $\quat Q_1$ and $\quat Q_2$ are two left transformation
quaternions that that represent some comparable transformation.
The additive difference between the two transformation quaternions has no
physical meaning and thus is not useful for analysis.
The normative mathematics of the transformation quaternions is
multiplication rather than addition.
A multiplicative rather than additive scheme is needed to
determine the "distance" between two quaternions.
\begin{definition}[Quaternion difference]\label{def:quat_err}
Given quaternions $\quat{Q}_1$ and $\quat{Q}_2$,
the difference between the two quaternions is defined as
\begin{equation}
  \Delta \quat Q \equiv \ 
                 \operatorname{acute} (\QxQ{\quat{Q}_2}{\quat{Q}_1^\conj})
  \label{eqn:quat_err_def}
\end{equation}
where \emph{acute} denotes that all components of the product
$\QxQ{\quat{Q}_2}{\quat{Q}_1^\conj}$
are to be negated when the scalar part of the product is negative.
\end{definition}

The magnitude of an error vector is often a more significant indicator
than is the direction of the error when analyzing errors in vectors.
Similarly, the single axis rotation angle computed from the quaternion
difference is often a more significant indicator than the direction of the
rotation when analyzing errors in transformation quaternions.

The difference between two vectors is also useful for computing
a weighted average:
$\bar{\vect x} = w  \Delta \vect x + \vect x_1$ where $w$ is some
weight factor between 0 and 1.
The arithmetic mean results when $w=1/2$.

Similarly, the weighted mean of two quaternions is
 \begin{definition}[Quaternion weighted mean]\label{def:quat_wmean}
Given quaternions $\quat{Q}_1$ and $\quat{Q}_2$,
the weighted mean of the two quaternions is defined as
\begin{equation}
  \bar{\quat Q} = \QxQ{(\Delta \quat Q)^w}{\quat{Q}_1}\label{eqn:quat_wmean_def}
\end{equation}
where $w$ is a weight factor between 0 and 1 and
$\Delta \quat Q$ is computed via equation~\eqref{eqn:quat_err_def}.
\end{definition}

\begin{align*}
\intertext{Let}
  \Delta \quat Q &= \quatsv{\Delta q_s}{\vect {\Delta q_v}}
\intertext{if $\abs{\Delta q_s} \approx 1$ then $(\Delta \quat Q)^w$ %
           can be approximated using the small angle approximations} \\
  \cos(w\theta) &\approx 1 + w^2(\cos\theta-1) \\
  \sin(w\theta) &\approx w\sin\theta \\
  (\Delta \quat Q)^w &\approx
    \quatsv{1 + w^2(\Delta q_s-1)}{w \vect {\Delta q_v}}
\end{align*}

