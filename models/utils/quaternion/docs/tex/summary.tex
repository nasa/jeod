%%%%%%%%%%%%%%%%%%%%%%%%%%%%%%%%%%%%%%%%%%%%%%%%%%%%%%%%%%%%%%%%%%%%%%%%%%%%%%%%
%
% Purpose: Quaternion model executive summary
%
% 
%
%%%%%%%%%%%%%%%%%%%%%%%%%%%%%%%%%%%%%%%%%%%%%%%%%%%%%%%%%%%%%%%%%%%%%%%%%%%%%%%%

\chapter*{Executive Summary}
Sir William Rowan Hamilton developed a four dimensional extension to the complex
numbers called quaternions in the mid 1800s. Like the reals and the complex
numbers, quaternions can be added, subtracted, multiplied, and divided.
Unlike the reals and complex numbers, quaternion multiplication and division
is not commutative. Care must be taken in specifying the order of the
factors when multiplying quaternions.

Quaternions, like complex numbers, can be represented as comprising real
imaginary parts. In the case of quaternions, there are three distinct roots
of -1, $i$, $j$, and $k$. The imaginary part of a quaternion can be viewed
as a three vector. (In fact, the use of $\hat i$, $\hat j$, and $\hat k$ as
canonical unit vectors in mathematics and physics was motivated by Hamilton's
quaternions.)

One widely used application of the quaternions is to represent rotations and
transformations in three dimensional space.
That application is the subject of the JEOD Quaternion model.
The full power of the quaternions is not needed for this application.
Unit quaternions (quaternions whose magnitude is identically one) suffice
for representing rotations in Euclidean three-space.
The principal operations of interest for this use of quaternions are
\begin{itemize}
\item Data representation.
Quaternions must be represented by some means to enable their use.
The JEOD quaternion model represents quaternions as comprising
a scalar real part and an imaginary vector part.
\item Multiplication.
A sequence of rotations or transformations in three space
maps to a product of quaternion representations of those rotations /
transformations.
\item Normalization.
The quaternions used to represent rotations / transformations in three space
should be unit quaternions. Normalizing a quaternion makes it a unit quaternion.
\item Conjugation.
One advantage of using unit quaternions is that the inverse is particularly
easy to compute: It is the quaternion's conjugate.
\item Conversion.
Quaternions are but one of many ways to represent rotations and transformations
in three space. Conversions between quaternions and other schemes can be
quite useful.
\end{itemize}
