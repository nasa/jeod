%%%%%%%%%%%%%%%%%%%%%%%%%%%%%%%%%%%%%%%%%%%%%%%%%%%%%%%%%%%%%%%%%%%%%%%%
%
% Purpose: Abstract for the Reference Frame Model
%
% 
%
%%%%%%%%%%%%%%%%%%%%%%%%%%%%%%%%%%%%%%%%%%%%%%%%%%%%%%%%%%%%%%%%%%%%%%%%%

\begin{abstract}

For a vehicle in orbit, there are many interrelated reference frames
that can be defined, often with respect to one another. Inertial frames,
planet fixed frames, vehicle body frames, and vehicle structural frames
are all important concepts when modeling a vehicle in planetary orbit.
Additionally, knowledge of the state of the vehicle, often with respect
to many interrelated reference frames, is critical to successful modeling
of the vehicle dynamics.

The JEOD \refframesDesc\ provides a structure for specifying
trees of reference frames, as well as the relative states between them.
Additionally, this model contains the tools for iterating over
a generic tree
structure, which allows for implementation of common tree algorithms
over other models.

Additionally, the \refframesDesc\ also provides a Reference Frame Manager class, used
to perform common Reference Frame related tasks, such as registering and searching
for reference frames, building a reference frame tree, and interacting with the reference
frame subscription functionality.

\end{abstract}
