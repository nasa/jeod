%%%%%%%%%%%%%%%%%%%%%%%%%%%%%%%%%%%%%%%%%%%%%%%%%%%%%%%%%%%%%%%%%%%%%%%%%%%%%%%%
% guide.tex
% User Guide for the Simulation Interface Model.
%
%
%%%%%%%%%%%%%%%%%%%%%%%%%%%%%%%%%%%%%%%%%%%%%%%%%%%%%%%%%%%%%%%%%%%%%%%%%%%%%%%%

\chapter{User Guide}\hyperdef{part}{user}{}\label{ch:user}

\section{Instructions for Simulation Users}
Simulation users interact with the model by means of the
\verb|jeod_sys| simulation object, which contains
the singular simulation interface object,
the singular message handler object, and
the singular memory manager object.
See section~\ref{sec:sim_developer_instructions} for details on this object.

The members of this simulation object mostly function without user intervention.
This section describes those cases where user intervention can be used.

\subsection{Memory Manager}
User control over the memory manager is limited to specifying the
memory manager debug level. The minimal setting results in the memory
manager reporting serious errors only. The maximal setting causes
every transaction to be reported.

Refer to the User Guide chapter of
\hypermodelrefinside{MEMORY}{part}{user} for details.

\subsection{Message Handler}
Users can direct the message handler to suppress messages that are
below some severity threshold and can to some extent control how
messages are formatted.
Refer to the User Guide chapter of
\hypermodelrefinside{MESSAGE}{part}{user} for details.

The Trick-specific message handler provided with this model adds the
ability to suppress messages by their code. For example, the
message handler reports JEOD-allocated memory that has not
been freed by the time the memory manager is destroyed with
the code \verb|utils/memory/corrupted_memory|. Messages tagged
with this code can be suppressed from the Python input file via
\begin{codeblock}
jeod_sys.message_handler.add_suppressed_code("utils/memory/corrupted_memory")
\end{codeblock}


\subsection{Checkpoint/Restart}
JEOD is checkpointable and restartable in the Trick 13 environment.
The standard \verb|jeod_sys| simulation object provided with JEOD
registers jobs that cause the \ModelDesc to create a JEOD checkpoint file
whenever a Trick checkpoint file is created.
The \ModelDesc saves JEOD content to this JEOD checkpoint file as a part of
the Trick checkpoint process.

The  \verb|jeod_sys| simulation object
similarly registers jobs that cause the \ModelDesc to restore JEOD content
from a JEOD checkpoint file as a part of the restart procress.
The \ModelDesc restores JEOD content from a JEOD checkpoint file as a part of
the Trick restart process.

Trick provides several mechanisms for creating checkpoint files and
for restoring from a checkpoint.
A set of post-initialization checkpoint files can be created via
\begin{codeblock}
trick_mm.mmw.set_post_init_checkpoint(1)
\end{codeblock}
Post-initialization and end checkpoint files can be created via similar
mechanisms.

This class of checkpoint files typically are not useful for restart.
Checkpoint files suitable for restart are created by Trick events
at scheduled to run at either the beginning or end of a Trick frame.
The following is the code used to create a checkpoint file in the
\verb|SET_test/RUN_prop_checkpoint| run
of the
\hypermodelref{EPHEMERIDES} simulation \verb|SIM_prop_planet_10|.
\begin{codeblock}
# Import the JEOD checkpoint/restart module.
import sys
import os
sys.path.append ('/'.join([os.getenv("JEOD_HOME"), "lib/jeod/python"]))
import jeod_checkpoint_restart

... (omitted code)

# Drop a checkpoint 60 days into the simulation.
jeod_checkpoint_restart.create_checkpoint (60*86400, 86400*(365*150+30))
\end{codeblock}

This uses the JEOD Python library file \verb|jeod_checkpoint_restart.py|
to do most of the work. The function \verb|create_checkpoint| in that
module takes two arguments. The first is the simulation time at which the
checkpoint is to be created. The second is the simulation time at which
a simulation restarted from this checkpoint file is to end.

The Trick and JEOD checkpoint files are named according to the time at which the
checkpoint is created. In the above example, the checkpoint files will be named
\verb|chkpnt_5184000.000000| and \verb|jeod_chkpnt_5184000.000000|. The
\verb|SET_test/RUN_prop_restart| run of the same simulation loads the checkpoint
files via the following:
\begin{codeblock}
# Import the JEOD checkpoint/restart module.
import sys
import os
sys.path.append ('/'.join([os.getenv("JEOD_HOME"), "lib/jeod/python"]))
import jeod_checkpoint_restart

... (omitted code)

# Restart the sim from the checkpoint data in SET_test/RUN_prop_checkpoint.
jeod_checkpoint_restart.restore_from_checkpoint (
  "SET_test/RUN_prop_checkpoint", "chkpnt_5184000.000000")
\end{codeblock}

\section{Instructions for Simulation Developers}
\label{sec:sim_developer_instructions}
\subsection{The \code{jeod\_sys} Simulation Object}
\label{sec:guide_jeod_sys}
All JEOD-based simulations must contain an instance of a
\verb|JeodSimulationInterface| object.
This object must be the first JEOD object that is constructed and
the last JEOD object that is destroyed during the course of execution of the
simulation.

Use
\begin{codeblock}
#include "JEOD_S_modules/jeod_sys.sm"
\end{codeblock}
where \verb|JEOD_S_modules| is a symbolic link to
\verb|$JEOD_HOME/sims/shared/Trick10/S_modules|.
To use a MessageHandler class  other than the TrickMessageHandler,
\verb|#define MESSAGE_HANDLER_CLASS| to the desired class name.

\subsection{Integration Jobs}
\label{sec:guide_integ}
The \DYNMANAGER\ provides the ability to monolithically integrate all of the
\verb|DynBody| objects that comprise a simulation.
Refer to the User Guide chapter of
\hypermodelrefinside{DYNMANAGER}{part}{user} for details.

The \ModelDesc provides a Trick-specific multiple integration group capability.
One way to use this capability is to use the JEOD S\_module
\verb|integ_loop.sm|. This defines the
\verb|JeodIntegLoopSimObject| simulation object class.
You will need to \verb|#include| this S\_module in your \Sdefine file:
\begin{codeblock}
#include "JEOD_S_modules/jeod_sys.sm"
\end{codeblock}
Unlike the other Trick 13 JEOD S\_modules,
this S\_module does not create an instance of simulation object class.
Your \Sdefine must declare these instances via
the  \verb|JeodIntegLoopSimObject| non-default constructor.
The following example is from the
\hypermodelref{EPHEMERIDES} simulation \verb|SIM_prop_planet_10|.
\begin{codeblock}
JeodIntegLoopSimObject fast_integ_loop (
   FAST_DYNAMICS,
   integ_constructor.selected_constr, integ_constructor.group,
   jeod_time.time_manager, dynamics.dyn_manager, env.gravity_manager,
   &sun, &jupiter, &saturn, nullptr);

JeodIntegLoopSimObject slow_integ_loop (
   SLOW_DYNAMICS,
   integ_constructor.selected_constr, integ_constructor.group,
   jeod_time.time_manager, dynamics.dyn_manager, env.gravity_manager,
   nullptr);
\end{codeblock}

The only public constructor for the \verb|JeodIntegLoopSimObject| is the
non-default constructor used in the above example. This constructor takes
six required arguments plus an indeterminate number of optional arguments.
The six required arguments are\begin{description}
\item[{\tt integ\_cycle}]
The time interval between calls to the integration loop's integrate function.

\item[{\tt integ\_cotr\_in}]
A pointer to the integration constructor that is to be used to create
integration artifacts for the integration group and the objects integrated
by the integration group.

\item[{\tt integ\_group\_in}]
An integration group that is to be used as a prototype for creating the
integration loop's integration group object.

\item[{\tt time\_manager\_in}]
The simulation-wide JEOD time manager.

\item[{\tt dyn\_manager\_in}]
The simulation-wide JEOD dynamics manager.

\item[{\tt grav\_model\_in}]
The simulation-wide JEOD gravity model.
\end{description}

Another option is to build a custom simulation object class.
This class must contain an object of type \verb|JeodDynbodyIntegrationLoop|.
That object needs to be constructed using the non-default constructor
for that class. The simulation object class must register
\verb|initialization|, \verb|derivative|, and \verb|integ_loop| jobs
as  the \verb|JeodIntegLoopSimObject| class does.
Use the \verb|JeodIntegLoopSimObject| as a guide for these efforts.

\section{Instructions for Model Developers}

This section describes the use of the model from the perspectives of
a user of the STL sequence container replacements,
a model developer who wishes to write a class that derives from
one of the provided checkpointable/restartable classes,
and a simulation interface developer who is porting the model outside of Trick.


\subsection{Using the Simulation Interface Model}
External users can selected public interfaces of the
global simulation interface object.
Most of these are well-described in the \ModelDesc API.

External users who wish to use the JEOD checkpoint/restart capability
can either
create/read from their own sections within the JEOD checkpoint file or
create/read from their own partitioned checkpoint file..
This sections that follow describes both of these options.

\subsection{Creating and Restoring From a Section in the JEOD Checkpoint File}
External users can write to the JEOD checkpoint file in a function
registered with Trick as a checkpoint or post\_checkpoint job.
The steps involved in creating a section in
the JEOD checkpoint file are:
\begin{itemize}
\item Create a \verb|SectionedOutputStream| object
by calling the static function \verb|get_checkpoint_writer|
in the class \verb|JeodSimulationInterface|.
\item Activate the \verb|SectionedOutputStream| object.
This activation will fail if another such output stream
is currently active.
\item Write to the \verb|SectionedOutputStream| object using
the standard \Cplusplus stream insertion operator.
\item Deactivate the \verb|SectionedOutputStream| object.
\end{itemize}
An abbreviated version of the above follows.
\begin{codeblock}
SectionedOutputStream writer (
   JeodSimulationInterface::get_checkpoint_writer (section_id));
if (! writer.activate()) {
   // Handle error
}
while (! done_writing) {
   writer << more_stuff_to_write();
}
writer.deactivate();
\end{codeblock}

Restoring from that section is accomplished in a function registered
with Trick as preload\_checkpoint or restart job.
The steps involved in restoring from a section in
the JEOD checkpoint file are:
\begin{itemize}
\item Create a \verb|SectionedInputStream| object
by calling the static function \verb|get_checkpoint_reader|
in the class \verb|JeodSimulationInterface|.
\item Activate the \verb|SectionedInputStream| object.
This activation will fail if another such input stream
is currently active.
\item Read from the \verb|SectionedInputStream| object using
\verb|std::getline| or
the \Cplusplus stream insertion operator.
\item Deactivate the \verb|SectionedInputStream| object.
\end{itemize}
An abbreviated version of the above follows.
\begin{codeblock}
SectionedInputStream reader (
   JeodSimulationInterface::create_checkpoint_reader (section_id));
if (! reader.activate()) {
   // Handle error
}
while (std::getline (reader, line)) {
   process_line(line);
}
reader.deactivate();
\end{codeblock}

Three restrictions apply to those who wish to write to or read from
the JEOD checkpoint file:
\begin{itemize}
\item The substring "JEOD" must not appear anywhere in the name of the section.
These names are reserved for use by JEOD.
\item You must ``play nice''. You need to close your output and input section
streams when you are done using them. Streams that are left open will prevent
JEOD from writing or reading its checkpoint file sections.
\item The checkpoint file is only available during the checkpoint and restart
processes.
To write to the JEOD checkpoint file you must do this in the context
of a checkpoint job with a phase greater than 0 or a
post\_checkpoint job with a phase less 65535.
To read from the JEOD checkpoint file you must do this in the context
of a preload\_checkpoint job with a phase greater than 0 or a
restart job with a phase less 65535.
\end{itemize}

\subsection{Creating and Restoring From a Custom Partitioned Checkpoint File}
An alternative to the above is to create your own partitioned checkpoint file.
This alternative removes the constraints on using the JEOD checkpoint file
but adds the burden yet another checkpoint file that needs to
be managed.

The class \verb|CheckPointOutputManager| provides the ability to create a
partitioned checkpoint file.
Programmatic users of this class should:
\begin{itemize}
\item Create a \verb|CheckPointOutputManager| object via the non-default
constructor. The non-default constructor takes three arguments:
the name of the file,
the string that denotes the start of a section, and
the string that denotes the end of a section.
This opens a \Cplusplus output stream to  the specified checkpoint file.
That stream is private; it is accessed indirectly by
\verb|SectionedOutputStream| objects created by the
\verb|CheckPointOutputManager| object.
\begin{codeblock}
checkpoint_writer = new CheckPointOutputManager (
   output_file_name, section_start, section_end);
\end{codeblock}

\item For each section to be created,
\begin{itemize}
\item Create a \verb|SectionedOutputStream| object
by calling the \verb|CheckPointOutputManager| object's
\verb|create_section_writer| member function.
\item Activate the \verb|SectionedOutputStream| object.
This activation will fail if some other \verb|SectionedOutputStream|
is currently active.
\item Write to the \verb|SectionedOutputStream| object using
the standard \Cplusplus stream insertion operator.
\item Deactivate the \verb|SectionedOutputStream| object.
\end{itemize}
\item An abbreviated version of the above follows.
\begin{codeblock}
SectionedOutputStream writer (
   checkpoint_writer->create_section_writer (section_id));
if (! writer.activate()) {
   // Handle error
}
while (! done_writing) {
   writer << more_stuff_to_write();
}
writer.deactivate();
\end{codeblock}

\item The output file is closed with the destruction of the
\verb|CheckPointOutputManager| object.
\end{itemize}


The class \verb|CheckPointInputManager| provides the ability to read
from a previously created partitioned checkpoint file.
Programmatic users of this class should:
\begin{itemize}
\item Create a \verb|CheckPointInputManager| object via the non-default
constructor. The non-default constructor takes three arguments:
the name of the file,
the string that denotes the start of a section, and
the string that denotes the end of a section.
This opens a \Cplusplus input stream to  the specified checkpoint file.
That stream is private; it is accessed indirectly by
\verb|SectionedInputStream| objects created by the
\verb|CheckPointInputManager| object.
\begin{codeblock}
checkpoint_reader = new CheckPointInputManager (
   output_file_name, section_start, section_end);
\end{codeblock}

\item For each section to be read,
\begin{itemize}
\item Create a \verb|SectionedInputStream| object
by calling the \verb|CheckPointInputManager| object's
\verb|create_section_reader| member function.
\item Activate the \verb|SectionedInputStream| object.
This activation will fail if some other \verb|SectionedInputStream|
is currently active.
\item Read from the \verb|SectionedInputStream| object using
\verb|std::getline| or
the \Cplusplus stream insertion operator.
\item Deactivate the \verb|SectionedInputStream| object.
\end{itemize}
\item An abbreviated version of the above follows.
\begin{codeblock}
SectionedInputStream reader (
   checkpoint_writer->create_section_reader (section_id));
if (! reader.activate()) {
   // Handle error
}
while (std::getline (reader, line)) {
   process_line(line);
}
reader.deactivate();
\end{codeblock}

\item The input file is closed with the destruction of the
\verb|CheckPointInputManager| object.
\end{itemize}


\subsection{Extending the Simulation Interface Model}
This section describes how to extend the data types defined by the
\ModelDesc.

The classes \verb|JeodSimulationInterface|,
\verb|JeodMemoryInterface|, and \verb|IntegratorInterface|
declares several pure virtual functions.
Usable derived class must provide implementations
of these pure virtual functions.
Dummy implementations can be provided for several of
these functions. For example, address to name and name to address
translation is only used during checkpoint and restart. There is no
need to provide a full-blown implementation of this functionality
if the target usage does not require checkpoint/restart.

A compliant instantiable class that derives from
\verb|JeodSimulationInterface| must contain
one instance of a class that derives from \verb|MessageHandler|,
one instance of a class that derives from \verb|JeodMemoryInterface|, and
one instance of a class that derives from \verb|JeodMemoryManager|.
These objects must be constructed in the above order
and destructed in the reverse order.
