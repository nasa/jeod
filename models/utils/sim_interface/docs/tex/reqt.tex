%%%%%%%%%%%%%%%%%%%%%%%%%%%%%%%%%%%%%%%%%%%%%%%%%%%%%%%%%%%%%%%%%%%%%%%%%%%%%%%%
% reqt.tex
% Requirements on the Simulation Interface Model
%
% 
%%%%%%%%%%%%%%%%%%%%%%%%%%%%%%%%%%%%%%%%%%%%%%%%%%%%%%%%%%%%%%%%%%%%%%%%%%%%%%%%

%----------------------------------
\chapter{Product Requirements}\hyperdef{part}{reqt}{}
\label{ch:reqt}
%----------------------------------

\requirement{Top-level requirement}
\label{reqt:toplevel}
\begin{description:}
\item[Requirement]
  The \ModelDesc shall meet the JEOD project requirements specified in
  the \JEODid\
  \hyperref{file:\JEODHOME/docs/JEOD.pdf}{part1}{reqt}
           {JEOD top-level document}.
\item[Rationale]
  This is a project-wide requirement.
\item[Verification]
  Inspection 
\end{description:}

\requirement{Hidden Data Visibility}
\label{reqt:hidden_data_visibility}
\begin{description:}
\item[Requirement]
  The \ModelDesc shall provide a generic (simulation engine agnostic) mechanism
  by which a model developer can make protected and private data visible to the
  simulation engine.
\item[Rationale]
  Simulation users at times need access to protected and private data.
  As one simulation user said, ``Murphy's Law laughs at and cosmic rays don't
  obey data encapsulation. I need to see, and sometimes modify, your hidden
  data. Tell me I can't have such access and my project won't use your tool.''

  Another motivating factor is the need to make JEOD checkpointable and
  restartable. The Trick simulation engine do most of the work with regard
  to checkpoint and restart if hidden data are visible to Trick. Making
  hidden data visible drastically reduces the burden of making JEOD
  checkpointable and restartable.
\item[Verification]
  Inspection, Test 
\end{description:}


\requirement{Allocated Data Visibility}
\label{reqt:allocated_data_visibility}
\begin{description:}
\item[Requirement]
  The \ModelDesc shall provide a generic (simulation engine agnostic) mechanism
  by which data allocated via the \MEMORY\ can be made visible to the
  simulation engine.
\item[Rationale]
  The same factors that motivated requirement~\ref{reqt:hidden_data_visibility}
  apply to this requirement.
\item[Verification]
  Inspection, Test 
\end{description:}


\requirement{Simulation Engine Interface}
\label{reqt:sim_engine_interface}
\begin{description:}
\item[Requirement]
  With one exception, the \ModelDesc shall be the only JEOD model that
  directly invokes simulation engine functions
  or that directly uses simulation engine data constructs.
\item[Exception]
  The \DYNMANAGER\ is allowed to use Trick 7 specific data constructs
  with regard to integration.
\item[Rationale]
  The need to support the Trick environment resulted in an ever-growing
  number of Trick-specific constructs sprinkled throughout JEOD.
  These widely dispersed Trick-specific constructs threatened to make JEOD
  unusable outside of the Trick environment.
  This growth in Trick-specific constructs and the associated risk of
  failing to be usable outside of Trick were the key factors that motivated
  the the creation of the \ModelDesc.
\item[Verification]
  Inspection, Test 
\end{description:}

\requirement{Integration Interface}
\label{reqt:integ_interface}
\begin{description:}
\item[Requirement]
  The \ModelDesc shall provide:
  \subrequirement{Generic.}
   Simulation engine agnostic mechanisms to make
  JEOD integration operable within a generic simulation environment.
  \subrequirement{Trick specific.}
  Trick-specific implementations of these mechanisms to make
  JEOD integration operable within the Trick simulation environment.
\item[Rationale]
  This is a derived requirement of requirement~\ref{reqt:sim_engine_interface}.

  JEOD integration can be used in simulation engines such as Trick
  that treat integration as an abstract process, but this requires
  communications between integration as performed by JEOD and
  integration as performed by the simulation engine.
\item[Verification]
  Inspection, Test 
\end{description:}


\requirement{Job Cycle}
\label{reqt:job_cycle}
\begin{description:}
\item[Requirement]
  The \ModelDesc shall provide a mechanism to obtain the time interval between
  successive calls by the simulation engine to the currently executing function.
\item[Rationale]
  This is a derived requirement of requirement~\ref{reqt:sim_engine_interface}.

  The physical processes modeled in some JEOD models require knowledge
  of the rate at which the process model is called.
\item[Verification]
  Inspection, Test 
\end{description:}


\requirement{Trick Message Handler}
\label{reqt:trick_message_handler}
\begin{description:}
\item[Requirement]
  The \ModelDesc shall provide a Trick-based implementation of the abstract
  class MessageHandler.
\item[Rationale]
  This is a derived requirement of requirement~\ref{reqt:sim_engine_interface}.

  None of the verification simulations would build without a functional
  message handler.
\item[Verification]
  Inspection, Test 
\end{description:}

\requirement{Checkpoint/\hspace{0pt} Restart}
\label{reqt:checkpoint_restart}
\begin{description:}
\item[Requirement]
  The \ModelDesc shall provide mechanisms that enable JEOD-based simulations
  to be checkpointable and restartable in the Trick 10 environment.
\item[Rationale]
  This is a requirement mandated by external users of JEOD.
\item[Verification]
  Inspection, Test 
\end{description:}


\requirement{Address/Name Translation}
\label{reqt:addr_name_xlate}
\begin{description:}
\item[Requirement]
  The \ModelDesc shall provide mechanisms to translate an address
  to a simulation engine symbolic name and to translate a symbolic name
  to an address.
\item[Rationale]
  This is a derived requirement of
  requirements~\ref{reqt:sim_engine_interface}
  and~\ref{reqt:checkpoint_restart},
  levied on the model by the \CONTAINER.
\item[Discussion]
  This capability is only used when the simulation interface object
  indicates that checkpoint/restart are enabled. An implementation that
  does not support checkpoint/restart can provide dummy implementations
  of this functionality.
\item[Verification]
  Inspection, Test 
\end{description:}



\requirement{Multiple Integration Groups}
\label{reqt:multiple_integ_groups}
\begin{description:}
\item[Requirement]
  The \ModelDesc shall provide a multiple integration group capability
  in the Trick 10 environment. In particular,
  \subrequirement{Multiple rates.}
  The model shall provide the ability to
  integrate different dynamic bodies at different rates / with different
  integration techniques.
  \subrequirement{Switchability.}
  The model shall provide the ability to
  move a dynamic body from one integration group to another.
\item[Rationale]
  This is a requirement mandated by external users of JEOD.
\item[Verification]
  Inspection, Test 
\end{description:}


\requirement{Extensibility}
\label{reqt:extensibility}
\begin{description:}
\item[Requirement]
  The \ModelDesc shall be architected in such a manner that
  allows it to be applied to simulation engines other than Trick.
\item[Rationale]
  That JEOD is to be usable outside of the Trick environment
  is a top-level JEOD requirement.
\item[Verification]
  Inspection
\end{description:}
