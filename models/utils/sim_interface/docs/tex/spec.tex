%%%%%%%%%%%%%%%%%%%%%%%%%%%%%%%%%%%%%%%%%%%%%%%%%%%%%%%%%%%%%%%%%%%%%%%%%%%%%%%%
% spec.tex
% Specification of the Simulation Interface Model
%
%
%%%%%%%%%%%%%%%%%%%%%%%%%%%%%%%%%%%%%%%%%%%%%%%%%%%%%%%%%%%%%%%%%%%%%%%%%%%%%%%%

\chapter{Product Specification}\hyperdef{part}{spec}{}\label{ch:spec}

\section{Conceptual Design}
\label{sec:conceptual_design}

\subsection{Overview}

JEOD must be usable in the Trick 13 environment,
and should be usable in other environments as well.
The decision to make the \ModelDesc was mostly an internal decision.
The model consolidates the interfaces between JEOD and Trick,
and potentially with other simulation engines, in one place.

The \ModelDesc is implemented in the form of
header files that configure JEOD for use in a particular simulation environment,
header files the define macros,
and header and source files that define classes.
The common theme that connects these disparate items is
that they represent interfaces between JEOD and the simulation engine.
These interfaces take a number of forms:
\begin{itemize}
\item Making a class's protected and private data members
  visible to the simulation engine.
\item Making data items allocated by JEOD models
  visible to the simulation engine.
\item Translating addresses to and from simulation engine symbolic names.
\item Checkpointing and restarting data allocations and \CONTAINER\ objects.
\item Making JEOD integration work within the context
  of a simulation engine's integration scheme.
\item Obtaining the rate at which the simulation engine calls
  the currently executing function.
\end{itemize}

The Trick environment is the primary implementation target of JEOD.
A number of the model header and source files pertain specifically to the Trick
simulation environment.

In addition to the above interfaces, the model provides a
Trick 13-specific capability to support multiple rates of integration.

\subsection{Configuration}
Some of the header files provided with the model configure JEOD for use in
a particular simulation environment. These configuration headers define
a set of \Cplusplus macros. Some of these macros are used in other model
header and source files. Others provide portable type definitions that can
be used by  model developers.

\subsection{Exposing Protected/Private Data Members to the Simulation Engine}
All JEOD models must be restartable from a previously created set of
checkpoint files. JEOD models rely primarily upon Trick 13 mechanisms to provide
this checkpoint / restart capability. Making the protected and private data
members of a class are visible to Trick enables Trick to checkpoint a large
portion of the members of all class instances that are visible to Trick.

To this end, the \ModelDesc provides two macros,
\verb|JEOD_MAKE_SIM_INTERFACES| and \verb|JEOD_NOMINALLY_PRIVATE|.
The former makes all of a class's data members visible
to the simulation engine.
All JEOD classes that have protected or private member data invoke this macro.
The latter is for templates that are base classes of various JEOD classes.

\subsection{Interfacing JEOD and Simulation Engine Integration}
JEOD has a well-developed concept of how to integrate the equations of motion,
but so do many simulation engines. The \ModelDesc provides mechanisms to
make JEOD integration work within the context of how the simulation engine
performs integration.

\subsection{Interfacing JEOD and Simulation Engine Memory Management}
The constructors and initialization-time functions of a number of the JEOD
models allocate memory dynamically. These allocations, along with registrations
of containers that need to be checkpointed and restarted, are the purview
of the JEOD \MEMORY. The allocations and type descriptions must be coordinated
with the simulation engine. The \ModelDesc relays information about the
allocations to the simulation engine and provides the memory model with
simulation engine descriptions of data types.

\subsection{Checkpoint/Restart}

JEOD checkpoint/restart capabilities come in two basic forms:
A generic checkpoint/restart text-based capability,
and a Trick 13-specific checkpoint restart capability that uses this generic
capability to augment Trick 13 checkpoint restart.

The generic capability extends \Cplusplus I/O to provide the ability to write
and read a partitioned checkpoint file. Each partition in the file looks like
a separate file from the perspective of start and end of file markers.
This partitioning reduces the number of files that need to be managed.

The Trick 13-specific capability augments existing Trick 13 text-based
checkpoint/restart, writing checkpoint data to a pair of partitions
in a partitioned JEOD checkpoint file during checkpoint
and restoring content from that checkpoint file on restart.
While Trick 13 can checkpoint and restore the contents of
externally allocated data,  it cannot checkpoint and restore the
allocations themselves. One of the partitions in the JEOD checkpoint file
describes the data allocated by JEOD models. The \CONTAINER\ Checkpointable
objects used throughout JEOD are opaque to the simulation engine.
The other partition in the JEOD checkpoint file contains the contents
of those Checkpointable objects.

\subsection{Trick Message Handler}
The \verb|MessageHandler| class is an abstract class.
The \ModelDesc provides a Trick-specific classes that derive from
the \verb|MessageHandler| and that are instantiable.

\subsection{Multiple Integration Groups}
Providing the ability to support multiple rates of integration, with each
rate group using its own integration technique, has been a long-standing desire
within the JEOD project.
Until recently the Trick environment limited the extent to which this could be
accomplished and there was no external impetus to provide this ability.
Trick 13 now provides the necessary infrastructure to support this feature,
and an external request for this feature now exists.
The \ModelDesc provides a Trick 13-specific
multiple integration group capability.

\section{Key Algorithms}
\label{sec:algorithms}

\subsection{Checkpoint/Restart}
The JEOD project assumed that the bulk of the checkpoint and restart
efforts are provided by the simulation engine. The two exceptions to
this are data allocated by JEOD and classes with data members that
are opaque to the simulation engine.

The \ModelDesc addresses these issues by recording the allocations and
opaque content in a segmented JEOD checkpoint file. The allocations and opaque
content are restored at restart time based on the contents of the
specified JEOD checkpoint file.
The algorithms for checkpointing and restoring Checkpointable objects
is specified in the \CONTAINER.

\subsection{Trick Name Lookup}
JEOD uses Trick name lookup to convert addresses to names at checkpoint
time and to convert names to addresses at restart time. The mechanisms
for performing these activities were the result of mutually agreed upon
negotiations between the Trick and JEOD development teams.

\subsection{Trick Memory Registration}
JEOD registers allocated memory and deregisters freed memory with Trick
per interfaces described in the Trick documentation.


\section{Interactions}
\label{sec:interactions}

\subsection{JEOD Models Used by the \ModelDesc}
The \ModelDesc uses the following JEOD models:
\begin{itemize}
\item\hypermodelref{CONTAINER}.\\
The Trick 13 memory interface maintains a registry of
\CONTAINER\ \verb|JeodCheckpointable| objects.
The model uses this registry as the basis for
checkpointing the contents of the checkpointable objects at checkpoint time
and for restoring the contents of the checkpointable objects at restart time.

\item\hypermodelref{DYNMANAGER} and \hypermodelref{DYNBODY}.\\
The JEOD-agnostic aspect of the Trick 13-specific multiple integration group
capability integrates a collection of Trick 13 simulation objects.
The JEOD-aware aspect of this capability augments this JEOD-agnostic capability,
using a \verb|DynamicsIntegrationGroup| object to integrate all
of the \verb|DynBody| objects contained in the subject simulation objects.
The set of \verb|DynBody| to be integrated is formed by interactions with the
simulation's \verb|DynManager| object.

\item\hypermodelref{INTEGRATION}.\\
The Trick simulation engine treats integration a special class of jobs.
Trick makes integration abstract. It repeatedly calls derivative class jobs and
then integration class jobs until the integration jobs indicate completion.
The \ModelDesc enables communication between the simulation engine and
JEOD integration in the form of the abstract class IntegratorInterface and
Trick-specific classes that derive from IntegratorInterface.

\item\hypermodelref{MEMORY}.\\
Every JEOD-based simulation must contain exactly one instance of a class that
derives from \verb|JeodSimulationInterface|. This class must contain or
construct an instance of a class that derives from \verb|JeodMemoryManager|.

The \ModelDesc reacts to calls from the memory manager regarding
allocations/deallocations of memory and registrations/deregistrations
of checkpointable objects.

\item\hypermodelref{MESSAGE}.\\
The \ModelDesc  mandates that a compliant \verb|JeodSimulationInterface|
must either contain or construct an instance of a class that derives from
\verb|MessageHandler|. This message handler must be created prior to the
memory manager described above.

As do almost all other models,
the \ModelDesc uses the message handler to generate messages.

\item\hypermodelref{NAMEDITEM}.\\
The Trick 13-specific memory interface calls \verb|NamedItem::demangle| to
demangle type names.

\item\hypermodelref{TIME}.\\
The JEOD-aware aspect of the Trick 13-specific multiple integration group
capability uses the simulation's \verb|TimeManager| object to create the
integration group and to update time at the start of an integration loop.
\end{itemize}

\subsection{JEOD Models That Use the \ModelDesc}
All JEOD models use the \ModelDesc to make data members visible to the
simulation engine. In addition to these uses,
\begin{itemize}
\item\hypermodelref{CONTAINER}.\\
The Container Model uses the \ModelDesc to translate addresses to
and from symbolic names.

\item\hypermodelref{DYNMANAGER}.\\
The \DYNMANAGER\ integrates the dynamic aspects of a typical JEOD-based
simulation.
It uses the \INTEGRATION\ to perform the integration, with the simulation engine
mediating the integration process.
Interactions between JEOD simulation and the simulation engine are handled
generically by the \ModelDesc class \verb|JeodIntegratorInterface|.

The \verb|DynManager| class contains an element of type
\verb|JEOD_SIM_INTEGRATOR_POINTER_TYPE|.
This macro defines a pointer to a simulation engine-specific integrator
structure, \verb|Trick::Integrator| in the case of a Trick-based simulation.
The Trick integration class job that calls the \verb|DynManager| instance's
integrate function should supply this member as the \verb|sup\_class\_data|
for the integration class job.

\item\hypermodelref{INTEGRATION}.\\
The Integration Model uses a \ModelDesc \verb|JeodIntegratorInterface| object to
coordinate integration as performed by JEOD with integration as
performed by the simulation engine.

\item\hypermodelref{MEMORY}.\\
The Memory Model uses the \ModelDesc to
\begin{itemize}
\item Coordinate JEOD and simulation engine descriptions of data types,
\item Register/deregister allocated memory with the simulation engine, and
\item Register/deregister Container Model objects with the \ModelDesc.
\end{itemize}
\end{itemize}


\section{Detailed Design}

\subsection{Configuration}
The header file \verb|config.hh| includes an installation-specific
configuration file that defines a set of macros used in other model headers.
The model provides installation-specific configuration files for
Trick 13 and for the Trick-independent test harness.

Developers who wish to use JEOD in a non-Trick simulation environment
must develop a custom installation-specific configuration file.

\subsection{Exposing Protected/Private Data Members to the Simulation Engine}
In a Trick environment, the macro \verb|JEOD_MAKE_SIM_INTERFACES|
declares \verb|InputProcessor| as a friend class and declares the
auto-generated \code{init\_attr\inanglebrackets{class\_name}}
as a friend function. To handle namespaces, the macro is overloaded to take in
up to two namespaces as arguments prior to the class name, and they are included
in the name of the auto-generated friend function.
Declaring \verb|InputProcessor| as a friend class signals Trick's ICG facility
to process protected and private data members. Declaring the function
generated by ICG as a friend function is needed to make that auto-generated
code compilable.

Developers who wish to use JEOD in a non-Trick simulation environment
that requires visibility to protected and private data must define a
custom version of \verb|JEOD_MAKE_SIM_INTERFACES|.


\subsection{The JEOD Simulation Interface Object}
Every JEOD-based simulation must have exactly one instance of a class that
that inherits from the abstract class \verb|JeodSimulationInterface|.
\subsubsection{Class JeodSimulationInterface}
This abstract class defines the basis for the interface between JEOD and a
simulation engine. A compliant derived class must contain one instance each
of a class that derives from \verb|MessageHandler| and a class that derives from
\verb|JeodMemoryManager|. The \verb|MessageHandler| object must be constructed
before the \verb|JeodMemoryManager| object; destruction must be performed in
reverse order.

\subsubsection{Class JeodSimulationInterfaceInit}
This class defines configuration data that can be used to
configure the \verb|JeodSimulationInterface| object's
message handler and memory manager data members.

\subsubsection{Class BasicJeodTrickSimInterface}
This class implements the required capabilities of the
generic \verb|JeodSimulationInterface| in a Trick simulation environment.
The class contains a reference to a message handler,
a memory interface object, and a memory manager object.
The message handler, being a reference, must be created before
the \verb|BasicJeodTrickSimInterface| object is constructed.

The Trick 13 \verb|JEODSysSimObject| provided with JEOD contains a
\verb|BasicJeodTrickSimInterface| data member. The MessageHandler object,
which defaults to a \verb|TrickMessageHandler| object, is passed as an argument
to the \verb|BasicJeodTrickSimInterface| non-default constructor.

\subsubsection{Class JeodTrickSimInterface}
This class derives from \verb|TrickMessageHandlerMixin| and
\verb|BasicJeodTrickSimInterface|.
The order of the inheritance ensures that the message handler is
constructed prior to the memory interface and the memory manager members.
Note that because the class \verb|TrickMessageHandlerMixin| derives from the
the TrickMessageHandler class, using a \verb|JeodTrickSimInterface| precludes
the use of a custom message handler. Developers who want to use their own
message handler should use the \verb|BasicJeodTrickSimInterface| rather than
the \verb|JeodTrickSimInterface|.

\subsection{Interfacing JEOD and Simulation Engine Integration}
\subsubsection{Configuration Macros}
The configuration file should define a series of macros that encapsulate
concepts related to integration. These integration configuration macros are:
\begin{description}
\item[\tt JEOD\_SIM\_INTEGRATOR\_FORWARD]
provides a forward declaration to the simulation engine's
integration data structure. This macro does not need to be defined
if such a forward declaration isn't needed.
In a Trick setting, this macro is defined via\newline
{\tt \#define JEOD\_SIM\_INTEGRATOR\_FORWARD %\textbackslash\newline
namespace Trick \{ class Integrator; \}}

\item[\tt JEOD\_SIM\_INTEGRATOR\_POINTER\_TYPE]
defines a pointer type that points to an instance of the
simulation engine's integration data structure.
In a Trick setting, this macro is defined via\newline
{\tt \#define JEOD\_SIM\_INTEGRATOR\_POINTER\_TYPE Trick::Integrator *}

\item[\tt JEOD\_SIM\_INTEGRATOR\_ENUM]
defines an enumeration or integral type that identifies
the type used by the simulation engine to identify an integration technique.
In a Trick setting, this macro is defined via\newline
{\tt \#define JEOD\_SIM\_INTEGRATOR\_ENUM Integrator\_type}
\end{description}

\subsubsection{Class JeodIntegratorInterface}
This abstract class derives from the \verb|er7_utils::IntegratorInterface|
class. The base class defines interfaces that the ER7 Utilities Integration
model requires of an interface to the simulation engine's integration module.
The class \verb|JeodIntegratorInterface| adds two additional pure virtual
interfaces used within JEOD.

Developers who wish to use JEOD in a non-Trick simulation environment
must define a custom class that derives from \verb|JeodIntegratorInterface|
to interface their simulation engine's concept of integration
with JEOD's concept of integration.

\subsection{Class TrickJeodIntegrator}
This class extends the \verb|Trick::Integrator| class for internal use by
JEOD. The class is needed because the Trick integration loop requires that
integration class jobs be associated with a Trick integration structure.
The loop uses this structure for various control purposes such as
determining whether the derivative jobs are to be called on the first
pass through the integration loop.

The class \verb|Trick::Integrator| declares two pure virtual member
functions, \verb|initialize| and \verb|integrate|.
Even though these functions are never used by JEOD integration, they
must be defined to make the class instantiable.
The class \verb|TrickJeodIntegrator| implements these as dummy functions.

\subsection{Class JeodTrickIntegrator}
This class derives from the \verb|JeodIntegratorInterface| class.
It implements all pure virtual interfaces declared in and inherited by
the parent class.
The class also contains a \verb|TrickJeodIntegrator| data member.
This inheritance and containment enables a \verb|JeodTrickIntegrator| instance
to be used as a common basis for Trick and ER7 Utilities integration.

\subsection{Interfacing JEOD and Simulation Engine Memory Management}
\subsubsection{Class JeodMemoryInterface}
This abstract class specifies several pure virtual methods that are needed to
interface the JEOD memory manager with the simulation engine. These include
\begin{itemize}
\item Functions pertaining to the JEOD versus simulation engine
representations of data types.
\item Functions for registering/deregistering JEOD data allocations
with the simulation engine.
\item Functions for translating addresses to and from symbolic names.
\item Functions for registering/deregistering JEOD container objects
with the simulation interface.
\item Functions for checkpointing/restoring those registered  container objects
at checkpoint or restart time.
\end{itemize}

Developers who wish to use JEOD in a non-Trick simulation environment
must define a custom class that derives from JeodMemoryInterface
to interface their simulation engine's concept of memory with JEOD.

\subsubsection{Class JeodTrickMemoryInterface}
This class derives from \verb|JeodMemoryInterface|. It provides implementations
of all of the pure virtual functions declared in the parent class.
Several of these are dummy implementations based on the assumption
that a \verb|JeodSimulationInterface| by default is not
checkpointable/restartable.

\subsubsection{Class JeodTrick10MemoryInterface}
This class derives from \verb|JeodTrickMemoryInterface|. It overrides the dummy
implementations provided in \verb|JeodTrickMemoryInterface| to make a
Trick 13 / JEOD-based simulation fully checkpointable and restartable.


\subsection{Checkpoint/Restart}
This subsection describes the classes that provide a generic checkpoint/restart
capability to create and restore from a partitioned checkpoint file.
A partitioned checkpoint file comprises comprises multiple sections delineated
by section markers, something along the lines of the following:
\begin{codeblock}
// ++++++++++++++++++++++++++++++++++++++++++ Start of section foo
Contents of section foo (multiple lines)
// ------------------------------------------ End of section foo

// ++++++++++++++++++++++++++++++++++++++++++ Start of section bar
Contents of section bar (multiple lines)
// ------------------------------------------ End of section bar

// ++++++++++++++++++++++++++++++++++++++++++ Start of section baz
Contents of section baz (multiple lines)
// ------------------------------------------ End of section baz
\end{codeblock}

A section within a checkpoint file looks like a ``file'' to the
external users of this capability. The writers automatically create
the section markers when an output section is opened and later closed.
The readers make the individual sections look like a ``file''.
The pseudo file appears to start just after the start of the section marker
and end just before the end of section marker.

The classes that comprise this capability take advantage of
the fact that \Cplusplus I/O streams and buffers,
unlike the \Cplusplus containers, were designed for extensibility.
Users of this capability can use the standard \Cplusplus stream insertion
and stream extraction operators to write to and read from a section
of the checkpoint file.


\subsubsection{Class CheckPointOutputManager}
The class \verb|CheckPointOutputManager| provides the ability to create
an output partitioned checkpoint file.
The non-default constructor creates a \Cplusplus output file stream to the
specified checkpoint file.
This stream is private to the \verb|CheckPointOutputManager| instance.
This stream is accessed indirectly via a \verb|SectionedOutputStream|
object.
The \verb|CheckPointOutputManager| member function \verb|create_section_writer|
creates a \verb|SectionedOutputStream|.

\subsubsection{Class SectionedOutputStream}
A \verb|SectionedOutputStream| is a \verb|std::ostream| that writes a section
of a partitioned checkpoint file. This class automatically writes the
start and end markers of the checkpoint file section.

The current implementation is a bit spare.
It does not provide buffering,
and it does not support \verb|seekp| or \verb|tellp|
to reposition the stream.
It does support the stream insertion operator.
It is this operator that external users should use to
write the contents of the checkpoint file section.

Note that most of the content of this class is private.
This class is not extensible and is intended to be used within the
context of a \verb|CheckPointOutputManager|.

\subsubsection{Class SectionedOutputBuffer}
A \verb|SectionedOutputBuffer| is a \verb|std::streambuf| that writes a section
of a partitioned checkpoint file. This stream buffer is created and used
by a \verb|SectionedOutputStream| object.

Note that with the exception of the destructor and the inherited members
from \verb|std::streambuf|, everything in this class is private.
This class is not extensible.

\subsubsection{CheckPointInputManager}
A \verb|CheckPointInputManager| provides the ability to read from a
previously created partitioned checkpoint file.
The non-default constructor creates a \Cplusplus input file stream to the
specified checkpoint file.
This stream is private to the \verb|CheckPointInputManager| instance.
This stream is accessed indirectly via a \verb|SectionedInputStream|
object.
The \verb|CheckPointOInputManager| member function \verb|create_section_reader|
creates a \verb|SectionedInputStream|.


\subsubsection{SectionedInputStream}
A \verb|SectionedInputStream| is a \verb|std::istream| that reads from a
section in a partitioned checkpoint file. This class indicates EOF when the
input pointer in the checkpoint file file buffer reaches the end of the section.

The current implementation is a bit spare.
It does not provide buffering,
it does not support operations that require buffering
such as \verb|peek|, \verb|putback|, and \verb|unget|,
and it does not support \verb|seekg| or \verb|tellg|
to reposition the stream.
It does support \verb|std::getline| the stream extraction operator.
It is this function or this operator that external users should use to
read the contents of the checkpoint file section.

Note that most of the content of this class is private.
This class is not extensible and is intended to be used within the
context of a CheckPointInputManager.

\subsubsection{SectionedInputBuffer}
A \verb|SectionedInputBuffer| is a \verb|std::streambuf| that reads from a
section in a partitioned checkpoint file. This class indicates EOF when the
input pointer in the checkpoint file file buffer reaches the end of the section.

Note that with the exception of the destructor and the inherited members
from \verb|std::streambuf|, everything in this class is private.
This class is not extensible.

\subsection{Trick Message Handler}
\subsubsection{TrickMessageHandler}
This class is an instantiable class that derives from the
\verb|SuppressedCodeMessageHandler| class, which in turn derives from
the \verb|MessageHandler| class.
The \verb|TrickMessageHandler| provides an implementation of the
\verb|process_message| member function declared as pure virtual in the base
\verb|MessageHandler| class.

JEOD classes that contain container objects should in general register those
containers with the simulation interface at construction time.
The TrickMessageHandler cannot do that because it needs to be constructed
prior to the memory interface and memory manager.
The class instead provides the method \verb|register_contents| which is
called by the \verb|BasicJeodTrickSimInterface| constructor.

\subsubsection{TrickMessageHandlerMixin}
This class contains a \verb|TrickMessageHandler| as a protected data member.
The class \verb|JeodTrickSimInterface| inherits from this class and from
\verb|BasicJeodTrickSimInterface|, in that order.


\subsection{Multiple Integration Groups}
The \ModelDesc provides a Trick 13-specific multiple integration group
capability. In JEOD 2.2, this capability was split into a JEOD-agnostic and
JEOD-aware parts. The JEOD-agnostic part is now a part of Trick 13. The
only aspect left in JEOD is the JEOD-aware part of the JEOD 2.2 implementation.

\subsubsection{JeodDynbodyIntegrationLoop}
This class inherits from the \verb|Trick::IntegLoopScheduler| class
and from the \INTEGRATION\ \verb|IntegrationGroupOwner| class to
 provide the JEOD-aware aspect of the multiple integration group
capability.
A \verb|JeodDynbodyIntegrationLoop| object integrates all of the
\verb|DynBody| objects that are contained in the sim objects to be integrated
by that \verb|JeodDynbodyIntegrationLoop| object.

A DynBody object that is an element of a Trick sim object is automatically moved
from one integration loop to another when the sim object that contains the
DynBody is moved to a different integration loop.
Note well: JEOD does not support the concept of DynBody objects that are not
members of a Trick simulation object. DynBody objects that are dynamically
allocated outside of the context of a containing Trick simulation object is
undefined behavior.

A \verb|JeodDynbodyIntegrationLoop| object also integrates all non-DynBody
integrable objects managed by the loop. These non-DynBody integrable objects
can be added to or removed from the integration loop directly via calls to
\verb|JeodDynbodyIntegrationLoop::add_integrable_object| and
\verb|JeodDynbodyIntegrationLoop::remove_integrable_object|.

The recommended practice is to instead associate integrable objects with
a DynBody object. The integrable objects associated with a DynBody object
are automatically added to the dynamics integration group that integrates
the DynBody object. In a future release or patch to JEOD, these associated
integrable objects will be moved along with the DynBody object as the
DynBody object is moved from one integration loop to another.

\clearpage
\boilerplateinventory
